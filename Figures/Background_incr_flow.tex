\tikzstyle{block} = [draw,fill=blue!20,node distance = 3.2cm, minimum size=2em]

% diameter of semicircle used to indicate that two lines are not connected
\tikzstyle{branch}=[fill,shape=circle,minimum size=3pt,inner sep=0pt]

\begin{tikzpicture}[>=latex']

    % Draw blocks, inputs and outputs

    % \foreach \y in {1,2,3,4,5} {
        % blocks
        \node[block] at (2,-1) (prod) {$SAT$};
        \node[block, name=stging, below=3.75cm of prod] {$SAT$};
        \node[block, name=devel, below=2.0cm of stging] {$SAT$};

        % input nodes
        % \node[left of = devel, xshift=-15pt] (input1) {$development$};
        % \node[left of = stging] (input2) {$\turnBlue{staging}$};
        \node[left of =prod,xshift=-10pt] (input3) {\pV{}};

        % inputs
        \draw[->] (input3) -- (prod);
        \draw[->] (prod) -- node[xshift = 70pt, align=left] {%
          $ \begin{aligned}
            \texttt{pop}  &\quad \eV{}\\
            \texttt{pop}  &\quad \cV{}\\
            \texttt{pop}  &\quad \bV{}\\
            \texttt{push} &\quad (\bV{} \vee \neg \iV{})  \\
            \texttt{push} &\quad \cV{} \\
            \texttt{push} &\quad (\gV{} \rightarrow \cV{}) \\
          \end{aligned}
          $
        } (stging);
\draw[->] (stging) -- node[xshift = 95pt, align=left] {%
  $\begin{aligned}
    & \texttt{resetAssertionStack}\\
    &\texttt{push} \quad \zV{} \leftrightarrow (a \wedge b \wedge c \wedge e)
    % \text{pop} &\quad \kf{pti} \rightarrow c_{i.1}\\
    % \text{pop} &\quad (\kf{spectre\_v2} \vee \kf{l1tf}) \leftrightarrow \\&\quad (c_{0} \wedge (\kf{nospec\_store\_bypass\text{-}disable} \rightarrow f_{j})\\
    % \text{pop} &\quad (c_{0.0} \wedge c_{1} \wedge \ldots c_{n})\\
  \end{aligned}
  $
} (devel);

        % outputs
        \draw[->] (prod.east) -- +(1.0,0);
        \draw[->] (stging.east) -- +(1.0,0);
        \draw[->] (devel.east) -- +(1.0,0);


        % \node at (3.5,-3) (result1) {$result$};
        % \node[right of = stging, xshift=15pt] (input2) {$result$};
        % \node[right of = devel, xshift=15pt] (input2) {$result$};
        % \node at (3.5,-1) (input3) {$result$};

        \node[right of = stging, xshift=38pt] (input2) {$result_{\qV{}}$};
        \node[right of = devel, xshift=38pt] (input2) {$result_{\rV{}}$};
        \node[xshift=19pt] at (3.8,-1) (input3) {$result_{\pV{}}$};
    % }
    % \node[block] at (2,-6) (block6) {$f_6$};
    % \draw[->] (block6.east) -- +(0.5,0);

    % % Calculate branch point coordinate
    % \path (input1) -- coordinate (branch) (block1);

    % % Define a style for shifting a coordinate upwards
    % % Note the curly brackets around the coordinate.
    % \tikzstyle{s}=[shift={(0mm,\radius)}]
    % % It would be natural to use the yshift or xshift option, but that does
    % % not seem to work when shifting coordinates.

    % \draw[->] (branch) node[branch] {}{ % draw branch junction
    %         \foreach \c in {2,3,4,5} {
    %             % Draw semicircle junction to indicate that the lines are
    %             % not connected. The intersection between the lines are
    %             % calculated using the convenient -| syntax. Since we want
    %             % the semicircle to have its center where the lines intersect,
    %             % we have to shift the intersection coordinate using the 's'
    %             % style to account for this.
    %             [shift only] -- ([s]input\c -| branch) arc(90:-90:\radius)
    %             % Note the use of the [shift only] option. It is not necessary,
    %             % but I have used it to ensure that the semicircles have the
    %             % same size regardless of scaling.
    %         }
    %     } |- (block6);
\end{tikzpicture}