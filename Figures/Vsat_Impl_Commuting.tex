\begin{tikzpicture}
  \matrix (m) [matrix of math nodes,row sep=3em,column sep=4em,minimum width=2em]
  {
    F_t(x) & F(x) \\
    A_t & A \\};
  \path[-stealth]
  (m-1-1) edge node [left] {$\mathcal{B}_X$} (m-2-1)
  edge [double] node [below] {$\mathcal{B}_t$} (m-1-2)
  (m-2-1.east|-m-2-2) edge node [below] {$\mathcal{B}_T$}
  node [above] {$\exists$} (m-2-2)
  (m-1-2) edge node [right] {$\mathcal{B}_T$} (m-2-2)
  edge [dashed,-] (m-2-1);
\end{tikzpicture}




% \begin{tikzpicture}
%   \begin{scope}[every node/.style={circle,thick,draw}]
%     \node (A) at (0,0) {A};
%     % \node (B) at (0,3) {B};
%     \node (B) [right =of A] {B};
%     \node (C) [below =of A] {C};
%     \node (D) [right =of C] {D};
%   \end{scope}

%   \begin{scope}[%>={Stealth[black]},
%     every node/.style={fill=white,circle},
%     every edge/.style={draw=black,thick}]
%     \path [->] (A) edge (B);
%     \path [->] (B) edge (D);
%     \path [->] (C) edge (D);
%     \path [->] (A) edge (C);
%   \end{scope}



% \tikzstyle{block}    = [draw,fill=white!20,node distance = 3.5cm,align=center]
% \tikzstyle{inEdge}   = [fill=white, text width=1cm]
% \tikzstyle{overEdge} = [midway,above]
% \tikzstyle{input}    = [fill=white!20,node distance = 2.2cm,align=center, text width=1cm]
% \tikzstyle{double} = [draw, anchor=text, rectangle split,rectangle split parts=2]
% % diameter of semicircle used to indicate that two lines are not connected
% \tikzstyle{branch}=[fill,shape=circle,minimum size=3pt,inner sep=0pt]
% \tikzstyle{pinstyle} = [pin edge={to-,thin,black}]
% \begin{center}
% \begin{tikzpicture}[>=latex]


% \end{tikzpicture}%
% \end{center}