%%%%%%%%%%%%%%%%%%%%%%%%%%%%%%% Packages %%%%%%%%%%%%%%%%%%%%%%%%%%%%%%%%%%%%%%%
\usepackage{graphics}
\usepackage{epsfig}
\usepackage{times}
\usepackage{amsmath}
\usepackage{amssymb}                 %% for more arrows like rightarrowtail
\usepackage{amsfonts}                %% \mathbb
\usepackage{mathtools}               %% for coloneqq and others
\usepackage{cases}                   %% better math brackets
\usepackage{mathpartir}              %%% for inference rules
\usepackage[nolist]{acronym}
\usepackage[colorlinks]{hyperref}                %% for auto references
\usepackage{xpunctuate}              %% for punctuation's after macros
\usepackage{cite}                    %% for bibliography ranges
\usepackage{subcaption}              %% For complex figures with subfigures/subcaptions
\usepackage{wrapfig}                 %% wrapping figures around text
\usepackage{listings}                %% for source code lstlisting
\usepackage[dvipsnames]{xcolor}                  %% For listings
\usepackage{caption}
\usepackage{todonotes}
\usepackage{tikz}                    %% assertion stack visualization
\usetikzlibrary{matrix}
\usetikzlibrary{arrows}
\usetikzlibrary{positioning}
\usetikzlibrary{shapes.multipart}
\usetikzlibrary{automata}

\usepackage{lib/cc}
\usepackage{lib/lambda}
\usepackage{lib/thesis}
%%%%%%%%%%%%%%%%%%%%%%%%%%%%%%% Packages Config %%%%%%%%%%%%%%%%%%%%%%%%%%%%%%%%

%%%%%%%%%%%%%%%% Hyperref config %%%%%%%%%%%%%%%%
\newcommand\myshade{85}
\colorlet{mylinkcolor}{NavyBlue}
\colorlet{mycitecolor}{Aquamarine}
% \colorlet{myurlcolor}{Aquamarine}

\hypersetup{
  linkcolor  = mylinkcolor!,
  citecolor  = mycitecolor,
  % urlcolor   = myurlcolor!\myshade!black,
  colorlinks = true,
}

\renewcommand{\chapterautorefname}{Chapter}%
\renewcommand{\sectionautorefname}{Section}%


%%%%%%%%%%%%%%%% Listings config %%%%%%%%%%%%%%%%
\lstset{
  frame=top,frame=bottom,
  basicstyle=\linespread{.75}\small\normalfont\sffamily,    % the size of the fonts that are used for the code
  stepnumber=1,                           % the step between two line-numbers. If it is 1 each line will be numbered
  numbersep=10pt,                         % how far the line-numbers are from the code
  tabsize=2,                              % tab size in blank spaces
  extendedchars=true,                     %
  breaklines=true,                        % sets automatic line breaking
  captionpos=t,                           % sets the caption-position to top
  mathescape=true,
  % stringstyle=\color{white}\ttfamily,
  % showspaces=false,
  % showtabs=false,
  % xleftmargin=17pt,
  % framexleftmargin=17pt,
  % framexrightmargin=17pt,
  % framexbottommargin=5pt,
  % framextopmargin=5pt,
  showstringspaces=false,
  escapeinside={(*@}{@*)},%
  literate = {-}{-}1
 }

\newcommand{\lstvdots}{%
  \raisebox{-1pt}[0pt][0pt]{%
    \scalebox{0.7}{\ensuremath{\vdots}}}%
  \hspace{-1.5pt}}