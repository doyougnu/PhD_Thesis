~\label{chapter:related-work}
Related work has been discussed throughout the previous sections. This section
collects related work that was not previously covered. To my knowledge, this
work is the first to translate the specific ideas of a nanopass architecture,
and a variational compiler, to the \ac{sat}/\ac{smt} domains. Furthermore, at
time of this writing this work is the first to investigate variational
concurrency.

% \paragraph{Incremental Solving}
% The main goal of variational solving is to solve multiple satisfiability
% problems efficiently. In the SAT community, this problem was mainly addressed
% before by means of incremental solving, which is supported by many modern SAT or
% SMT solvers. With incremental solving, we can feed a solver with formulas,
% compute satisfiability, and then compute satisfiability again after adding or
% removing formulas. This way, the solver can reuse learned knowledge about the
% part of the formulas that remains in the solver. However, when applying
% incremental solving in such an adhoc fashion, it is likely that the full reuse
% potential is not exploited. In contrast, with variational solving, we create a
% variational formula first, such that variation becomes explicit and can be
% utilized as all variation points are known prior to the solving. While we have
% experienced speed-ups of variational solving over incremental solving, it
% requires that all those satisfiability problems are known in advance.

% \paragraph{Similar Solvers, Related Techniques}
%
This work is most similar to \cite{Visser:2012:GRR:2393596.2393665}, which also
constructs a \ac{sat} solver that exploits shared terms and prevents redundant
computation. However, the projects differ in important ways. Visser et al.'s
solver is oriented for program analysis and does not use incremental \ac{sat}
solving. Rather, it uses heuristics to find canonical forms of sliced programs,
and caches solver results on these canonical forms in a key-value
store~\cite{redis}. In contrast, variational \ac{sat} solving is domain
agnostic, solves \ac{sat} problems expressed in \ac{vpl}, returns a variational
model, and uses incremental \ac{sat} solving.


Variational \ac{sat} solving is the latest in a line of work that uses the
choice calculus to investigate variation as a computational phenomena. The
choice calculus has been successfully applied to diverse areas of computer
science, such as databases~\cite{ATW17dbpl,ATW18poly},
graphics~\cite{ES18diagrams}, data
structures~\cite{MMWWK17vamos,Walk14onward,SE17fosd,EWC13fosd}, type
systems~\cite{CCEW18popl,CCW18icfp,CEW12icfp,CEW14toplas}, error
messages~\cite{CES17jvlc,CE14popl,CEW12icfp,CES14hcc}, and now satisfiability
solving.
%
Our use of choices is similar to the concept of
\textit{facets}~\cite{austin2012multiple} and \textit{faceted
execution}~\cite{Schmitz2018FacetedSM,Micinski2018AbstractingFE,10.1145/2465106.2465121},
which have been successfully applied to information-flow security and
policy-agnostic programming.
%
% Facets, like choices, provide a syntactic form to parameterize a value or
% expression in a language, but differ from choices by having specific meaning
% in the domain of information security, namely to hide information between
% public and private information consumers. Choices, in contrast, are given
% meaning by the object language (i.e., the application domain.), thus one can
% view choices as a syntactic abstraction of facets, and may recover faceted
% execution through the construction of a variational interpreter.

The use of compiler optimization techniques in \ac{sat} solvers is not novel,
for example \ac{cse} and variable elimination has been successfully implemented
in \ac{sat} solvers\cite{10.1007/11499107_5,10.1007/978-3-319-23219-5_23}. Other
pre-processing techniques such as removing blocked
clauses~\cite{10.5555/1928380.1928406}, and detecting
autarkies~\cite{10.1007/978-3-540-79719-7_18} have been very successful at
automatically increasing performance of the \ac{sat} or \ac{smt} solver. From
this perspective, our use of choices is a pre-processing technique to speed up
incremental solving over sets of related problems.

Some solvers such as z3~\cite{10.1007/978-3-540-78800-3_24}, allow users to
program the heuristics used by the solver to find choose efficient solving
techniques, z3 in particular calls these constructs \emph{strategies}.
Strategies are thus similar to many optimizing compiler
techniques~\cite{10.5555/6448}, only applied to the \ac{sat} domain

% \paragraph{Applications for Variational Solving}
%
% \itodo{Thomas: Ideally, we would give an overview on applications that could
% benefit from incremental/variational solving. For example: program analysis
% \cite{VGD:FSE12}. Another application, namely reasoning about software
% variability, is described below in more detail.}
%
% Software variability, as explored in this paper, is a natural application domain
% for our work. The variability of SPLs or configurable software is often reduced
% to propositional logic~\cite{B05,CW07,MWCC08} for analysis
% purposes~\cite{BSRC10,TAK+:CSUR14,GBT+19}. Many analyses have been implemented
% using \ac{} solving~\cite{TAK+:CSUR14}, including feature-model
% analysis~\cite{BSRC10,GBT+19}, parsing~\cite{KGR+:OOPSLA11}, dead-code
% analysis~\cite{TLSS:EuroSys11}, code simplification~\cite{RGA+:ICSE15}, type
% checking~\cite{TBKC07}, consistency checking~\cite{CP06}, dataflow
% analysis~\cite{LKA+:ESECFSE13}, model checking~\cite{CCS+13}, variability-aware
% execution~\cite{NKN:ICSE14}, testing~\cite{MMCA:IST14}, product
% sampling~\cite{MKR+:ICSE16,VAT+:SPLC18}, product
% configuration~\cite{SIMA:ASE13}, optimization of non-functional
% properties~\cite{SRK+:SQJ12}, and variant-preserving
% refactoring~\cite{FMS+:SANER17}. While each of these analyses gives rise to
% multiple \ac{sat} problems for even a single analysis run, the authors typically
% do not discuss how they are solved. We argue that many could benefit from
% variational solving.

% More generally, any scenario that involves solving many related \ac{sat}
% problems, and where all of these problems are known or can be generated in
% advance, is a potential application for variational \ac{sat} solving.
% %
% Such situations arise in program analysis~\cite{VGD:FSE12}, and especially in
% \emph{speculative} program analyses that involve generating and exploring huge
% numbers of variations of a program, for example, as in
% counterfactual~\cite{CE14popl} and migrational~\cite{CCW18icfp,CCEW18popl}
% typing. Furthermore, we believe that variational solving provides a basis for
% such speculative analyses on feature models.
%
% where Generating type-error suggestions can be reduced to a variational
% \ac{} problem with choices isolating describe variants where a suggestion is
% applied. \itodo{jeff: other applications: variational route planning,
% variational circuit layout, variational (or counterfactual) scheduling}

%%% Local Variables:
%%% mode: latex
%%% TeX-master: "../../thesis"
%%% End: