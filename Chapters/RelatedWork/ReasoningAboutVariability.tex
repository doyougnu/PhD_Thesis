~\label{section:related-work:reasoning-about-variability}
%
Since \ac{sat} solving is so common in software variability applications, many
strategies have been developed to reduce effort in this domain.

Similar to variational formulas, \citet{NMS+:GPCE18} encode several versions of
a feature model in a single formula. We reuse their benchmark as part of our
evaluation as described in
\autoref{section:case-studies:experimental-methodology}; a direct comparison
with their approach is nuanced and discussed in
\autoref{section:case-studies:results-and-discussion}.
%
While their work focuses on feature-model analysis only, variational formulas
and variational solving can be applied to many application areas.

In the context of family-based type checking~\citep{TAK+:CSUR14}, others have
discussed merging multiple \ac{sat} problems into one.
%
Most work in this area use a \emph{local} approach where \ac{sat} problems are
solved as they are encountered during typing; in contrast, \emph{global}
approaches collect \ac{sat} checks into a single problem that is solved at the
end of the analysis.
%
While the global approach improves efficiency by increasing reuse of learned
clauses in the solver, it loses the ability to identify \emph{which} variants
contain type errors~\citep{ASLK:FOSD10,HZS:SCP11}.
%
% Through variational models,
Variational solving can achieve the reuse benefits of the global approach
without sacrificing the precision of the local approach.


Since the size of \ac{sat} problems in software variability applications is often
dominated by the feature model, researchers tried to reduce the size of
satisfiability problems by delaying consideration of the feature model until
after the analysis and only using it rule out false
positives~\citep{BMB+:PLDI13,CCS+13,LKA+:ESECFSE13}, a technique known as late
feature-model consideration~\citep{TAK+:CSUR14}.
%
\citet{BMB+:PLDI13} found that this technique increases the overall efficiency
of static analysis~\citep{BMB+:PLDI13}, while \citet{CCS+13} found that it
actually decreases efficiency of family-based model checking. Variational
solving is orthogonal to these approaches since the feature model can be
excluded from a variational formula and then used later to rule out false
positives.


Feature models can also be reduced in size to speed up analyses, for example,
by slicing~\citep{ACLF:ASE11,KST+:SPLC16} or decomposition~\citep{SKT+:ICSE16}.
It is largely unexplored how much such reductions can improve efficiency, but the
analysis will still involve multiple similar \ac{sat} problems, which can
benefit from variational solving.


A final approach is to avoid \ac{sat} problems by using modal implications
graphs~\citep{KTS+:ICSE18}, which support faster reasoning. The idea is to
encode as many software variability constraints as possible in such graphs,
then use a \ac{sat} solver only for the remaining constraints.
%
The construction of modal implication graphs already requires solving \ac{sat}
problems, but this cost is amortized if many \ac{sat} queries will be solved
during the analysis, as \citet{KTS+:ICSE18} found for configuration processes.

Lastly, our idea of representing variation in a non-traditional formula (a
\ac{vpl} formula in our case) is similar to the approach by
\cite{10.1145/3442391.3442405}, which uses quantified boolean formulas to encode
variation, and quantified boolean \ac{sat} solvers to detect anomalies in
context-aware feature models. Notably, this approach has the benefit of avoiding
incremental \ac{sat} solving altogether.



%%% Local Variables:
%%% mode: latex
%%% TeX-master: "../../thesis"
%%% End: