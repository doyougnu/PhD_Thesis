\label{section:related-work:variational-systems}
%
Variational \ac{sat} solving is the latest in a line of work that uses the
choice calculus to investigate variation as a computational phenomena. This body
of work ranges from data structures, to graphics, to full fledged systems such
as the system presented in this thesis. Due to the nature of variational
problems, many variational or variation-aware systems employ \ac{sat} and
\ac{smt} solvers. We collect and discuss these contributions here beginning with
variational data structures.

There is relatively little work on variational data structures.
\citet{EWC13fosd} describes a general strategy for constructing a variational
data structure. \citet{Walk14onward} expands on this strategy and attempts to
formalize ad-hoc implementations used in variational systems such as
TypeChef~\cite{KKHL:FOSD10} and SuperC~\cite{GG:PLDI12}. For this section we
focus on recent advancements implementing performent variational stacks and
lists. The goal for variational data structures is to construct a data structure
which describes a set of non-variational data structures efficiently. The
variational artifact is the implementation of the variational data structure,
and the variants in this domain are the plain versions of the data structure or
plain values that result from operating on the data structure. The challenge
then is to devise a variational data structure that describes and contains the
variation, and provides a set of operations to manipulate the data structure
that are as close to the performance of their plain counterparts as possible.

A fundamental tension in this domain is exemplified by work on variational
stacks by~\citet{MMWWK17vamos}. Meng et al.\ identify two kinds of possible
variational stacks depending on the location of variation on may have either: a
stack of choices, or a choice of stacks. However, their analysis on a general
implementation strategy was inconclusive, rather they found that depending on
the implementation strategy runtime performance could be affected by as much as
20\%. Furthermore, the variation in their experiment is coarse grained, \ie{},
the sharing ratio is high. Thus, Meng et al.\ utilized heuristics (optimizations
in their paper) which further improved performance for both implementations by
43\%. Utilizing heuristics was also found to be a successful strategy in
Meinicke's PhD dissertation which we address below.

The work on variational stacks yields an alternative implementation strategy for
variational \ac{sat} solvers. We have carefully designed our variational
\ac{sat} and \ac{smt} solvers with a goal to utilize a plain base solver. We
could have done otherwise and implemented a variational solver directly. With
variational stacks the variational solver could utilize a variational assertion
stack and we would avoid the need for a zipper in choice removal. Such an
implementation is worth considering although by developing an independent solver
we lose any benefits brought by the \ac{sat}/\ac{smt} communities and lose the
general recipe for constructing a variation-aware system \emph{using} its plain
counterpart.

Similar to variational stacks, \citet{SE17fosd} successfully implemented
variational lists. Smeltzer and Erwig devise six implementations of variational
lists with one implementation, the \emph{suffix list} coming from previous
work~\cite{EW11gttse}. Smeltzer and Erwig's study leads to some surprising
results. Out of their six implementations they found that for some
implementations, simple functions such as \texttt{head} (which returns the first
element of the list) are slower than the brute force counterpart, because the
implementation may be required to traverse the whole list to resolve the
variation. However, they do conclude that one implementation a \emph{segment
  list} yields reasonable performance given the data in their study. The segment
list is an interesting result as the idea behind the design is to encode
variation as a \emph{sequence of segments}, where a segment is either a choice
or a sequence of plain elements. This idea should sound familiar as accumulation
and symbolic values are essentially pointers to sequences of plain terms.
Smeltzer and Erwig also observe that the sharing ratio has a measurable impact
on performance (a finding we also observed) and thus minimizing or manipulating
choices to increase the ratio is important; a result that has also been observed
in software product-lines by~\citet{ARW+:ICSE13} and~\citet{KRE+:FOSD12}.

In addition to data structures there has been research on applications of the
choice calculus to graphics~\citep{ES18diagrams},type
systems~\citep{CCEW18popl,CCW18icfp,CEW:TOPLAS14,CEW12icfp}, and error
messages~\citep{CES17jvlc,CE14popl,CEW12icfp,CES14hcc}. For the remainder of
this section we focus on variational or variation-aware systems.

This work is not the first to construct a variational or variation-aware system.
Notably, \citet{KKHL:FOSD10} used the choice calculus and variational data
structures to type check every possible Linux kernel. Thus constructing a
variational parsing~\cite{KGR+:OOPSLA11}, a variational
lexer~\cite{Kastner11partialpreprocessing}, type
system\cite{LKA+:ESECFSE13,KOE:OOPSLA12} and control-flow and data-flow
analyses\cite{LKA+:ESECFSE13}. Similarly, \citet{GG:PLDI12} variationally parse
the Linux kernel by utilizing variational data structures and \emph{choice
  nodes} in the abstract syntax tree. TypeChef is notable for several reasons:
its implementation is a direct inspiration for our baseline algorithm \vTop{}
which uses an incremental \ac{sat} solver but only exhibits sharing
\emph{before} a choice is discovered. This kind of sharing, called \emph{prefix
  sharing} by \citet{SE17fosd} is the de-facto standard in software product-line
applications which employ incremental \ac{sat} solvers. Given the results of
this thesis, large performance gains are possible if our results are
Representative with the use of a variational \ac{sat} or \ac{smt} solver.
TypeChef is also notable for its two step approach, first it parses source code
to find \cpp{if-def} annotations, and stores these in files called
\emph{presence conditions}. Presence conditions are isomorphic to variation
contexts, both are \pl{} formula's over dimensions (or conditions of the
\cpp{if-def}) which determine a variant. Using the presence conditions, TypeChef
annotates choice nodes to determine which variant the leaves of the choice node
belong. Then TypeChef extends the symbol table of a C program to contain types
which are conditional based on the presence conditions. This allows a variable's
type to change from one variant to another. Each type checking operation is then
lifted to handle the variational cases and then type checking checks the
variation-aware types to ensure every each variant type checks. Similar to our
use of variation contexts, TypeChef allows a \emph{variability model} which
specifies variants that should be type checked by conjoining the model with the
presence conditions.

In his PhD dissertation, \citet{M14,JensDebugging} constructs a variational
interpreter called VarexJ and a variational bytecode transformer called VarexC,
to achieve a variational execution and debugging framework. The framework tries
to maximize sharing in two ways: It directly utilizes the choice calculus to
represent local points of variation and achieve a \emph{fine-grained approach},
this allows the framework to share program states and keep a unified heap.
Additionally, the framework achieves instruction-level sharing among
control-flows between variants. It achieves this by implementing a variational
scheduler, which seeks to order the execution of program statements to optimize
sharing. We achieve this same effect through the interaction between
accumulation, evaluation and choice removal with the wrapped primitive
operations. Interestingly, Meinicke identifies redundant \ac{sat} calls as a
major bottleneck in the variational execution framework. Specifically they
determine redundant \rn{check-sat} calls as the most expensive. To reduce the
redundant calls the variational execution framework caches calls to the solver
thus only employing the solver for new queries. Chu-Pan Wong used VarexJ to do
speculative mutation testing and automated program repair~\cite{ChuPanThesis}.

% Lastly, Meinicke's work was put to use by \citet{ChuPanThesis} to induce
% \emph{speculative variation} for automatic program repair and mutation testing.
% We only touch on this work briefly as it applies a variational system rather
% than creating one. However, the work utilizes the aforementioned variational
% execution framework and a \ac{sat} solver to reduce the search space of
% interesting mutants to test.

Lastly, choice calculus has been successfully applied to databases to construct
a complete approach for variational databases including a variational database
management system, a variational query language, and variational tables.
\citet{ATW17dbpl,ATW18poly} constructs a variational database in an interesting
way. Ataei et al. add choices to relational algebra to define a variational
query language for a variational relational database. The variational query
language serves as the variational artifact similar to the role of \ac{vpl} in
the variational \ac{sat} and \ac{smt} solvers. Ataei et al. specifically choose
to avoid adding choices to variational tables, instead opting to apply
\emph{annotations} to the table schema, table attributes, and table tuples. The
annotations are \pl{} formulas and are derived from the dimensions of choices in
the variational query language. Annotations that are attached to an aspect of
the database, such as a schema, attribute or tuple are called \emph{presence
  conditions} following the work on TypeChef. Annotations which are not attached
but describe possible variants are called \emph{feature expressions}.

This careful design has several desirable properties: the separation between the
variational aspects of the system and the database engine allows the database
engine to remain mostly unchanged. Thus, Ataei et al. avoid implementing low
level details such as a variational \texttt{B+-Tree} or file system. However the
system is still memory efficient: Elements which are shared between variants are
represented a single time in the database. To realize an element is shared, a
\ac{sat} solver is called on the presence conditions for that element. For
example, imagine an attribute that belongs to two variants $\kf{A}$ and
$\kf{B}$, to encode that this attribute belongs to these two variants is
expressed in the presence condition as a disjunction, $\kf{A \vee B}$. Thus,
Ataei et al.'s system is a mixed approach; the query language embeds choices to
explicitly represent local points of variation. The underlying object language
(the database in this case) lacks a primitive operation to handle variation such
as the \rn{push}/\rn{pop} commands in our work. Hence, Ataei et al. choose to
\emph{realize} variation in the database through indirection based on
annotations and \ac{sat} solving. Thereby enabling a full fledged variational
database without requiring substantive changes to the entire database
implementation. Therefore, Ataei et al.'s system is more expressive than the
variational \ac{sat} and \ac{smt} solver's presented in this thesis because it
can express dependencies between variants through presence conditions, while our
approach is limited to express dependencies by nesting choices.



%%% Local Variables:
%%% mode: latex
%%% TeX-master: "../../thesis"
%%% End: