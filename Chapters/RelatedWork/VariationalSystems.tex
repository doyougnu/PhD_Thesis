~\label{section:related-work:variational-systems}
%

Variational \ac{sat} solving is the latest in a line of work that uses the
choice calculus to investigate variation as a computational phenomena. This body
of work ranges from data structures, to graphics, to full fledged systems such
as the system presented in this thesis. We collect and discuss these
contributions here beginning with variational data structures.

There is relatively little work on variational data structures.
\citet{EWC13fosd} describes a general strategy for constructing a variational
data structure. \citet{Walk14onward} expands on this strategy and attempts to
formalize ad-hoc implementations used in variational systems such as
TypeChef~\cite{KKHL:FOSD10} and SuperC~\cite{GG:PLDI12}. For this section we
focus on recent advancements implementing performent variational stacks and
lists. The goal for variational data structures is to construct a data structure
which describes a set of non-variational data structures efficiently. The
variational artifact is the implementation of the variational data structure,
and the variants in this domain are the plain versions of the data structure or
plain values that result from operating on the data structure. The challenge
then is to devise a variational data structure that describes and contains the
variation, and provides a set of operations to manipulate the data structure
that are as close to the performance of their plain counterparts as possible.

A fundamental tension in this domain is exemplified by work on variational
stacks by~\citet{MMWWK17vamos}. Meng et al.\ identify that two kinds of
variational stacks are possible depending on the location of variation on may
have either: a stack of choices, or a choice of stacks. However, their analysis
on a general implementation strategy was inconclusive, rather they found that
depending on the implementation strategy runtime performance could be affected
by as much as 20\%. Furthermore, the variation in their experiment is coarse
grained, \ie{}, the sharing ratio is high. Thus, Meng et al.\ utilized
heuristics (optimizations in their paper) which further improved performance for
both implementations by 43\%.

The work on variational stacks yields an alternative implementation strategy for
variational \ac{sat} solvers. We have carefully designed our variational
\ac{sat} and \ac{smt} solvers with a goal to utilize a plain base solver. We
could have done otherwise and implemented a variational solver directly. With
variational stacks the variational solver could utilize a variational assertion
stack and we would avoid the need for a zipper in choice removal. Such an
implementation is worth considering although by developing an independent solver
we lose any benefits brought by the \ac{sat}/\ac{smt} communities and lose the
general recipe for constructing a variation-aware system \emph{using} its plain
counterpart.

Similar to variational stacks, \citet{SE17fosd} successfully implemented
variational lists. Smeltzer and Erwig devise six implementations of variational
lists with one implementation, the \emph{suffix list} coming from previous
work~\cite{EW11gttse}. Smeltzer and Erwig's study leads to some surprising
results. Out of their six implementations they found that for some
implementations, simple functions such as \texttt{head} (which returns the first
element of the list) are slower than the brute force counterpart, because the
implementation may be required to traverse the whole list to resolve the
variation. However, they do conclude that one implementation a \emph{segment
  list} yields reasonable performance given the data in their study. The segment
list is an interesting result as the idea behind the design is to encode
variation as a \emph{sequence of segments}, where a segment is either a choice
or a sequence of plain elements. This idea should sound familiar as accumulation
and symbolic values are essentially pointers to sequences of plain terms.
Smeltzer and Erwig also observe that the sharing ratio has a measurable impact
on performance (a finding we also observed) and thus minimizing or manipulating
choices to increase the ratio is important; a result that has also been observed
in software product-lines by~\citet{ARW+:ICSE13} and~\citet{KRE+:FOSD12}.

The
choice calculus has been successfully applied to diverse areas of computer
science, such as databases~\citep{ATW17dbpl,ATW18poly},
graphics~\citep{ES18diagrams}, data
structures~\citep{Walk14onward,EWC13fosd} type
systems~\citep{CCEW18popl,CCW18icfp,CEW:TOPLAS14,CEW12icfp}, error
messages~\citep{CES17jvlc,CE14popl,CEW12icfp,CES14hcc}, variational execution
systems~\citep{10.1145/3276487,M14} and now satisfiability solving.


%%% Local Variables:
%%% mode: latex
%%% TeX-master: "../../thesis"
%%% End: