~\label{sec:vsmt}

\TODO{TODOs}
\begin{itemize}
\item \TODO{worked example}
\item \TODO{Plain model alignment}
\item \TODO{Proof read}
\item \TODO{Claims align with introduction}
\end{itemize}

In this section we describe an extension of variational satisfiability solving
to variational \ac{smt} solving.
%
% We generalize variational models to handle any type of value and discuss
% further extensions to other background theories, such as an array theory, in
% \autoref{sec:future-work}.
%
\ac{smt} solvers generalize \ac{sat} solvers through the use of
\emph{background theories} that allow the solver to reason about values and
constructs outside the Boolean domain. The SMTLIB2 standard defines seven such
background theories: \texttt{Core} (Boolean theory), \texttt{ArraysEx},
\texttt{FixedSizeBitVectors}, \texttt{FloatingPoint}, \texttt{Ints},
\texttt{Reals}, and \texttt{Real\_Ints}. In this section, we use integer
arithmetic (\texttt{Ints}) as an example \ac{smt} extension for variational
\ac{smt} solving. However, the technique illustrated here can be used to extend
the approach to any theory that adds a regular language to the \ac{smt} domain.
%
\eric{Not sure what ``adds a regular language to the SMT domain'' means.}
%
Extending the variational solving algorithm to context sensitive
theories, such as \texttt{FixedSizeBitVectors} and \texttt{ArraysEx} are open
research questions.

\begin{figure}
  \begin{subfigure}[t]{\linewidth}
    \centering
    \begin{syntax}
  i & \in{} & \mathbb{Z} & \textit{Integers} \\
  t_{i} & \Coloneqq{} & r_{i} \quad|\quad i & \textit{Integer variables and literals} \\
  [1.5ex]

  ar & \Coloneqq{} & t_{i}    & \textit{Terminal} \\
  & | & -\ ar       & \textit{Arithmetic Negation} \\
  & | & ar\ -\ ar       & \textit{Subtraction} \\
  & | & ar\ +\ ar     & \textit{Addition} \\
  & | & ar\ *\ ar     & \textit{Multiplication} \\
  & | & ar\ \div\ ar     & \textit{Division} \\
  & | & \chc[D]{ar,ar} & \textit{Choice} \\
\end{syntax}
    \caption{Syntax of Integer arithmetic extension.}%
    \label{fig:arith:stx}
  \end{subfigure}
  % \vfill
  \begin{subfigure}[t]{\linewidth}
    \begin{syntax}
  t & \Coloneqq{} & r \quad|\quad \tru{} \quad|\quad \fls{}
  & \textit{Variables and Boolean literals} \\
  [1.5ex]
  \otimes & \Coloneqq{} & < \,\;|\;\; \leq \;\:\,|\;\; \geq \,\;|\;\;  > \,\;|\  \equiv & \emph{Binary relations}\\
  [1.5ex]


  f & \Coloneqq{} & t    & \textit{Terminal} \\
  & | & \neg{} f       & \textit{Negate} \\
  & | & f \vee{} f     & \textit{Or} \\
  & | & f \wedge{} f   & \textit{And} \\
  & | & ar \otimes{} ar  & \emph{Integer comparisons} \\
  & | & \chc[D]{f,f} & \textit{Choice} \\
\end{syntax}

    \centering
    \caption{Syntax of extended \vpl{}.}%
    \label{fig:arith:vpl}
  \end{subfigure}
  \caption{Formal definition of extended \vpl{}.}%
  \label{fig:ex:vpl}
\end{figure}

\paragraph{Syntax}
%
\autoref{fig:arith:stx} defines the syntax of the integer arithmetic extension,
which consists of integer variables, integer literals, a set of standard
operators, and choices.
%
The sets of Boolean and arithmetic variables are disjoint, thus an expression
such as $(\kf{s < 10) \wedge (s \vee p})$, where $s$ occurs as both an integer
and Boolean variable is disallowed.
%
The syntax of the language prevents type errors and expressions that do not
yield Boolean values. For example, $\chc[D]{1,2} \wedge p$ is syntactically
invalid.
%
% Thus, the language is purposefully limited to arithmetic expressions that
% have an inequality at the root of the expression, such as: $\kf{g} =
% (\chc[A]{1, 2} + j \geq 2) \vee a \wedge \chc[A]{c, d}$.
%
% The property of synchronization is maintained.
%
Choices in the same dimension are synchronized across Boolean and arithmetic
sub-expressions, for example, the expression
%
$\kf{g} = (\chc[A]{1, 2} + j \geq 2) \vee (a \wedge \chc[A]{c,d})$
%
represents two variants:
%
$\sem[A_{\tru{}}]{g} = (1 + j \geq 2) \vee (a \wedge c)$ and
$\sem[A_{\fls{}}]{g} = (2 + j \geq 2) \vee (a \wedge d)$.

\begin{figure}
  \centering
\begin{tabular}{r@{~~:~~}l@{~~\OB{\to}~~}ll}
\pnot
  & $(\aStore{},s)$
  & $(\aStore{},s)$
  & \emph{Negate a symbolic value} \\
\pand
  & $(\aStore{},s,s)$
  & $(\aStore{},s)$
  & \emph{Conjunction of symbolic values} \\
\por
  & $(\aStore{},s,s)$
  & $(\aStore{},s)$
  & \emph{Disjunction of symbolic values} \\
\pneg
  & $(\aStore{},s)$
  & $(\aStore{},s)$
  & \emph{Negate an arithmetic symbolic value} \\
\padd
  & $(\aStore{},s,s)$
  & $(\aStore{},s)$
  & \emph{Add symbolic values} \\
\psub
  & $(\aStore{},s,s)$
  & $(\aStore{},s)$
  & \emph{Subtract symbolic values} \\
\pdiv
  & $(\aStore{},s,s)$
  & $(\aStore{},s)$
  & \emph{Divide symbolic values} \\
\pmult
  & $(\aStore{},s,s)$
  & $(\aStore{},s)$
  & \emph{Multiply symbolic values} \\
\plt
  & $(\aStore{},s,s)$
  & $(\aStore{},s)$
  & \emph{Less than over symbolic values} \\
\plte
  & $(\aStore{},s,s)$
  & $(\aStore{},s)$
  & \emph{Less than equals over symbolic values} \\
\pgt
  & $(\aStore{},s,s)$
  & $(\aStore{},s)$
  & \emph{Greater than over symbolic values} \\
\pgte
  & $(\aStore{},s,s)$
  & $(\aStore{},s)$
  & \emph{Greater than equals over symbolic values} \\
\peq
  & $(\aStore{},s,s)$
  & $(\aStore{},s)$
  & \emph{Arithmetic equivalence over symbolic values} \\
\pspawn
  & $(\aStore{},r)$
  & $(\aStore{},s)$
  & \emph{Create symbolic value based on a boolean variable} \\
\pspawni
  & $(\aStore{},r_{i})$
  & $(\aStore{},s)$
  & \emph{Create symbolic value based on a arithmetic variable} \\
\passert
  & $(\eStore{},\aStore{},s)$
  & $\eStore{}$
  & \emph{Assert a symbolic value to the solver} \\
\pmodel
  & $(\eStore{},\aStore{})$
  & $m$
  & \emph{Get a model for the current solver state}
\end{tabular}

  \caption{Assumed base solver primitive operations}%
  \label{fig:vsmt:inf:prim}
\end{figure}
%
\begin{figure}
  \begin{mathpar}
\inferrule*[right=\acRef]
  { \pwrap{\pspawn}(\aStore{},r) = (\aStore{}', s) }
  { (\aStore{}, r) \accumulation (\aStore{}', s) }

  \inferrule*[right=\acRefI]
  { \pwrap{\pspawni}(\aStore{},r_{i}) = (\aStore{}', s) }
  { (\aStore{}, r_{i}) \accumulation (\aStore{}', s) }

\inferrule*[right=\acNotS]
  { (\aStore{}, \eIL) \accumulation (\aStore{}', s) \\
    \pwrap{\pnot}(\aStore{}', s) = (\aStore{}'', s') }
  { (\aStore{}, \neg{} \eIL) \accumulation (\aStore{}'', s') }

  \inferrule*[right=\acUnaryS]
  { \op \in \integerUnary{} \\
    \lookup{\op}{\integerUnary{}} \coloneqq \pwrap{\op} \\
    (\aStore{}, \eAR) \accumulation (\aStore{}', s) \\
    \pwrap{\op}(\aStore{}', s) = (\aStore{}'', s') }
  { (\aStore{}, \op\; \eAR) \accumulation (\aStore{}'', s') }

\inferrule*[right=\acBoolS]
 { \op \in \boolFuncs{} \\
   \lookup{\op}{\boolFuncs{}} \coloneqq \pwrap{\op} \\
  (\aStore{}, \eIL*{1}) \accumulation (\aStore{1}, s_1) \\
    (\aStore{1}, \eIL*{2}) \accumulation (\aStore{2}, s_2) \\
    \pwrap{\op}(\aStore{2}, s_1, s_2) = (\aStore{3}, s_3) }
  { (\aStore{}, \eIL*{1}\; \op\; \eIL*{2}) \accumulation{} (\aStore{3}, s_3) }

  \inferrule*[right=\acArithS]
  { \op \in \integerFuncs{} \\
    \lookup{\op}{\integerFuncs{}} \coloneqq \pwrap{\op} \\
    (\aStore{}, \eAR*{1}) \accumulation (\aStore{1}, s_1) \\
    (\aStore{1}, \eAR*{2}) \accumulation (\aStore{2}, s_2) \\
    \pwrap{\op}(\aStore{2}, s_1, s_2) = (\aStore{3}, s_3) }
  { (\aStore{}, \eAR*{1}\; \op\; \eAR*{2}) \accumulation{} (\aStore{3}, s_3) }

  \inferrule*[right=\acInEqS]
  { \op \in\ \inequalities{} \\
    \lookup{\op}{\inequalities} \coloneqq \pwrap{\op} \\
    (\aStore{}, \eAR*{1}) \accumulation (\aStore{1}, s_1) \\
    (\aStore{1}, \eAR*{2}) \accumulation (\aStore{2}, s_2) \\
    \pwrap{\op}(\aStore{2}, s_1, s_2) = (\aStore{3}, s_3) }
  { (\aStore{}, \eAR*{1}\; \op\; \eAR*{2}) \accumulation{} (\aStore{3}, s_3) }

\inferrule*[right=\acChc]
  { }
  {(\aStore,\ \chc[D]{e_1,e_2}) \accumulation (\aStore{},\chc[D]{e_1,e_2})}

\inferrule*[right=\acChcI]
  { }
  {(\aStore,\ \chc[D]{ar_{1},ar_{2}}) \accumulation (\aStore{},\chc[D]{ar_{1},ar_{2}})}

\inferrule*[right=\acNotV]
  { (\aStore{}, \eIL) \accumulation (\aStore{}', \eIL') }
  { (\aStore{}, \neg{} \eIL) \accumulation (\aStore{}', \neg \eIL') }

  \inferrule*[right=\acUnaryV]
  { (\aStore{}, \eIL) \accumulation (\aStore{}', \eIL') }
  { (\aStore{}, \integerUnary \eIL) \accumulation (\aStore{}', \integerUnary \eIL') }

\inferrule*[right=\acBoolV]
  { (\aStore{}, \eIL_1) \accumulation (\aStore{1}, \eIL*{1}') \\
    (\aStore{1}, \eIL_2) \accumulation (\aStore{2}, \eIL*{2}') }
  { (\aStore{}, \eIL_1 \boolFuncs{} \eIL_2) \accumulation{} (\aStore{2}, \eIL*{1}' \boolFuncs{} \eIL*{2}') }

  \inferrule*[right=\acArithV]
  { (\aStore{}, \eAR*{1}) \accumulation (\aStore{1}, \eAR*{1}') \\
    (\aStore{1}, \eAR*{2}) \accumulation (\aStore{2}, \eAR*{2}') }
  { (\aStore{}, \eAR*{1} \integerFuncs{} \eAR*{2}) \accumulation{} (\aStore{2}, \eAR*{1}' \integerFuncs{} \eAR*{2}') }

  \inferrule*[right=\acInEqV]
  { (\aStore{}, \eAR*{1}) \accumulation (\aStore{1}, \eAR*{1}') \\
    (\aStore{1}, \eAR*{2}) \accumulation (\aStore{2}, \eAR*{2}') }
  { (\aStore{}, \eAR*{1} \inequalities{} \eAR*{2}) \accumulation{} (\aStore{2}, \eAR*{1}' \inequalities{} \eAR*{2}') }

\end{mathpar}

  \caption{Accumulation inference rules}%
  \label{fig:vsmt:inf:acc}
\end{figure}
%
\begin{figure}
  \begin{mathpar}
  %%% Computation rules
  \inferrule*[Right=Ev-Tm]
  { \texttt{Assert}((\eStore{}, \aStore{}), $t$) = \eStore{}' }
  %%% --------------------------------------------------------------
  {(\eStore{}, \aStore{}, t) \evaluation{} (\eStore{}', \aStore{}, \unit{}) }


  \inferrule*[Right=Ev-Sym]
  { \texttt{Assert}((\eStore{}, \aStore{}), s) = \eStore{}' }
  %%% --------------------------------------------------------------
  { (\eStore{}, \aStore{}, s) \evaluation{} (\eStore{}, \Delta, \unit{}) }
\\

\inferrule*[Right=Ev-Model]
{ \texttt{GetModel}(\aStore{}, \eStore{}) = m }
%%% --------------------------------------------------------------
{ (\aStore{}, \eStore{}, \unit{}) \evaluation{} m }
\\

\inferrule*[Right=Ev-Chc]
{ }
%%% --------------------------------------------------------------
{(\aStore{}, \eStore{}, \chc[D]{e_{1},e_{2}}) \evaluation{} (\aStore{}, \eStore{}, \chc[D]{e_{1},e_{2}})}
\end{mathpar}

\begin{mathpar}
  %%%\unit{} elimination rules
  \inferrule*[Right=Ev-UL]
  { \boolFuncs{} = \texttt{And}  }
  %%% --------------------------------------------------------------
  { (\Theta,\unit{} \boolFuncs{} v) \evaluation{} (\Theta, v) }


  \inferrule*[Right=Ev-UR]
  { \boolFuncs{} = \texttt{And} }
  %%% --------------------------------------------------------------
  { (\Theta, v \boolFuncs{} \unit{}) \evaluation{} (\Theta, v)}
\end{mathpar}

\begin{mathpar}
  \inferrule*[Right=Ev-BU]
  { (\aStore{},\neg{} v) \accumulation{} (\aStore{}', v') }
  %%% --------------------------------------------------------------
  { (\eStore{},\aStore{}, \neg{} v) \evaluation{} (\eStore{},\aStore{}', v') }


  \inferrule*[Right=Ev-IU]
  { (\aStore{}, \integerUnary{} v) \accumulation{} (\aStore{}', v') }
  %%% --------------------------------------------------------------
  { (\eStore{},\aStore{}, \integerUnary{} v) \evaluation{} (\eStore{},\aStore{}', v') }
\end{mathpar}

\begin{mathpar}
  \inferrule*[Right=Ev-And]
  { \boolFuncs{} = \texttt{And} \\
    (\Theta, v_{1}) \evaluation{} (\Theta', v_{1}') \\
    (\Theta', v_{2}) \evaluation{} (\Theta'', v_{2}')
  }
  %%% --------------------------------------------------------------
  { (\Theta, v_{1} \boolFuncs{} v_{2}) \evaluation{} (\Theta'', v_{1}' \boolFuncs{} v_{2}') }


    \inferrule*[Right=Ev-AccB]
    { \boolFuncs \neq \texttt{And} \\
      (\aStore{}, v_{1}) \accumulation{} (\aStore{}', v_{1}') \\
      (\aStore{}',v_{2}) \accumulation{} (\aStore{}'', v_{2}')
    }
%%% --------------------------------------------------------------
    { (\Gamma, \Delta, v_{1} \boolFuncs{} v_{2}) \evaluation{} (\Gamma, \Delta'', v_{1}'
      \boolFuncs{} v_{2}')}
\end{mathpar}%

\begin{mathpar}
  \inferrule*[Right=Ev-AccIB]
  {
    (\aStore{}, v_{1}) \accumulation{} (\aStore{}', v_{1}') \\
    (\aStore{}',v_{2}) \accumulation{} (\aStore{}'', v_{2}')
  }
  %%% --------------------------------------------------------------
  { (\Gamma, \Delta, v_{1} \inequalities{} v_{2}) \evaluation{} (\Gamma, \Delta'', v_{1}' \inequalities{} v_{2}')}
\end{mathpar}%
  \caption{Evaluation inference rules}%
  \label{fig:vsmt:inf:eval}
\end{figure}
%
\begin{figure}
  \begin{mathpar}
\inferrule*[right=\crEval]
  { (\eStore{},\aStore{},v) \evaluation (\eStore{}',\aStore{}',\unit) \\
    \texttt{Combine}(\vmodel{},\pmodel(\aStore{},\eStore{})) = \vmodel{}' }
  { (\crCtx, \inRoot, v) \choiceRemoval \vmodel{}' }

\inferrule*[right=\crChcT]
  { (D,\true)\in C \\
    (\crCtx, z, e_1) \choiceRemoval \vmodel{}' }
  { (\crCtx, z, \chc[D]{e_1,e_2} \choiceRemoval \vmodel{}' }

\inferrule*[right=\crChcF]
  { (D,\false)\in C \\
    (\crCtx, z, e_2) \choiceRemoval \vmodel{}' }
  { (\crCtx, z, \chc[D]{e_1,e_2} \choiceRemoval \vmodel{}' }

\inferrule*[right=\crChc]
  { D\notin\dom{C} \\
    (C\cup(D,\true),\eStore{},\aStore{},\vmodel{}, z, e_1)
      \choiceRemoval \vmodel{1} \\
    (C\cup(D,\false),\eStore{},\aStore{},\vmodel{}', z, e_2)
      \choiceRemoval \vmodel{2} }
  { (\crCtx, z, \chc[D]{e_1,e_2} \choiceRemoval \vmodel{2} }

\inferrule*[right=\crNot]
  { (\crCtx, \inNot{z}, v) \choiceRemoval \vmodel{}' }
  { (\crCtx, z, \neg v) \choiceRemoval \vmodel{}' }

\inferrule*[right=\crNotIn]
  { (\aStore{}, \neg s) \accumulation (\aStore{}', s') \\
    (\crCtx, z, s') \choiceRemoval \vmodel{}' }
  { (\crCtx, \inNot{z}, s) \choiceRemoval \vmodel{}' }

\inferrule*[right=\crAnd]
  { (\crCtx, \inAndL{z}{v_2}, v_2) \choiceRemoval \vmodel{}' }
  { (\crCtx, z, v_1 \wedge v_2) \choiceRemoval \vmodel{}' }

\inferrule*[right=\crAndL]
  { (\crCtx, \inAndR{s}{z}, v) \choiceRemoval \vmodel{}' }
  { (\crCtx, \inAndL{z}{v}, s) \choiceRemoval \vmodel{}' }

\inferrule*[right=\crAndR]
  { (\aStore{}, s_1 \wedge s_2) \accumulation (\aStore{}', s_3) \\
    (\crCtx, z, s_3) \choiceRemoval \vmodel{}' }
  { (\crCtx, \inAndR{s_1}{z}, s_2) \choiceRemoval \vmodel{}' }

\inferrule*[right=\crOr]
  { (\crCtx, \inOrL{z}{v_2}, v_2) \choiceRemoval \vmodel{}' }
  { (\crCtx, z, v_1 \vee v_2) \choiceRemoval \vmodel{}' }

\inferrule*[right=\crOrL]
  { (\crCtx, \inOrR{s}{z}, v) \choiceRemoval \vmodel{}' }
  { (\crCtx, \inOrL{z}{v}, s) \choiceRemoval \vmodel{}' }

\inferrule*[right=\crOrR]
  { (\aStore{}, s_1 \vee s_2) \accumulation (\aStore{}', s_3) \\
    (\crCtx, z, s_3) \choiceRemoval \vmodel{}' }
  { (\crCtx, \inOrR{s_1}{z}, s_2) \choiceRemoval \vmodel{}' }
\end{mathpar}

  \caption{Choice removal inference rules}%
  \label{fig:vsmt:inf:chc}
\end{figure}


\paragraph{Semantics}
%
\autoref{fig:vsmt:inf:prim} shows a set of primitive operations that the base
solver is assumed to support. We use the nonterminals in the grammar as
metavariables to range over operations. The naming scheme of the metavariables
is to use \uop{}, \bop{}, and \sop{} to indicate unary, binary, and solver
functions in the base solver domain, and use subscripts to indicate the
argument type and (in the case of binary operations) result type of the
operations they range over.
%
For example, integer inequality is ranged over by \bib{}, indicating that
inequality is a binary function whose arguments are of type \integers{} and
whose result is of type \booleans{}.
% When a result is the same type as the arguments we elide the result
% subscript, such as in Boolean and Integer negation, and
%
We do not show subscripts for \sop{} as these operations are run for their side
effects in the base solver.
%
\itodo{not sure if this is a good strategy, we would have to show models and a
unit value for side effects if we didn't elide these though}.

\itodo{also not sure about the unary symbol, but I think making the
  variational level all symbols and using text for base solver level is a good
  idea, open to suggestions here}
%
\eric{Can we define these directly in the grammar in Fig.~6? I think this is
confusing and overwhelming in its current form.}
%
Similarly we define metavariables for these
functions in the variational \ac{smt} domain. We use \inequalities{} to
represent binary relations over integers, \integerFuncs{} to represent
arithmetic binary functions such as addition, \integerUnary{} for unary
arithmetic functions, and \boolFuncs{} to represent binary logical connectives.
Thus a term such as $\kf{(i < j) \wedge a}$ is represented as
$\kf{\bbb{}(\bib{}(i,j),a)}$ in the base solver domain, and $(i \inequalities{}
j) \boolFuncs{} a$ in the variational \ac{smt} solver domain.


\NOTE{The previous mini-section is called \emph{Semantics}, but it just
introduces a set of primitive operations and some syntax for referring to
them.}


\NOTE{The rules in Figure~\ref{fig:vsmt:inf:acc} are passing the store to the
primitive operations. Why this is makes sense when you read the explanation of
the store, but the syntax is inconsistent with the explanation of how the
metavariables are used in the previous paragraph.}

\paragraph{Accumulation}
%
With primitive operations and metavariables defined we specify accumulation in
\autoref{fig:vsmt:inf:acc}. Since the metavariable $\kf{s}$ has two meanings:
sub-trees of \texttt{IL} in the variational \ac{smt} domain and sequences of
clauses in the base solver, we treat it as overloaded.
%
\eric{I only see one definition of $s$ (in Sec.~4). Can we refactor to avoid
this problem?}
%
An implementation of the rules would require a store which maps symbolics at
the variational level to terms or sequences at the base solver level.

Accumulation is represented as a binary relation with \accumulation{}. The rules
follow a simple pattern: \texttt{Ac-Chc} skips any choices, \texttt{Ac-Gen} and
\texttt{Ac-Geni} provide a method to inject references into the symbolic domain,
\texttt{Ac-Ref} and \texttt{Ac-Refi} cache references to ensure the same
reference is mapped to the same symbolic, and the rest of the rules provide
operations on symbolic terms, \eg{}, \texttt{Ac-SBinB}, or are congruence rules
such as \texttt{Ac-BinI}. We elide rules which process formulas composed of
constants such $\kf{(\tru{} \wedge \fls{})}$ or $(1 + 2 + 3)$. In cases such as
$1 + 2 < \kf{i}$, constants are reduced and treated as references, thus this
formula becomes $3 < \kf{i}$ and is accumulated to $s_{3} < \kf{i}$.

Accumulation maintains a store, \aStore{}, to track and cache symbolic terms.
For example, given formula such as: $g = a \wedge (a \wedge b)$, \texttt{Ac-Gen}
will spawn only two new references, one for $\kf{a}$ and one for $\kf{b}$, and
\texttt{Ac-Ref} ensures the same symbolic will represent the $a$ reference. This
will produce $g = s_{a} \wedge (s_{a} \wedge s_{b})$, because we $g$ contains
two boolean connective \texttt{Ac-BinB} will be called twice beginning with the
inner conjunction. \texttt{Ac-BinB} will combine $s_{a}$ and $s_{b}$ into a new
symbolic $s_{ab}$, update the store to \aStore{}'. The new store will include
entries for both references \textit{and} symbolic references, thus, in this
example \aStore{}' contains $\kf{a} \rightarrow s_{a}$, $\kf{b} \rightarrow
s_{b}$, and $\kf{s_{ab}} \rightarrow (s_{a} \wedge s_{b})$. Finally
\texttt{Ac-BinB} will repeat the last procedure on the outermost conjunction
adding a new entry to the symbolic store.


\NOTE{Figure~\ref{fig:vsmt:inf:eval} is mixing up the syntax of the IL and the
operations of the base solver. For example, several of the rules have
conditions like $\otimes=\mathtt{And}$, but $\mathtt{And}$ is not part of the
syntax of the IL. Also, in cases like this, we should just use the
corresponding operation directly in the conclusion of the rule rather than
expressing it as a side condition.}


\NOTE{Given $\aeStore{}=(\aStore{},\eStore{})$, the structure of the judgment
is inconsistent since the tuple is often shown ``in-lined'' in the parent
tuple. I suspect it'd be simpler to just not introduce $\aeStore{}$ at all, but
if it is used, then when both $\aStore{}$ and $\eStore{}$ are needed, they
should be shown as a tuple.}

\paragraph{Evaluation}
%
Evaluation is defined in \autoref{fig:vsmt:inf:eval} as a relation of the form
(\aeStore{}, v) \evaluation{} (\aeStore{}, v), where \aeStore{} = (\eStore{},
\aStore{}) and \eStore{} represents the base solver state. The rules
\texttt{EV-TM} and \texttt{Ev-Sym} push new clauses to the base solver using the
primitive assert operation. \texttt{Ev-Model} calls for a plain model from the
base solver, only once a variant is fully reduced to \unit{}. \texttt{Ev-Chc}
skips choices, \texttt{Ev-UL} and \texttt{Ev-UR} implement left and right unit,
reducing conjunctions where one side has been processed by the base solver. Of
special note is the difference between the \texttt{Ev-AccB} and \texttt{Ev-And}
rules. While \texttt{Ev-And} is a straightforward congruence rule,
\texttt{Ev-AccB} instead processes its arguments using accumulation
(\accumulation{}). Disjunctions are a source of back-tracking in variational
solving, and thus the solver cannot evaluate the left-hand side without
evaluating the right, both of which may contain choices, hence evaluation must
switch to accumulation, as we informally described in the previous subsection.
This problem is repeated for inequalities as well. \texttt{Ev-AccIB} switches to
accumulation as one side of an inequality cannot be processed without knowledge
of the adjacent side. Thus, evaluation contains no rules for arithmetic.

\itodo{part of me wants to say: with these rules if we input a CNF formula then
  we'll get the incremental pattern that is desirable as stated in the
  background...not sure if this is the place though.}

\NOTE{Is the forgotten $\aStore{1}$ correct in the bottom three rules of
Figure~\ref{fig:vsmt:inf:chc}?}

\paragraph{Choice removal}
Choice removal is defined in \autoref{fig:vsmt:inf:chc} as a relation between
the evaluation/accumulation stores (\aeStore{}), the configuration
(\configuration{}), and terms in IL\@. Furthermore, we track the current
variational model as part of the 4-tuple. The vast majority of rules are either
commutative versions of the presented rules; such as \texttt{CR-RB} which is
\texttt{CR-LB} but with a choice as the left child of \boolFuncs{}, or the same
rules over different operators, such as \texttt{CR-LIB} which is \texttt{CR-LB}
only for \inequalities{}; thus we only present a subset.

The interesting rules are \texttt{Gen} and \texttt{Sym} which use evaluation to
query for a plain model, and construct a new variational model through the
\texttt{Combine} function. \texttt{CR-LB} ensure the property of
synchronization; when a choice is observed as the right child of a boolean
operator, and the dimension has a value in the configuration (in this case
\true{}), then the proper alternative (in this case the left alternative) of the
choice is retrieved. \texttt{CR-IB-ChcR} removes choices when the choice is not
present in the configuration. We present the version of \texttt{CR-IB-ChcR} for
\inequalities{}; the same rule exists for \boolFuncs{}, \integerFuncs{}, and for
choices as the left children of \inequalities{}. The assertion stack counter,
$\kf{i}$, is incremented indicating that all recursive processing occurs in a
new \texttt{push}/\texttt{pop} context. Each configuration is updated to process
both alternatives, \true{} for the left and \false{} for the right alternative.
Both alternatives eventually conclude to a \unit{} and thus a variational model,
which are combined to a final result.

The remaining rules are congruence rules that recursively call accumulation
after a choice has been found, and new terms are introduced as the result of a
replacing a choice with an alternative. Careful readers will recognize that the
provided rules can easily become stuck. For example, given the formula $a \vee{}
(b \leq{} \chc[D]{p,q})$ the rules cannot further reduce the formula due to the
disjunction and inequality, and the choice cannot be accumulated. What is
required is to find the choice while storing the \emph{context} around the
choice. We leave this as an implementation detail, the prototype variational
solvers utilize a Huet zipper~\cite{huet_1997} data structure to capture this
context\footnote{that the Huet zipper has been successful implies delimited
  continuations~\itodo{cite} may be an alternative and efficient method to
  capture the context}, searches the variational core until a choice is in the
focus position, and then applies a choice removal rule such as
\texttt{Cr-IB-ChcR} or \texttt{Cr-LB}.

\paragraph{Derivation of a Variational Core}
Consider the query formula $\kf{h = ((1 + 2 < (i - \chc[A]{k,l})) \wedge a)
  \wedge (\chc[A]{c,\neg b} \vee b)}$; derivation of the variational core
$\kf{h}$ begins with evaluation and all stores \aStore{}, \eStore{} initialized
to empty. When a sure inputs a \vc{} the configuration, \configuration{}, is
initialized to it, otherwise \configuration{} is initialized to empty.

\texttt{Ev-And} is the only applicable rule, matching \boolFuncs{} with $\wedge$
at the root of $\kf{h}$. Thus, $v_{1} = \kf{((1 + 2 < (i -
  \chc[A]{k,l})) \wedge a)}$, and $v_{2} = (\chc[A]{c,\neg b} \vee b)$. We
traverse $v_{1}$ first, leading to a recursive application of \texttt{Ev-And}.
We denote the recursive levels with a tick mark $'$, thus \newline{} $\kf{v_{1}' = (1 + 2 <
  (i - \chc[A]{k,l}))}$ is the recursive left child and the right child is
$\kf{v_{2}' = a}$.

\texttt{Ev-AccIB} matches \inequalities{}with the $<$ at the root of $v_{1}'$ and
switches to accumulation. $v_{2}'$ is a terminal, will match \texttt{Ev-Tm}, be
sent to the base solver, and replaced with \unit{}. \texttt{Ev-Tm} updates
\eStore{}, recording the interaction and yields $(\eStore{v_{2}'},\aStore{},
\unit{})$, where $\eStore{v_{2}'} = \{\kf{a}\} \cup \eStore{}$ as the result for
$v_{2}'$.

Accumulation on $v_{1}'$ matches \inequalities{} to $<$, applying
\texttt{Ev-AccIB} yields two recursive cases: $v_{1}'' = 1 + 2$; and $v_{2}'' =
\kf{i - \chc[A]{k,l}}$. \itodo{v1'' will be turned into a constant but we don't
  show rules for constants because they aren't interesting, should we? We could
  also transform Ac-Ref and Ac-Refi to work on t and $t_{i}$. Thoughts?}.
%
$v_{1}''$ will be preprocessed to the value 3, and accumulated to a symbolic with
\texttt{Ac-Refi} yielding $(\aStore{v_{1}''}, s_{3})$ where $\aStore{v_{1}''} =
\{(3, \kf{s_{3}})\} \cup \aStore{}$. $v_{2}''$ is the interesting case.
\texttt{Ac-BinI} will match $-$ at the root node, $\kf{i}$ will be accumulated
to $\kf{s_{i}}$ with \texttt{Ac-Refi} and the choice is skipped with
\texttt{Ac-Chc}. Hence we have $(\aStore{v_{2}''}, s_{i} - \chc[A]{k,l}$), where
$\aStore{v_{2''}} = (\{\kf{i, s_{i}}\} \cup \aStore{v_{1}''})$ as the result for
$v_{2}''$. Note that the stores, \aeStore{}, are threaded through from the left
child to the right child and thus can only monotonically increase until the
query formula is processed.

With results for $v_{1}''$, $v_{2}''$, and $v_{2}'$ the recursive calls can finally
resolve. $v_{1}'$ yields $(\aStore{v_{1}'}, \kf{s_{3}} < s_{i} - \chc[A]{k,l})$,
where \aStore{v_{1}'} $= \{(\kf{i}, s_{i}), (3, s_{3})\}$, $v_{2}'$'s result
only manipulated \eStore{} and thus $v_{1}$'s result is $(\eStore{v_{2}'},
\aStore{v_{1}'}, (\kf{s_{3}} < s_{i} - \chc[A]{k,l}) \wedge \unit{})$, which can
be further reduced by \texttt{Ev-UR} to \newline{} $(\eStore{v_{2}'},
\aStore{v_{1}'}, (\kf{s_{3}} < s_{i} - \chc[A]{k,l}))$.

This process is repeated for $v_{2} = (\chc[A]{c,\neg b} \vee b)$ with the final
stores from processing $v_{1}$. The only rule that matches $\vee$ is
\texttt{Ev-AccB}, thus $v_{2}$ is processed in accumulation. Accumulation
matches on the disjunction and applies \texttt{Acc-BinB} with $v_{1}' =
\chc[A]{c,\neg b}$ and $v_{2}' = \kf{b}$. The choice, by \texttt{Ac-Chc}, is
skipped over; $\kf{b}$, by \texttt{Ac-Gen} will be converted to a symbolic
$s_{b}$ yielding $(\eStore{v_{2}}, \aStore{v_{2}}, \chc[A]{c,\neg b} \vee
s_{b})$, where $(\aStore{v_{2}} = \{(\kf{b}, \kf{s_{b}})\} \cup
\aStore{v_{1}'})$, and $\eStore{v_{2}} = \{\kf{a}\}$ as the result for $v_{2}$.
With both $v_{1}$ and $v_{2}$ processed the variational core for $\kf{h}$ is
found to be $\kf{h_{core}} = (s_{3} < (s_{i} - \chc[A]{k,l})) \wedge
(\chc[A]{c,\neg b} \vee s_{b})$ with stores $\eStore{h_{core}} = \{(a)\}$,
$\aStore{h_{core}} = \{\kf{(b,s_{b}), (i,s{i}), (3,s_{3})}\}$.%

\paragraph{Solving the variational core}
Solving the variational core begins with choice removal and proceeds with
recursive calls to evaluation and consequently accumulation. We assume an empty
configuration for the remainder of the example because the \vc{} case is a
sub-case. The computation rules which remove choices, such as \texttt{Cr-LB},
and \texttt{Cr-IB-ChcR}, require a choice in the child node of a binary
relation, however $h_{core}$'s immediate child nodes are binary relations
themselves, $<$ on the left, and $\vee{}$ on the right. We use a zipper to
manipulate the core such that a choice is in position for removal, while the
remainder of the core is held in a context, a variational \sat{} solving may
instead choose to migrate choices according to Boolean equivalency laws.

Assuming $\kf{\chc[A]{k,l}}$ is found to be the focus, then the left version of
\texttt{Cr-IB-ChcR}, \texttt{Cr-IB-ChcL} would apply. Clearly $D \notin
\configuration{}$, thus a recursive case for each alternative, beginning with
the left alternative $e_{1}$, is performed. Several changes occur: the assertion
stack is incremented; indicating a push for the next call to evaluation, the
configuration mutates to account for the selection, and $e_{1}$ is translated
into IL and replaces the choice, thereby introducing a \textit{new} plain term:
$\kf{l}$. Thus, the recursive call for the left alternative is $(s_{3} < (s_{i}
- k)) \wedge (\chc[A]{c,\neg b} \vee s_{b})$ where $\configuration{\kf{L}} =
\{(\kf{A}, \true{})\}$, and $\kf{i_{\kf{L}} = 1}$. Similarly the right
alternative is $(s_{3} < (s_{i} - l)) \wedge (\chc[A]{c,\neg b} \vee s_{b})$
with $\configuration{\kf{R}} = \{(\kf{A}, \false{})\}$, and $\kf{i_{R} = 1}$.

With the choice removed the rules are no longer stuck. \texttt{Cr-BinB} will
apply to both alternatives because their root node, $\wedge$ matches
\boolFuncs{}. We walk through the processing of the left alternative in detail,
the right alternative follows the same procedure. \texttt{Cr-BinB} produces two
calls to accumulation with $v_{1} = (s_{3} < (s_{i} - k))$, and $v_{2} =
\chc[A]{c,\neg b} \vee s_{b}$, $v_{2}$ is still stuck and will thus be returned,
$v_{1}$ is no longer stuck will be fully reduced to a symbolic term.

Accumulation will apply \texttt{Ac-BinIB} with $v_{1}' = s_{3}$ and $v_{2}' = s_{i}
- k$. $v_{1}'$ is already accumulated and will be returned, \texttt{Ac-BinI}
will be applied to $v_{2}'$, will translate $\kf{k}$ to a symbolic $s_{k}$ via
\texttt{Ac-GenI}, and update \aStore{h_{core}} to $\aStore{h_{L}} = \{(k, s_{k})
\cup \aStore{h_{core}}\}$. Thus we have $v_{2}'$ accumulated to $v_{2}' = s_{i}
- s_{k}$ which allows an application of the computation rule \texttt{Ac-SBinI}
to produce a single symbolic, $v_{2}' = s_{i-k}$ with $\aStore{h_{L}} =
\kf{\{(s_{i} - s_{k}, s_{i-k}), (k,s_{k})\} \cup \aStore{h_{core}}}$. The
recursion continues to unwind with the result of \texttt{Ac-BinIB} as $v_{1}' =
s_{3} < s_{i-k}$, the rule \texttt{Ac-SBinIB} can be applied yielding the result
for $v_{1}$ as $v_{1} = s_{s_{3}<s_{i-k}}$ with store $\aStore{h_{L}} =
\kf{\{(s_{3} < s_{i-k}, s_{s_{3}<s_{i-k}}), (s_{i} - s_{k}, s_{i-k}),
  (k,s_{k})\} \cup \aStore{h_{core}}}$.

With $v_{1}$ accumulated we have a new variational core $s_{s_{3}<s_{i-k}} \wedge
(\chc[A]{c, \neg b} \vee s_{b})$, only this time, depending on the alternative,
\configuration{} has enough information to configure $\kf{A}$. Again, we must
find a choice in the focus in order to proceed, once \chc[A]{c, \neg b} is in
focus \texttt{Cr-RB} (the right version of \texttt{Cr-LB}) will be applied.
$\kf{A} \in \configuration{L}$ and so the left alternative $\kf{c}$ will replace
the choice for $s_{s_{3}<s_{i-k}} \wedge (c \vee s_{b})$. This formula will
switch into accumulation due to \texttt{Cr-BinB} and be processed to a single
symbolic similarly to $s_{s_{3}<s_{i-k}}$. Once the symbolic has been created,
the \texttt{Sym} rule calls evaluation which performs the assertion stack
manipulation, writes the symbolic to the base solver. A model is generated with
the \texttt{Gen} rule and combined with an empty variational model. With the
model for the \true{} variant of $\kf{A}$ the process backtracks to compute the
false variant.

%
\paragraph{Variational \ac{smt} models}
%
\begin{figure}[h]
    \centering
    \begin{subfigure}[t]{\textwidth}
  \begin{tabbing}
    \qquad \quad \= $\iV{} \rightarrow$ -1 \\
    \> $\cV{} \rightarrow$ 0 \\
    \\
    \\
    \quad $C_{FF}$ = \{(\AV{}, \false{}), (\BV{}, \false{})\} \\
  \end{tabbing}
\end{subfigure}%
\begin{subfigure}[t]{\textwidth}
  \begin{tabbing}
    \\
    \\
    \qquad \quad \= $\aV{} \rightarrow$ \tru{} \\
    \\
    \quad $C_{FF}$ = \{(\AV{}, \fls{}), (\BV{}, \tru{})\} \\
  \end{tabbing}
\end{subfigure}%
\newline
\begin{subfigure}[t]{\textwidth}
  \begin{tabbing}
    \qquad \quad \= $\iV{} \rightarrow$ 0 \\
    \> $\cV{} \rightarrow$ 1 \\
    \\
    \\
    \quad $C_{FT}$ = \{(\AV{}, \tru{}), (\BV{}, \fls{})\} \\
  \end{tabbing}
\end{subfigure}%
\begin{subfigure}[t]{\textwidth}
  \begin{tabbing}
    \qquad \quad \= $\iV{} \rightarrow$ 0 \\
    \> $\cV{} \rightarrow$ 0 \\
    \\
    \> $\bV{} \rightarrow$ -10 \\
    \quad $C_{TT}$ = \{(\AV{}, \tru{}), (\BV{}, \tru{})\} \\
  \end{tabbing}
\end{subfigure}%
    \caption{Possible plain models for variants of $\kf{f}$. \itodo{correct these}}%
    \label{fig:vsmt:models:plain}
\end{figure}
%
To support \ac{smt} theories, variational models must be abstract enough to
handle values other than Booleans. Functionally, variational \ac{smt} models
must satisfy several constraints: the variational \ac{smt} model must be more
memory efficient than storing all models returned by the solver naively. The
variational model must allow users to find satisfying values for a variant. The
model must allow users to find all variants a variable has a particular value or
range of values.

Furthermore, several useful properties of variational models should be
maintained: The model is non-variational; thus the user does not need to
understand the choice calculus to understand their results. The model produces
results that can be fed into a plain \ac{sat} solver (or \ac{smt} solver in the
extension). The model can be built incrementally and without regard to the
ordering of results, because it forms a commutative monoid under $\vee$.

To maintain these properties and satisfy the functional requirements, our
strategy for variational \ac{smt} models is to create a mapping of variables to
\ac{smt} expressions. By virtue of this strategy, variables are disallowed from
changing types across the set of variants and hence disallowed from changing
type as the result of a choice. For any variable in the model, we assume the
type returned by the base solver is correct, and store the satisfying value in a
linked list constructed \emph{if-statements}\footnote{Also called a
  church-encoded list}. Specifically, we utilize the function $ite : \mathbb{B}
\rightarrow T \rightarrow T$ from the SMTLIB2 standard to construct the list.
All variables are initialized as undefined (\emph{Und}) until a value is
returned from the base solver for a variant. To ensure the correct value of a
variable corresponds to the appropriate variant, we translate the configuration
which determines the variant to a variation context, and place the appropriate
value in the \emph{then} branch. For example, a possible entry for $\kf{j}$ in
the variational model of $\kf{g}$ would be $j \rightarrow (\kf{ite}\ \kf{A}\ 1\
(\kf{ite}\ \neg\kf{A}\ 0\ \kf{Und}))$.

\autoref{fig:vsmt:models:plain} show possible plain models for $\kf{f}$ with the
corresponding variational \ac{smt} model display in
\autoref{fig:vsmt:models:var}. We've added line breaks to emphasize the branches
the $\kf{then}$ and $\kf{else}$ branches of the $\kf{ite}$ SMTLIB2 primitive.

\begin{figure}[h]
  \centering
  \begin{subfigure}[t]{\textwidth}
  \begin{tabbing}
  \hquad \= $\_Sat \rightarrow (\neg \AV{} \wedge \neg \BV{}) \vee (\neg \AV{} \wedge \BV{}) \vee (\AV{} \wedge \neg \BV{}) \vee (\AV{} \wedge \BV)$ \\
  \> \iV{}\quad\hspace{1.7ex}\=$\rightarrow$ ($\kf{ite}$ ($\AV{} \wedge \BV{}$)\; \=2 \\
  \> \> \> ($\kf{ite}$\; ($\AV{} \wedge \neg \BV{}$)\; \=0 \\
  \> \> \> \> ($\kf{ite}$\; ($\neg \AV{} \wedge \neg \BV{}$)\; -1 $\kf{Und}$))) \\

  \> \cV{}\quad\hspace{1.7ex}\=$\rightarrow$ ($\kf{ite}$ ($\AV{} \wedge \BV{}$)\; \=0 \\
  \> \> \> ($\kf{ite}$\; ($\AV{} \wedge \neg \BV{}$)\; \=0 \\
  \> \> \> \> ($\kf{ite}$\; ($\neg \AV{} \wedge \BV{}$)\; \=0  \\
  \> \> \> \> \> ($\kf{ite}$\; ($\neg \AV{} \wedge \neg \BV{}$)\; \=0 $\kf{Und}$))) \\

  \> \aV{}\quad\hspace{1.7ex}\=$\rightarrow$ ($\kf{ite}$ ($ \AV{} \wedge \BV{}$)\; \=\tru{} \\
  \> \> \> ($\kf{ite}$ ($\neg \AV{} \wedge \BV{}$) \fls{} $\kf{Und}$))\\


  \> \bV{}\quad\hspace{1.7ex}\=$\rightarrow$ ($\kf{ite}$ ($\AV{} \wedge \BV{}$)\; \=3 \\
  \> \> \> ($\kf{ite}$\; ($\AV{} \wedge \neg \BV{}$)\; \=-10 \\
  \> \> \> \> ($\kf{ite}$\; ($\neg \AV{} \wedge \BV{}$)\; \=0  \\
  \> \> \> \> \> ($\kf{ite}$\; ($\neg \AV{} \wedge \neg \BV{}$)\; \=4 $\kf{Und}$))) \\
\end{tabbing}
\end{subfigure}

  \caption{Variational model corresponding to the plain models in
    \autoref{fig:vsmt:models:plain}.}%
  \label{fig:vsmt:models:var}
\end{figure}

This formulation maintains the functional requirements of the model. We maintain
a special variable $\_Sat$ to track the variants that were found satisfiable. In
this case all variants are satisfiable and thus we have four clauses over
dimensions in disjunctive normal form. If a user has a configuration then they
only need to perform substitution to determine the value of a variable under
that configuration. For example, if the user were interested in the value of
\iV{} in the $\{(\AV{}, \tru{}), (\BV{}, \tru{})\}$ variant they would
substitute the configuration into \vc{\iV{}} and recover 0 from the first
$\kf{ite}$ case. To find the variants at which a variable has a value, a user
may employ a \ac{smt} solver, add \vc{\iV{}} as a constraint, and query for a
model.

This maintains the desirable properties of variational \ac{sat} models while
allowing any type specified in the SMTLIB2 standard. The variational \ac{smt}
model does not require knowledge of choice calculus or variation, it is still
monoidal---although not a commutative monoid---and can be built in any order as
long as there are no duplicate variants; a scenario that is impossible by the
property of synchronization on choices. However, variational \ac{sat} models
clearly compressed results by preventing duplicate values with constant
variables, the variational \ac{smt} model allows for duplicate values, if those
values are parameterized by disjoint variants. For example, both \iV{} and \cV{}
contain duplicate values, but only one: \iV{} is easy to check in $O(1)$ time as
the duplicates are sequential in \vc{\iV{}} and can thus be checked during model
construction. Such a case would be easily avoided in an implementation by
tracking the values a variable has been assigned in all variants. However, we
desire to keep variational models as simple as possible and therefore only
present the minimum required machinery.



%%% Local Variables:
%%% mode: latex
%%% TeX-master: "../thesis"
%%% End:
