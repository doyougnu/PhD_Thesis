Over the last two decades, satisfiability and satisfiability-modulo theory
(SAT/SMT) solvers have grown powerful enough to be general purpose reasoning
engines employed in various areas of software engineering and computer science.
However, most practical use cases of SAT/SMT solvers require not just solving a
single SAT/SMT problem, but solving sets of related SAT/SMT problems. This
discrepancy was directly addressed by the SAT/SMT community with the invention
of incremental SAT/SMT solving. However, incremental SAT/SMT solvers require
end-users to hand write a program which dictates terms in a set that are shared
between problems and terms which are unique. By placing the onus on end-users to
write a program, incremental solvers couple the end-users' solution to the
end-users' exact sequence of SAT/SMT problems---making the solution overly
specific---and require the end-user to write extra infrastructure to coordinate
or handle the results. In this thesis, I apply results from research on
\emph{variational} programming languages to the domain of SAT/SMT solvers to
automate this interaction, creating the first variational SAT/SMT solver. I
demonstrate numerous benefits to this approach: End-users need only identify the
set of SAT/SMT problems to solve rather than identify the set \emph{and} provide
a program. Otherwise difficult optimizations can now be automatically detected
and applied. Through use of variational constructs, the variational SAT/SMT can
be made asynchronous and both single threaded and multi-threaded versions of
variational SAT/SMT solvers are more performent in their expected use case.

%%% Local Variables:
%%% mode: latex
%%% TeX-master: "../main"
%%% End: