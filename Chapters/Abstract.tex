Over the last two decades, satisfiability and satisfiability-modulo theory
(SAT/SMT) solvers have grown powerful enough to be general purpose reasoning
engines throughout software engineering and computer science. However, most
practical use cases of SAT/SMT solvers require not just solving a single SAT/SMT
problem, but solving sets of related SAT/SMT problems. This discrepancy was
directly addressed by the SAT/SMT community with the invention of incremental
SAT/SMT solving. However, incremental SAT/SMT solvers require end-users to hand
write a program which dictates the terms that are shared between problems and
terms which are unique. By placing the onus on end-users, incremental solvers
couple the end-users' solution to the end-users' exact sequence of SAT/SMT
problems---making the solution overly specific---and require the end-user to
write extra infrastructure to coordinate or handle the results.

My work is concerned with proliferating the concept of variation as a
computational concept similar to the concept of loops or objects, thereby
obtaining improved usability, performance, and software reuse. Variation
encapsulates the idea of \emph{similar yet different} objects in a domain. This
dissertation applies theory from \emph{variational} programming to the domain of
SAT/SMT solvers to solve the aforementioned problems and demonstrate the
benefits of variation as a computational concept.

This thesis formalizes a variational propositional logic and specifies the first
variational SAT solver. I show numerous benefits gained with this approach:
%
A variational solver is more declarative than a SAT solver as previously
necessary control flow operations are no longer required.
%
As a result, end-users need only identify the set of SAT problems of interest
rather than identify the set \emph{and} direct the solver via the control flow
operations.
%
With the concept of variation previously difficult optimizations can be
syntactically detected and applied before the solver runtime.
%
When solving a set of related problems a variational solver promotes more reuse
of information and can thus be more performent than a non-variational solver in
its use case.

Furthermore, \todo{double check these}the thesis explores several extensions and
reflect on lessons learned from implementing a large scale variation-aware
system. It defines a general algorithm to construct a single variational string
from a set of non-variational strings. It extends variational SAT solving to
variational SMT solving, and with variational constructs, it shows that a
variational SAT/SMT solver can trivially  be made asynchronous.

%%% Local Variables:
%%% mode: latex
%%% TeX-master: "../thesis"
%%% End: