Over the last two decades, satisfiability and satisfiability-modulo theory
(SAT/SMT) solvers have grown powerful enough to be general purpose reasoning
engines throughout software engineering and computer science. However, most
practical use cases of SAT/SMT solvers require not just solving a single SAT/SMT
problem, but solving sets of related SAT/SMT problems. This discrepancy was
directly addressed by the SAT/SMT community with the invention of incremental
SAT/SMT solving. However, incremental SAT/SMT solvers require end-users to hand
write a program which dictates the terms that are shared between problems and
terms which are unique. By placing the onus on end-users, incremental solvers
couple the end-users' solution to the end-users' \emph{exact} sequence of
SAT/SMT problems---making the solution overly specific---and require the
end-user to write extra infrastructure to coordinate or handle the results.

This dissertation argues that the aforementioned problems result from accidental
complexity produced by solving a problem that is \emph{variational} without the
concept of \emph{variation}, similar to problematic use of \rn{goto} statements
in the absence of \rn{while} loop constructs. To demonstrate the argument, this
thesis applies theory from \emph{variational} programming to the domain of
SAT/SMT solvers to create the first variational SAT solver and solve
aforementioned problems.
%
To do so, the thesis formalizes a variational propositional logic and specifies
variational SAT solving as a compiler; which compiles variational SAT problems
to non-variational SAT that are processed by an industrial strength SAT solver.
It shows that the compiler is an instance of a variational fold and uses that
fact to extend the variational SAT solver to a variational SMT solver.

% Finally, it defines a general algorithm to construct a single
% variational string from a set of non-variational strings.

%%% Local Variables:
%%% mode: latex
%%% TeX-master: "../thesis"
%%% End: