%
The syntax of variational propositional logic is given in
\autoref{fig:cc:stx}. It extends the propositional formula notation of \pl{}
with a single new connective called a \emph{choice} from the choice calculus.
%
A choice $\chc[D]{f_1,f_2}$ represents either $f_1$ or $f_2$ depending on the
Boolean value of its \emph{dimension} $D$. We call $f_1$ and $f_2$ the
\emph{alternatives} of the choice.
%
Although dimensions are Boolean variables, the set of dimensions is disjoint
from the set of variables from \pl{}, which may be referenced in the leaves of
a formula. We use lowercase letters to range over variables and uppercase
letters for dimensions.

The syntax of \ac{vpl} does not include derived logical connectives, such as
$\rightarrow$ and $\leftrightarrow$. However, such forms can be defined
from other primitives and are assumed throughout the thesis.


%%% Local Variables:
%%% mode: latex
%%% TeX-master: "../../thesis"
%%% End:
