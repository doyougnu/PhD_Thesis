One of the most important aspects of any programming language is the ability to
control complexity, especially as software written in that language grow. Novel
and recent work on \emph{variation} and \emph{variational
  programming}~\cite{EW11gttse,EW11tosem,HW16fosd,CEW16ecoop,Walk14onward}
attempt to control complexity that is induced in software when many
\emph{similar yet distinct} kinds of software must coexist. For example, the
same piece of software is often created and \emph{ported} to other platforms,
thus creating many similar, yet distinct kinds of that software which must be
maintained. Such instances of \emph{variation} are ubiquitous: Web applications
are tested on development, staging, and then production servers; programming
languages maintain backwards compatibility across versions and so do libraries;
databases evolve over time and locale; and device drivers must work with
different processors. However, the theory of variation and variational
programming has been successful in small systems, it has yet to be tested in a
performance demanding practical domain. In the words of Joe
Armstrong\cite{armstrongThesis}, ``No theory is complete without proof that the
ideas work in practice''; this is the central project of this thesis, to put the
ideas of \emph{variation} and \emph{variational programming}to the test in the
practical domain of \ac{sat}.

I begin by identifying and defining instances of variation in \acl{sat} problems
and \acl{sat} solvers. The thesis major contributions of this thesis is the
formalization of a \emph{\ac{vpl}} and \emph{variational \acl{sat}} and the
construction of a \emph{variational \ac{sat} solver}. Variational \ac{sat}
solving is then

In this thesis I use \ac{sat} solving to explore the characteristics a domain
must have to benefit from the concept of variation. I show that given a
non-variational system a \emph{variational-aware} system can be constructed by
using the non-variational system as a backend engine.


I then demonstrate that the theory ``works'' through
performance and usability improvements to classes of problems which previously
required hand-written solutions.


%%% Local Variables:
%%% mode: latex
%%% TeX-master: "../thesis"
%%% End: