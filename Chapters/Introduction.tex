One of the most important aspects of any programming language is the ability to
control complexity, especially as software written in that language grows. The
burgeoning field of \emph{variation theory} and \emph{variational
  programming}~\cite{EW11gttse,EW11tosem,HW16fosd,CEW16ecoop,Walk14onward}
attempt to control complexity that is induced in software when many
\emph{similar yet distinct} kinds of software must coexist. For example, the
same piece of software is often \emph{ported} to other platforms, creating many
similar, yet distinct instances of that software which must be maintained. Such
instances of variation are ubiquitous: Web applications are tested on multiple
servers; programming languages maintain backwards compatibility and so do
software libraries; databases evolve over time and locale; and device drivers
must work with different processors. Variation theory and variational
programming has been successful in small systems, yet it has not been tested in
a performance demanding practical domain. In the words of Joe
Armstrong\cite{armstrongThesis}, ``No theory is complete without proof that the
ideas work in practice''; this is the central project of this thesis, to put the
ideas of \emph{variation} and \emph{variational programming}to the test in the
practical domain of \ac{sat}.

The major contribution of the thesis is the formalization of a \emph{\ac{vpl}},
\emph{variational \acl{sat}}, and the construction of a \emph{variational
  \ac{sat} solver}. The thesis makes several other contributions. It extends
variational \acl{sat} to variational \acl{smt}. It demonstrates reusable
techniques and architecture for constructing \emph{variational or
  variation-aware} systems using their non-variational counterparts. It shows
that, with the concept of variation, the variational \ac{smt} and \ac{sat}
solvers can be trivially parallelized. Lastly, the thesis provides a general
algorithm to construct variational strings from a set of non-variational strings
and argues for the proliferation of variation theory to other domains in
computer science.

%%% Local Variables:
%%% mode: latex
%%% TeX-master: "../thesis"
%%% End: