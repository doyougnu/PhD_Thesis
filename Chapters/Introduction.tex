One of the most important aspects of any programming language is the ability to
control complexity, especially as software written in that language grows. The
burgeoning field of \emph{variation theory} and \emph{variational
  programming}~\cite{EW11gttse,EW11tosem,HW16fosd,CEW16ecoop,Walk14onward}
attempt to control complexity which is induced into a software artifact when
many \emph{similar yet distinct} kinds of the same software artifact must
coexist. For example, software is often \emph{ported} to other platforms,
creating similar, yet distinct instances of that software which must be
maintained. Such instances of variation are ubiquitous: Web applications are
tested on multiple servers; programming languages maintain backwards
compatibility and so do software libraries; databases evolve over time, locale
and data; and device drivers must work with varying processors and
architectures. Variation theory and variational programming have been successful
in small systems\todo{cite variational data structures and images}, yet it has
not been tested in a performance demanding practical domain. In the words of Joe
Armstrong\cite{armstrongThesis}, ``No theory is complete without proof that the
ideas work in practice''; this is the central project of this thesis, to put the
ideas of \emph{variation} and \emph{variational programming} to the test in the
practical domain of \ac{sat}.

The major contribution of this thesis is the formalization of a \emph{\ac{vpl}},
\emph{variational \acl{sat}}, and the construction of a \emph{variational
  \ac{sat} solver}. In the next section I motivate the use of variation theory
and variational techniques in \acl{sat}. In addition to work on variational
\ac{sat} several other contributions are made. The thesis extends variational
\acl{sat} to variational \ac{smt}. It demonstrates reusable techniques and
architecture for constructing \emph{variational or variation-aware} systems
using the non-variational counterparts of these systems for other domains. It
shows that, with the concept of variation, the variational \ac{smt} and \ac{sat}
solvers can be trivially parallelized. Lastly, the thesis provides a general
algorithm to construct variational strings from a set of non-variational strings
and argues for the proliferation of variation theory to other domains in
computer science.


%%% Local Variables:
%%% mode: latex
%%% TeX-master: "../thesis"
%%% End: