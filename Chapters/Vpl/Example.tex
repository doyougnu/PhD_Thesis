%
To demonstrate the application of \ac{vpl}, we encode the incremental example
from \autoref{chapter:background}. Our goal is to construct a single variational
formula that encodes the related plain formulas \pV{}, \qV{}, \rV{}. Ideally,
this variational formula should maximize sharing among the plain formulas in
order to avoid redundant analyses during a variational solving. We reproduce the
formulas below for the convenience:
%
\begin{align*}
  p =\ a \wedge b \wedge c \wedge e && q=\ a \wedge (b \vee \neg \iV{}) \wedge c \wedge (\gV{} \rightarrow c) && r =~z \leftrightarrow (a \wedge b \wedge c \wedge e)
\end{align*}

%
Every set of plain formulas can be encoded as a variational formula
systematically by first constructing a nested choice containing all of the
individual variables as alternatives, then factoring out shared subexpressions
by applying the laws in \autoref{fig:cc:eqv}.
%
Unfortunately, the process of globally minimizing a variational formula in this
way is hard\footnote{\label{vpl:bdd}. We hypothesize that it is equivalent to
  BDD minimization, which is NP-complete, but the equivalence has not been
  proved; see~\cite{Walk14onward}.} since we must apply an arbitrary number of
laws right-to-left in order to set up a particular sequence of left-to-right
applications that factor out commonalities.

Due to the difficulty of minimization, we instead demonstrate how one can build
such a formula \emph{incrementally}.
%
Our variational formula will use the dimensions \dimP, \dimQ, \dimR{} to
represent the respective variants. Unique portions of each variant will be
nested into the left alternative so that the unique portion is considered when
the dimension is bound to \true in the configuration.

%
We begin by combining \pV{} and \rV{} since the differences\todo{section on
  encoding problem}\footnote{There are many ways to assess the difference of two
  formulas. For now the reader may consider it reducible to the Levenshtein
  distance of two strings~\cite{Levenshtein_SPD66}. We return to this discussion
  in \ldots{}} between the two are smaller than between other pairs of feature
models in our example. Feature models may be combined in any order as long as
the variants in the resulting formula correspond to their plain counterparts.
The only change between \pV{} and \rV{} is the addition of \zV{} and thus we
wrap the leaf in a choice with dimension \dimR. This yields the following
variational formula.
%
\begin{equation*}
  v_{pr} = \chc[\dimR]{\zV, \tru{}} \leftrightarrow a \wedge b \wedge c \wedge e
\end{equation*}
%
%
We exploit the fact that $\wedge$ forms a monoid with \tru{} to recover a
formula equivalent to \pV{} for configurations where \dimR{} is \false.

Next we combine $f$ with \qV{} to obtain a variational formula that the
propositional formulas \pV{}, \qV{}, \rV{}. As before, every change
in \ldots{} is wrapped in a choice in dimension \dimQ. The choice in \ldots{} is
nested in the right alternative of a choice in \ldots{} because that change is not
present in \ldots{}:
%
%
Now that we have constructed the variational formula we need to ensure that it
encodes all variants of interest and nothing else. In this example, this is
relatively easy to confirm by enumerating all total configurations involving
\ldots{} and \ldots{}. However, we'll return to the general case in the discussion
of variational models in \autoref{chapter:vsat}.


%%% Local Variables:
%%% mode: latex
%%% TeX-master: "../../thesis"
%%% End:
