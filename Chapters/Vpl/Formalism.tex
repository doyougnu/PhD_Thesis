\label{section:vpl:formalism}
%
Having defined the syntax and semantics of \ac{vpl} the rest of this chapter
will define useful functions and properties. We conclude the chapter with an
example of encoding a set of \pl{} formulas to a single \ac{vpl} formula.
%
First define useful functions to retrieve interesting aspects of \ac{vpl}
formulas.

\begin{definition}[Dimensions]
  Given a formula $f \in \ac{vpl}$, let $\dimensions{f}$ be the set of unique
  dimensions in the formula:  $\dimensions{f} = \{ D\ |\ D \in f \}$.
\end{definition}

For example, \dimensions{\chc[A]{p,q}\wedge\chc[B]{r,s}} = $\{\kf{A,B}\}$ and
\dimensions{\chc[A]{p,q}\wedge\chc[A]{r,s}} = $\{\kf{A}\}$.
%
Similarly we define a notion of \emph{cardinality} over \ac{vpl} formulas.
%
\begin{definition}[Dimension-cardinality]
  The dimension-cardinality of a formula $f \in \ac{vpl}$ is the cardinality of
  the set of unique dimensions in a formula. We use the following notation as
  shorthand: $\dcardinality{f} = |{\dimensions{f}}|$.
\end{definition}

Similarly to $\kf{Dimensions}$ it is useful to have projections from a \ac{vpl}
formula to possible variants:
%
\begin{definition}[Variants]
  Given a formula $f \in \ac{vpl}$, let $\variants{f}$ be the set of all
  possible \emph{variants} of $f$. Thus, $\variants{f} = \{\ v\ |\ \exists
  C.\ \sem[C]{f} = v \,\}$
\end{definition}
%
and we can define a projection for all plain variants as well:
%
\begin{definition}[\pl{} Variants]
  Given a formula $f \in \ac{vpl}$, let $\variants*{f}$ be the set of all
  possible \emph{plain variants} of $f$. Thus, $\variants*{f} = \{\ v\ |\ \exists C.\
  \sem[C]{f} = v,\ v \in \pl{} \,\}$
\end{definition}
%
Using $\kf{Dimensions}$ we can define a more precise property on configurations.
%
\begin{definition}[Minimal Configuration]
  We say a configuration $\kf{C}$ is minimal with respect to some formula $f \in
  \ac{vpl}$ if and only if $ \sem[C]{f} = v$ where $v \in
  \variants*{f}$ and $\nexists \prime*{C} .\ \sem[\prime*{C}]{f} = v$
  and $\cardinality{\prime*{C}} < \cardinality{C}$.
\end{definition}

Note if \fV{} is plain then $\fV{} = \kf{v}$ and $\kf{C} = \varnothing$ but is
still minimal. One may think of a minimal configuration as a total configuration
with \emph{nothing extra}. For example, the configuration $C = \{\kf{(A,\true),
  (B,\false),(E,\true)}\}$ is total with respect to the formula $f =
\chc[A]{p,q}\wedge\chc[B]{r,s}$ because $C$ eliminates all choices in $f$.
However $\kf{C}$ is not minimal with respect to $f$ because there exists a
configuration $\prime*{C} = \set{(A,\true), (B,\false)}$ that is total, produces
the same variant and is smaller, \ie{}, $\cardinality{\prime*{C}} <
\cardinality{C}$. Similarly, consider a formula such as $\chc[A]{\pV{},
  \chc[B]{\qV{}, \rV{}}}$ where the minimal configuration changes depending on
the variant. In this case, the configuration $C = \set{(A, \true), (B, \false)}$
is not minimal for the \set{(A,\true)} variant, however $\kf{C} =
\set{(A,\true)}$ is, but $\kf{C}$ is minimal for every \set{(A,\false)} variant.

With these functions and definitions we can prove that \ac{vpl} reduces to
\pl{}:
%
\begin{theorem}[\ac{vpl} reducible to \pl{}]%
  \label{thm:vpltopl}
  For any configuration $C$ and any formula $f \in \ac{vpl}$, if $C$ is total
  with respect to $f$, then $\sem[C]{f} \in \pl{}$
\end{theorem}
%
\begin{proof}
  This follows directly from the semantics of configuration in
  \autoref{fig:cc:cfg}, the definition of a total configuration, and
  \autoref{lemma:vpl:deterministic}. The proof is by structural induction on
  $\kf{f}$ and case analysis. The only interesting case is the case for choices.
  Since $C$ is total we have $C : D \rightarrow \booleans{}$ instead of $C : D
  \rightarrow \booleans{}_{\bot}$, thus the last case for choices, where
  $\kf{C(D)} = \bot$, can never happen and therefore configuration of a formula
  $\kf{f}$ with a total configuration is a total function. The other base case
  is over terminals, which are in \pl{} by definition, and each other case
  propagate the total configuration to a base case. Thus each choice and its'
  alternatives are recursively reified for $\kf{f}$, and by definition a
  \ac{vpl} formula which lacks choices is $\in \pl{}$.
\end{proof}


%%% Local Variables:
%%% mode: latex
%%% TeX-master: "../../thesis"
%%% End: