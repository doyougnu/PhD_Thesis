\begin{definition}[Choice]
  A Choice \chc[D]{\e{1},\e{2}} is a binary connective that denotes a local
  point of variation between two possible \textit{alternatives}; \e{1}, and
  \e{2}, with a single Boolean variable, $D$, called a \textit{dimension}.
\end{definition}

\begin{definition}[Dimension]
  A Dimension is a Boolean variable that represents a choice and is used to
  determine an alternative during configuration. For some formula $f$, let
  $\kf{Dimensions(f)}$ denote the dimensions present in a \ac{vpl} formula.
\end{definition}

\begin{definition}[Alternative]
  An alternative is a projection from a choice to one of the two possible
  \ac{vpl} formulas that the choice simultaneously represents.
\end{definition}

\begin{definition}[Valid Configuration]
  We say configuration $C$ is valid with respect to some formula $f$ iff
  $\kf{Dom(C)}\ \cap\ \kf{Dimensions(\kf{f})\ \neq \varnothing}$.
\end{definition}

\begin{definition}[Total Configuration]
    \label{tot:conf}
    For any valid configuration $C$, with respect to formula $\kf{f}$, let
    $C$ be total iff $\kf{Dimensions(f) \subseteq Dom(C)}$.
\end{definition}
%
\begin{corollary}[Partial Configuration]
  For any valid configuration $C$ with respect to formula $f$, let $C$ be
  partial iff $C$ is not total.
\end{corollary}
\begin{definition}[Variant]
  Let $f$ be a variant of a formula $e$ iff there exists some
  valid, total configuration $C$ such that $\sem[C]{e} = f$.
\end{definition}

Because variants are defined only with total configurations, all variants cannot
contain choices and are hence called \textit{plain}:

\begin{definition}[Plain formula]
  For any formula $e \in\ \ac{vpl}$, let $e$ be plain iff $\kf{Dimensions(e)}
  = \varnothing$
\end{definition}
%
\begin{lemma}[Variants are plain]
  By \autoref{thm:cclToPl} and the fact that variants are found via total
  configurations
\end{lemma}


%%% Local Variables:
%%% mode: latex
%%% TeX-master: "../../thesis"
%%% End: