\label{section:vpl:formalism}
%
Having defined the syntax and semantics of \ac{vpl} the rest of this chapter
will define useful functions and properties. We conclude the chapter with an
example of encoding a set of \pl{} formulas to a single \ac{vpl} formula.
%
First define useful functions to retrieve interesting aspects of \ac{vpl}
formulas.

\begin{definition}[Dimensions]
  Given a formula $f \in \ac{vpl}$, let $\dimensions{f}$ be the set of unique
  dimensions in the formula:  $\dimensions{f} = \{ D\ |\ D \in f \}$.
\end{definition}

For example, \dimensions{\chc[A]{p,q}\wedge\chc[B]{r,s}} = $\{\kf{A,B}\}$ and
\dimensions{\chc[A]{p,q}\wedge\chc[A]{r,s}} = $\{\kf{A}\}$.
%
Similarly we define a notion of \emph{cardinality} over \ac{vpl} formulas.
%
\begin{definition}[Dimension-cardinality]
  The dimension-cardinality or d-cardinality of a formula $f \in \ac{vpl}$ is
  the cardinality of the set of unique dimensions in a formula. We use the
  following notation as shorthand: $\dcardinality{f} = |{\dimensions{f}}|$.
\end{definition}

Similarly to $\kf{Dimensions}$ another useful function is $\kf{Variants}$:

\begin{definition}[Variants]
  Given a formula $f \in \ac{vpl}$, let $\variants{f}$ be the set of all
  possible \emph{plain variants} of $f$. Thus, $\variants{f} = \{\ v\ |\ \exists C.\
  v = \sem[C]{f},\ v \in \pl{} \,\}$
\end{definition}

Using $\kf{Dimensions}$ we can now define a more precise property on
configurations.
%
\begin{definition}[Minimal Configuration]
  We say a configuration $C$ is minimal with respect to some formula $f \in
  \ac{vpl}$ iff $\domain{C}\ \cap\ \kf{Dimensions(\kf{f})\ \neq \varnothing}$.
\end{definition}

One may think of a minimal configuration as a total configuration with
\emph{nothing extra}. For example, the configuration $C = \{\kf{(A,\true),
  (B,\false),(E,\true)}\}$ is total with respect to the formula $f =
\chc[A]{p,q}\wedge\chc[B]{r,s}$ because $C$ eliminates all choices in $f$.
However $\kf{C}$ is not minimal with respect to $f$ as $\kf{Dom(C)}\ \cap\
\kf{Dimensions(\kf{f})\ = \{\kf{E}\}}$, since $C$ contains an extra binding for
$E$ that is not needed to configure $f$.

With these functions and definitions we can prove useful lemmas and theorems.
%
We'll begin by proving that the configuration semantics over a \ac{vpl} formula
is \emph{confluent}, \ie{}, each configuration with respect to a \ac{vpl}
formula precisely specifies one variant:
%
\begin{lemma}[\ensuremath{\sem[C]{f}} is deterministic]
  \label{lemma:vpl:deterministic}
  For any configuration $C$, if \ensuremath{\semL f \semR_{C} = g}, and
  \ensuremath{\semL f \semR_{C} = h}, then $\kf{g} = \kf{h}$.
\end{lemma}
%
\begin{proof}
  As shown in \autoref{fig:cc:cfg}, there is only one applicable case for each
  relation in \ac{vpl}. Furthermore, by the property of synchronization, each
  choice must be configured to the same alternative, thus \sem[C]{f} is
  deterministic for some $\kf{C}$ and $\kf{f}$.
\end{proof}

Confluence is given directly by \autoref{lemma:vpl:deterministic} because
\sem[C]{f} cannot produce different results. We state it here because it is a
necessary property for the proof of variational preservation in
\autoref{chapter:vsat}.
%
\begin{theorem}[\ensuremath{\sem[C]{f}} is Confluent]
  \label{thm:vpl:confluent}
  For any configuration $C$, if \ensuremath{\semL f \semR_{C} = g}, and
  \ensuremath{\semL f \semR_{C} = h}, then there exists a $\kf{C'}$ and
  $\kf{f'}$ such that \ensuremath{\semL g \semR_{C'} = f'} and \ensuremath{\semL
    h \semR_{C'} = f'}
\end{theorem}
%
\begin{proof}
  By structural induction on the relation \sem[C]{f} proceeding by case
  analysis. From \autoref{lemma:vpl:deterministic} if \sem[C]{f} = $\kf{g}$ and
  \sem[C]{f} = $\kf{h}$ then $\kf{h = g}$. Therefore for some \prime*{C} and
  \prime*{f}, $\sem[C]{f} = \sem[\prime*{C}]{g} = \sem[\prime*{C}]{h} =
  \prime*{f}$.
\end{proof}
%
Similarly, clearly \ac{vpl} reduces to \pl{}:
%
\begin{theorem}[\ac{vpl} reducible to \pl{}]
  \label{thm:vpltopl}
  For any configuration $C$ and any formula $f \in \ac{vpl}$, if $C$ is total
  with respect to $f$, then $\sem[C]{f} \in \pl{}$
\end{theorem}
%
\begin{proof}
  This follows directly from the semantics of configuration in
  \autoref{fig:cc:cfg}, the definition of a total configuration, and
  \autoref{lemma:vpl:deterministic}. The proof is by structural induction on
  $\kf{f}$ and case analysis. The only interesting case is the case for choices.
  Since $C$ is total we have $C : D \rightarrow \booleans{}$ instead of $C : D
  \rightarrow \booleans{}_{\bot}$, thus the last case for choices, where
  $\kf{C(D)} = \bot$, can never happen and therefore configuration of a formula
  $\kf{f}$ with a total configuration is a total function. The other base case
  is over terminals, which are in \pl{} by definition, and each other case
  propagate the total configuration to a base case. Thus each choice and its'
  alternatives are recursively reified for $\kf{f}$, and by definition a
  \ac{vpl} formula which lacks choices is $\in \pl{}$.
\end{proof}


%%% Local Variables:
%%% mode: latex
%%% TeX-master: "../../thesis"
%%% End: