%
Having defined the syntax and semantics of \ac{vpl} the rest of this chapter
will define useful functions and properties of \ac{vpl} formulas. Lastly, we
conclude the chapter with an example of encoding a set of \pl{} formulas to a
single \ac{vpl} formula.

We begin with useful functions to retrieve interesting aspects of \ac{vpl}
formulas.

\begin{definition}[Dimensions]
  Given a formula $f \in \ac{vpl}$, let $\dimensions{f}$ be the set of unique
  dimensions in the formula. Thus, $\dimensions{f} = \{ D\ |\ D \in f \}$.
\end{definition}

For example, \dimensions{\chc[A]{p,q}\wedge\chc[B]{r,s}} = $\{\kf{A,B}\}$ and
\dimensions{\chc[A]{p,q}\wedge\chc[A]{r,s}} = $\{\kf{A}\}$.
%
Similarly we define a notion of \emph{cardinality} over \ac{vpl} formulas.
%
\begin{definition}[Dimension-cardinality]
  The dimension-cardinality or d-cardinality of a formula $f \in \ac{vpl}$ is
  the cardinality of the set of unique dimensions in a formula. We use the
  following notation as shorthand: $\dcardinality{f} = |{\dimensions{f}}|$.
\end{definition}

Similarly to $\kf{Dimensions}$ another useful function is $\kf{Variants}$:

\begin{definition}[Variants]
  Given a formula $f \in \ac{vpl}$, let $\variants{f}$ be the set of all
  possible \emph{plain variants} of $f$. Thus, $\variants{f} = \{\ v\ |\ \exists C.\
  v = \sem[C]{f},\ v \in \pl{} \,\}$
\end{definition}

Using $\kf{Dimensions}$ we can now define a more precise property on
configurations.
%
\begin{definition}[Valid Configuration]
  We say a configuration $C$ is valid with respect to some formula $f \in
  \ac{vpl}$ iff $\kf{Dom(C)}\ \cap\ \kf{Dimensions(\kf{f})\ \neq \varnothing}$.
\end{definition}

One may think of a valid configuration as a total configuration with
\emph{nothing extra}. For example, the configuration $C = \{\kf{(A,\true),
  (B,\false),(E,\true)}\}$ is total with respect to the formula $f =
\chc[A]{p,q}\wedge\chc[B]{r,s}$ because $C$ eliminates all choices in $f$.
However $\kf{C}$ is not valid with respect to $f$ because $\kf{Dom(C)}\ \cap\
\kf{Dimensions(\kf{f})\ = \{\kf{E}\}}$, and thus in this case $C$ contains one
extra binding for $E$, than needed with respect to $f$.

\begin{definition}[Variant]
  Let $f$ be a variant of a formula $e$ iff there exists some
  valid, total configuration $C$ such that $\sem[C]{e} = f$.
\end{definition}

Because variants are defined only with total configurations, all variants cannot
contain choices and are hence called \textit{plain}:

\begin{definition}[Plain formula]
  For any formula $e \in\ \ac{vpl}$, let $e$ be plain iff $\kf{Dimensions(e)}
  = \varnothing$
\end{definition}
%
\begin{lemma}[Variants are plain]
  By \autoref{thm:cclToPl} and the fact that variants are found via total
  configurations
\end{lemma}
%
% \begin{theorem}[\ccl{} reducible to \pl{}]
%   \label{thm:cclToPl}
%   For any configuration $C$ and any formula $e$, if $C$ is
%   valid and total with respect to $e$, then $\sem[C]{e} \in \pl{}$
% \end{theorem}
%
% \begin{proof}
%   This follows directly from the semantics of configuration in
%   Figure~\ref{fig:bkg:cc:cfg}, and Definition~\ref{tot:conf}. The proof is done
%   by structural induction and case analysis; because we have a total
%   configuration, and the configuration semantic function is a total function,
%   every choice and its' configured alternatives, will be recursively reified for
%   $e$. Then by the definition of \ccl{} a formula which lacks choices is by
%   definition in \pl{}.
% \end{proof}

%%% Local Variables:
%%% mode: latex
%%% TeX-master: "../../thesis"
%%% End: