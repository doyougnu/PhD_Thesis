\label{section:vpl:semantics}
%
Conceptually, a variational formula represents several propositional logic
formulas at once, which can be obtained by resolving all of the choices. For
software product-line researchers, it is useful to think of \ac{vpl} as analogous
to \cpp{ifdef}-annotated \pl{}, where choices correspond to a
disciplined~\cite{LKA:AOSD11} application of \cpp{ifdef} annotations.
%
From a logical perspective, following the many-valued logic of
Kleene~\cite{kleene1968introduction,Rescher1969-RESML}, the intuition behind
\ac{vpl} is that a choice is a placeholder for two equally possible alternatives
that is deterministically resolved by reference to an external environment.
%
In this sense, \ac{vpl} deviates from other many-valued logics, such as modal
logic~\cite{sep-logic-modal}, because a choice \emph{waits} until there is
enough information in an external environment to choose an alternative (i.e.,
until the formula is \emph{configured}).

The \emph{configuration semantics} of \ac{vpl} is given in
\autoref{fig:cc:cfg} and describes how choices are eliminated from a
formula. The semantics is parameterized by a \emph{configuration}\ $C$, which is
a partial function from dimensions to Boolean values.
%
The first four cases of the semantics simply propagate configuration down the
formula, terminating at the leaves. The case for choices is the interesting one:
if the dimension of the choice is defined in the configuration, then the choice
is replaced by its left or right alternative corresponding to the associated
value of the dimension in the configuration. If the dimension is undefined in
the configuration, then the choice is left intact and configuration propagates
into the choice's alternatives.

If a configuration $C$ eliminates all choices in a formula $f$, we call $C$
\emph{total} with respect to $f$. If $C$ does \emph{not} eliminate all choices
in $f$ (i.e., a dimension used in $f$ is undefined in $C$), we call $C$
\emph{partial} with respect to $f$.
%
We call a choice-free formula \emph{plain}, and call the set of all plain
formulas that can be obtained from $f$ (by configuring it with every possible
total configuration) the \emph{variants} of $f$.

To illustrate the semantics of \ac{vpl}, consider the formula
$p\wedge\chc[A]{q,r}$, which has two variants: $p\wedge q$ when $C(A)=\true$
and $p\wedge r$ when $C(A)=\false$.
%
From the semantics, it follows that choices in the same dimension are
\emph{synchronized} while choices in different dimensions are
\emph{independent}. For example, $\chc[A]{p,q}\wedge\chc[B]{r,s}$ has four
variants, while $\chc[A]{p,q}\wedge\chc[A]{r,s}$ has only two ($p\wedge r$ and
$q\wedge s$).
%
It also follows from the semantics that nested choices in the same dimension
contain redundant alternatives; that is, inner choices are \emph{dominated} by
outer choices in the same dimension. For example, $\chc[A]{p,\chc[A]{r,s}}$ is
equivalent to $\chc[A]{p,s}$ since the alternative $r$ cannot be reached by any
configuration.
%
As the previous example illustrates, the representation of a \ac{vpl} formula is
not unique; that is, the same set of variants may be encoded by different
formulas. \autoref{fig:cc:eqv} defines a set of equivalence laws for
\ac{vpl} formulas. These laws follow directly from the configuration semantics in
\autoref{fig:cc:cfg} and can be used to derive semantics-preserving
transformations of \ac{vpl} formulas.
%
For example, we can simplify the formula $\chc[A]{p\vee q, p\vee r}$ by first
applying the \rn{Or} law to obtain $\chc[A]{p,p}\vee\chc[A]{q,r}$, then applying
the \rn{Idemp} law to the first argument to obtain $p\vee\chc[A]{q,r}$ in which
the redundant $p$ has been factored out of the choice.
%

%%% Local Variables:
%%% mode: latex
%%% TeX-master: "../../thesis"
%%% End:
