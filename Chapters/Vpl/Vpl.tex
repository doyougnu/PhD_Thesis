~\label{chapter:vpl}
%
In this chapter, we present the logic of variational satisfiability problems.
%
The logic is a conservative extension of classic two-valued logic (\pl{})
with a \emph{choice} construct from the choice
calculus~\cite{EW11tosem,Walk13thesis}, a formal language for describing
variation. We call the new logic \acs{vpl}, short for variational propositional
logic, and refer to \ac{vpl} expressions as \emph{variational formulas}.
%
This chapter defines the syntax and semantics of \ac{vpl} and concludes with a
set of definitions, lemmas and theorems for the logic.
%
\begin{figure}
  \begin{subfigure}[t]{\linewidth}
    \centering
    \begin{syntax}
  % D & \Coloneqq & \text{(any dimension name)} & \textit{Dimension} \\ \\
  % t & \Coloneqq & r & \textit{Variable reference} \\
  % & | & \mathit{T} & \textit{True} \\
  % & | & \mathit{F} & \textit{False} \\ \\

  t & \Coloneqq & r \quad|\quad \tru \quad|\quad \fls
    & \textit{Variables and Boolean literals} \\[1.5ex]

  f & \Coloneqq & t    & \textit{Terminal} \\
    & | & \neg f       & \textit{Negate} \\
    & | & f \vee f     & \textit{Or} \\
    & | & f \wedge f   & \textit{And} \\
    & | & \chc[D]{f,f} & \textit{Choice} \\
\end{syntax}

    \caption{Syntax of \ac{vpl}.}%
    \label{fig:cc:stx}
  \end{subfigure}
%
  \begin{subfigure}[t]{\linewidth}
    \begin{align*}
  \sem[]{\cdot} &: f\rightarrow C \rightarrow f
    \qquad\qquad \text{where } C = D\rightarrow\mathbb{B}_\bot \\
  \sem[C]{t}             &= t \\
  \sem[C]{\neg f}        &= \neg \sem[C]{f} \\
  \sem[C]{f_1\wedge f_2} &= \sem[C]{f_1}\wedge\sem[C]{f_2}\\
  \sem[C]{f_1\vee f_2}   &= \sem[C]{f_1}\vee\sem[C]{f_2}\\
  \sem[C]{\chc[D]{f_1,f_2}} &=
    \begin{cases}
      \sem[C]{f_1}                       & C(D) = \true \\
      \sem[C]{f_2}                       & C(D) = \false \\
      \chc[D]{\sem[C]{f_1},\sem[C]{f_2}} & C(D) = \bot \\
    \end{cases}
\end{align*}

    \centering
    \caption{Configuration semantics of \ac{vpl}.}%
    \label{fig:cc:cfg}
  \end{subfigure}
%
  \begin{subfigure}[t]{\linewidth}
    \begin{align*}
  \chc[D]{f,f}
    & \equiv f
    & \rn{Idemp} \\
  \chc[D]{\chc[D]{f_1,f_2},f_3}
    & \equiv \chc[D]{f_1,f_3}
    & \rn{Dom-L} \\
  \chc[D]{f_1,\chc[D]{f_2,f_3}}
    & \equiv \chc[D]{f_1,f_3}
    & \rn{Dom-R} \\
  \chc[D_1]{\chc[D_2]{f_1,f_2},\chc[D_2]{f_3,f_4}}
    & \equiv \chc[D_2]{\chc[D_1]{f_1,f_3},\chc[D_1]{f_2,f_4}}
    & \rn{Swap} \\
  \chc[D]{\neg f_1,\neg f_2}
    & \equiv \neg\chc[D]{f_1,f_2}
    & \rn{Neg} \\
  \chc[D]{f_1\vee f_3,\;f_2\vee f_4}
    & \equiv \chc[D]{f_1,f_2}\vee\chc[D]{f_3,f_4}
    & \rn{Or} \\
  \chc[D]{f_1\wedge f_3,\;f_2\wedge f_4}
    & \equiv \chc[D]{f_1,f_2}\wedge\chc[D]{f_3,f_4}
    & \rn{And} \\
  \chc[D]{f_1\wedge f_2,f_1}
    & \equiv f_1\wedge\chc[D]{f_2,\tru}
    & \rn{And-L} \\
  \chc[D]{f_1\vee f_2,f_1}
    & \equiv f_1\vee\chc[D]{f_2,\fls}
    & \rn{Or-L} \\
  \chc[D]{f_1,f_1\wedge f_2}
    & \equiv f_1\wedge \chc[D]{\tru,f_2}
    & \rn{And-R} \\
  \chc[D]{f_1,f_1\vee f_2}
    & \equiv f_1\vee\chc[D]{\fls,f_2}
    & \rn{Or-R}
\end{align*}

    \centering
    \caption{\ac{vpl} equivalence laws.}%
    \label{fig:cc:eqv}
  \end{subfigure}
\caption{Formal definition of \ac{vpl}.}%
\label{fig:cc}
\end{figure}
%
%
\section{Syntax}
~\label{section:syntax}
\label{section:vpl:syntax}
%
The syntax of variational propositional logic is given in
\autoref{fig:cc:stx}. It extends the propositional formula notation of \pl{}
with a single new connective called a \emph{choice} from the choice calculus.
%
A choice $\chc[D]{f_1,f_2}$ represents either $f_1$ or $f_2$ depending on the
Boolean value of its \emph{dimension} $D$. We call $f_1$ and $f_2$ the
\emph{alternatives} of the choice.
%
Although dimensions are Boolean variables, the set of dimensions is disjoint
from the set of variables from \pl{}, which may be referenced in the leaves of
a formula. We use lowercase letters to range over variables and uppercase
letters for dimensions.

The syntax of \ac{vpl} does not include derived logical connectives, such as
$\rightarrow$ and $\leftrightarrow$. However, such forms can be defined
from other primitives and are assumed throughout the thesis.


%%% Local Variables:
%%% mode: latex
%%% TeX-master: "../../thesis"
%%% End:


\section{Semantics}
~\label{section:semantics}
\label{section:vpl:semantics}
%
Conceptually, a variational formula represents several propositional logic
formulas at once, which can be obtained by resolving all of the choices. For
software product-line researchers, it is useful to think of \ac{vpl} as analogous
to \cpp{ifdef}-annotated \pl{}, where choices correspond to a
disciplined~\cite{LKA:AOSD11} application of \cpp{ifdef} annotations.
%
From a logical perspective, following the many-valued logic of
Kleene~\cite{kleene1968introduction,Rescher1969-RESML}, the intuition behind
\ac{vpl} is that a choice is a placeholder for two equally possible alternatives
that is deterministically resolved by reference to an external environment.
%
In this sense, \ac{vpl} deviates from other many-valued logics, such as modal
logic~\cite{sep-logic-modal}, because a choice \emph{waits} until there is
enough information in an external environment to choose an alternative (i.e.,
until the formula is \emph{configured}).

The \emph{configuration semantics} of \ac{vpl} is given in
\autoref{fig:cc:cfg} and describes how choices are eliminated from a
formula. The semantics is parameterized by a \emph{configuration}\ $C$, which is
a partial function from dimensions to Boolean values.
%
The first four cases of the semantics simply propagate configuration down the
formula, terminating at the leaves. The case for choices is the interesting one:
if the dimension of the choice is defined in the configuration, then the choice
is replaced by its left or right alternative corresponding to the associated
value of the dimension in the configuration. If the dimension is undefined in
the configuration, then the choice is left intact and configuration propagates
into the choice's alternatives.

If a configuration $C$ eliminates all choices in a formula $f$, we call $C$
\emph{total} with respect to $f$. If $C$ does \emph{not} eliminate all choices
in $f$ (i.e., a dimension used in $f$ is undefined in $C$), we call $C$
\emph{partial} with respect to $f$.
%
We call a choice-free formula \emph{plain}, and call the set of all plain
formulas that can be obtained from $f$ (by configuring it with every possible
total configuration) the \emph{variants} of $f$.

To illustrate the semantics of \ac{vpl}, consider the formula
$p\wedge\chc[A]{q,r}$, which has two variants: $p\wedge q$ when $C(A)=\true$
and $p\wedge r$ when $C(A)=\false$.
%
From the semantics, it follows that choices in the same dimension are
\emph{synchronized} while choices in different dimensions are
\emph{independent}. For example, $\chc[A]{p,q}\wedge\chc[B]{r,s}$ has four
variants, while $\chc[A]{p,q}\wedge\chc[A]{r,s}$ has only two ($p\wedge r$ and
$q\wedge s$).
%
It also follows from the semantics that nested choices in the same dimension
contain redundant alternatives; that is, inner choices are \emph{dominated} by
outer choices in the same dimension. For example, $\chc[A]{p,\chc[A]{r,s}}$ is
equivalent to $\chc[A]{p,s}$ since the alternative $r$ cannot be reached by any
configuration.
%
As the previous example illustrates, the representation of a \ac{vpl} formula is
not unique; that is, the same set of variants may be encoded by different
formulas. \autoref{fig:cc:eqv} defines a set of equivalence laws for
\ac{vpl} formulas. These laws follow directly from the configuration semantics in
\autoref{fig:cc:cfg} and can be used to derive semantics-preserving
transformations of \ac{vpl} formulas.
%
For example, we can simplify the formula $\chc[A]{p\vee q, p\vee r}$ by first
applying the \rn{Or} law to obtain $\chc[A]{p,p}\vee\chc[A]{q,r}$, then applying
the \rn{Idemp} law to the first argument to obtain $p\vee\chc[A]{q,r}$ in which
the redundant $p$ has been factored out of the choice.
%

%%% Local Variables:
%%% mode: latex
%%% TeX-master: "../../thesis"
%%% End:


\section{Background example}
~\label{section:background}
%
To demonstrate the application of \ac{vpl} and conclude the chapter, we encode
the incremental example from \autoref{chapter:background}. Our goal is to
construct a single variational formula that encodes the related plain formulas
\pV{}, \qV{}, \rV{}. Ideally, this variational formula should maximize sharing
among the plain formulas in order to avoid redundant analyses during a
variational solving. We reproduce the formulas below for the convenience:
%
\begin{align*}
  p =\ a \wedge b \wedge c \wedge e && q=\ a \wedge (b \vee \neg \iV{}) \wedge c \wedge (\gV{} \rightarrow c) && r =~z \leftrightarrow (a \wedge b \wedge c \wedge e)
\end{align*}

%
Every set of plain formulas can be encoded as a variational formula
systematically by first constructing a nested choice containing all of the
individual variables as alternatives, then factoring out shared subexpressions
by applying the laws in \autoref{fig:cc:eqv}.
%
Unfortunately, the process of globally minimizing a variational formula in this
way is hard\footnote{\label{vpl:bdd}. We hypothesize that it is equivalent to
  BDD minimization, which is NP-complete, but the equivalence has not been
  proved; see~\cite{Walk14onward}.} since we must apply an arbitrary number of
laws right-to-left in order to set up a particular sequence of left-to-right
applications that factor out commonalities.

Due to the difficulty of minimization, we instead demonstrate how one can build
such a formula \emph{incrementally}.
%
Our variational formula will use the dimensions \dimP, \dimQ, \dimR{} to
represent the respective variants. Unique portions of each variant will be
nested into the left alternative so that the unique portion is considered when
the dimension is bound to \true in the configuration.

%
We begin by combining \pV{} and \rV{} since the differences\todo{section on
  encoding problem}\footnote{There are many ways to assess the difference of two
  formulas. For now the reader may consider it reducible to the Levenshtein
  distance of two strings~\cite{Levenshtein_SPD66}. We return to this discussion
  in \ldots{}} between the two are smaller than between other pairs of feature
models in our example. Feature models may be combined in any order as long as
the variants in the resulting formula correspond to their plain counterparts.
The only change between \pV{} and \rV{} is the addition of \zV{} and thus we
wrap the leaf in a choice with dimension \dimR. This yields the following
variational formula.
%
\begin{equation*}
  f_{\pV\rV} = \chc[\dimR]{\zV, \tru{}} \leftrightarrow (a \wedge b \wedge c \wedge e)
\end{equation*}
%
%
We exploit the fact that $\wedge$ forms a monoid with \tru{} to recover a
formula equivalent to \pV{} for configurations where \dimR{} is \false.

Next we combine $f_{\pV\rV}$ with \qV{} to obtain a variational formula that encodes the
propositional formulas \pV{}, \qV{}, \rV{}. There are two sub-trees that must be
wrapped in choices. First, we must encode the difference between $b \vee \neg i$
from \qV{} and $b$. Second, we must ensure synchronization and thus use the same
dimension to encode the difference between $g \rightarrow c$ and $e$. Thus the
resulting variational formula is:
%
\begin{equation*}
  f = \chc[\dimR]{\zV, \tru{}} \leftrightarrow (a \wedge \chc[\dimQ]{b \vee \neg i, b} \wedge c \wedge \chc[\dimQ]{g \rightarrow c, e})
\end{equation*}
%
Now that we have constructed the variational formula we need to ensure that it
encodes all variants of interest and nothing else. Notice that only 2 dimensions
are used to encode 3 variants, because \dcardinality{f} = 2 we have are 4
possible variants and thus one extra variant. We can observe this by enumerating
the variants and possible configurations:
%
\begin{align*}
  \pV &=\ \tru{} \leftrightarrow (a \wedge b \wedge c \wedge e)                          & C = \{(\dimR, \false), (\dimQ, \false)\} \\
  \qV &=\ \tru{} \leftrightarrow (a \wedge (b \vee \neg \iV{}) \wedge c \wedge (\gV{} \rightarrow c))   & C = \{(\dimR, \false), (\dimQ, \true)\} \\
  \rV &=~z \leftrightarrow (a \wedge b \wedge c \wedge e)                               & C = \{(\dimR, \true), (\dimQ, \false)\} \\
  \kf{extra} &=~z \leftrightarrow (a \wedge (b \vee \neg \iV{}) \wedge c \wedge (\gV{} \rightarrow c)) & C = \{(\dimR, \true), (\dimQ, \true)\}
\end{align*}
%

Notice the $\kf{extra}$ variant and that \pV{} and \qV{} are only recovered
through equivalency laws from propositional logic. While it is undesirable that
there exists extra variants, the important constraint: $\kf{fs} = \{ v\ |\ v \in
\{ \pV, \qV, \rV\} \}$, $fs \subseteq \variants{f}$ is satisfied. We'll return
to the case of \todo{maybe superfluous?}extra variants in the next chapter as a
variational \ac{sat} solver must only solve variants of interest.



%%% Local Variables:
%%% mode: latex
%%% TeX-master: "../../thesis"
%%% End:


%%% Local Variables:
%%% mode: latex
%%% TeX-master: "../../thesis"
%%% End:
