~\label{sec:vsat} A core contributions of the proposed thesis is the design and
architecture of a variational satisfiability solver. In this section, I review
the architecture of a prototype solver called \vsat{} and conclude the section
by summarizing work left to do on \autoref{phase-change-deliverable}, \ie{},
assessing the hardness of variational satisfiability problems.

\subsection{Design}
Research on \ac{sat} and incremental \ac{sat} is fast moving with novel \ac{sat}
solvers regularly competing in the International \ac{sat}
Competition~\cite{interSatComp}, to leverage these results we design variational
satisfiability solving to target the incremental interface specified in the
SMT-LIB2~\cite{BarFT-RR-17} standard. Targeting the standard provides several
benefits: an implementation of a variational solver is free to choose any
SMT-LIB2~\cite{BarFT-RR-17} compliant solver ranging from experimental solvers
such as cvc4~\cite{10.1007/978-3-642-22110-1_14}, to industrial strength solvers
such as z3~\cite{10.1007/978-3-540-78800-3_24}. Since the implementation is
\ac{sat} solver agnostic, a variational solver can be run as a
\emph{meta-solver}, \ie{}, a \ac{sat} solver that interleaves, or simultaneously
uses several \ac{sat} solvers to solve a \ac{sat} problem, especially in
asynchronous workloads. Lastly, any result from the \ac{sat} or \ac{smt}
community that enters the standard is supported, this includes new \ac{smt}
theories.

In addition to coupling to SMTLIB2, the variational \ac{sat} solver should not
place further burden on the end-user. Thus users of a variational solver should
not be required to understand the choice calculus in order to interpret their
results, a variational solver's output should not contain choices. This design
principle in combination with \autoref{encoding-strat-deliverable} completes the
approach and would allow an end-users to receive the benefits of this work
without additional upfront cost.

\subsection{Architecture}

This section provides an informal description of variational satisfiability
solving and variational models, I provide the formalization in the next section.
A variational satisfiability solver is a compiler from the domain of variational
formulas to SMT-LIB2 programs. Throughout this section, I use SMTLIB2 snippets
to describe variational solving concepts in terms of an incremental solver.
While I target SMTLIB2, conforming to the standard is not an essential
requirement. Any solver that exposes an incremental API as defined by
minisat~\cite{10.1007/978-3-319-09284-3_16} can be used to implement variational
satisfiability solving following the same architecture and semantics.

\begin{figure}
  \begin{subfigure}[t]{0.5\linewidth}
    \tikzstyle{block}    = [draw,fill=white!20,node distance = 3.5cm,align=center]
\tikzstyle{inEdge}   = [fill=white, text width=1cm]
\tikzstyle{overEdge} = [midway,above]
\tikzstyle{input}    = [fill=white!20,node distance = 2.2cm,align=center, text width=1cm]
\tikzstyle{double} = [draw, anchor=text, rectangle split,rectangle split parts=2]
% diameter of semicircle used to indicate that two lines are not connected
\tikzstyle{branch}=[fill,shape=circle,minimum size=3pt,inner sep=0pt]
\tikzstyle{pinstyle} = [pin edge={to-,thin,black}]
\begin{center}
\begin{tikzpicture}[>=latex]

  % The loop
  \node[block]  (solve) {Reification\\Engine};
  \node[block, below right=2.05cm and 3.125cm of solve] (vmodel) {VModel\\Constructor};
  \node[block, below=2.05cm of solve] (vcore) {Reduction\\Engine};
  \node[block, right=of solve] (base) {Base\\Solver};

  % input nodes
  \node (input) [input, left=0.7cm of vcore] {$\kf{Query}$ $\kf{formula}$};
  \node (vf)    [input, left=0.7cm of solve] {$\kf{variant}$ $\kf{formula}$};


  % outputs
  \node[input, right=0.75cm of vmodel] (output) {$\kf{VModel}$};

  \draw
  %% Inputs
  (input) edge[->] node[overEdge] {} (vcore)
  (vf)    edge[dotted,->] node[overEdge] {} (solve)

  %% outputs
  (vmodel) edge[->] node[overEdge] {} (output)
  (base)   edge[->,sloped] node[overEdge,left,rotate=89,xshift=3.0cm]
  {$\kf{plain~models}$} (vmodel)
  (base)   edge[->,sloped,bend right=10] node[overEdge,right,rotate=-90] {} (vmodel)
  (base)   edge[->,sloped,bend right=20] node[overEdge,right,rotate=-90] {} (vmodel)
  (base)   edge[->,sloped,bend left=10] node[overEdge,right,rotate=-90] {} (vmodel)
  (base)   edge[->,sloped,bend left=20] node[overEdge,right,rotate=-90] {} (vmodel)

  %% Loop
  (solve) edge[->,sloped, bend left=25] node[overEdge,right,text width=5.5cm,rotate=90] {$s$, $v_{1} \vee v_{2}$, $v_{1} \wedge v_{2}$} (vcore)
  (vcore) edge[->,sloped, bend left=25] node[overEdge,left,rotate=-90] {$\kf{VCore}$, $\unit{}$} (solve)
  (solve) edge[->,in=100,out=160,loop, min distance=20mm] node[overEdge,xshift=-2em,yshift=0.2em] {$v \wedge \sem[C]{\chc[D]{e_{1}, e_{2}}}$} (solve)
  (solve) edge[dashed,->,in=70,out=15,loop, min distance=20mm] node[overEdge,xshift=4.75em,yshift=0.2em,text width = 5.5cm] {$v \wedge \sem[C\ \cup\ \{(D, \true)\}]{\chc[D]{e_{1}, e_{2}}}$, $v \wedge \sem[C\ \cup\ \{(D, \false)\}]{\chc[D]{e_{1}, e_{2}}}$} (solve)

  %% Loop escape
  (solve) edge[->] node[overEdge] {} (base)
  (solve) edge[->,sloped,bend right=5] node[overEdge] {} (base)
  (solve) edge[->,sloped,bend right=10] node[overEdge] {} (base)
  (solve) edge[->,sloped,bend left=5] node[overEdge] {} (base)
  (solve) edge[->,sloped,bend left=10] node[overEdge] {\unit{}} (base)
  % (vmodel) edge[->,sloped,bend right=10]  node[overEdge] {$r$, $s$, $t$} (base)
  % (vmodel) edge[->,sloped,bend left=10]  node[overEdge,xshift=0.1cm] {$\neg v$, $v_{1} \vee v_{2}$} (acc)
  % (vmodel) edge[->, loop above]  node[overEdge] {$\unit{} \wedge v$, $v \wedge \unit{}$} (vmodel)
  % (acc) edge[->, loop below] node[overEdge,below] {$r$, $\neg s$, $s_{1} \wedge s_{2}$, $s_{1} \vee s_{2}$} (acc)
  % (acc) edge[->,sloped, bend left=25] node[overEdge,below] {$s$} (vmodel)
  % (base) edge[->,sloped, bend right=25] node[overEdge,below] {$\unit{}$} (vmodel)
  % (vmodel) edge[->] node[overEdge] {} (vcore)
  ;

\end{tikzpicture}%
\end{center}

    \vspace{3.5ex}
    \caption{System overview of a variational solver.}%
    \label{impl:overview}
  \end{subfigure}
  \begin{subfigure}[t]{0.5\linewidth}
    \tikzstyle{block}    = [draw,fill=white!20,node distance = 1.75cm,align=center]
\tikzstyle{inEdge}   = [fill=white, text width=1cm]
\tikzstyle{overEdge} = [midway,above]
\tikzstyle{input}    = [fill=white!20,node distance = 2.2cm,align=center, text width=1cm]
\tikzstyle{double} = [draw, anchor=text, rectangle split,rectangle split parts=2]
% diameter of semicircle used to indicate that two lines are not connected
\tikzstyle{branch}=[fill,shape=circle,minimum size=3pt,inner sep=0pt]
\tikzstyle{pinstyle} = [pin edge={to-,thin,black}]
\begin{center}
\begin{tikzpicture}[>=latex]

  % input nodes
  \node[input] (input) {$\kf{Query}$ $\kf{formula}$};
  \node[block, right= 0.55cm of input] (ilInput) {to IL};

  % The loop
  \node[block, right=0.55cm of ilInput] (eval) {Evaluation};
  \node[block, below right=1.5cm of eval] (acc) {Accumulation};
  \node[block, below left=1.5cm of eval] (base) {Base\\Solver};

  % output
  \node[input, above left=0.75cm and 0.05 of eval] (vcore) {};

  % nowhere node
  \node[input, above right=0.75cm and 0.05cm of eval] (nowhere) {};

  % escape edges
  \draw
  %% inputs
  (input) edge[->] node[overEdge] {} (ilInput)
  (ilInput) edge[->] node[overEdge] {} (eval)

  %% Eval
  (eval) edge[->,sloped,bend right=10]  node[overEdge] {$r$, $s$, $t$} (base)
  (eval) edge[->,sloped,bend left=10]  node[overEdge,xshift=0.1cm] {$\neg v$, $v_{1} \vee v_{2}$} (acc)
  (eval) edge[->, loop above]  node[overEdge, text width=1cm] {$v_{1} \wedge v_{2}$} (eval)
  (eval) edge[<-,sloped,bend right=25] node[overEdge,right,rotate=-57,text width=1.5cm,xshift=0.5em] {$s$, $v_{1} \vee v_{2}$, $v_{1} \wedge v_{2}$} (nowhere)

  %% acc
  (acc) edge[->, loop below, transform canvas={xshift=6mm}] node[overEdge,below, text width = 1cm] {$\neg v$, $v_{1} \vee v_{2}$, $v_{1} \wedge v_{2}$} (acc)
  (acc) edge[->, loop below, transform canvas={xshift=-6mm}] node[overEdge,below, text width = 1cm] {$r$, $\neg s$, $s_{1} \vee s_{2}$, $s_{1} \wedge s_{2}$} (acc)
  (acc) edge[->,sloped, bend left=25] node[overEdge,below, text width = 1.3cm] {$s$, $v_{1} \wedge v_{2}$, $\ v_{1}\vee v_{2}$} (eval)
  (base) edge[->,sloped, bend right=25] node[overEdge,below] {$\unit{}$} (eval)

  %% output
  (eval) edge[->,sloped, bend left = 25] node[overEdge,left,rotate=54] {$\kf{VCore}$} (vcore) ;

\end{tikzpicture}%
\end{center}

    \caption{Overview of the reduction engine.}
    \label{impl:vcore}
  \end{subfigure}
  \caption{}
\end{figure}

A \ac{vpl} formula is solved using a recursive approach, decoupling the handling
of plain terms from the handling of variational terms. The idea is to define a
process to evaluate plain terms and skip choices, then define another process
that can only configures choices thus introducing new plain terms to the formula
that can be recursively processed. The base case is a variant, at which point a
model can be queried and the assertion stack can be popped to backtrack to solve
another variant.

I present an overview of a variational solver as a state diagram in
\autoref{impl:overview} that operates on the input's abstract syntax tree.
Labels on incoming edges denote inputs to a state and labels on outgoing edges
denote return values; we show only inputs for recursive edges; labels separated
by a comma share the edge. We omit labels that can be derived from the logical
properties of connectives, such as commutativity of $\vee$ and $\wedge$.
Similarly, we omit base case edge labels for choices and describe these cases in
the text. The solver has four subsystems: The \emph{reduction engine} processes
plain terms and generates a formula ready for reification called a
\emph{variational core}. The \emph{reification engine} configures choices in a
variational core. The \textit{base solver} is the incremental solver used to
produce plain models. Finally, the \emph{variational model constructor}
synthesizes a single variational model from the set of plain models returned by
the base solver.

The solver inputs a \vpl{} formula, called a \emph{query formula}, and an
optional input, called a \emph{variation context} (\vc{}). A \vc{} is a
propositional formula of dimensions that restricts the solver to a subset of
variants.
%
The variational solver translates the query formula to a formula in an
intermediate language (IL) that the reduction and reification engines operate
over; its syntax is given below.
%
\[
  v \hquad\Coloneqq\hquad \unit{}
  \hquad|\hquad t
  \hquad|\hquad s
  \hquad|\hquad \neg v
  \hquad|\hquad v \wedge v
  \hquad|\hquad v \vee v
  \hquad|\hquad \chc[D]{e,e}
\]
%
The IL includes two kinds of terminals not present in the input query formulas:
plain sub-terms that can be reduced symbolically will be replaced by a
\emph{symbolic reference} $s$, and sub-terms that have been sent to the base
solver will be represented by the unit value \unit{}.
%
Note that choices contain unprocessed expressions ($e$) as alternatives.


\paragraph{Derivation of a Variational Core}%
\label{ssec:impl:accum}

A variational core is an IL formula that captures the variational structure of
a query formula. Plain terms will either be placed on the assertion stack or
will be symbolically reduced, leaving only logical connectives, symbolic
references, and choices.
%
Consider the query formula $f = ((a \wedge b) \wedge \chc[A]{e_{1}, e_{2}}) \wedge
((p \wedge \neg q) \vee \chc[B]{e_{3}, e_{4}})$. Translated to an IL formula,
$f$ has four references ($a$, $b$, $p$, $q$) and two choices. The reduction
engine shown in \autoref{impl:vcore} will produce a variational core that will
assert $(a \wedge b)$ in the base solver, thus pushing it onto the assertion
stack and create a symbolic reference for $(p \wedge \neg q)$. This is done in
two states: \emph{evaluation}, which issues commands to the base solver to
process plain terms, and \emph{accumulation} which is called by evaluation to
create symbolic references.

Generating the core begins with evaluation. Evaluation will match on the root
node: $\wedge$, of $f$ and recur following the $v_1 \wedge v_2$ edge, where
%
$v_1=(a\wedge b)\wedge\chc[A]{e_1,e_2}$ and
$v_2=(p\wedge \neg q) \vee \chc[B]{e_3, e_4}$.
%
The recursion processes the left child first. Thus, evaluation will again
match on $\wedge$ of $v_{1}$ creating another recursive call with $v_{1}' =
(a\wedge b)$ and $v_{2}' = \chc[A]{e_1,e_2}$. Finally, the base case is reached
with a last recursive call where $v_{1}'' = a$, and $v_{2}'' = b$. At the base
case both $a$ and $b$ are references, thus evaluation will send $a$ to the
base solver, following the $\kf{r, s, t}$ edge, which returns $\unit{}$ for
the left child. The right child follows the same process yielding $\unit{}
\wedge \unit{}$; since the assertion stack implicitly conjuncts all assertions,
$\unit{} \wedge \unit{}$ will be further reduced to $\unit{}$ and returned as
the result of $v_{1}'$,
indicating that both children have been pushed to the base solver. This
leaves $v_{1}' = \unit{}$ and $v_{2}' = \chc[A]{e_1,e_2}$. $v_{2}'$ is a base
case for choices and cannot be reduced in evaluation, and so $\unit{} \wedge
\chc[A]{e_1,e_2}$, will be reduced to just $\chc[A]{e_1,e_2}$ as the result for
$v_{1}$.

In evaluation, conjunctions can be split because of the behavior of the
assertion stack and the and-elimination property of $\wedge$. Disjunctions and
negations cannot be split in this way because both cannot be performed if a child
node has been lost to the solver, e.g., $\neg \unit{}$. Thus, in accumulation, we
construct symbolic terms to represent entire sub-trees, ensuring information is
not lost, but still allowing for the sub-tree to be evaluated if it is sound to
do so.

The right child, $v_2=(p\wedge \neg q) \vee \chc[B]{e_3, e_4}$ requires
accumulation. Evaluation will match on the root $\vee$, and send $(p\wedge \neg
q) \vee \chc[B]{e_3, e_4}$ to accumulation via the $v_{1} \vee v_{2}$ edge.
Accumulation has two recursive edges, one to create symbolic references (with
labels $r, s, \hdots$), and one to recur to values. Accumulation matches the
root $\vee$ and recurs on the self-loop with edge $v_{1} \vee v_{2}$, $v_{1} =
(p\wedge \neg q)$, and $v_{2} = \chc[B]{e_3, e_4}$. Processing the left child
first, accumulation will recur again with $v_{1}' = p$ and $v_{2}' = \neg q$.
$v_{1}' = p$ is a base case for references, thus a unique symbolic reference
$s_{p}$ is generated for $p$, following the self-loop with label $r$ and
returned as the result for $v_{1}'$. $v_{2}'$ will follow the self-loop with
label $\neg v$ to recur through $\neg$ to $q$, where a symbolic term $s_{q}$
will be generated and returned. This yields $\neg s_{q}$, which follows the
$\neg s$ edge to be processed into a new symbolic term, yielding the result for
$v_{2}'$ as $s_{\neg q}$. With both results $v_{1} = s_{p}\wedge s_{\neg q}$,
accumulation will match on $\wedge$ \emph{and} both $s_{p}$ and $s_{\neg q}$ to
accumulate the entire sub-tree to a single symbolic term, $s_{s_{p} \wedge
  s_{\neg q}}$, which will be returned as the result for $v_{1}$. $v_{2}$ is a
base case, hence accumulation will return $s_{s_{p} \wedge s_{\neg q}} \vee
\chc[B]{e_3, e_4}$ to evaluation. Evaluation will conclude with
$\chc[A]{e_1,e_2}$ as the result for the left child of $\wedge$ and $s_{s_{p}
  \wedge s_{\neg q}} \vee \chc[B]{e_3, e_4}$ for the right child, yielding
$\chc[A]{e_1,e_2} \wedge (s_{s_{p} \wedge s_{\neg q}} \vee \chc[B]{e_3, e_4})$ as
the variational core of $\kf{f}$.

A variational core is derived to save redundant work. If solved naively, plain
sub-formulas of $f$, such as $a \wedge b$ and $p \wedge \neg q$, would be
processed once for each variant even though they are unchanged. To save
computation evaluation moves sub-formulas into the solver state to be reused
among different variants, and accumulation caches sub-formulas that cannot be
immediately evaluated to be evaluated later.

Symbolic references are variables in the reduction engine's memory that
represent a sub-tree of the query formula. From the perspective of the base
solver a symbolic reference represents a set of programmatic statements. For
example, $s_{pq}$ represents three declarations in the base solver:
%
\begin{lstlisting}[columns=flexible,keepspaces=true]
(declare-const p Bool)           ;; $s_{pq}$ represents
(declare-const q Bool)           ;; several declarations
(declare-fun $s_{ab}$ () Bool (or p (not c)))
\end{lstlisting}

The program language shown here is lisp~\cite{10.1145/367177.367199} as defined
in the SMTLIB2 standard. Function application begins with an open parenthesis,
where the first symbol in the parenthesis is the name of the function, every
symbol after the first is an argument to that function. In the above snippet, we
see three function calls, two which declare constants in the program for \pV{}
and \qV{}, both with type $\kf{Bool}$, and one which declares a new function
which takes no input and returns a $\kf{Bool}$.

Similar to symbolic references, a variational core is a sequence of statements
in the base solver with holes $\Diamond$. For example, the representation of
$\kf{VCore_{f}}$:
%
\begin{lstlisting}[columns=flexible,keepspaces=true]
(assert (and a b))                ;; $a \wedge b$ on the assertion stack
(declare-const $\Diamond_{A}$)                ;; choice A
 $\vdots$                                ;; many declares may occur
(assert $\Diamond_{A}$)                       ;; many assertions may occur
 $\vdots$                                ;; $s_{pq}$
(declare-fun $s_{pq}$ () Bool (and p q))
(declare-const $\Diamond_{B}$)                ;; choice B
 $\vdots$
(assert (or $s_{ab}$ $\Diamond_{B}$))               ;; assert waiting on $\sem[C]{\chc[B]{e_{3},e_{4}}}$
\end{lstlisting}
%
Each hole is filled by configuring a choice and may require multiple statements
to process the alternative as the alternative could introduce several new
variables or functions.

\paragraph{Solving the Variational Core}

The reduction engine performs the work at each recursive step. Whereas the
reification engine defines transitions between the recursive steps by
manipulating the configuration. In \vpl{}, a configuration was formalized as a
function, for variational solvers we use a set of tuples $\{\kf{(D \times
  \mathbb{B})}\}$. \autoref{impl:overview} shows two self-loops for the
reification engine corresponding to the reification of choices. The edges from
the reification engine to the reduction engine are transitions taken after a
choice is removed, where new plain terms have been introduced and thus a new
core is derived. If the user supplied a variation context, then it is used to
construct an initial configuration. Finally, a model is called from the base
solver when the reduction engine returns \unit{}, indicating that a variant has
been found.

We display a subset of edges of the reification engine using the $\wedge$
connective. In general, these edges will be duplicated for each binary logical
connective, e.g., $\vee$. The left edge, is taken when a choice is observed in
the variational core: $v \wedge \sem[C]{\chc[D]{e_{1}, e_{2}}}$ and $D \in C$.
This edge reduces choices with dimension $D$ to an alternative, which are then
translated to IL\@. The right edge is dashed to indicate assertion stack
manipulation, and is taken when $D \notin C$. For this edge, the configuration
is mutated for both alternatives: $C \cup \{(D, \tru{})\}$, and $C \cup \{(D,
\fls{})\}$, and the recursive call is wrapped with a \texttt{push}, and
\texttt{pop} command. To the base solver, this branching is a linear sequence of
assertion stack manipulations that performs backtracking behavior, for example
the representation of $\kf{f}$ is:
%
\begin{lstlisting}[columns=flexible,keepspaces=true]
 $\vdots$          ;; declares and assertions from VCore
(push 1)    ;; a configuration on B has occurred
 $\vdots$          ;; new declarations for left alternative
(declare-fun $s$ () Bool (or $s_{pq}$ $[\Diamond_{B} \rightarrow s_{B_{T}}]\Diamond_{B}$))  ;; fill
(assert $s$)
 $\vdots$          ;; recursive processing
(pop 1)     ;; return for the right alternative
(push 1)    ;; repeat for right alternative
\end{lstlisting}
%
Where the hole $\Diamond_{B}$\footnote{the notation $[x \rightarrow v]p$ should
  be read ``replace all free occurrences of $\kf{x}$ in $\kf{p}$ with
  $\kf{v}$'', it derives from work on explicit substitution from the programming
  languages community~\cite{10.5555/509043}}, will be filled with a newly
defined variable $s_{B_{T}}$ that represents the left alternative's formula.

\paragraph{Variational Models}%
\label{ssec:vmodels}
%
Classic \ac{sat} models map variables to Boolean values; variational models map
variables to variational contexts that record the variants where the variable
was assigned \tru{}. The variational context for a variable $r$ is denoted as
\vc{r}, and a variational model reserves a special variable called \SatVar{} to
track the configurations that were found satisfiable.
%
\begin{figure}[h]
  \centering
  \begin{subfigure}[t]{\textwidth}
  \begin{tabbing}
    \qquad \quad \= $\aV{} \rightarrow$ \tru{} \\
    \> $\bV{} \rightarrow$ \fls{} \\
    \> $\cV{} \rightarrow$ \tru{} \\
    \> $\pV{} \rightarrow$ \tru{} \\
    \> $\qV{} \rightarrow$ \fls{} \\
    \quad $C_{FF}$ = \{(\AV{}, \fls{}), (\BV{}, \fls{})\} \\
  \end{tabbing}
\end{subfigure}%
\begin{subfigure}[t]{\textwidth}
  \begin{tabbing}
    \qquad \quad \= $\aV{} \rightarrow$ \tru{} \\
    \> $\bV{} \rightarrow$ \fls{} \\
    \> $\cV{} \rightarrow$ \tru{} \\
    \> $\pV{} \rightarrow$ \tru{} \\
    \> $\qV{} \rightarrow$ \fls{} \\
    \quad $C_{FT}$ = \{(\AV{}, \fls{}), (\BV{}, \tru{})\} \\
  \end{tabbing}
\end{subfigure}%
\begin{subfigure}[t]{\textwidth}
  \begin{tabbing}
    \qquad \quad \= $\aV{} \rightarrow$ \tru{} \\
    \> $\bV{} \rightarrow$ \fls{} \\
    \\
    \> $\pV{} \rightarrow$ \fls{} \\
    \> $\qV{} \rightarrow$ \tru{} \\
    \quad $C_{TT}$ = \{(\AV{}, \tru{}), (\BV{}, \tru{})\} \\
  \end{tabbing}
\end{subfigure}%
  \caption{Possible plain models for variants of $\kf{f}$.}%
  \label{fig:models:plain}
\end{figure}
\begin{figure}[h]
  \centering
  \begin{subfigure}[t]{\textwidth}
  \begin{tabbing}
  \qquad \qquad \= $\_Sat \rightarrow\ (\neg \AV{} \wedge \neg \BV{}) \vee (\neg \AV{} \wedge \BV{}) \vee (\AV{} \wedge \BV)$ \\
  \> $\aV{} \rightarrow\ (\neg \AV{} \wedge \neg \BV{}) \vee (\neg \AV{} \wedge \BV{}) \vee (\AV{} \wedge \BV)$ \\
  \> $\bV{} \rightarrow\ \fls{}$ \\
  \> $\cV{} \rightarrow\ (\neg{} \AV{} \wedge{} \neg{} \BV{}) \vee{} (\neg{} \AV{} \wedge{} \BV{})$ \\
  \> $\pV{} \rightarrow\ (\neg \AV{} \wedge \neg \BV{}) \vee (\neg \AV{} \wedge \BV{})$ \\
  \> $\qV{} \rightarrow\ (\AV{} \wedge \BV)$
\end{tabbing}
\end{subfigure}

  \caption{Variational model corresponding to the plain models in
    \autoref{fig:models:plain}.}%
  \label{fig:models:var}
\end{figure}
%
As an example, consider an altered version of the query formula from the
previous section $f = ((\aV{} \wedge \neg \bV{}) \wedge \chc[A]{\aV{}
  \rightarrow \neg \pV{}, \cV{}}) \wedge ((\pV{} \wedge \neg \qV{}) \vee
\chc[B]{\qV{}, \pV{}})$. We can easily see that one variant, with configuration
$\{(\AV{},\tru{}), (\BV{},\fls{})\}$ is unsatisfiable. If the remaining variants
are satisfiable, then three models would result, as illustrated in
\autoref{fig:models:plain}; the corresponding variational model is shown in
\autoref{fig:models:var}.

We see that \Satfmf{} consists of three disjuncted terms, one for each
satisfiable variant. Variational models are flexible; a satisfiable assignment
of the query formula can be found by calling \sat{} on \Satfmf{}. Assuming the
model $C_{FT} = \{(\AV{}, \fls{}), (\BV{}, \tru{})\}$ is returned, one can find
a variable's value through substitution with the configuration; for example,
substituting the returned model on \vc{c} yields:
%
\begin{align*}
  \cV{} \rightarrow\ & (\neg \AV{} \wedge \neg \BV{}) \vee (\neg \AV{} \wedge \BV{}) & \text{\vc{} for \cV{}} \\
  \cV{} \rightarrow\ & (\neg \fls{} \wedge \neg \tru{}) \vee (\neg \fls{} \wedge \tru{}) & \text{Substitute \fls{} for \AV{}, \tru{} for \BV{}} \\
  \cV{} \rightarrow\ & \tru{} & \text{Result}
\end{align*}%
%
Furthermore, to find variants where a variable \cV{} is satisfiable reduces to
$\kf{SAT(\vc{\cV{}})}$

Variational models are constructed incrementally by merging each new plain model
returned by the solver into the variational model. A merge requires the current
configuration, the plain model, and current \vc{} of a variable. Variables are
initialized to \fls{}. For each variable $i$ in the model, if $i$'s assignment
is \tru{} in the plain model, then the configuration is translated to a
variation context and disjuncted with \vc{i}. For example, to merge the
$C_{FT}$'s plain model to the variational model in \autoref{fig:models:var},
$C_{FT}$'s configuration is converted to $\neg \AV{} \wedge \BV{}$. This clause
is disjuncted for variables assigned \tru{} in the plain model: \vc{\aV{}},
\vc{\cV{}}, and \vc{\pV{}}, even if they are new (e.g., \cV{}). Variables
assigned \fls{} are skipped, thus \vc{\qV{}} remains \fls{}. For example, in the
next model $C_{TT}$, \cV{} is \fls{} thus \vc{\cV{}} remains unaltered, while
\vc{\qV{}} flips to \tru{} hence \vc{\qV{}} records $\AV{} \wedge \BV{}$.
Variables such as \bV{}, whose \vc{}'s stay \fls{} are called \textit{constant}.


Variational models are constructed in \ac{dnf}, and form a monoid under with
$\vee$ as the semigroup operation, and \fls{} as the unit value. I take note of
this for mathematically inclined readers because it has important ramifications
for the asynchronous version of variational satisfiability solvers.

\subsection{Formalization of Variational Satisfiability Solving}
This subsection presents a selection of inference rules that specify behavior
described in the previous subsection. Many inference rules are similar to others
due to commutativity of boolean operators and thus I only present an interesting
subset. These inference rules will be significantly altered in the proposed
thesis to fully generalize to \ac{smt} problems.

\begin{figure}
  \begin{mathpar}
  %%% Computation rules
  \inferrule*[Right=Ac-Gen]
  { r \notin \kf{dom}(\Delta) \\
    \texttt{spawn($\Delta, r$)} = (\Delta', s)}
  {(\Delta, r) \mapsto (\Delta', s)}

  \inferrule*[Right=Ac-Ref]
  { \Delta(r) = s}
  {(\Delta, r) \mapsto (\Delta, s)}
  \\
  %%% Negation rules
  \inferrule*[Right=Ac-Neg]
  { \texttt{negate($\Delta, s$)} = (\Delta', s')}
  { (\Delta, \neg s) \mapsto (\Delta', s')}
\\
%   \hspace{0.7cm}
%   \inferrule*[Right=Ac-C]
%   { }
%   {(\Delta, \chc[D]{e_{1}, e_{2}}) \mapsto (\Delta, \chc[D]{e_{1}, e_{2}})}
%
% \begin{mathpar}
%   \inferrule*[Right=Ac-Neg-C]
%   { }
%   { (\Delta, \neg \chc[D]{e_{1},e_{2}}) \mapsto (\Delta, \chc[D]{\neg e_{1}, \neg e_{2}}) }
% % \end{mathpar}
%   \hspace{0.9cm}
%
% % \begin{mathpar}
%   \inferrule*[Right=Ac-Unit]
%   { }
%   { (\Delta, \bullet) \mapsto (\Delta, \unit{})}
% \end{mathpar}
%
  \inferrule*[Right=Ac-SOr]
  { \texttt{or($\Delta, s_{1}, s_{2}$)} = (\Delta', s') }
  {(\Delta, s_{1} \vee s_{2}) \mapsto (\Delta', s')}

  \inferrule*[Right=Ac-SAnd]
  { \texttt{and($\Delta, s_{1}, s_{2}$)} = (\Delta', s') }
  {(\Delta, s_{1} \wedge s_{2}) \mapsto (\Delta', s')}
\\
%
% \begin{mathpar}
%   \hspace{-1.5cm}
%   \inferrule*[Right=Ac-DM-VOr]
%   { }
%   { (\Delta, \neg (e_{1} \vee e_{2})) \mapsto (\Delta, \neg e_{1} \wedge
%     \neg e_{2}) }
% \end{mathpar}
%
% \begin{mathpar}
%   \hspace{-1.5cm}
%   \inferrule*[Right=Ac-DM-VAnd]
%   { }
%   { (\Delta, \neg (e_{1} \wedge e_{2})) \mapsto (\Delta, \neg e_{1}
%     \vee \neg e_{2}) }
% \end{mathpar}
%
%%% And with Choice rules
% \begin{mathpar}
%   \hspace{-1.5cm}
%   \inferrule*[Right=Ac-VAnd-ChcL]
%   {\Delta, e \mapsto \Delta', s}
%   { \Delta, \chc[D]{e_{1},e_{2}} \wedge e \mapsto \Delta', \chc[D]{e_{1},e_{2}} \wedge s}
% \end{mathpar}
%
% \begin{mathpar}
%   \hspace{-1.5cm}
%   \inferrule*[Right=Ac-VAnd-ChcR]
%   {\Delta, e \mapsto \Delta', s}
%   { \Delta, e \wedge \chc[D]{e_{1},e_{2}} \mapsto \Delta', s \wedge \chc[D]{e_{1},e_{2}} }
% \end{mathpar}
%
  %%% Or with Choice rules
% \begin{mathpar}
%   \hspace{-1.5cm}
%   \inferrule*[Right=Ac-VOr-ChcL]
%   {\Delta, e \mapsto \Delta', s}
%   { \Delta, \chc[D]{e_{1},e_{2}} \vee e \mapsto \Delta' \,
%     \chc[D]{e_{1},e_{2}} \vee s}
% \end{mathpar}
%
% \begin{mathpar}
%   \hspace{-1.5cm}
%   \inferrule*[Right=Ac-VOr-ChcR]
%   {\Delta, e \mapsto \Delta', s}
%   { \Delta, e \vee \chc[D]{e_{1},e_{2}} \mapsto \Delta', s \vee \chc[D]{e_{1},e_{2}} }
% \end{mathpar}
%
%%% Congruence rules with symbolic execution
% \end{mathpar}
%
% \begin{mathpar}
%   \hspace{-1.5cm}
%   \inferrule*[Right=Ac-VOr-Value]
%   {(\Delta, e_{1}) \mapsto (\Delta_{1}, s_{1}) \\ (\Delta_{1}, e_{2}) \mapsto
%     (\Delta_{2}, s_{2}) \\  (\Delta_{2}, s_{1} \vee s_{2}) \leadsto
%     (\Delta', s)}
%   { (\Delta, e_{1} \vee e_{2}) \mapsto (\Delta', s)}
% \end{mathpar}
%
% \begin{mathpar}
%   %%% Congruence rules with symbolic execution
%   \inferrule*[Right=Ac-VAnd]
%   {(\Delta, v_{1}) \mapsto (\Delta_{1}, v_{1}) \\ (\Delta_{1}, v_{2}) \mapsto
%     (\Delta', v_{2})}
%   {(\Delta, v_{1} \wedge v_{2}) \mapsto (\Delta', v_{1} \wedge v_{2})}
%
%   \inferrule*[Right=Ac-VOr]
%   {(\Delta, v_{1}) \mapsto (\Delta_{1}, v_{1}) \\
%     (\Delta_{1}, v_{2}) \mapsto (\Delta', v_{2})}
%   {(\Delta, v_{1} \vee v_{2}) \mapsto (\Delta', v_{1} \vee v_{2})}
% \end{mathpar}%
%
\end{mathpar}
%
% \begin{mathpar}
% \inferrule*[Right=Ac-And]
% {(\Delta, v_{1}) \mapsto (\Delta_{1}, v_{1}') \\ (\Delta_{1}, v_{2}) \mapsto (\Delta_{2}, v_{2}')}
% % ---------------------------------------------------------------
% { (\Delta, v_{1} \wedge v_{2}) \mapsto (\Delta_{2}, v_{1}' \wedge v_{2}') }
% \end{mathpar}
% %
% \begin{mathpar}
% \inferrule*[Right=Ac-Or]
% {(\Delta, v_{1}) \mapsto (\Delta_{1}, v_{1}') \\ (\Delta_{1}, v_{2}) \mapsto (\Delta_{2}, v_{2}')}
% % ---------------------------------------------------------------
% { (\Delta, v_{1} \vee v_{2}) \mapsto (\Delta_{2}, v_{1}' \vee v_{2}') }

% \end{mathpar}

  \caption{Selected accumulation semantics on IL formulas.}%
  \label{impl:accum}
\end{figure}

Accumulation is defined in \autoref{impl:accum} as a relation of the form
$(\Delta,v)\mapsto(\Delta,v)$, where $\Delta$ is a symbolic store, and $v$ is
the syntactic category representing the set of all possible IL formulas; the
tuples on the left and right are interpreted as input and output, respectively.
Accumulation interacts with the symbolic execution engine via primitive
operations represented in \texttt{typewriter} font.
%
The \rn{Acc-Gen} rule generates new symbolic references using \texttt{spawn},
which are looked up by \rn{Acc-Ref}. The \rn{Acc-And} and \rn{Acc-Or} rules
reduce symbolic sub-formulas to a new symbolic reference.
%
This selection of rules do the work of reducing formulas to symbolic references.
The remaining rules simply push negation down expressions and propagate
accumulation over the $\wedge$ and $\vee$-connectives.

%
\begin{figure}
  \begin{mathpar}
  %%% Computation rules
  \inferrule*[Right=Ev-Term]
  { \texttt{assert($(\Gamma, \Delta), t$)} = \Gamma' }
  {(\Gamma, \Delta, t) \rightarrowtail (\Gamma', \Delta, \unit{}) }

  \inferrule*[Right=Ev-Sym]
  { \texttt{assert($(\Gamma, \Delta), s$)} = \Gamma'}
  { (\Gamma, \Delta, s) \rightarrowtail (\Gamma', \Delta, \unit{}) }
\end{mathpar}
% \inferrule*[Right=Ev-Chc]
%   { }
%   {(\Theta, \chc[D]{e_{1}, e_{2}}) \rightarrowtail (\Theta, \chc[D]{e_{1}, e_{2}})}
%
%   \hspace{0.5cm}
%   \inferrule*[Right=Ev-Unit]
%   { }
%   { (\Theta,\unit{}) \rightarrowtail (\Theta, \unit{}) }

  \begin{mathpar}
  %%%\unit{} elimination rules
  \inferrule*[Right=Ev-UL]
  {  }
  { (\Theta,\unit{} \wedge v) \rightarrowtail (\Theta, v) }

  \inferrule*[Right=Ev-UR]
  { }
  { (\Theta, v \wedge\unit{}) \rightarrowtail (\Theta, v)}
\end{mathpar}
\begin{mathpar}
%   \inferrule*[Right=Ev-DM-VOr]
%   { (\Theta ,\neg e_{1} \wedge \neg e_{2}) \rightarrowtail (\Theta', v)}
%   { (\Theta, \neg (e_{1} \vee e_{2})) \rightarrowtail (\Theta', v)}
%
%   \hspace{1cm}
%   \inferrule*[Right=Ev-DM-VAnd]
%   { (\Theta, \neg e_{1} \vee \neg e_{2}) \rightarrowtail (\Theta', v)}
%   { (\Theta, \neg (e_{1} \wedge e_{2})) \rightarrowtail (\Theta', v)}
%
  %%% Negation rules
  % \inferrule*[Right=Ev-NegS]
  % { (\Delta, \neg s) \mapsto (\Delta', s') \\
  %   ((\Gamma, \Delta'), s') \rightarrowtail (\Theta, \unit{})
  % }
  % { ((\Gamma, \Delta), \neg s) \rightarrowtail (\Theta, \unit{})}
%   \inferrule*[Right=Ev-Neg-Chc]
%   { }
%   { (\Theta, \neg \chc[D]{e_{1},e_{2}}) \rightarrowtail (\Theta, \chc[D]{\neg e_{1}, \neg e_{2}}) }
%
  \inferrule*[Right=Ev-Neg]
  { (\Delta,\neg v) \mapsto (\Delta', v')
    % ((\Gamma, \Delta') ,\neg v) \rightarrowtail (\Theta, v')
  }
  { (\Gamma, \Delta, \neg v) \rightarrowtail (\Gamma, \Delta', v') }
\end{mathpar}
  %%% And with Choice rules
% \begin{mathpar}
%   \hspace{-1.5cm}
%   \inferrule*[Right=Ev-And-ChcL]
%   { \Gamma, \Delta,e \rightarrowtail \Gamma', \Delta',e'}
%   { \Gamma, \Delta,\chc[D]{e_{1},e_{2}} \wedge e \rightarrowtail \Gamma',
%     \Delta',\chc[D]{e_{1},e_{2}} \wedge e' }
% \end{mathpar}
% \begin{mathpar}
%   \hspace{-1.5cm}
%   \inferrule*[Right=Ev-And-ChcR]
%   { \Gamma, \Delta,e \rightarrowtail \Gamma', \Delta',e'}
%   { \Gamma, \Delta,e \wedge \chc[D]{e_{1},e_{2}} \rightarrowtail \Gamma',
%     \Delta',e' \wedge \chc[D]{e_{1},e_{2}} }
%   % {l \rightarrowtail l'}
%   % { l \wedge \chc[D]{e_{1},e_{2}} \rightarrowtail l' \wedge \chc[D]{e_{1},e_{2}}}
% \end{mathpar}
%   %%% Or with Choice rules
% \begin{mathpar}
%   \hspace{-1.5cm}
%   \inferrule*[Right=Ev-Or-ChcL]
%   {\Delta,e \mapsto \Delta',v}
%   { \Gamma, \Delta,\chc[D]{e_{1},e_{2}} \vee e \rightarrowtail \Gamma,
%     \Delta',\chc[D]{e_{1},e_{2}} \vee v}
% \end{mathpar}
% \begin{mathpar}
%   \hspace{-1.5cm}
%   \inferrule*[Right=Ev-Or-ChcR]
%   {\Delta,e \mapsto \Delta',v}
%   { \Gamma, \Delta,e \vee \chc[D]{e_{1},e_{2}} \rightarrowtail \Gamma,
%     \Delta',v \vee \chc[D]{e_{1},e_{2}} }
% \end{mathpar}
% \begin{mathpar% }
%   \inferrule*[Right=Ev-Or-Value]
%   {(\Delta, e_{1}) \mapsto (\Delta_{1}, s_{1}) \\
%     (\Delta_{1}, e_{2}) \mapsto (\Delta_{2}, s_{2}) \\
%     (\Delta_{2}, s_{1} \vee s_{2}) \mapsto (\Delta_{s}, s) \\
%     ((\Gamma, \Delta_{s}), s) \twoheadrightarrow (\Theta,\unit{})
%   }
%   { ((\Gamma, \Delta),e_{1} \vee e_{2}) \rightarrowtail (\Theta,\unit{})}
% \end{mathpar}
  %%% Congruence rules with symbolic execution
\begin{mathpar}
  \inferrule*[Right=Ev-Or]
  { (\Delta, v_{1}) \mapsto (\Delta', v_{1}') \\
    (\Delta',v_{2}) \mapsto (\Delta'', v_{2}')}
  { (\Gamma, \Delta, v_{1} \vee v_{2}) \rightarrowtail (\Gamma, \Delta'', v_{1}' \vee v_{2}')}
\end{mathpar}%
\begin{mathpar}
  \inferrule*[Right=Ev-And]
  {(\Theta, v_{1}) \rightarrowtail (\Theta', v_{1}') \\
    (\Theta', v_{2}) \rightarrowtail (\Theta'', v_{2}')
  }
  { (\Theta, v_{1} \wedge v_{2}) \rightarrowtail (\Theta'', v_{1}' \wedge v_{2}')
  }
\end{mathpar}%

  \caption{Selected evaluation semantics over \vpl{} formulas.}%
  \label{impl:eval}
\end{figure}

Evaluation is defined in \autoref{impl:eval} as a relation of the form
$(\Theta,v)\rightarrowtail(\Theta,v)$, where $\Theta=(\Gamma,\Delta)$ and
$\Gamma$ represents the base solver state.
%
As before, we show only a significant subset of the rules here. The rules
\rn{Ev-Term} and \rn{Ev-Sym} push new clauses to the base solver using the
primitive \texttt{assert} operation. The \rn{Ev-UL} and \rn{Ev-UR} implement
left and right unit, reducing conjunctions where one side has been processed by
the base solver.
%
Of special note is the difference between the \rn{Ev-Or} and \rn{Ev-And} rules.
While \rn{Ev-And} is a straightforward congruence rule, \rn{Ev-Or} instead
processes its arguments using accumulation ($\mapsto$). Disjunctions are a
source of back-tracking in variational solving, and thus the solver cannot
evaluate the left-hand side without evaluating the right, both of which may
contain choices, hence evaluation must switch to accumulation, as we informally
described in the previous subsection.

\begin{figure}
  % \begin{tabbing}
%   % \begin{align*}
%   {\sc CoreChoices}$\ :\ C \rightarrow vm \rightarrow ev \rightarrow vm$ \\
%   {\sc CoreChoices}$\ C\ mdls\ v$\\
%   \qquad case $v$ of \\
%   \qquad \qquad \= Unit \qquad \qquad \= = return \mdls{} \\
%   \> \chc[d]{e,e'} \> = CoreChoicesHelper C \dimd{} \mdls{} (Evaluate e) (Evaluate e') \\
%   \> \chc[d]{e,e'} $\wedge\ ev\ $ \> = do \\
%   \> \> \quad \=  $vE \leftarrow$ Evaluate $e$ \\
%   \> \> \> $vE' \leftarrow$ Evaluate $e'$ \\
%   \> \> \> CoreChoicesHelper C \dimd{} \mdls{} ($vE \wedge\ ev$) ($vE' \wedge ev$) \\
%   \> $ev\ \wedge\ $\chc[d]{e,e'} \> = do \\
%   \> \> \quad \=  $vE \leftarrow$ Evaluate $e$ \\
%   \> \> \> $vE' \leftarrow$ Evaluate $e'$ \\
%   \> \> \> CoreChoicesHelper C \dimd{} \mdls{} ($ev \wedge\ vE$) ($ev \wedge eV'$) \\
%   \> \chc[d]{e,e'} $\vee\ ev\ $ \> = do \\
%   \> \> \quad \=  $vE \leftarrow$ Evaluate $e$ \\
%   \> \> \> $vE' \leftarrow$ Evaluate $e'$ \\
%   \> \> \> CoreChoicesHelper C \dimd{} \mdls{} ($vE \vee\ ev$) ($vE' \vee ev$) \\
%   \> $ev\ \vee\ $\chc[d]{e,e'} \> = do \\
%   \> \> \quad \=  $vE \leftarrow$ Evaluate $e$ \\
%   \> \> \> $vE' \leftarrow$ Evaluate $e'$ \\
%   \> \> \> CoreChoicesHelper C \dimd{} \mdls{} ($ev \vee\ vE$) ($ev \vee eV'$) \\
%   \> $ev$ \> = do \\
%   \> \> \quad \= $vE \leftarrow$ FindPChoice (Evaluate $ev$) \\
%   \> \> \> solveChoices C \dimd{} \mdls{} ($vE$)  \\
% \end{tabbing}
\begin{mathpar}
  %%% Computation rules
  \inferrule*[Right=Gen]
  { \texttt{getModel($\Gamma, \Delta$)} = m' }
  { (C, \Theta, m, \unit{}) \Downarrow_{i} (C, \Theta, \oplus(C, m', m), \unit{}) }
% \end{mathpar}
\hspace{0.5cm}
% \begin{mathpar}
  \inferrule*[Right=Sym]
  { (\Theta, s) \rightarrowtail (\Theta', \unit{})}
  { (C, \Theta, m, s) \Downarrow_{i} (C, \Theta', m, \unit{})
  }
\end{mathpar}
%
\begin{mathpar}
% \inferrule*[Right=Cr-And]
% { (\Phi, v_{1}) \Downarrow_{i} (\Phi_{1}, v_{1}') \\
%     (\Phi_1, v_{2}) \Downarrow_{1} (\Phi', v_{2}') \\
%   }
%   { (\Phi, v_{1} \wedge v_{2}) \Downarrow_{i} (\Phi', v_{1}' \wedge v_{2}')
%   }
% \\%
%   \hspace{0.5cm}
  %% solve a symbolic reference execute
  \inferrule*[Right=Cr-Or]
  { (\Delta, v_{1} \vee v_{2}) \mapsto (\Delta', v)}
  { (C, \Gamma, \Delta, m, v_{1} \vee v_{2}) \Downarrow_{i} (C, \Gamma, \Delta',
    m, v) }
\\%
\inferrule*[Right=Cr-And-T]
{
  (D, \true) \in C \\
  (C, \Theta, m, v \wedge \texttt{toIR$(e_{1})$}) \Downarrow_{i} (C, \Theta, m, v')
}
{
  (C, \Theta, m, v \wedge \chc[D]{e_{1}, e_{2}}) \Downarrow_{i} (C, \Theta, m, v')
}

  % \inferrule*[Right=Cr-And-F]
  % {
  %   C\ d = False \\
  %   ((\Gamma, \Delta), e_{2}) \rightarrowtail ((\Gamma_{2}, \Delta_{2}), v_{2}) \\
  %   ((C, \Gamma_{2}, \Delta_{2}), v \wedge v_{2}) \Downarrow_{i} m
  % }
  % {
  %   ((C, \Gamma, \Delta), v \wedge \chc[d]{e_{1}, e_{2}}) \Downarrow_{i} m
  % }
%
  % \inferrule*[Right=Cr-Or-T]
  % {
  %   C(d) = True \\
  %   ((\Gamma, \Delta), e_{1}) \mapsto ((\Gamma_{1}, \Delta_{1}), v_{1}) \\
  %   ((C, \Gamma_{1}, \Delta_{1}), v \vee v_{1}) \Downarrow_{i} m
  % }
  % {
  %   ((C, \Gamma, \Delta), v \vee \chc[d]{e_{1}, e_{2}}) \Downarrow_{i} m
  % }
  %
\inferrule*[Right=Cr-Or-T]
{
  (D, \true) \in C \\
  (C, \Theta, m, v \vee \texttt{toIR$(e_{1})$}) \Downarrow_{i} (C, \Theta, m, v')
}
{
  (C, \Theta, m, v \vee \chc[D]{e_{1}, e_{2}}) \Downarrow_{i} (C, \Theta, m, v')
}
  % \inferrule*[Right=Cr-Or-F]
  % {
  %   C(D) = false \\
  %   ((C, \Gamma, \Delta), v \vee \texttt{toIR$(e_{2})$}) \Downarrow_{i} m
  % }
  % {
  %   ((C, \Gamma, \Delta), v \vee \chc[D]{e_{1}, e_{2}}) \Downarrow_{i} m
  % }
\\
%   \inferrule*[Right=Cr-COr]
%   {
%     \hspace{1.3cm}
%     ((C \cup \{(D, \kernfix{true})\}, \Gamma, \Delta), v \vee \chc[D]{e_{1}, e_{2}}) \Downarrow_{i+1} m_{1} \\
%     D \notin C \\
%     ((C \cup \{(D, \kernfix{false})\}, \Gamma, \Delta), v \vee \chc[D]{e_{1}, e_{2}}) \Downarrow_{i+1} m_{2} \\
%   }
%   {
%     ((C, \Gamma, \Delta), v \vee \chc[D]{e_{1}, e_{2}}) \Downarrow_{i} m_{1}
%     \oplus m_{2}
%   }
% \\
  \inferrule*[Right=Cr-CAnd]
  {
    % \hspace{1.5cm}
    D \notin C \\
    % \hspace{-0.5cm}
    (C \cup \{(D, \true)\}, \Theta, m,  v \wedge \chc[D]{e_{1}, e_{2}})
    \Downarrow_{i+1} (C', \Theta', m',  \unit{}) \\
    \hspace{2.25cm}
    (C \cup \{(D, \false)\}, \Theta, m,  v \wedge \chc[D]{e_{1}, e_{2}}) \Downarrow_{i+1} (C'', \Theta'', m'',  \unit{})\\
  }
  {
    (C, \Theta, m, v \wedge \chc[D]{e_{1}, e_{2}}) \Downarrow_{i} (C'',
    \Theta'', m' \cup m'',  \unit{})
  }
%
\end{mathpar}

  \caption{Selected variational solving semantics on cores.}%
  \label{impl:choice-eval}
\end{figure}

Solving the core is defined in \autoref{impl:choice-eval}, as a relation
%
$(C, \Gamma, \Delta, m,v) \Downarrow_{i} (C, \Gamma, \Delta, m, v)$, where $C$,
is a set which represents the configuration of the \vpl{} formula, and $m$
represents the variational model, which is initialized as empty.
%
The count of \texttt{push}'s on the assertion stack are represented with the
counter $i$. The solving process reifies choices by manipulating the
configuration and uses accumulation and evaluation to process terms. The \vc{}
input to the solver pre-populates the configuration, thereby restricting the
solver to a subset of variants. When no \vc{} is input, the configuration is
initialized as empty. \rn{Cr-CAnd} processes novel choices by manipulating the
configuration and performing a \texttt{push} in the base solver; resulting
variational models are merged via an element-wise $\vee$, shown as $\cup$.
Choices are removed through \rn{Cr-And-T} and \rn{Cr-Or-T} by selection on the
alternative. Once the choice is removed, the nested clauses are translated to
the intermediate language through the \texttt{toIL} primitive and processed by
accumulation and evaluation. A model is called from the base solver with
\rn{Gen}, once the core, and thus query formula is reduced to \unit{}. Note
again, \rn{Cr-Or} switches to accumulation to ensure sound results, we have
omitted congruence rules such as \rn{Cr-And}. Similarly, we omit rules which are
commutative versions of those shown here, namely: rules which process the left
branch of connectives, rules which select the $\kf{false}$ alternatives,
and the \rn{Or} version of \rn{Cr-CAnd}.

\subsection{Research Plan}

\paragraph{Estimating the difficulty of finding satisfiability} This item will
produce a method to determine the \emph{hardness} of solving a \ac{vpl} formula
for satisfiability. A well known result in the random-\ac{sat} community is the
phenomenon of a \emph{phase transition}~\cite{Gent94thesat} in randomly
generated \ac{sat} problems. The phase transition of \ac{sat} problems is an
inflection point in the probability of finding a satisfying assignment as the
ratio of clauses to variables is varied. Conceptually, one may think of the
phase transition as a method to estimate the difficultly in solving a \ac{sat}
problem. If there are many variables relative to clauses, then the \ac{sat}
problem is likely easy to solve as it is under-constrained, in contrast, if
there are too few variables relative to clauses then it is over-constrained and
thus easy to compute as unsatisfiable.

Difficult problems are balanced with respect to the clause variable ratio and
thus are at the phase transition point of a high probability of finding
satisfiability to finding unsatisfiability. Estimating the difficulty of a
\ac{sat} problem is thus theoretically useful to isolate sets of problems to
study and progress the state of the art, but also practically useful, because an
end-user may estimate the difficulty of a problem and choose to \emph{not} solve
it.

This item will replicate the analysis from the random-\ac{sat} community to
determine if the phase transition exists for variational satisfiability
problems. It is clear that the phase transition exists for variants, but having
a method to assess the phase transition point \emph{in terms of} \ac{vpl}
formulas would provide the aforementioned benefits for variational
satisfiability solvers.

%%% Local Variables:
%%% mode: latex
%%% TeX-master: "../main"
%%% End: