\label{section:motivation}
%
%% get to the problem:
%%% incremental satisfiability solvers solve sets of related problems efficiently
%%% but the interface could be improved in two ways
%%% first the interface require hand programming the solver, leading to
%%% brittle solutions
%%% secondly, the specifies control flow operations that dictate runtime
%%% behavior, thus any static optimizations enabled by domain knowledge cannot
%%% be used.
%
Classic \ac{sat}, which solves the boolean satisfiability
problem~\cite{10.5555/1550723} has been one of the largest success stories in
computer science over the last two decades. Although \ac{sat} solving is known
to be NP-complete~\cite{10.1145/800157.805047}, \ac{sat} solvers based on
\ac{cdcl}~\cite{Marques-Silva:1999:GSA:304491.304506,Silva:1997:GNS:244522.244560,10.5555/1867406.1867438}
have been able to solve boolean formulae with millions variables quickly enough
for use in real-world applications~\cite{10.5555/1557461}. Leading to their
proliferation into several fields of scientific inquiry ranging from software
engineering to
Bioinformatics~\cite{10.1007/11814948_16,10.1007/978-3-642-31612-8_12}.

The majority of research in the \ac{sat} community focuses on solving a single
\ac{sat} problem as fast as possible, yet many practical applications of
\ac{sat}
solvers~\cite{silva1997robust,10.1007/3-540-44798-9_4,10.1145/378239.379019,10.1145/1698759.1698762,Een_asingle-instance,een2003temporal,10.5555/1998496.1998520}
require solving a set of related \ac{sat}
problems~\cite{10.1007/3-540-44798-9_4, silva1997robust, een2003temporal}. To
take just one example, \ac{spl} utilizes \ac{sat} solvers for a diverse range of
analyses including: automated feature model
analysis~\cite{useBTRC05,GBT+19,TAK+:CSUR14}, feature model
sampling~\cite{MKR+:ICSE16,VAT+:SPLC18}, anomaly
detection~\cite{AKTS:FOSD16,KAT:TR16,MNS+:SPLC17}, and dead code
analysis~\cite{TLSS:EuroSys11}.

This misalignment between the \ac{sat} research community and the practical use
cases of \ac{sat} solvers is well known. To address the misalignment, modern
solvers attempt to propagate information from one solving instance, on one
problem, to future instances in the problem set. Initial attempts focused on
\ac{cs}~\cite{10.1007/3-540-44798-9_4,10.1145/378239.379019} where learned
clauses from one problem in the problem set are propagated forward to future
problems. Although, modern solvers are based on a major breakthrough that
occurred with \emph{incremental \ac{sat} under assumptions}, introduced in
Minisat~\cite{10.1007/978-3-540-24605-3_37}.

Incremental \ac{sat} under assumptions, made two major contributions: a
performance contribution, where information including learned clauses, restart
and clause-detection heuristics are carried forward. A usability contribution;
Minisat exposed an interface that allowed the end-user to directly program the
solver. Through the interface the user can add or remove clauses and dictate
which clauses or variables are shared and which are unique to the problem set,
thus directly addressing the practical use case of \ac{sat} solvers.

Despite the its success, the incremental interface introduced a programming
language that required an extra input, the set of \ac{sat} problems, \emph{and}
a program to direct the solver with side-effectual statements. This places
further burden on the end-user: the system is less-declarative as the user must
be concerned with the internals of the solver. A new class of errors is possible
as the input program could misuse the introduced side-effectual statements. By
requiring the user to direct the solver, the users' solution is specific to the
exact set of satisfiability problems at hand, thus the programmed solution is
specific to the problem set and therefore to the solver input. Should the user
be interested in the assignment of variables under which the problem at hand was
found to be satisfiable, then the user must create additional infrastructure to
track results; which again couples to the input and is therefore difficult to
reuse.

We argue that solving a set of related \ac{sat} problems \emph{is a variational
  programming problem} and that by directly addressing the problem's variational
nature the incremental \ac{sat} interface and performance can be improved. The
essence of variational programming is a formal language called the \emph{choice
  calculus}. With the choice calculus, sets of problems in the \ac{sat} domain
can be expressed syntactically as a single \emph{variational artifact}. The
benefits are numerous:
\begin{enumerate}
\item The side-effectual statements are hidden from the user, recovering the
  declarative nature of non-incremental \ac{sat} solving.
\item Malformed programs built around the control flow operators become
  syntactically impossible.
\item The end-user's programmed solution is decoupled from the specific problem
  set, increasing software reuse.
\item The solver has enough syntactic information to produce results which
  previously required extra infrastructure constructed by the end-user.
\item Previously difficult optimizations can be syntactically detected and
  applied before the runtime of the solver.
\end{enumerate}


This work is applied programming language theory in the domain of satisfiability
solvers. Due to the ubiquity of satisfiability solvers estimating the impact is
difficult although the surface area of possible applications is large. For
example, many analyses in the software product-lines community use incremental
\ac{sat} solvers. By creating a variational \ac{sat} solver such analyses
directly benefit from this work, and thus advance the state of the art. For
researchers in the incremental satisfiability solving community, this work
serves as an avenue to construct new incremental \ac{sat} solvers which
efficiently solve classes of problems that deal with variation.

For researchers studying variation the significance and impact is several fold.
By utilizing results in variational research, this work adds validity to
variational theory and serves as an empirical case study. At the time of this
writing, and to my knowledge, this work is the first to directly use results in
the variational research community to parallelize a variation unaware tool. Thus
by directly handling variation, this work demonstrates direct benefits to be
gained for researchers in other domains and magnifies the impact of any results
produced by the variational research community. Lastly, the result of my thesis,
a variational \ac{sat} solver, provides a new logic and tool to reason about
variation itself.

%%% Local Variables:
%%% mode: latex
%%% TeX-master: "../../thesis"
%%% End: