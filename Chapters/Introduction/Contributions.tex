\label{section:contributions}
%
The goal of this thesis is to use variation theory to formalize and construct a
variational satisfiability solver that understands and can solve \ac{sat}
problems that contain \emph{variational values} in addition to boolean values.
It is our desire that the work not only be of theoretical interest but of
practical use. Thus, the thesis provides numerous examples of variational
\ac{sat} and variational \ac{smt} problems to motivate and demonstrate the
solver. The rest of this section outlines the thesis and expands on the
contributions of each chapter:

  \begin{enumerate}
  \item \autoref{chapter:background} (\emph{Background}) provides the necessary
    material for a reader to understand the contributions of the thesis. This
    section provides an overview of \acl{sat}, \acl{smt} solving, incremental
    \ac{sat} and \ac{smt} solving. Several important concepts are introduced:
    First, the definition of satisfiability and the boolean satisfiability
    problem. Second, the internal data structure that incremental \ac{sat}
    solvers use to provide incrementality, and the operations that manipulate
    the incremental solver and form the basis of variational \acl{sat}. Third,
    the definition of the output of a \ac{sat} or \ac{smt} solver which has
    implications for variational \acl{sat} and variational \ac{smt}.

  \item \autoref{chapter:vpl} (\emph{Variational Propositional Logic})
    introduces a variational logic that a variational \ac{sat} solver operates
    upon. This section introduces the essential aspects of variation using
    propositional logic and in the process presents the first instance of a
    \emph{\recipe{} recipe} to construct a \emph{variation-aware} system using a
    non-variational version of that system. Several variational concepts are
    defined and formalized which are used throughout the thesis, such as
    \emph{variant}, \emph{configuration} and \emph{variational artifact}.
    Lastly, the section proves theorems that are required to prove the soundness
    of variational satisfiability solving. Major portions of this section are
    adapted from published~\cite{YWT:SPLC20}.

  \item \autoref{chapter:vsat} (\emph{Variational Satisfiability Solving}) makes
    the central contribution of the thesis. In this chapter we define the
    general approach and architecture of a variational \ac{sat} solver. The
    general approach is the second presentation of the aforementioned recipe; in
    this case, using a \ac{sat} solver rather than propositional logic. This
    section is an expanded version of published peer-reviewed
    work~\cite{YWT:SPLC20}. It provides a rationale for our design and makes
    several important contributions:
    \begin{enumerate}
    \item The definition of the variational satisfiability problem.
    \item A formal semantics of variational satisfiability solving. A
      variational \ac{sat} problem is an encoding of the problem in variational
      propositional logic that is translated to an incremental \ac{sat} program
      which is suitable for execution on an incremental \ac{sat} solver.
    \item A formal definition of several concepts such as a \emph{variational
        core} which are transferable to domains other than \ac{sat}. Variational
      cores are key to our approach, and enable the preservation of shared terms
      between variants.
    \item A definition of a \emph{variational compiler}. The compiler is defined
      as a variational fold which is the basis for the performance gains
      presented in the thesis. The folding algorithm has three phases to ensure
      non-variational terms are shared across \ac{sat} problems and plain terms
      are processed before variational terms, thus mitigating redundant
      computation.
    \item A definition of the variational output that is returned to the user.
      The output presents several unique challenges that must be overcome while
      still being useful for the user. We present and consider these
      concerns and provide a solution.
    \end{enumerate}

  \item \autoref{chapter:vsmt} (\emph{Variational Satisfiability-Modulo Theory
      Solving}) extends the variational solving algorithm to consider \ac{smt}
    theories and propositions which include numeric values such as Integers and
    Reals in addition to Booleans. We present the requisite extensions to the
    variational propositional logic, the variational compiler and solving
    algorithm, and extend the output to support types other than just Booleans.
    We demonstrate that our method fully generalizes to the core \ac{smt}
    theories in the \acl{smtlib} standard.

  \item \autoref{chapter:case-studies} (\emph{Case Studies}) The central project
    of this thesis is to evaluate the ideas of variational programming in
    \acl{sat}. Having defined and constructed a variational \ac{sat} and
    \ac{smt} solver, this chapter empirically assesses the prototype variational
    solvers. This chapter is adapted from work currently under
    review~\cite{VSATJournal}.

  \item \autoref{chapter:related-work} (\emph{Related Work}) is split into two
    sections. First, this work is related to numerous \ac{sat} solvers that
    attempt to reuse information, solve sets of \ac{sat} problems, and implement
    incremental \ac{sat} solving. We situate this work in the context of these
    solvers. Second, this thesis is part of a lineage of recent variation-aware
    systems, thus this section collects this research and provides a comparison
    of our method to create a variation-aware system with previous methods.

  \item \autoref{chapter:conclusion} (\emph{Conclusion and Future Work})
    summarizes the contributions of the thesis and relates the work to the
    central project of the thesis. In addition to the conceptual point, numerous
    areas of future work are discussed, such as further variational extensions,
    faster implementation strategies and the possibility to reuse our findings
    to create a variational
    Prolog~\cite{wielemaker:2011:tplp,EarlyLogicProgramming}.
  \end{enumerate}



%%% Local Variables:
%%% mode: latex
%%% TeX-master: "../../thesis"
%%% End: