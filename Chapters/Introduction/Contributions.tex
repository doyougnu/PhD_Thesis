~\label{section:contributions}

The high-level goal of this thesis is to use variation theory to formalize and
construct a variational satisfiability solver that understands and can solve
\ac{sat} problems that contain \emph{variational values} in addition to boolean
values. It is our desire that the work not only be of theoretical interest but
of practical use. Thus, the thesis provides numerous examples of variational
\ac{sat} and variational \ac{smt} problems to motivate and demonstrate the
solver. The rest of this section outlines the thesis and expands on the
contributions of each chapter:

  \begin{enumerate}
  \item \autoref{chapter:background} (\emph{Background}) provides the necessary
    material for a reader to understand the contributions of the thesis. This
    section provides an overview of \acl{sat}, \acl{smt} solving, incremental
    \ac{sat} and \ac{smt} solving . Several important concepts are introduced:
    The definition of satisfiability and definition of the boolean
    satisfiability problem. The internal data structure incremental \ac{sat}
    solvers utilize to provide incrementality, and the side-effectual operations
    which manipulate the incremental solver and form the basis of variational
    \acl{sat}. Lastly, the definition of the output of a \ac{sat} or \ac{smt}
    solver which has implications for variational \acl{sat} and variational
    \ac{smt}.

  \item \autoref{chapter:vpl} (\emph{Variational Propositional Logic})
    introduces a variational logic that a variational \ac{sat} solver operates
    upon. This section introduces the essential aspects of variation using
    propositional logic and in the process presents the first instance of a
    \emph{\recipe{} recipe} to construct a \emph{variation-aware}
    system using a non-variational version of that system. Several variational
    concepts are defined and formalized which are used throughout the thesis,
    such as \emph{variant}, \emph{configuration} and \emph{variational
      artifact}. Lastly, the section proves theorems that are central to proof
    of the soundness of variational satisfiability solving.

  \item \autoref{chapter:vsat} (\emph{Variational Satisfiability Solving}) makes
    the central contribution of the thesis. In this chapter we define the
    general approach and architecture of a variational \acl{sat}. The general
    approach is the second presentation of the aforementioned recipe; in this
    case using a \ac{sat} solver rather than propositional logic. This section
    provides a rationale for this design and makes several important
    contributions:
    \begin{enumerate}
    \item An operational semantics of variational satisfiability solving. A
      variational \ac{sat} problem is a description of the problem in
      variational propositional logic that is translated to an incremental
      \ac{sat} program which is suitable for execution on an incremental
      \ac{sat} solver.
    \item A formal definition of concepts such as a \emph{variational core}
      which are transferable to domains other than \ac{sat}. Variational cores
      are instances of var
    \item A definition of a \emph{variational transpiler}. The transpiler is
      defined as a variational fold which is the basis for the performance gains
      presented in the thesis. The folding algorithm is defined in three phases
      to ensure that non-variational terms are shared across \ac{sat} problems
      and thus redundant computation is mitigated.
    \item A definition of a variational output that is returned to the user. The
      output presents several unique challenges that must be overcome while
      still being useful for the end user. We present and consider these
      concerns and provide a salable solution.
    \end{enumerate}

  \item \autoref{chapter:vsmt} (\emph{Variational Satisfiability-Modulo Theory
      Solving}) extends the variational solving algorithm to consider \ac{smt}
    theories and propositions which include numeric values such as Integers and
    Reals in addition to Booleans. We present the requisite extensions to the
    variational propositional logic, the variational transpiler and solving
    algorithm, and extend the output to support types other than just Booleans.
    While these extensions are relatively straightforward, complete support of
    \ac{smt} theories would need to include \texttt{BitVector} and
    \texttt{Arrays}. Supporting these theories is not straightforward, thus the
    chapter concludes by outlining the problem space for adding these features.

  \item \autoref{chapter:case-studies} (\emph{Case Studies}) The central project
    of this thesis is to evaluate the ideas of variational programming in
    \acl{sat}. Having defined and constructed a variational \ac{sat} and
    \ac{smt} solver in the previous chapters this chapter empirically evaluates
    the solvers. The first section demonstrates performance improvements in
    variational \ac{sat}'s intended use case. The second section present
    variational versions of classic \ac{sat} problems such as: \todo{fill after
      making examples}\ldots{} and serves as a tutorial for identifying,
    writing, and solving variational \ac{sat} problems with a variational
    \ac{sat} solver.

  \item \autoref{chapter:related-work} (\emph{Related Work}) is split into two
    sections. First, this thesis is situated into a lineage of recent
    variational-aware systems, thus this section collects this research and
    provides a comparison of our method to create a variational-aware system
    with previous methods. Second, this work is related to numerous \ac{sat}
    solvers that attempt to reuse information, solve sets of \ac{sat} problems
    and implement incremental \ac{sat} solving. We situate this work in the
    context of these solvers and compare their methods.


  \item \autoref{chapter:conclusion} \todo{rewrite, not really sure what do put
      here yet, writers block}(\emph{Conclusion and Future Work}) summarizes the
    contributions of the thesis, discusses possible extensions to variational
    \ac{sat} solving and creating variational-aware system.
  \end{enumerate}

% \begin{enumerate}
% \item\label{vpl-deliverable} \emph{Variational propositional logic}: \ac{sat}
%   solvers input and operate on sentences in propositional
%   logic\cite{10.5555/1324777}. Variational satisfiability solvers, in order to
%   reason about variation, must input sentences in a propositional logic that is
%   \textit{variational}, \ie{} a many-valued logic~\cite{Rescher1969-RESML} which
%   contains variational values, called \emph{choices} in addition to boolean
%   values.

%   The formulation of \ac{vpl} is requisite and central to the high level goal of
%   designing a variational satisfiability solver. Furthermore, \ac{vpl} serves
%   two other functions: It provides an avenue for future work through the
%   formalization of variation in the domain of propositional logic for
%   variational satisfiability solvers. It provides a foundation for research on
%   variation in propositional logic outside of the considerations of
%   satisfiability solvers.

%   This work is nearly complete. The logic has been formalized and successfully
%   used in a prototype variational solver\cite{10.1145/3382025.3414965}.
%   \autoref{sec:vpl} introduces \ac{vpl} and describes the following
%   contributions which are directly enabled by it:

%   \item \checkmark{} A set of variation preserving equivalences. Similar to the well known
%     propositional logic equivalences, such as DeMorgan's law, these equivalences
%     allow a variational solver to refactor input possibly yielding simpler
%     variational sentences.
%   \item\label{encoding-strat-deliverable} An efficient algorithm for translating
%     a set of propositional formulae into a single \ac{vpl} formula. The
%     prototype variational \ac{sat} solver used a naive algorithm, and
%     preliminary results showed that the encoding impacts solver performance.
%     Hence, finding a more efficient encoding algorithm is desirable. This work
%     is yet to be done but there are two promising routes forward. First, a naive
%     algorithm which interleaves syntactic equivalences to produce a \ac{vpl}
%     formula that is easier to solve. Second, an algorithm similar to Huffman
%     codes\cite{4051119} to translate the \ac{sat} problems into a data
%     structure, then use heuristics to select high quality candidates to combine.
%     With such an algorithm the end-user of the variational solver only needs to
%     input their problem sequence rather than a \ac{vpl} formula.
%   \end{enumerate}

% \item\label{vsat-deliverable} \emph{A variational satisfiability solver}: This
%   is the central contribution of my thesis. It is completed and is published in
%   a peer-reviewed conference~\cite{10.1145/3382025.3414965} paper. Preliminary
%   results are promising but based on only two case studies from the \ac{spl}
%   community.

%   \autoref{sec:vsat} discusses these results and provides an overview of the
%   variational solving algorithm. The following contributions are based on this
%   work:
%   \begin{enumerate}
%   \item \checkmark{} Formalization of a variational \ac{sat} solving algorithm
%     that inputs a \ac{vpl} formula and outputs a \emph{variational model}.
%   \item \checkmark{} Formalization of variational models; that is satisfying
%     assignments of values to variables in input formula that succinctly
%     represent results in the context of variation.
%   \item \checkmark{} A method for determining the amount of variation in a given
%     \ac{vpl} formula.
%   \item\label{phase-change-deliverable} A method for determining the relative
%     hardness of a \ac{vpl} formula based on work in the random-\ac{sat}
%     community~\cite{Gent94thesat}. This item is orthogonal to all other items
%     and thus can be done in parallel.
%   \end{enumerate}

% \item\label{vsmt-deliverable} \emph{A concurrent variational \ac{smt} solver}:
%   Contingent on \autoref{vsat-deliverable}, \emph{satisfiability modulo
%     theories} extends \ac{sat} solvers such that they are able to reason about
%   logical formulas in combination to \textit{background theories}, such as
%   arithmetic or arrays. Furthermore, with variation statically represented in a
%   \ac{vpl} formula, the \ac{sat} or \ac{smt} procedure can be made asynchronous
%   leading to speedups on multi-core machines. The approach is to change the
%   semantics of a choice; in the prototype \ac{sat} solver each choice blocks
%   future \ac{sat} problems from being solved, by creating an asynchronous
%   solving algorithm these future problems are unblocked and can be processed
%   earlier.

%   This item is an extension of the central contributions of the thesis. There
%   are two extensions to the previous work to construct a variational \ac{smt}
%   solver and one to make it asynchronous.

%   First, the extensions to \ac{vpl} abstract logical connectives in \ac{vpl}
%   allowing for theories which conclude to a Boolean value, such as arithmetic
%   inequalities, and thus can be reasoned about in a \ac{smt} solver. Second,
%   variational models are similarly extended, rather than assuming only Boolean
%   values, the extension allows for polymorphic results through the use of
%   SMTLIB2 compliant functions.

%   Third, I extend the semantics of choices in the variational \ac{sat} solver to
%   include atomic concurrent operations. When a choice is observed the solver
%   state is copied and sent to a thread with instructions to compute continue the
%   computation.

%   This work is completed but unpublished. \autoref{sec:vsmt} expands on this
%   item and discusses the evaluation of the prototype variational \ac{smt} solver
%   with additional case studies. The following summarizes the expected
%   contributions:
%   \begin{enumerate}
%   \item \checkmark{} Formalize the extension of \ac{vpl} with \ac{smt} theories.
%   \item \checkmark{} Formalize the extension of variational models to express
%     \ac{smt} results.
%   \item \checkmark{} Formalize the asynchronous variational solving algorithm
%   \item\label{nanopass-deliverable} A set of optimizations based on work on
%     nanopass compilers~\cite{10.1145/2500365.2500618} from the scheme
%     programming language community~\cite{r7rs-scheme}. The goal is to leverage
%     \ac{vpl} equivalence rules and other compiler optimizations, such as
%     inlining, on SMTLIB2 programs, thus optimizing variational SMT programs. The
%     prototype variational \ac{smt} solver is architected as a nanopass compiler
%     and thus is able to perform optimizations as a single pass over the input
%     formula. However, no optimizations are performed as of yet, although all
%     requisite items for this work to begin are done.
%   \item \label{eval-deliverable}An empirical evaluation of solver performance. The empirical evaluation
%     will reuse the datasets the prototype \ac{sat} solver was evaluated on. In
%     addition, three new data sets will be added, two by harvesting \ac{sat}
%     problems from work on variational lexing, parsing, and type
%     checking~\cite{KKHL:FOSD10} real world software such as
%     Busybox~\cite{busybox} and the Linux kernel~\cite{linux}, and one by
%     generating variational \ac{smt} problems. This dataset will be used several
%     times in the thesis and will be made public. First, as a foundation to test
%     the encoding strategies from \autoref{encoding-strat-deliverable}. Second,
%     to evaluation the optimizations from \autoref{nanopass-deliverable} and
%     third, to evaluate the performance of the single threaded and multi-threaded
%     variational \ac{smt} solver. This work is partially complete, random
%     generation of variational \ac{sat} and \ac{smt} problems is done, as is the
%     Busybox dataset. The remaining work is to scale the logging solution to
%     handle the Linux kernel.
%   \end{enumerate}

% \item \label{proof-deliverable} \textit{Proof of variation preservation}: A
%   proof of variation preservation is a proof that the results of the variational
%   solvers are sound, \ie{}, for any variant $\kf{v}$, if a variational solver
%   finds $\kf{VSat(v)} = \kf{True}$, then $\kf{Sat(v)} = \kf{True}$. Both
%   \autoref{vsat-deliverable} and \autoref{vsmt-deliverable} are verified sound
%   via property-based testing~\cite{10.1145/351240.351266} but the variational
%   solving algorithm itself has not been proven sound up to the soundness of the
%   underlying incremental solver. This work is in progress using the proof
%   assistant Agda\cite{10.1145/2841316} and is expected to yield such a proof.
%   \autoref{sec:vsmt} discusses this item further and lists the specific tasks
%   left to do.
% \end{enumerate}


%%% Local Variables:
%%% mode: latex
%%% TeX-master: "../../thesis"
%%% End: