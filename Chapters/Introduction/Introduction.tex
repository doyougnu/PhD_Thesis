\label{chapter:introduction}%
%
Controlling complexity is a central goal of any programming language, especially
as software written in that language grows. The burgeoning field of
\emph{variation theory} and \emph{variational
  programming}~\cite{EW11gttse,EW11tosem,HW16fosd,CEW16ecoop,Walk14onward}
attempt to control a kind of complexity which is induced into software when many
\emph{similar yet distinct} kinds of the same software must coexist. For
example, software is often \emph{ported} to other platforms, creating similar,
yet distinct instances of that software which must be maintained. Such instances
of variation are ubiquitous: Web applications are tested on multiple servers;
programming languages maintain backwards compatibility and so do software
libraries; databases evolve over time, locale and data; and device drivers must
work with varying processors and architectures. Variation theory and variational
programming have been successful in small
systems~\cite{ES18diagrams,SE17fosd,MMWWK17vamos}, yet it has not been tested in
a performance demanding practical domain. In the words of Joe
Armstrong\cite{armstrongThesis}, ``No theory is complete without proof that the
ideas work in practice''; this is the project of this thesis, to put the ideas
of \emph{variation} and \emph{variational programming} to the test in the
practical domain of \ac{sat}.

The major contribution of this thesis is the formalization of a
\emph{\ac{vpl}},\todo{this is the place for variational system definition}
\emph{variational \acl{sat}}, and the construction of a \emph{variational
  \ac{sat} solver}. In the next section I motivate the use of variation theory
and variational techniques in \acl{sat}. In addition to work on variational
\ac{sat} several other contributions are made. The thesis extends variational
\acl{sat} to variational \ac{smt}. It demonstrates reusable techniques and
architecture for constructing \emph{variational or variation-aware} systems
using the non-variational counterparts of these systems for other domains. It
shows that, with the concept of variation, the variational \ac{smt} and \ac{sat}
solvers can be trivially parallelized. Lastly, the thesis provides a general
algorithm to construct variational strings from a set of non-variational strings
and argues for the proliferation of variation theory to other domains in
computer science.

\section{Motivation and Impact}
\label{section:motivation}
%
%% get to the problem:
%%% incremental satisfiability solvers solve sets of related problems efficiently
%%% but the interface could be improved in two ways
%%% first the interface require hand programming the solver, leading to
%%% brittle solutions
%%% secondly, the specifies control flow operations that dictate runtime
%%% behavior, thus any static optimizations enabled by domain knowledge cannot
%%% be used.
%
Classic \ac{sat}, which solves the boolean satisfiability problem~\cite{BBH+09}
has been one of the largest success stories in computer science over the last
two decades. Although \ac{sat} solving is known to be
NP-complete~\cite{10.1145/800157.805047}, \ac{sat} solvers based on
\ac{cdcl}~\cite{Marques-Silva:1999:GSA:304491.304506,Silva:1997:GNS:244522.244560,10.5555/1867406.1867438}
have been able to solve boolean formulae with millions variables quickly enough
for use in real-world applications~\cite{10.5555/1557461}. Leading to their
proliferation into several fields of scientific inquiry ranging from software
engineering to
Bioinformatics~\cite{10.1007/11814948_16,10.1007/978-3-642-31612-8_12}.

The majority of research in the \ac{sat} community focuses on solving a single
\ac{sat} problem as fast as possible, yet many practical applications of
\ac{sat}
solvers~\cite{silva1997robust,10.1007/3-540-44798-9_4,10.1145/378239.379019,10.1145/1698759.1698762,Een_asingle-instance,een2003temporal,10.5555/1998496.1998520}
require solving a set of related \ac{sat}
problems~\cite{10.1007/3-540-44798-9_4, silva1997robust, een2003temporal}. To
take just one example, \ac{spl} utilizes \ac{sat} solvers for a diverse range of
analyses including: automated feature model
analysis~\cite{useBTRC05,GBT+19,TAK+:CSUR14}, feature model
sampling~\cite{MKR+:ICSE16,VAT+:SPLC18}, anomaly
detection~\cite{AKTS:FOSD16,KAT:TR16,MNS+:SPLC17}, and dead code
analysis~\cite{TLSS:EuroSys11}.

This misalignment between the \ac{sat} research community and the practical use
cases of \ac{sat} solvers is well known. To address the misalignment, modern
solvers attempt to propagate information from one solving instance, on one
problem, to future instances in the problem set. Initial attempts focused on
\ac{cs}~\cite{10.1007/3-540-44798-9_4,10.1145/378239.379019} where learned
clauses from one problem in the problem set are propagated forward to future
problems. Although, modern solvers are based on a major breakthrough that
occurred with \emph{incremental \ac{sat} under assumptions}, introduced in
\texttt{Minisat}~\cite{10.1007/978-3-540-24605-3_37}.

Incremental \ac{sat} under assumptions, made two major contributions: a
performance contribution, where information including learned clauses, restart
and clause-detection heuristics are carried forward. A usability contribution;
\texttt{Minisat} exposed an interface that allowed the end-user to directly
program the solver. Through the interface the user can add or remove clauses and
dictate which clauses or variables are shared and which are unique to the
problem set, thus directly addressing the practical use case of \ac{sat}
solvers.

Despite the its success, the incremental interface introduced a programming
language that required an extra input, the set of \ac{sat} problems, \emph{and}
a program to direct the solver with side-effectual statements. This places
further burden on the end-user: the system is less-declarative as the user must
be concerned with the internals of the solver. A new class of errors is possible
as the input program could misuse the introduced side-effectual statements. By
requiring the user to direct the solver, the users' solution is specific to the
exact set of satisfiability problems at hand, thus the programmed solution is
specific to the problem set and therefore to the solver input. Should the user
be interested in the assignment of variables under which the problem at hand was
found to be satisfiable, then the user must create additional infrastructure to
track results; which again couples to the input and is therefore difficult to
reuse.

We argue that solving a set of related \ac{sat} problems \emph{is a variational
  programming problem}, and by directly addressing the problem's variational
nature, the incremental \ac{sat} interface and performance can be improved. The
essence of variational programming is a formal language called the \emph{choice
  calculus}. With the choice calculus, local points of variation are able to be
explicitly represented. Furthermore, sets of problems in the \ac{sat} domain can
be expressed syntactically as a single \emph{variational artifact}. The benefits
are numerous:
\begin{enumerate}
\item The side-effectual statements are hidden from the user, recovering the
  declarative nature of non-incremental \ac{sat} solving.
\item Malformed programs built around the control flow operators become
  syntactically impossible.
\item The end-user's programmed solution is decoupled from the specific problem
  set, increasing software reuse.
\item The solver has enough syntactic information to produce results which
  previously required extra infrastructure constructed by the end-user.
\item Previously difficult optimizations can be syntactically detected and
  applied before the runtime of the solver.
\end{enumerate}


This work is applied programming language theory in the domain of satisfiability
solvers. Due to the ubiquity of satisfiability solvers estimating the impact is
difficult although the surface area of possible applications is large. For
example, many analyses in the software product-lines community use incremental
\ac{sat} solvers. By creating a variational \ac{sat} solver such analyses
directly benefit from this work, and thus advance the state of the art. For
researchers in the incremental satisfiability solving community, this work
serves as an avenue to construct new incremental \ac{sat} solvers which
efficiently solve classes of problems that deal with variation.

For researchers studying variation the significance and impact is several fold.
By utilizing results in variational research, this work adds validity to
variational theory and serves as an empirical case study. At the time of this
writing, and to my knowledge, this work is the first to directly use results in
the variational research community to parallelize a variation unaware tool. Thus
by directly handling variation, this work demonstrates direct benefits to be
gained for researchers in other domains and magnifies the impact of any results
produced by the variational research community. Lastly, the result of this
thesis, a variational \ac{sat} solver, provides a tool to reason about variation
itself.

%%% Local Variables:
%%% mode: latex
%%% TeX-master: "../../thesis"
%%% End:

\section{Contributions and Outline of this Thesis}
\label{section:contributions}
%
The goal of this thesis is to use variation theory to formalize and construct a
variational satisfiability solver that understands and can solve \ac{sat}
problems that contain \emph{variational values} in addition to boolean values.
It is our desire that the work not only be of theoretical interest but of
practical use. Thus, the thesis provides numerous examples of variational
\ac{sat} and variational \ac{smt} problems to motivate and demonstrate the
solver. The rest of this section outlines the thesis and expands on the
contributions of each chapter:

  \begin{enumerate}
  \item \autoref{chapter:background} (\emph{Background}) provides the necessary
    material for a reader to understand the contributions of the thesis. This
    section provides an overview of \acl{sat}, \acl{smt} solving, incremental
    \ac{sat} and \ac{smt} solving. Several important concepts are introduced:
    First, the definition of satisfiability and the boolean satisfiability
    problem. Second, the internal data structure that incremental \ac{sat}
    solvers use to provide incrementality, and the operations that manipulate
    the incremental solver and form the basis of variational \acl{sat}. Third,
    the definition of the output of a \ac{sat} or \ac{smt} solver which has
    implications for variational \acl{sat} and variational \ac{smt}.

  \item \autoref{chapter:vpl} (\emph{Variational Propositional Logic})
    introduces a variational logic that a variational \ac{sat} solver operates
    upon. This section introduces the essential aspects of variation using
    propositional logic and in the process presents the first instance of a
    \emph{\recipe{} recipe} to construct a \emph{variation-aware} system using a
    non-variational version of that system. Several variational concepts are
    defined and formalized which are used throughout the thesis, such as
    \emph{variant}, \emph{configuration} and \emph{variational artifact}.
    Lastly, the section proves theorems that are required to prove the soundness
    of variational satisfiability solving. Major portions of this section are
    adapted from published~\cite{YWT:SPLC20}.

  \item \autoref{chapter:vsat} (\emph{Variational Satisfiability Solving}) makes
    the central contribution of the thesis. In this chapter we define the
    general approach and architecture of a variational \ac{sat} solver. The
    general approach is the second presentation of the aforementioned recipe; in
    this case, using a \ac{sat} solver rather than propositional logic. This
    section is an expanded version of published peer-reviewed
    work~\cite{YWT:SPLC20}. It provides a rationale for our design and makes
    several important contributions:
    \begin{enumerate}
    \item The definition of the variational satisfiability problem.
    \item A formal semantics of variational satisfiability solving. A
      variational \ac{sat} problem is an encoding of the problem in variational
      propositional logic that is translated to an incremental \ac{sat} program
      which is suitable for execution on an incremental \ac{sat} solver.
    \item A formal definition of several concepts such as a \emph{variational
        core} which are transferable to domains other than \ac{sat}. Variational
      cores are key to our approach, and enable the preservation of shared terms
      between variants.
    \item A definition of a \emph{variational compiler}. The compiler is defined
      as a variational fold which is the basis for the performance gains
      presented in the thesis. The folding algorithm has three phases to ensure
      non-variational terms are shared across \ac{sat} problems and plain terms
      are processed before variational terms, thus mitigating redundant
      computation.
    \item A definition of the variational output that is returned to the user.
      The output presents several unique challenges that must be overcome while
      still being useful for the user. We present and consider these
      concerns and provide a solution.
    \end{enumerate}

  \item \autoref{chapter:vsmt} (\emph{Variational Satisfiability-Modulo Theory
      Solving}) extends the variational solving algorithm to consider \ac{smt}
    theories and propositions which include numeric values such as Integers and
    Reals in addition to Booleans. We present the requisite extensions to the
    variational propositional logic, the variational compiler and solving
    algorithm, and extend the output to support types other than just Booleans.
    We demonstrate that our method fully generalizes to the core \ac{smt}
    theories in the \acl{smtlib} standard.

  \item \autoref{chapter:case-studies} (\emph{Case Studies}) The central project
    of this thesis is to evaluate the ideas of variational programming in
    \acl{sat}. Having defined and constructed a variational \ac{sat} and
    \ac{smt} solver, this chapter empirically assesses the prototype variational
    solvers. This chapter is adapted from work currently under
    review~\cite{VSATJournal}.

  \item \autoref{chapter:related-work} (\emph{Related Work}) is split into two
    sections. First, this work is related to numerous \ac{sat} solvers that
    attempt to reuse information, solve sets of \ac{sat} problems, and implement
    incremental \ac{sat} solving. We situate this work in the context of these
    solvers. Second, this thesis is part of a lineage of recent variation-aware
    systems, thus this section collects this research and provides a comparison
    of our method to create a variation-aware system with previous methods.

  \item \autoref{chapter:conclusion} (\emph{Conclusion and Future Work})
    summarizes the contributions of the thesis and relates the work to the
    central project of the thesis. In addition to the conceptual point, numerous
    areas of future work are discussed, such as further variational extensions,
    faster implementation strategies and the possibility to reuse our findings
    to create a variational
    Prolog~\cite{wielemaker:2011:tplp,EarlyLogicProgramming}.
  \end{enumerate}



%%% Local Variables:
%%% mode: latex
%%% TeX-master: "../../thesis"
%%% End:

%%% Local Variables:
%%% mode: latex
%%% TeX-master: "../../thesis"
%%% End: