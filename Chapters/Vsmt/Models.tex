~\label{section:vsmt:models}
% %
\begin{figure}[h]
    \centering
    \begin{subfigure}[t]{\textwidth}
  \begin{tabbing}
    \qquad \quad \= $\iV{} \rightarrow$ -1 \\
    \> $\cV{} \rightarrow$ 0 \\
    \\
    \\
    \quad $C_{FF}$ = \{(\AV{}, \false{}), (\BV{}, \false{})\} \\
  \end{tabbing}
\end{subfigure}%
\begin{subfigure}[t]{\textwidth}
  \begin{tabbing}
    \\
    \\
    \qquad \quad \= $\aV{} \rightarrow$ \tru{} \\
    \\
    \quad $C_{FF}$ = \{(\AV{}, \fls{}), (\BV{}, \tru{})\} \\
  \end{tabbing}
\end{subfigure}%
\newline
\begin{subfigure}[t]{\textwidth}
  \begin{tabbing}
    \qquad \quad \= $\iV{} \rightarrow$ 0 \\
    \> $\cV{} \rightarrow$ 1 \\
    \\
    \\
    \quad $C_{FT}$ = \{(\AV{}, \tru{}), (\BV{}, \fls{})\} \\
  \end{tabbing}
\end{subfigure}%
\begin{subfigure}[t]{\textwidth}
  \begin{tabbing}
    \qquad \quad \= $\iV{} \rightarrow$ 0 \\
    \> $\cV{} \rightarrow$ 0 \\
    \\
    \> $\bV{} \rightarrow$ -10 \\
    \quad $C_{TT}$ = \{(\AV{}, \tru{}), (\BV{}, \tru{})\} \\
  \end{tabbing}
\end{subfigure}%%
    \caption{Possible plain models for variants of $\kf{f}$.}%
    \label{fig:vsmt:models:plain}
\end{figure}
% %
To support \ac{smt} theories, variational models must be abstract enough to
handle values other than Booleans. Functionally, variational \ac{smt} models
must satisfy several constraints: the variational \ac{smt} model must be more
memory efficient than storing all models returned by the solver naively. The
variational model must allow users to find satisfying values for a variant. The
model must allow users to find all variants a variable has a particular value or
range of values.

Furthermore, several useful properties of variational models should be
maintained: The model is non-variational; thus the user does not need to
understand the choice calculus to understand their results. The model produces
results that can be fed into a plain \ac{sat} solver (or \ac{smt} solver in the
extension). The model can be built incrementally and without regard to the
ordering of results, because it forms a commutative monoid under $\vee$.

To maintain these properties and satisfy the functional requirements, our
strategy for variational \ac{smt} models is to create a mapping of variables to
\ac{smt} expressions. By virtue of this strategy, variables are disallowed from
changing types across the set of variants and hence disallowed from changing
type as the result of a choice. For any variable in the model, we assume the
type returned by the base solver is correct, and store the satisfying value in a
linked list constructed \emph{if-statements}\footnote{Also called a
  church-encoded list}. Specifically, we utilize the function $ite : \mathbb{B}
\rightarrow T \rightarrow T$ from the SMTLIB2 standard to construct the list.
All variables are initialized as undefined (\emph{Und}) until a value is
returned from the base solver for a variant. To ensure the correct value of a
variable corresponds to the appropriate variant, we translate the configuration
which determines the variant to a variation context, and place the appropriate
value in the \emph{then} branch. For example, a possible entry for $\kf{j}$ in
the variational model of $\kf{g}$ would be $j \rightarrow (\kf{ite}\ \kf{A}\ 1\
(\kf{ite}\ \neg\kf{A}\ 0\ \kf{Und}))$.

\autoref{fig:vsmt:models:plain} show possible plain models for $\kf{f}$ with the
corresponding variational \ac{smt} model display in
\autoref{fig:vsmt:models:var}. We've added line breaks to emphasize the branches
the $\kf{then}$ and $\kf{else}$ branches of the $\kf{ite}$ SMTLIB2 primitive.

\begin{figure}[h]
  \centering
  \begin{subfigure}[t]{\textwidth}
  \begin{tabbing}
  \hquad \= $\_Sat \rightarrow (\neg \AV{} \wedge \neg \BV{}) \vee (\neg \AV{} \wedge \BV{}) \vee (\AV{} \wedge \neg \BV{}) \vee (\AV{} \wedge \BV)$ \\
  \> \iV{}\quad\hspace{1.7ex}\=$\rightarrow$ ($\kf{ite}$ ($\AV{} \wedge \BV{}$)\; \=2 \\
  \> \> \> ($\kf{ite}$\; ($\AV{} \wedge \neg \BV{}$)\; \=0 \\
  \> \> \> \> ($\kf{ite}$\; ($\neg \AV{} \wedge \neg \BV{}$)\; -1 $\kf{Und}$))) \\

  \> \cV{}\quad\hspace{1.7ex}\=$\rightarrow$ ($\kf{ite}$ ($\AV{} \wedge \BV{}$)\; \=0 \\
  \> \> \> ($\kf{ite}$\; ($\AV{} \wedge \neg \BV{}$)\; \=0 \\
  \> \> \> \> ($\kf{ite}$\; ($\neg \AV{} \wedge \BV{}$)\; \=0  \\
  \> \> \> \> \> ($\kf{ite}$\; ($\neg \AV{} \wedge \neg \BV{}$)\; \=0 $\kf{Und}$))) \\

  \> \aV{}\quad\hspace{1.7ex}\=$\rightarrow$ ($\kf{ite}$ ($ \AV{} \wedge \BV{}$)\; \=\tru{} \\
  \> \> \> ($\kf{ite}$ ($\neg \AV{} \wedge \BV{}$) \fls{} $\kf{Und}$))\\


  \> \bV{}\quad\hspace{1.7ex}\=$\rightarrow$ ($\kf{ite}$ ($\AV{} \wedge \BV{}$)\; \=3 \\
  \> \> \> ($\kf{ite}$\; ($\AV{} \wedge \neg \BV{}$)\; \=-10 \\
  \> \> \> \> ($\kf{ite}$\; ($\neg \AV{} \wedge \BV{}$)\; \=0  \\
  \> \> \> \> \> ($\kf{ite}$\; ($\neg \AV{} \wedge \neg \BV{}$)\; \=4 $\kf{Und}$))) \\
\end{tabbing}
\end{subfigure}

  \caption{Variational model corresponding to the plain models in
    \autoref{fig:vsmt:models:plain}.}%
  \label{fig:vsmt:models:var}
\end{figure}

This formulation maintains the functional requirements of the model. We maintain
a special variable $\_Sat$ to track the variants that were found satisfiable. In
this case all variants are satisfiable and thus we have four clauses over
dimensions in disjunctive normal form. If a user has a configuration then they
only need to perform substitution to determine the value of a variable under
that configuration. For example, if the user were interested in the value of
\iV{} in the $\{(\AV{}, \tru{}), (\BV{}, \tru{})\}$ variant they would
substitute the configuration into \vc{\iV{}} and recover 0 from the first
$\kf{ite}$ case. To find the variants at which a variable has a value, a user
may employ a \ac{smt} solver, add \vc{\iV{}} as a constraint, and query for a
model.

This maintains the desirable properties of variational \ac{sat} models while
allowing any type specified in the SMTLIB2 standard. The variational \ac{smt}
model does not require knowledge of choice calculus or variation, it is still
monoidal---although not a commutative monoid---and can be built in any order as
long as there are no duplicate variants; a scenario that is impossible by the
property of synchronization on choices. However, variational \ac{sat} models
clearly compressed results by preventing duplicate values with constant
variables, the variational \ac{smt} model allows for duplicate values, if those
values are parameterized by disjoint variants. For example, both \iV{} and \cV{}
contain duplicate values, but only one: \iV{} is easy to check in $O(1)$ time as
the duplicates are sequential in \vc{\iV{}} and can thus be checked during model
construction. Such a case would be easily avoided in an implementation by
tracking the values a variable has been assigned in all variants. However, we
desire to keep variational models as simple as possible and therefore only
present the minimum required machinery.


%%% Local Variables:
%%% mode: latex
%%% TeX-master: "../../thesis"
%%% End: