~\label{section:vsmt:models}
% %
\begin{figure}[h]
    \centering
    \begin{subfigure}[t]{\textwidth}
  \begin{tabbing}
    \qquad \quad \= $\iV{} \rightarrow$ -1 \\
    \> $\cV{} \rightarrow$ 0 \\
    \\
    \\
    \quad $C_{FF}$ = \{(\AV{}, \false{}), (\BV{}, \false{})\} \\
  \end{tabbing}
\end{subfigure}%
\begin{subfigure}[t]{\textwidth}
  \begin{tabbing}
    \\
    \\
    \qquad \quad \= $\aV{} \rightarrow$ \tru{} \\
    \\
    \quad $C_{FF}$ = \{(\AV{}, \fls{}), (\BV{}, \tru{})\} \\
  \end{tabbing}
\end{subfigure}%
\newline
\begin{subfigure}[t]{\textwidth}
  \begin{tabbing}
    \qquad \quad \= $\iV{} \rightarrow$ 0 \\
    \> $\cV{} \rightarrow$ 1 \\
    \\
    \\
    \quad $C_{FT}$ = \{(\AV{}, \tru{}), (\BV{}, \fls{})\} \\
  \end{tabbing}
\end{subfigure}%
\begin{subfigure}[t]{\textwidth}
  \begin{tabbing}
    \qquad \quad \= $\iV{} \rightarrow$ 0 \\
    \> $\cV{} \rightarrow$ 0 \\
    \\
    \> $\bV{} \rightarrow$ -10 \\
    \quad $C_{TT}$ = \{(\AV{}, \tru{}), (\BV{}, \tru{})\} \\
  \end{tabbing}
\end{subfigure}%%
    \caption{Possible plain models for variants of $\kf{f}$.}%
    \label{fig:vsmt:models:plain}
\end{figure}
% %
We have thus far covered accumulation, evaluation, and choice removal. However,
to support \ac{smt} theories, variational models must be abstract enough to
handle values other than Booleans. Functionally, variational \ac{smt} models
must satisfy several constraints: the variational \ac{smt} model must be more
memory efficient than storing all models returned by the solver naively. The
variational model must allow users to find satisfying values for a variant. The
model must allow users to find all variants in which a variable has a particular
value or range of values.

Furthermore, several useful properties of variational models should be
maintained: The model is non-variational; the user should not need to understand
the choice calculus to understand their results. The model produces results that
can be fed into a plain \ac{sat} or \ac{smt} solver. The model can be built
incrementally and without regard to the ordering of results. Variational
\ac{sat} models guaranteed this last constraint by forming commutative monoid
under $\vee$; a technique which we cannot replicate for variational \ac{smt}
models.

To maintain these properties and satisfy the functional requirements, our
strategy for variational \ac{smt} models is to create a mapping of variables to
\ac{smt} expressions. By virtue of this strategy, variables are disallowed from
changing types across variants and hence disallowed from changing type as the
result of a choice. For any variable in the model, we assume the type returned
by the base solver is correct, and store the satisfying value in a linked list
constructed \emph{if-statements}\footnote{Also called a church-encoded list}.
Specifically, we utilize the function $ite : \mathbb{B} \rightarrow T
\rightarrow T$ from the SMTLIB2 standard to construct the list. All variables
are initialized as undefined (\undefined) until a value is returned from the
base solver for a variant. To ensure the correct value of a variable corresponds
to the appropriate variant, we translate the configuration which determines the
variant to a variation context, and place the appropriate value in the
\emph{then} branch.

Consider the following variational \ac{smt} problem extended with an integer
arithmetic theory: $f = ((\chc[A]{\iV{}, 13} - \cV{}) < (\bV + 10)) \rightarrow
\chc[B]{\aV{}, \cV{} > \iV{}}$. \fV{} contains two unique choices, $\kf{A}$,
$\kf{B}$, and thus represents four variants. In this case, the expression is
under-constrained and so each variant will be found satisfiable.
% 
\begin{figure}[h]
  \centering
  \begin{subfigure}[t]{\textwidth}
  \begin{tabbing}
    \qquad \quad \= $\iV{} \rightarrow$ -1 \\
    \> $\cV{} \rightarrow$ 0 \\
    \\
    \\
    \quad $C_{FF}$ = \{(\AV{}, \fls{}), (\BV{}, \fls{})\} \\
  \end{tabbing}
\end{subfigure}%
\begin{subfigure}[t]{\textwidth}
  \begin{tabbing}
    \\
    \\
    \qquad \quad \= $\aV{} \rightarrow$ \tru{} \\
    \\
    \quad $C_{FF}$ = \{(\AV{}, \fls{}), (\BV{}, \tru{})\} \\
  \end{tabbing}
\end{subfigure}%
\begin{subfigure}[t]{\textwidth}
  \begin{tabbing}
    \qquad \quad \= $\iV{} \rightarrow$ 0 \\
    \> $\cV{} \rightarrow$ 1 \\
    \\
    \\
    \quad $C_{FT}$ = \{(\AV{}, \tru{}), (\BV{}, \fls{})\} \\
  \end{tabbing}
\end{subfigure}%
\begin{subfigure}[t]{\textwidth}
  \begin{tabbing}
    \qquad \quad \= $\iV{} \rightarrow$ 0 \\
    \> $\cV{} \rightarrow$ 0 \\
    \\
    \> $\bV{} \rightarrow$ -10 \\
    \quad $C_{TT}$ = \{(\AV{}, \tru{}), (\BV{}, \tru{})\} \\
  \end{tabbing}
\end{subfigure}%
  \caption{Possible plain models for variants of $\kf{f}$.}%
  \label{fig:vsmt:models:plain}
\end{figure}
\begin{figure}[h]
  \centering
  \begin{subfigure}[t]{\textwidth}
  \begin{tabbing}
  \qquad \qquad \= $\_Sat \rightarrow (\neg \AV{} \wedge \neg \BV{}) \vee (\neg \AV{} \wedge \BV{}) \vee (\AV{} \wedge \neg \BV{}) \vee (\AV{} \wedge \BV)$ \\
  \> \iV{}\quad\hspace{1.7ex}\=$\rightarrow$ ($\kf{ite}$ ($\AV{} \wedge \BV{}$)\; \=0 \\
  \> \> \> ($\kf{ite}$\; ($\AV{} \wedge \neg \BV{}$)\; \=0 \\
  \> \> \> \> ($\kf{ite}$\; ($\neg \AV{} \wedge \neg \BV{}$)\; -1 $\kf{Undefined}$))) \\

  \> \cV{}\quad\hspace{1.7ex}\=$\rightarrow$ ($\kf{ite}$ ($\AV{} \wedge \BV{}$)\; \=0 \\
  \> \> \> ($\kf{ite}$\; ($\AV{} \wedge \neg \BV{}$)\; \=1 \\
  \> \> \> \> ($\kf{ite}$\; ($\neg \AV{} \wedge \neg \BV{}$)\; 0 $\kf{Undefined}$))) \\

  \> \aV{}\quad\hspace{1.7ex}\=$\rightarrow$ ($\kf{ite}$ ($\neg \AV{} \wedge \BV{}$)\; \tru{} $\kf{Undefined}$)\\
  \> \bV{}\quad\hspace{1.7ex}\=$\rightarrow$ ($\kf{ite}$ ($\AV{} \wedge \BV{}$)\; -10 $\kf{Undefined}$)
\end{tabbing}
\end{subfigure}

  \caption{Variational model corresponding to the plain models in
    \autoref{fig:vsmt:models:plain}.}%
  \label{fig:vsmt:models:var}
\end{figure}

\autoref{fig:vsmt:models:plain} show possible plain models for $\kf{f}$ with the
corresponding variational \ac{smt} model presented in
\autoref{fig:vsmt:models:var}. We've added line breaks to emphasize the
$\kf{then}$ and $\kf{else}$ branches of the $\kf{ite}$ SMTLIB2 primitive. 

This formulation maintains the functional requirements and desirable properties
of the variational \ac{sat} models. The variable $\_Sat$ is used to track the
variants that were found satisfiable, just as in the variational \ac{sat}
solver. In this case, all variants are satisfiable and thus we have four clauses
over dimensions in disjunctive normal form. If a user has a configuration then
they only need to perform substitution to determine the value of a variable
under that configuration. For example, if the user were interested in the value
of \iV{} in the $\{(\AV{}, \tru{}), (\BV{}, \tru{})\}$ variant they would
substitute the configuration into the result for \iV{} and recover $2$ from the
first $\kf{ite}$ case. To find the variants at which a variable has a value, a
user may employ a \ac{smt} solver, add the entry for \iV{} as a constraint, and
query for a model.
%
This specification of variational \ac{smt} models does not require knowledge of
choice calculus or variation, it is still monoidal---although not a commutative
monoid---and can be built in any order as long as there are no duplicate
variants; a scenario that is impossible by the property of synchronization on
choices.


However, there are some notable differences. Where variational \ac{sat} models
clearly compressed results by preventing duplicate values with constant
variables, the variational \ac{smt} model allows for duplicate values, if those
values are produced our of order. For example, both models for \iV{} and \cV{}
contain duplicate values. The \iV{} model has duplicate $-1$ and the \cV{} model
contains duplicate $0$. However only one: \cV{} is easy to check in \bigOof{1}
time; each call to \rn{Combine} could check the last immediate value to prevent
duplicate branches. In contrast, the duplicate $-1$'s for \iV{} occur in
variants that are likely to occur with several other plain models between them,
namely the models for the $C_{\tru\fls}$ and $C_{\fls\fls}$ variants. Hence, a
check during \rn{Combine} would require \bigOof{n} time, where $n$ is the number
of satisfiable variants that \iV{} occurs in. While such a case is easily
avoided in an implementation by tracking the values a variable has been
previously assigned, we provide only a minimum specification and thus leave the
details to an implementation.
%
Lastly, the use of \undefined{} may seem unattractive. While all bindings in the
model end with an \undefined{}, a binding cannot result in an \undefined{} as
that would imply a variant that was found to be satisfiable but was not
satisfiable, and hence would be indicative of a bug in the variational solver
implementation. Thus mathematically inclined readers may observe that the monoid
variational \ac{smt} models form corresponds to the free monoid formed by
church-encoded lists.

%%% Local Variables:
%%% mode: latex
%%% TeX-master: "../../thesis"
%%% End: