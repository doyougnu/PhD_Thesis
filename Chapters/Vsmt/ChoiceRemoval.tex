~\label{section:vsmt:choice-removal}
% 
\begin{figure}
  % \[
%   \zipper \hquad\Coloneqq\hquad \inRoot
%   \hquad|\hquad \inNot{\zipper}
%   \hquad|\hquad \inUnary{\zipper}
%   \hquad|\hquad \inBoolL{\zipper}{\,\eIL}
%   \hquad|\hquad \inBoolR{s}{\zipper}
%   \hquad|\hquad \inArithL{\zipper}{\,\eIL}
%   \hquad|\hquad \inArithR{s}{\zipper}
%   \hquad|\hquad \inInEqL{\zipper}{\,\eIL}
%   \hquad|\hquad \inInEqR{s}{\zipper}
% \]
\begin{syntax}
  \zipper & \Coloneqq{} & \inRoot  \\
  & | & \inNot{\zipper}            \\
  & | & \inUnary{\zipper}          \\
  & | & \inBoolL{\zipper}{\,\eIL}   \\
  & | & \inBoolR{s}{\zipper}       \\
  & | & \inArithL{\zipper}{\,\eIL}  \\
  & | & \inArithR{s}{\zipper}      \\
  & | & \inInEqL{\zipper}{\,\eIL}   \\
  & | & \inInEqR{s}{\zipper}       \\
\end{syntax}
  \caption{Variational \ac{smt} zipper context}
  \label{fig:vsmt:zipper}
\end{figure}
%
\begin{figure}
  \begin{mathpar}
\inferrule*[right=\crEval]
  { (\eStore{},\aStore{},v) \evaluation (\eStore{}',\aStore{}',\unit) \\
    \texttt{Combine}(\vmodel{},\pmodel(\aStore{},\eStore{})) = \vmodel{}' }
  { (\crCtx, \inRoot, v) \choiceRemoval \vmodel{}' }

\inferrule*[right=\crChcT]
  { (D,\true)\in C \\
    (\crCtx, z, e_1) \choiceRemoval \vmodel{}' }
  { (\crCtx, z, \chc[D]{e_1,e_2} \choiceRemoval \vmodel{}' }

\inferrule*[right=\crChcF]
  { (D,\false)\in C \\
    (\crCtx, z, e_2) \choiceRemoval \vmodel{}' }
  { (\crCtx, z, \chc[D]{e_1,e_2} \choiceRemoval \vmodel{}' }

\inferrule*[right=\crChc]
  { D\notin\dom{C} \\
    (C\cup(D,\true),\eStore{},\aStore{},\vmodel{}, z, e_1)
      \choiceRemoval \vmodel{1} \\
    (C\cup(D,\false),\eStore{},\aStore{},\vmodel{}', z, e_2)
      \choiceRemoval \vmodel{2} }
  { (\crCtx, z, \chc[D]{e_1,e_2} \choiceRemoval \vmodel{2} }

\inferrule*[right=\crNot]
  { (\crCtx, \inNot{z}, v) \choiceRemoval \vmodel{}' }
  { (\crCtx, z, \neg v) \choiceRemoval \vmodel{}' }

\inferrule*[right=\crNotIn]
  { (\aStore{}, \neg s) \accumulation (\aStore{}', s') \\
    (\crCtx, z, s') \choiceRemoval \vmodel{}' }
  { (\crCtx, \inNot{z}, s) \choiceRemoval \vmodel{}' }

\inferrule*[right=\crAnd]
  { (\crCtx, \inAndL{z}{v_2}, v_2) \choiceRemoval \vmodel{}' }
  { (\crCtx, z, v_1 \wedge v_2) \choiceRemoval \vmodel{}' }

\inferrule*[right=\crAndL]
  { (\crCtx, \inAndR{s}{z}, v) \choiceRemoval \vmodel{}' }
  { (\crCtx, \inAndL{z}{v}, s) \choiceRemoval \vmodel{}' }

\inferrule*[right=\crAndR]
  { (\aStore{}, s_1 \wedge s_2) \accumulation (\aStore{}', s_3) \\
    (\crCtx, z, s_3) \choiceRemoval \vmodel{}' }
  { (\crCtx, \inAndR{s_1}{z}, s_2) \choiceRemoval \vmodel{}' }

\inferrule*[right=\crOr]
  { (\crCtx, \inOrL{z}{v_2}, v_2) \choiceRemoval \vmodel{}' }
  { (\crCtx, z, v_1 \vee v_2) \choiceRemoval \vmodel{}' }

\inferrule*[right=\crOrL]
  { (\crCtx, \inOrR{s}{z}, v) \choiceRemoval \vmodel{}' }
  { (\crCtx, \inOrL{z}{v}, s) \choiceRemoval \vmodel{}' }

\inferrule*[right=\crOrR]
  { (\aStore{}, s_1 \vee s_2) \accumulation (\aStore{}', s_3) \\
    (\crCtx, z, s_3) \choiceRemoval \vmodel{}' }
  { (\crCtx, \inOrR{s_1}{z}, s_2) \choiceRemoval \vmodel{}' }
\end{mathpar}

  \caption{Variational \ac{smt} choice removal inference rules}%
  \label{fig:vsmt:chc}
\end{figure}
% 
\begin{figure}
  \ContinuedFloat
  \begin{mathpar}
  \inferrule*[right=\crUnary]
  { (\crCtx, \inUnary{\zipper}, \eIL) \choiceRemoval \vmodel{}' }
  { (\crCtx, \zipper, \integerUnary \eIL) \choiceRemoval \vmodel{}' }

  \inferrule*[right=\crUnaryIn]
  { (\aStore{}, \integerUnary s) \accumulation (\aStore{}', s') \\
    (\crCtx, \zipper, s') \choiceRemoval \vmodel{}' }
  { (\crCtx, \inUnary{\zipper}, s) \choiceRemoval \vmodel{}' }

\inferrule*[right=\crInEq]
  { (\crCtx, \inInEqL{z}{\eIL*{2}}, \eIL*{1}) \choiceRemoval \vmodel{}' }
  { (\crCtx, \zipper, \eIL*{1} \inequalities{} \eIL*{2}) \choiceRemoval \vmodel{}' }

\inferrule*[right=\crInEqL]
  { (\crCtx, \inInEqR{s}{\zipper}, \eIL) \choiceRemoval \vmodel{}' }
  { (\crCtx, \inInEqL{\zipper}{\eIL}, s) \choiceRemoval \vmodel{}' }

\inferrule*[right=\crInEqR]
  { (\aStore{}, s_1 \inequalities{} s_2) \accumulation (\aStore{}', s_3) \\
    (\crCtx, \zipper, s_3) \choiceRemoval \vmodel{}' }
  { (\crCtx, \inInEqR{s_1}{\zipper}, s_2) \choiceRemoval \vmodel{}' }

\inferrule*[right=\crArith]
  { (\crCtx, \inArithL{z}{\eIL*{2}}, \eIL*{1}) \choiceRemoval \vmodel{}' }
  { (\crCtx, \zipper, \eIL*{1} \integerFuncs{} \eIL*{2}) \choiceRemoval \vmodel{}' }

\inferrule*[right=\crArithL]
  { (\crCtx, \inArithR{s}{\zipper}, \eIL) \choiceRemoval \vmodel{}' }
  { (\crCtx, \inArithL{\zipper}{\eIL}, s) \choiceRemoval \vmodel{}' }

\inferrule*[right=\crArithR]
  { (\aStore{}, s_1 \integerFuncs{} s_2) \accumulation (\aStore{}', s_3) \\
    (\crCtx, \zipper, s_3) \choiceRemoval \vmodel{}' }
  { (\crCtx, \inArithR{s_1}{\zipper}, s_2) \choiceRemoval \vmodel{}' }
\end{mathpar}

  \caption{Variational \ac{smt} choice removal inference rules}
  \label{fig:vsmt:chc-cont}
\end{figure}
%
With accumulation and evaluation complete we turn to choice removal.
%
Our strategy is similar to accumulation; we generalize the zipper context over
Boolean, arithmetic and inequality relations using the syntactic categories
defined in \autoref{fig:vsmt:categories}.
%
Conceptually, choice removal remains a variational left fold that builds the
zipper context until a symbolic value is the focus. at which point rules of the
form \rn{C-*-InL} \emph{switch} the fold to process the right child of the
relations, and rules of the form \rn{C-*-InR} call accumulation to reduce the
relation over symbolic values. We formally specify generalized choice removal in
\autoref{fig:vsmt:chc}. The heart of choice removal remains the same, the rules
\crEval, \crChc, \crChcT, and \crChcF are reproduced for the new zipper context
\zipper but are semantically identical to the specialized versions. The
remaining rules are generalized versions of the \ac{sat} rules to handle each
syntactic category the variational solver can process.

For each syntactic category we define three kinds of rules which form a template
to extend choice removal to new background theories: First, we have rules which
determine what to do when encountering a binary or unary relation for the first
time. As a design choice this is defined to proceed into the left child. For
example \crNot{}, \crBool{}, and \crInEq{} initiate the left fold by storing the
relation in the zipper and focusing the left child $\eIL*{1}$.
%
Second, we define rules which recur down the left child of the relations until a
symbolic value results from accumulation. For example, \crInEqL{}, \crBoolL{}
and \crArithL{} all move the focused symbol value $s$ to the zipper context
allowing choice removal to proceed to the right children of the same relation.
%
Lastly, we have computation rules which perform the fold on the relation by
calling accumulation. For example, \crUnaryIn, \crArithR and \crInEqR call
accumulation to process the symbolic value and reduce the given relation. In
effect, accumulation \emph{encapsulates} the semantics of the relations,
evaluation propagates accumulation and performs code generation in the base
solver, and choice removal alters the configuration, maintains evaluation
contexts, and removes choices introducing new plain terms to the formula.

%%% Local Variables:
%%% mode: latex
%%% TeX-master: "../../thesis"
%%% End: