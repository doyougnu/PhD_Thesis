~\label{section:vsmt:choice-removal}
%
\begin{figure}
  \begin{mathpar}
\inferrule*[right=\crEval]
  { (\eStore{},\aStore{},v) \evaluation (\eStore{}',\aStore{}',\unit) \\
    \texttt{Combine}(\vmodel{},\pmodel(\aStore{},\eStore{})) = \vmodel{}' }
  { (\crCtx, \inRoot, v) \choiceRemoval \vmodel{}' }

\inferrule*[right=\crChcT]
  { (D,\true)\in C \\
    (\crCtx, z, e_1) \choiceRemoval \vmodel{}' }
  { (\crCtx, z, \chc[D]{e_1,e_2} \choiceRemoval \vmodel{}' }

\inferrule*[right=\crChcF]
  { (D,\false)\in C \\
    (\crCtx, z, e_2) \choiceRemoval \vmodel{}' }
  { (\crCtx, z, \chc[D]{e_1,e_2} \choiceRemoval \vmodel{}' }

\inferrule*[right=\crChc]
  { D\notin\dom{C} \\
    (C\cup(D,\true),\eStore{},\aStore{},\vmodel{}, z, e_1)
      \choiceRemoval \vmodel{1} \\
    (C\cup(D,\false),\eStore{},\aStore{},\vmodel{}', z, e_2)
      \choiceRemoval \vmodel{2} }
  { (\crCtx, z, \chc[D]{e_1,e_2} \choiceRemoval \vmodel{2} }

\inferrule*[right=\crNot]
  { (\crCtx, \inNot{z}, v) \choiceRemoval \vmodel{}' }
  { (\crCtx, z, \neg v) \choiceRemoval \vmodel{}' }

\inferrule*[right=\crNotIn]
  { (\aStore{}, \neg s) \accumulation (\aStore{}', s') \\
    (\crCtx, z, s') \choiceRemoval \vmodel{}' }
  { (\crCtx, \inNot{z}, s) \choiceRemoval \vmodel{}' }

\inferrule*[right=\crAnd]
  { (\crCtx, \inAndL{z}{v_2}, v_2) \choiceRemoval \vmodel{}' }
  { (\crCtx, z, v_1 \wedge v_2) \choiceRemoval \vmodel{}' }

\inferrule*[right=\crAndL]
  { (\crCtx, \inAndR{s}{z}, v) \choiceRemoval \vmodel{}' }
  { (\crCtx, \inAndL{z}{v}, s) \choiceRemoval \vmodel{}' }

\inferrule*[right=\crAndR]
  { (\aStore{}, s_1 \wedge s_2) \accumulation (\aStore{}', s_3) \\
    (\crCtx, z, s_3) \choiceRemoval \vmodel{}' }
  { (\crCtx, \inAndR{s_1}{z}, s_2) \choiceRemoval \vmodel{}' }

\inferrule*[right=\crOr]
  { (\crCtx, \inOrL{z}{v_2}, v_2) \choiceRemoval \vmodel{}' }
  { (\crCtx, z, v_1 \vee v_2) \choiceRemoval \vmodel{}' }

\inferrule*[right=\crOrL]
  { (\crCtx, \inOrR{s}{z}, v) \choiceRemoval \vmodel{}' }
  { (\crCtx, \inOrL{z}{v}, s) \choiceRemoval \vmodel{}' }

\inferrule*[right=\crOrR]
  { (\aStore{}, s_1 \vee s_2) \accumulation (\aStore{}', s_3) \\
    (\crCtx, z, s_3) \choiceRemoval \vmodel{}' }
  { (\crCtx, \inOrR{s_1}{z}, s_2) \choiceRemoval \vmodel{}' }
\end{mathpar}

  \caption{Choice removal inference rules}%
  \label{fig:vsmt:inf:chc}
\end{figure}
%
With accumulation and evaluation we turn to choice removal. Our strategy is to
generalize the zipper context over the Boolean, arithmetic and inequality
relations using the syntactic categories and machinery we employed to generalize
accumulation. The extended zipper is given by the following grammar
% 
\begin{figure}
  % \[
%   \zipper \hquad\Coloneqq\hquad \inRoot
%   \hquad|\hquad \inNot{\zipper}
%   \hquad|\hquad \inUnary{\zipper}
%   \hquad|\hquad \inBoolL{\zipper}{\,\eIL}
%   \hquad|\hquad \inBoolR{s}{\zipper}
%   \hquad|\hquad \inArithL{\zipper}{\,\eIL}
%   \hquad|\hquad \inArithR{s}{\zipper}
%   \hquad|\hquad \inInEqL{\zipper}{\,\eIL}
%   \hquad|\hquad \inInEqR{s}{\zipper}
% \]
\begin{syntax}
  \zipper & \Coloneqq{} & \inRoot  \\
  & | & \inNot{\zipper}            \\
  & | & \inUnary{\zipper}          \\
  & | & \inBoolL{\zipper}{\,\eIL}   \\
  & | & \inBoolR{s}{\zipper}       \\
  & | & \inArithL{\zipper}{\,\eIL}  \\
  & | & \inArithR{s}{\zipper}      \\
  & | & \inInEqL{\zipper}{\,\eIL}   \\
  & | & \inInEqR{s}{\zipper}       \\
\end{syntax}
  \caption{variational \ac{smt} zipper context}
  \label{fig:vsmt:zipper}
\end{figure}

% 
% Choice removal is defined in \autoref{fig:vsmt:inf:chc} as a relation between
% the evaluation/accumulation stores (\aStore{}, \eStore{}), the configuration
% (\configuration{}), and terms in IL\@. Furthermore, we track the current
% variational model as part of the 4-tuple. The vast majority of rules are either
% commutative versions of the presented rules; such as \rn{CR-RB} which is
% \rn{CR-LB} but with a choice as the left child of \boolFuncs{}, or the same
% rules over different operators, such as \rn{CR-LIB} which is \rn{CR-LB}
% only for \inequalities{}; thus we only present a subset.

% The interesting rules are \rn{Gen} and \rn{Sym} which use evaluation to
% query for a plain model, and construct a new variational model through the
% \rn{Combine} function. \rn{CR-LB} ensure the property of
% synchronization; when a choice is observed as the right child of a boolean
% operator, and the dimension has a value in the configuration (in this case
% \true{}), then the proper alternative (in this case the left alternative) of the
% choice is retrieved. \rn{CR-IB-ChcR} removes choices when the choice is not
% present in the configuration. We present the version of \rn{CR-IB-ChcR} for
% \inequalities{}; the same rule exists for \boolFuncs{}, \integerFuncs{}, and for
% choices as the left children of \inequalities{}. The assertion stack counter,
% $\kf{i}$, is incremented indicating that all recursive processing occurs in a
% new \rn{push}/\rn{pop} context. Each configuration is updated to process
% both alternatives, \true{} for the left and \false{} for the right alternative.
% Both alternatives eventually conclude to a \unit{} and thus a variational model,
% which are combined to a final result.

% The remaining rules are congruence rules that recursively call accumulation
% after a choice has been found, and new terms are introduced as the result of a
% replacing a choice with an alternative. Careful readers will recognize that the
% provided rules can easily become stuck. For example, given the formula $a \vee{}
% (b \leq{} \chc[D]{p,q})$ the rules cannot further reduce the formula due to the
% disjunction and inequality, and the choice cannot be accumulated. What is
% required is to find the choice while storing the \emph{context} around the
% choice. We leave this as an implementation detail, the prototype variational
% solvers utilize a Huet zipper~\cite{huet_1997} data structure to capture this
% context\footnote{that the Huet zipper has been successful implies delimited
%   continuations~\itodo{cite} may be an alternative and efficient method to
%   capture the context}, searches the variational core until a choice is in the
% focus position, and then applies a choice removal rule such as
% \rn{Cr-IB-ChcR} or \rn{Cr-LB}.


%%% Local Variables:
%%% mode: latex
%%% TeX-master: "../../thesis"
%%% End: