~\label{section:vsmt:choice-removal}
%
Choice removal is defined in \autoref{fig:vsmt:inf:chc} as a relation between
the evaluation/accumulation stores (\aStore{}, \eStore{}), the configuration
(\configuration{}), and terms in IL\@. Furthermore, we track the current
variational model as part of the 4-tuple. The vast majority of rules are either
commutative versions of the presented rules; such as \rn{CR-RB} which is
\rn{CR-LB} but with a choice as the left child of \boolFuncs{}, or the same
rules over different operators, such as \rn{CR-LIB} which is \rn{CR-LB}
only for \inequalities{}; thus we only present a subset.

The interesting rules are \rn{Gen} and \rn{Sym} which use evaluation to
query for a plain model, and construct a new variational model through the
\rn{Combine} function. \rn{CR-LB} ensure the property of
synchronization; when a choice is observed as the right child of a boolean
operator, and the dimension has a value in the configuration (in this case
\true{}), then the proper alternative (in this case the left alternative) of the
choice is retrieved. \rn{CR-IB-ChcR} removes choices when the choice is not
present in the configuration. We present the version of \rn{CR-IB-ChcR} for
\inequalities{}; the same rule exists for \boolFuncs{}, \integerFuncs{}, and for
choices as the left children of \inequalities{}. The assertion stack counter,
$\kf{i}$, is incremented indicating that all recursive processing occurs in a
new \rn{push}/\rn{pop} context. Each configuration is updated to process
both alternatives, \true{} for the left and \false{} for the right alternative.
Both alternatives eventually conclude to a \unit{} and thus a variational model,
which are combined to a final result.

The remaining rules are congruence rules that recursively call accumulation
after a choice has been found, and new terms are introduced as the result of a
replacing a choice with an alternative. Careful readers will recognize that the
provided rules can easily become stuck. For example, given the formula $a \vee{}
(b \leq{} \chc[D]{p,q})$ the rules cannot further reduce the formula due to the
disjunction and inequality, and the choice cannot be accumulated. What is
required is to find the choice while storing the \emph{context} around the
choice. We leave this as an implementation detail, the prototype variational
solvers utilize a Huet zipper~\cite{huet_1997} data structure to capture this
context\footnote{that the Huet zipper has been successful implies delimited
  continuations~\itodo{cite} may be an alternative and efficient method to
  capture the context}, searches the variational core until a choice is in the
focus position, and then applies a choice removal rule such as
\rn{Cr-IB-ChcR} or \rn{Cr-LB}.
