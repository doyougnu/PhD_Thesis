\label{chapter:vsmt}
%
We have covered the basics of variational satisfiability solving. In this
chapter we generalize the variational solving procedure to variational \ac{smt}
solving.
% %
\ac{smt} solvers generalize \ac{sat} solvers through the use of \emph{background
  theories} that allow the solver to reason about values and constructs outside
the Boolean domain. The SMTLIB2 standard defines seven such background theories:
\rn{Core} (Boolean theory), \rn{ArraysEx}, \rn{FixedSizeBitVectors},
\rn{FloatingPoint}, \rn{Ints}, \rn{Reals}, and \rn{Real\_Ints}. In this chapter,
we use integer arithmetic (\rn{Ints}) as an example \ac{smt} extension for
variational \ac{smt} solving. Extensions for other background theories are
similar to the \rn{Ints} extension with the exception of the array theory. The
array theory presents unique challenges due to interactions with choices; we
conclude the section by presenting the array extension thus recovering the most
popular \ac{smt} background theories in the variational solver.
% %
\begin{figure}
  \centering
  \begin{syntax}
  i & \in{} & \mathbb{Z} & \textit{Integers} \\
  t_{i} & \Coloneqq{} & r_{i} \quad|\quad i & \textit{Integer variables and literals} \\
  [1.5ex]

  ar & \Coloneqq{} & t_{i}    & \textit{Terminal} \\
  & | & -\ ar       & \textit{Arithmetic Negation} \\
  & | & ar\ -\ ar       & \textit{Subtraction} \\
  & | & ar\ +\ ar     & \textit{Addition} \\
  & | & ar\ *\ ar     & \textit{Multiplication} \\
  & | & ar\ \div\ ar     & \textit{Division} \\
  & | & \chc[D]{ar,ar} & \textit{Choice} \\
\end{syntax}
  \caption{Syntax of Integer arithmetic extension.}%
  \label{fig:arith:stx}
\end{figure}
%
\begin{figure}
\begin{syntax}
  t & \Coloneqq{} & r \quad|\quad \tru{} \quad|\quad \fls{}
  & \textit{Variables and Boolean literals} \\
  [1.5ex]
  \otimes & \Coloneqq{} & < \,\;|\;\; \leq \;\:\,|\;\; \geq \,\;|\;\;  > \,\;|\  \equiv & \emph{Binary relations}\\
  [1.5ex]


  f & \Coloneqq{} & t    & \textit{Terminal} \\
  & | & \neg{} f       & \textit{Negate} \\
  & | & f \vee{} f     & \textit{Or} \\
  & | & f \wedge{} f   & \textit{And} \\
  & | & ar \otimes{} ar  & \emph{Integer comparisons} \\
  & | & \chc[D]{f,f} & \textit{Choice} \\
\end{syntax}

\centering
\caption{Syntax of extended \ac{vpl}.}%
\label{fig:arith:vpl}
\end{figure}
%
\section{Variational propositional logic extensions and primitives}
~\label{section:vsmt:primitives}
%
\begin{figure}
  \centering
\begin{tabular}{r@{~~:~~}l@{~~\OB{\to}~~}ll}
\pnot
  & $(\aStore{},s)$
  & $(\aStore{},s)$
  & \emph{Negate a symbolic value} \\
\pand
  & $(\aStore{},s,s)$
  & $(\aStore{},s)$
  & \emph{Conjunction of symbolic values} \\
\por
  & $(\aStore{},s,s)$
  & $(\aStore{},s)$
  & \emph{Disjunction of symbolic values} \\
\pneg
  & $(\aStore{},s)$
  & $(\aStore{},s)$
  & \emph{Negate an arithmetic symbolic value} \\
\padd
  & $(\aStore{},s,s)$
  & $(\aStore{},s)$
  & \emph{Add symbolic values} \\
\psub
  & $(\aStore{},s,s)$
  & $(\aStore{},s)$
  & \emph{Subtract symbolic values} \\
\pdiv
  & $(\aStore{},s,s)$
  & $(\aStore{},s)$
  & \emph{Divide symbolic values} \\
\pmult
  & $(\aStore{},s,s)$
  & $(\aStore{},s)$
  & \emph{Multiply symbolic values} \\
\plt
  & $(\aStore{},s,s)$
  & $(\aStore{},s)$
  & \emph{Less than over symbolic values} \\
\plte
  & $(\aStore{},s,s)$
  & $(\aStore{},s)$
  & \emph{Less than equals over symbolic values} \\
\pgt
  & $(\aStore{},s,s)$
  & $(\aStore{},s)$
  & \emph{Greater than over symbolic values} \\
\pgte
  & $(\aStore{},s,s)$
  & $(\aStore{},s)$
  & \emph{Greater than equals over symbolic values} \\
\peq
  & $(\aStore{},s,s)$
  & $(\aStore{},s)$
  & \emph{Arithmetic equivalence over symbolic values} \\
\pspawn
  & $(\aStore{},r)$
  & $(\aStore{},s)$
  & \emph{Create symbolic value based on a boolean variable} \\
\pspawni
  & $(\aStore{},r_{i})$
  & $(\aStore{},s)$
  & \emph{Create symbolic value based on a arithmetic variable} \\
\passert
  & $(\eStore{},\aStore{},s)$
  & $\eStore{}$
  & \emph{Assert a symbolic value to the solver} \\
\pmodel
  & $(\eStore{},\aStore{})$
  & $m$
  & \emph{Get a model for the current solver state}
\end{tabular}

  \caption{Assumed base solver primitive operations for \evpl{}}%
  \label{fig:vsmt:primops}
\end{figure}
%
\begin{figure}
  \centering
  \begin{subfigure}[t]{\linewidth}
    \newcommand{\wrappedprimspacer}{\mbox{\hspace{1.8cm}} & \\[-3ex]}
\centering
\begin{align*}
\pwrap{\pspawni}(\aStore{},r_{i}) &=
                               \begin{cases}
                                 \wrappedprimspacer
                                 (\aStore{},s) & (r_{i},s) \in \aStore{} \\
                                 \pspawni(\aStore{},r_{i}) & \mathit{otherwise}
                               \end{cases} \\
\pwrap{\pneg}(\aStore{},s) &=
                               \begin{cases}
                                 \wrappedprimspacer
                                 (\aStore{},s') & (-s, s') \in \aStore{} \\
                                 \pneg(\aStore{},s) & \mathit{otherwise}
                               \end{cases} \\
\pwrap{\padd}(\aStore{},s_1,s_2) &=
                                 \begin{cases}
                                   \wrappedprimspacer
                                   (\aStore{},s_3) & (s_1 + s_2, s_3) \in \aStore{} \\
                                   \padd(\aStore{},s_1,s_2) & \mathit{otherwise}
                                 \end{cases} \\
\pwrap{\psub}(\aStore{},s_1,s_2) &=
                                \begin{cases}
                                  \wrappedprimspacer
                                  (\aStore{},s_3) & (s_1 - s_2, s_3) \in \aStore{} \\
                                  \psub(\aStore{},s_1,s_2) & \mathit{otherwise}
                                \end{cases} \\
\pwrap{\pdiv}(\aStore{},s_1,s_2) &=
                                 \begin{cases}
                                   \wrappedprimspacer
                                   (\aStore{},s_3) & (s_1 \div s_2, s_3) \in \aStore{} \\
                                   \pdiv(\aStore{},s_1,s_2) & \mathit{otherwise}
                                 \end{cases} \\
  \pwrap{\pmult}(\aStore{},s_1,s_2) &=
                                     \begin{cases}
                                       \wrappedprimspacer
                                       (\aStore{},s_3) & (s_1 * s_2, s_3) \in \aStore{} \\
                                       \pmult(\aStore{},s_1,s_2) & \mathit{otherwise}
                                     \end{cases}
\end{align*}

    \caption{Wrapped arithmetic primitives.}%
    \label{fig:vsmt:primops:arithmetic}
  \end{subfigure}
  \vfill
  \begin{subfigure}[t]{\linewidth}
    \newcommand{\wrappedprimspacer}{\mbox{\hspace{1.8cm}} & \\[-3ex]}
\centering
\begin{align*}
  \pwrap{\plt}(\aStore{},s_1,s_2) &=
                                     \begin{cases}
                                       \wrappedprimspacer
                                       (\aStore{},s_3) & (s_1 < s_2, s_3) \in \aStore{} \\
                                       \plt(\aStore{},s_1,s_2) & \mathit{otherwise}
                                     \end{cases} \\
  \pwrap{\plte}(\aStore{},s_1,s_2) &=
                                     \begin{cases}
                                       \wrappedprimspacer
                                       (\aStore{},s_3) & (s_1 \leq s_2, s_3) \in \aStore{} \\
                                       \plte(\aStore{},s_1,s_2) & \mathit{otherwise}
                                     \end{cases} \\
  \pwrap{\pgt}(\aStore{},s_1,s_2) &=
                                     \begin{cases}
                                       \wrappedprimspacer
                                       (\aStore{},s_3) & (s_1 > s_2, s_3) \in \aStore{} \\
                                       \pgt(\aStore{},s_1,s_2) & \mathit{otherwise}
                                     \end{cases} \\
  \pwrap{\pgte}(\aStore{},s_1,s_2) &=
                                    \begin{cases}
                                      \wrappedprimspacer
                                      (\aStore{},s_3) & (s_1 \geq s_2, s_3) \in \aStore{} \\
                                      \pgte(\aStore{},s_1,s_2) & \mathit{otherwise}
                                    \end{cases} \\
  \pwrap{\peq}(\aStore{},s_1,s_2) &=
                                     \begin{cases}
                                       \wrappedprimspacer
                                       (\aStore{},s_3) & (s_1 \equiv s_2, s_3) \in \aStore{} \\
                                       \peq(\aStore{},s_1,s_2) & \mathit{otherwise}
                                     \end{cases} \\
\end{align*}

    \caption{Wrapped inequality  primitives.}%
    \label{fig:vsmt:primops:inequality}
  \end{subfigure}
  \caption{Wrapped \ac{smt} primitives.}
\end{figure}
%
\begin{figure}
  \begin{syntax}
  % U_{\booleans{}} & : \booleans{} \rightarrow{} \booleans{} & \textit{Integers} \\
  \neg & \Coloneqq{} & \texttt{Not}    & \emph{Boolean negation} \\
  [1.5ex]
  % _{\booleans{}} & : \booleans{} \rightarrow{} \booleans{} & \textit{Integers} \\
  \integerUnary & \Coloneqq{} & \texttt{Negate} & \emph{Negation} \\
  [1.5ex]

  \boolFuncs & \Coloneqq{} & \texttt{And} & \textit{Conjunction} \\
                      &      |      & \texttt{Or}  & \textit{Disjunction} \\
  [1.5ex]

  \inequalities & \Coloneqq{} & \texttt{LT}   & \textit{Less than} \\
                      &     |       & \texttt{GT}   & \textit{Greater than} \\
                      &     |       & \texttt{LTE}  & \textit{Less than Equal} \\
                      &     |       & \texttt{GTE}  & \textit{Greater than Equal} \\
                      &     |       & \texttt{Eqv}  & \textit{Equivalency} \\
  [1.5ex]

  \integerFuncs & \Coloneqq{} & \texttt{Add}   & \textit{Addition} \\
                      &     |       & \texttt{Sub}   & \textit{Subtraction} \\
                      &     |       & \texttt{Mult}  & \textit{Multiplication} \\
                      &     |       & \texttt{Div}  & \textit{Division} \\
                      &     |       & \texttt{Mod}  & \textit{Modulus} \\
\end{syntax}

  \caption{Syntactic categories of primitive operations}%
  \label{fig:vsmt:categories}
\end{figure}

In order to construct a variational \ac{smt} solver we must first extend
\ac{vpl} to include non-Boolean values. \ac{vpl} included two kinds of
relations: relations such as $\neg$ and $\vee$ which required accumulation in
the presence of variation, and relations such as $\wedge$ which required no
special handling. Unfortunately, in the presence of variation there are no
relations such as $\wedge$ for the \ac{smt} theories. Thus we add support for
each theory except arrays through accumulation. Our strategy to extend \ac{vpl}
to \evpl{} is to add the appropriate cases to the syntax of \ac{vpl}, extend the
intermediate language, add the requisite primitive operations, and then extend
the inference rules of accumulation and choice removal.

The \evpl{} syntax is presented in \autoref{fig:arith:stx}. \evpl{} includes
syntax of the integer arithmetic extension, which consists of integer variables,
integer literals, a set of standard operators, and choices.
%
The sets of Boolean and arithmetic variables are disjoint, thus an expression
such as $(\kf{s < 10) \wedge (s \vee p})$, where $s$ occurs as both an integer
and Boolean variable is disallowed.
%
The syntax of the language prevents type errors and expressions that do not
yield Boolean values. For example, $\chc[D]{1,2} \wedge p$ is syntactically
invalid.
%
Thus, the language is purposefully limited to arithmetic expressions that
have an inequality at the root of the expression, such as: $\kf{g} =
(\chc[A]{1, 2} + j \geq 2) \vee a \wedge \chc[A]{c, d}$.
%
Choices in the same dimension are synchronized across Boolean and arithmetic
sub-expressions, for example, the expression
%
$\kf{g} = (\chc[A]{1, 2} + j \geq 2) \vee (a \wedge \chc[A]{c,d})$
% %
represents two variants:
%
$\sem[\{(A, \true)\}]{g} = (1 + j \geq 2) \vee (a \wedge c)$ and
$\sem[\{(A, \false)\}]{g} = (2 + j \geq 2) \vee (a \wedge d)$.

Similarly to \autoref{chapter:vsat}, we define the assumed primitive operations
of the base solver in \autoref{fig:vsmt:primops}, and wrapped versions for new
operators in \autoref{fig:vsmt:primops:arithmetic} and
\autoref{fig:vsmt:primops:inequality}. The wrapped versions are defined
identically as the wrapped primitives in \autoref{fig:vsat:primwrapped} and
serve the same purpose.
%
From the perspective of the variational solver, operations such as addition,
division, and subtraction only differ in the primitive operation emitted to the
base solver. Thus, we define syntactic categories over like operations in
\autoref{fig:vsmt:categories}. Notice that the categories correspond to the
respective type of each operation. For example, the boolean categories
encapsulate operations which take two boolean expressions and return a boolean
expression, similarly the inequality category encapsulate operators which take
numeric expressions and return boolean expressions. Further \ac{smt} extensions
would directly copy this pattern, that is, defining a syntactic category of
\rn{FixedSizeBitVector} or \rn{Reals} operators. Similarly, while we present
only a single arithmetic unary function $-$, other arithmetic unary functions
would be straightforward to add. For example, to include a modulus operator
$\kf{mod}$, one would define the wrapped primitive, and add the operator to the
appropriate syntactic category without requiring any modification to the
inference rules or intermediate languages.

Just as \ac{vpl} was extended the intermediate language must be extended. First
we must add cases for inequality operations, and second we must define an
intermediate language for the arithmetic domain.
%
\autoref{fig:vsmt:il} defines the intermediate arithmetic language \eAR, and the
extended intermediate language \eIL. The syntax of both intermediate languages
follow directly from \evpl and should be unsurprising. The only important
difference from IL is that \eAR cannot express a \unit{} value. This is a
purposeful design decision; recall that a \unit{} represents a term that has
been sent to the base solver. Thus if \unit{} were in \eAR then expressions such
as $\unit{} + 2$ would be expressible in \eAR, however because all arithmetic
formula's require accumulation the only possible result of
evaluation/accumulation on arithmetic expressions is either a choice or a
symbolic term, not a \unit{}. Hence, we syntactically avoid classes bugs by
eliding the \unit{} value in \eAR.
% 
\begin{figure}
  \[
  \eIL \hquad \Coloneqq\hquad \unit{}
  \hquad|\hquad t
  \hquad|\hquad r
  \hquad|\hquad s
  \hquad|\hquad \neg \eIL
  \hquad|\hquad \eIL \boolFuncs{} \eIL
  \hquad|\hquad \eAR \inequalities{} \eAR 
  \hquad|\hquad \chc[D]{e,e}
\]
\[
  \eAR \hquad \Coloneqq\hquad i
  \hquad|\hquad r_{i}
  \hquad|\hquad s
  \hquad|\hquad \integerUnary \eAR
  \hquad|\hquad \eAR \integerFuncs \eAR
  \hquad|\hquad \chc[D]{ar,ar}
\]



% \begin{syntax}
%   \eIL & \Coloneqq{} & \unit{} & | & t   & | & r  & | & s      & | & \neg{} \eIL
%   & | & \eIL \boolFuncs{} \eIL & | & ar \inequalities ar & | & \chc[D]{f,f} \\
% \end{syntax}
% [1.5ex]
% \eAR & \Coloneqq{} & i       & | & r_{i} & | & s  & | & - \eAR & | & \eAR \integerFuncs{} \eAR & | & \chc[D]{f,f} 
  \caption{Extended intermediate language syntax}%
  \label{fig:vsmt:il}
\end{figure}

%%% Local Variables:
%%% mode: latex
%%% TeX-master: "../../thesis"
%%% End:%

\section{Accumulation}
~\label{section:vsmt:accumulation}
%
\begin{figure}
  \begin{mathpar}
\inferrule*[right=\acRef]
  { \pwrap{\pspawn}(\aStore{},r) = (\aStore{}', s) }
  { (\aStore{}, r) \accumulation (\aStore{}', s) }

  \inferrule*[right=\acRefI]
  { \pwrap{\pspawni}(\aStore{},r_{i}) = (\aStore{}', s) }
  { (\aStore{}, r_{i}) \accumulation (\aStore{}', s) }

\inferrule*[right=\acNotS]
  { (\aStore{}, \eIL) \accumulation (\aStore{}', s) \\
    \pwrap{\pnot}(\aStore{}', s) = (\aStore{}'', s') }
  { (\aStore{}, \neg{} \eIL) \accumulation (\aStore{}'', s') }

  \inferrule*[right=\acUnaryS]
  { \op \in \integerUnary{} \\
    \lookup{\op}{\integerUnary{}} \coloneqq \pwrap{\op} \\
    (\aStore{}, \eAR) \accumulation (\aStore{}', s) \\
    \pwrap{\op}(\aStore{}', s) = (\aStore{}'', s') }
  { (\aStore{}, \op\; \eAR) \accumulation (\aStore{}'', s') }

\inferrule*[right=\acBoolS]
 { \op \in \boolFuncs{} \\
   \lookup{\op}{\boolFuncs{}} \coloneqq \pwrap{\op} \\
  (\aStore{}, \eIL*{1}) \accumulation (\aStore{1}, s_1) \\
    (\aStore{1}, \eIL*{2}) \accumulation (\aStore{2}, s_2) \\
    \pwrap{\op}(\aStore{2}, s_1, s_2) = (\aStore{3}, s_3) }
  { (\aStore{}, \eIL*{1}\; \op\; \eIL*{2}) \accumulation{} (\aStore{3}, s_3) }

  \inferrule*[right=\acArithS]
  { \op \in \integerFuncs{} \\
    \lookup{\op}{\integerFuncs{}} \coloneqq \pwrap{\op} \\
    (\aStore{}, \eAR*{1}) \accumulation (\aStore{1}, s_1) \\
    (\aStore{1}, \eAR*{2}) \accumulation (\aStore{2}, s_2) \\
    \pwrap{\op}(\aStore{2}, s_1, s_2) = (\aStore{3}, s_3) }
  { (\aStore{}, \eAR*{1}\; \op\; \eAR*{2}) \accumulation{} (\aStore{3}, s_3) }

  \inferrule*[right=\acInEqS]
  { \op \in\ \inequalities{} \\
    \lookup{\op}{\inequalities} \coloneqq \pwrap{\op} \\
    (\aStore{}, \eAR*{1}) \accumulation (\aStore{1}, s_1) \\
    (\aStore{1}, \eAR*{2}) \accumulation (\aStore{2}, s_2) \\
    \pwrap{\op}(\aStore{2}, s_1, s_2) = (\aStore{3}, s_3) }
  { (\aStore{}, \eAR*{1}\; \op\; \eAR*{2}) \accumulation{} (\aStore{3}, s_3) }

\inferrule*[right=\acChc]
  { }
  {(\aStore,\ \chc[D]{e_1,e_2}) \accumulation (\aStore{},\chc[D]{e_1,e_2})}

\inferrule*[right=\acChcI]
  { }
  {(\aStore,\ \chc[D]{ar_{1},ar_{2}}) \accumulation (\aStore{},\chc[D]{ar_{1},ar_{2}})}

\inferrule*[right=\acNotV]
  { (\aStore{}, \eIL) \accumulation (\aStore{}', \eIL') }
  { (\aStore{}, \neg{} \eIL) \accumulation (\aStore{}', \neg \eIL') }

  \inferrule*[right=\acUnaryV]
  { (\aStore{}, \eIL) \accumulation (\aStore{}', \eIL') }
  { (\aStore{}, \integerUnary \eIL) \accumulation (\aStore{}', \integerUnary \eIL') }

\inferrule*[right=\acBoolV]
  { (\aStore{}, \eIL_1) \accumulation (\aStore{1}, \eIL*{1}') \\
    (\aStore{1}, \eIL_2) \accumulation (\aStore{2}, \eIL*{2}') }
  { (\aStore{}, \eIL_1 \boolFuncs{} \eIL_2) \accumulation{} (\aStore{2}, \eIL*{1}' \boolFuncs{} \eIL*{2}') }

  \inferrule*[right=\acArithV]
  { (\aStore{}, \eAR*{1}) \accumulation (\aStore{1}, \eAR*{1}') \\
    (\aStore{1}, \eAR*{2}) \accumulation (\aStore{2}, \eAR*{2}') }
  { (\aStore{}, \eAR*{1} \integerFuncs{} \eAR*{2}) \accumulation{} (\aStore{2}, \eAR*{1}' \integerFuncs{} \eAR*{2}') }

  \inferrule*[right=\acInEqV]
  { (\aStore{}, \eAR*{1}) \accumulation (\aStore{1}, \eAR*{1}') \\
    (\aStore{1}, \eAR*{2}) \accumulation (\aStore{2}, \eAR*{2}') }
  { (\aStore{}, \eAR*{1} \inequalities{} \eAR*{2}) \accumulation{} (\aStore{2}, \eAR*{1}' \inequalities{} \eAR*{2}') }

\end{mathpar}

  \caption{Accumulation inference rules}%
  \label{fig:vsmt:inf:acc}
\end{figure}
%
%
With primitive operations and metavariables defined we specify accumulation in
\autoref{fig:vsmt:inf:acc}. Since the metavariable $\kf{s}$ has two meanings:
sub-trees of \rn{IL} in the variational \ac{smt} domain and sequences of
clauses in the base solver, we treat it as overloaded.
%
An implementation of the rules would require a store which maps symbolics at
the variational level to terms or sequences at the base solver level.

Accumulation is represented as a binary relation with \accumulation{}. The rules
follow a simple pattern: \rn{Ac-Chc} skips any choices, \rn{Ac-Gen} and
\rn{Ac-Geni} provide a method to inject references into the symbolic domain,
\rn{Ac-Ref} and \rn{Ac-Refi} cache references to ensure the same
reference is mapped to the same symbolic, and the rest of the rules provide
operations on symbolic terms, \eg{}, \rn{Ac-SBinB}, or are congruence rules
such as \rn{Ac-BinI}. We elide rules which process formulas composed of
constants such $\kf{(\tru{} \wedge \fls{})}$ or $(1 + 2 + 3)$. In cases such as
$1 + 2 < \kf{i}$, constants are reduced and treated as references, thus this
formula becomes $3 < \kf{i}$ and is accumulated to $s_{3} < \kf{i}$.

Accumulation maintains a store, \aStore{}, to track and cache symbolic terms.
For example, given formula such as: $g = a \wedge (a \wedge b)$, \rn{Ac-Gen}
will spawn only two new references, one for $\kf{a}$ and one for $\kf{b}$, and
\rn{Ac-Ref} ensures the same symbolic will represent the $a$ reference. This
will produce $g = s_{a} \wedge (s_{a} \wedge s_{b})$, because we $g$ contains
two boolean connective \rn{Ac-BinB} will be called twice beginning with the
inner conjunction. \rn{Ac-BinB} will combine $s_{a}$ and $s_{b}$ into a new
symbolic $s_{ab}$, update the store to \aStore{}'. The new store will include
entries for both references \textit{and} symbolic references, thus, in this
example \aStore{}' contains $\kf{a} \rightarrow s_{a}$, $\kf{b} \rightarrow
s_{b}$, and $\kf{s_{ab}} \rightarrow (s_{a} \wedge s_{b})$. Finally
\rn{Ac-BinB} will repeat the last procedure on the outermost conjunction
adding a new entry to the symbolic store.


%%% Local Variables:
%%% mode: latex
%%% TeX-master: "../../thesis"
%%% End:%

\section{Evaluation}
~\label{section:vsmt:evaluation}
%
\begin{figure}
  \begin{mathpar}
  %%% Computation rules
  \inferrule*[Right=Ev-Tm]
  { \texttt{Assert}((\eStore{}, \aStore{}), $t$) = \eStore{}' }
  %%% --------------------------------------------------------------
  {(\eStore{}, \aStore{}, t) \evaluation{} (\eStore{}', \aStore{}, \unit{}) }


  \inferrule*[Right=Ev-Sym]
  { \texttt{Assert}((\eStore{}, \aStore{}), s) = \eStore{}' }
  %%% --------------------------------------------------------------
  { (\eStore{}, \aStore{}, s) \evaluation{} (\eStore{}, \Delta, \unit{}) }
\\

\inferrule*[Right=Ev-Model]
{ \texttt{GetModel}(\aStore{}, \eStore{}) = m }
%%% --------------------------------------------------------------
{ (\aStore{}, \eStore{}, \unit{}) \evaluation{} m }
\\

\inferrule*[Right=Ev-Chc]
{ }
%%% --------------------------------------------------------------
{(\aStore{}, \eStore{}, \chc[D]{e_{1},e_{2}}) \evaluation{} (\aStore{}, \eStore{}, \chc[D]{e_{1},e_{2}})}
\end{mathpar}

\begin{mathpar}
  %%%\unit{} elimination rules
  \inferrule*[Right=Ev-UL]
  { \boolFuncs{} = \texttt{And}  }
  %%% --------------------------------------------------------------
  { (\Theta,\unit{} \boolFuncs{} v) \evaluation{} (\Theta, v) }


  \inferrule*[Right=Ev-UR]
  { \boolFuncs{} = \texttt{And} }
  %%% --------------------------------------------------------------
  { (\Theta, v \boolFuncs{} \unit{}) \evaluation{} (\Theta, v)}
\end{mathpar}

\begin{mathpar}
  \inferrule*[Right=Ev-BU]
  { (\aStore{},\neg{} v) \accumulation{} (\aStore{}', v') }
  %%% --------------------------------------------------------------
  { (\eStore{},\aStore{}, \neg{} v) \evaluation{} (\eStore{},\aStore{}', v') }


  \inferrule*[Right=Ev-IU]
  { (\aStore{}, \integerUnary{} v) \accumulation{} (\aStore{}', v') }
  %%% --------------------------------------------------------------
  { (\eStore{},\aStore{}, \integerUnary{} v) \evaluation{} (\eStore{},\aStore{}', v') }
\end{mathpar}

\begin{mathpar}
  \inferrule*[Right=Ev-And]
  { \boolFuncs{} = \texttt{And} \\
    (\Theta, v_{1}) \evaluation{} (\Theta', v_{1}') \\
    (\Theta', v_{2}) \evaluation{} (\Theta'', v_{2}')
  }
  %%% --------------------------------------------------------------
  { (\Theta, v_{1} \boolFuncs{} v_{2}) \evaluation{} (\Theta'', v_{1}' \boolFuncs{} v_{2}') }


    \inferrule*[Right=Ev-AccB]
    { \boolFuncs \neq \texttt{And} \\
      (\aStore{}, v_{1}) \accumulation{} (\aStore{}', v_{1}') \\
      (\aStore{}',v_{2}) \accumulation{} (\aStore{}'', v_{2}')
    }
%%% --------------------------------------------------------------
    { (\Gamma, \Delta, v_{1} \boolFuncs{} v_{2}) \evaluation{} (\Gamma, \Delta'', v_{1}'
      \boolFuncs{} v_{2}')}
\end{mathpar}%

\begin{mathpar}
  \inferrule*[Right=Ev-AccIB]
  {
    (\aStore{}, v_{1}) \accumulation{} (\aStore{}', v_{1}') \\
    (\aStore{}',v_{2}) \accumulation{} (\aStore{}'', v_{2}')
  }
  %%% --------------------------------------------------------------
  { (\Gamma, \Delta, v_{1} \inequalities{} v_{2}) \evaluation{} (\Gamma, \Delta'', v_{1}' \inequalities{} v_{2}')}
\end{mathpar}%
  \caption{Evaluation inference rules}%
  \label{fig:vsmt:inf:eval}
\end{figure}
%
\todo{make into subsection or paragraph? Only 1 page long...}Evaluation's
purpose is to assert symbolic terms in the base solver if it is safe to do so.
Thus, the extensions to evaluation are minimal as accumulation is performing the
majority of the work in creating the symbolic terms.

Variational \ac{smt} evaluation is defined in \autoref{fig:vsmt:inf:eval}. The
only change is the addition of \evInEq{} corresponding to the addition of
inequalities to \ac{vpl}. Just as \evOr{} skips over un-accumulated
disjunctions, \evInEq{} skips over un-accumulated inequalities as evaluating
inside an inequality is unsound in the base solver. Since evaluation calls
\evAcc{} to accumulate relations if \evAndL, \evAndR, and \evAnd don't apply,
variational \ac{smt} evaluation simply relies on accumulation to progress. The
special cases for conjunctions are maintained in order to sequence evaluation
from left to right to take advantage of the behavior of the assertion stack and
to propagate accumulation across conjunctions, just as in variational
satisfiability evaluation.


%%% Local Variables:
%%% mode: latex
%%% TeX-master: "../../thesis"
%%% End:%

\section{Choice removal}
\label{section:vsmt:choice-removal}
%
\begin{figure}
  % \[
%   \zipper \hquad\Coloneqq\hquad \inRoot
%   \hquad|\hquad \inNot{\zipper}
%   \hquad|\hquad \inUnary{\zipper}
%   \hquad|\hquad \inBoolL{\zipper}{\,\eIL}
%   \hquad|\hquad \inBoolR{s}{\zipper}
%   \hquad|\hquad \inArithL{\zipper}{\,\eIL}
%   \hquad|\hquad \inArithR{s}{\zipper}
%   \hquad|\hquad \inInEqL{\zipper}{\,\eIL}
%   \hquad|\hquad \inInEqR{s}{\zipper}
% \]
\begin{syntax}
  \zipper & \Coloneqq{} & \inRoot  \\
  & | & \inNot{\zipper}            \\
  & | & \inUnary{\zipper}          \\
  & | & \inBoolL{\zipper}{\,\eIL}   \\
  & | & \inBoolR{s}{\zipper}       \\
  & | & \inArithL{\zipper}{\,\eIL}  \\
  & | & \inArithR{s}{\zipper}      \\
  & | & \inInEqL{\zipper}{\,\eIL}   \\
  & | & \inInEqR{s}{\zipper}       \\
\end{syntax}
  \caption{Variational \ac{smt} zipper context}
  \label{fig:vsmt:zipper}
\end{figure}
%
\begin{figure}
  \begin{mathpar}
\inferrule*[right=\crEval]
  { (\eStore{},\aStore{},v) \evaluation (\eStore{}',\aStore{}',\unit) \\
    \texttt{Combine}(\vmodel{},\pmodel(\aStore{},\eStore{})) = \vmodel{}' }
  { (\crCtx, \inRoot, v) \choiceRemoval \vmodel{}' }

\inferrule*[right=\crChcT]
  { (D,\true)\in C \\
    (\crCtx, z, e_1) \choiceRemoval \vmodel{}' }
  { (\crCtx, z, \chc[D]{e_1,e_2} \choiceRemoval \vmodel{}' }

\inferrule*[right=\crChcF]
  { (D,\false)\in C \\
    (\crCtx, z, e_2) \choiceRemoval \vmodel{}' }
  { (\crCtx, z, \chc[D]{e_1,e_2} \choiceRemoval \vmodel{}' }

\inferrule*[right=\crChc]
  { D\notin\dom{C} \\
    (C\cup(D,\true),\eStore{},\aStore{},\vmodel{}, z, e_1)
      \choiceRemoval \vmodel{1} \\
    (C\cup(D,\false),\eStore{},\aStore{},\vmodel{}', z, e_2)
      \choiceRemoval \vmodel{2} }
  { (\crCtx, z, \chc[D]{e_1,e_2} \choiceRemoval \vmodel{2} }

\inferrule*[right=\crNot]
  { (\crCtx, \inNot{z}, v) \choiceRemoval \vmodel{}' }
  { (\crCtx, z, \neg v) \choiceRemoval \vmodel{}' }

\inferrule*[right=\crNotIn]
  { (\aStore{}, \neg s) \accumulation (\aStore{}', s') \\
    (\crCtx, z, s') \choiceRemoval \vmodel{}' }
  { (\crCtx, \inNot{z}, s) \choiceRemoval \vmodel{}' }

\inferrule*[right=\crAnd]
  { (\crCtx, \inAndL{z}{v_2}, v_2) \choiceRemoval \vmodel{}' }
  { (\crCtx, z, v_1 \wedge v_2) \choiceRemoval \vmodel{}' }

\inferrule*[right=\crAndL]
  { (\crCtx, \inAndR{s}{z}, v) \choiceRemoval \vmodel{}' }
  { (\crCtx, \inAndL{z}{v}, s) \choiceRemoval \vmodel{}' }

\inferrule*[right=\crAndR]
  { (\aStore{}, s_1 \wedge s_2) \accumulation (\aStore{}', s_3) \\
    (\crCtx, z, s_3) \choiceRemoval \vmodel{}' }
  { (\crCtx, \inAndR{s_1}{z}, s_2) \choiceRemoval \vmodel{}' }

\inferrule*[right=\crOr]
  { (\crCtx, \inOrL{z}{v_2}, v_2) \choiceRemoval \vmodel{}' }
  { (\crCtx, z, v_1 \vee v_2) \choiceRemoval \vmodel{}' }

\inferrule*[right=\crOrL]
  { (\crCtx, \inOrR{s}{z}, v) \choiceRemoval \vmodel{}' }
  { (\crCtx, \inOrL{z}{v}, s) \choiceRemoval \vmodel{}' }

\inferrule*[right=\crOrR]
  { (\aStore{}, s_1 \vee s_2) \accumulation (\aStore{}', s_3) \\
    (\crCtx, z, s_3) \choiceRemoval \vmodel{}' }
  { (\crCtx, \inOrR{s_1}{z}, s_2) \choiceRemoval \vmodel{}' }
\end{mathpar}

  \caption{Variational \ac{smt} choice removal inference rules}%
  \label{fig:vsmt:chc}
\end{figure}
%
\begin{figure}
  \ContinuedFloat
  \begin{mathpar}
  \inferrule*[right=\crUnary]
  { (\crCtx, \inUnary{\zipper}, \eIL) \choiceRemoval \vmodel{}' }
  { (\crCtx, \zipper, \integerUnary \eIL) \choiceRemoval \vmodel{}' }

  \inferrule*[right=\crUnaryIn]
  { (\aStore{}, \integerUnary s) \accumulation (\aStore{}', s') \\
    (\crCtx, \zipper, s') \choiceRemoval \vmodel{}' }
  { (\crCtx, \inUnary{\zipper}, s) \choiceRemoval \vmodel{}' }

\inferrule*[right=\crInEq]
  { (\crCtx, \inInEqL{z}{\eIL*{2}}, \eIL*{1}) \choiceRemoval \vmodel{}' }
  { (\crCtx, \zipper, \eIL*{1} \inequalities{} \eIL*{2}) \choiceRemoval \vmodel{}' }

\inferrule*[right=\crInEqL]
  { (\crCtx, \inInEqR{s}{\zipper}, \eIL) \choiceRemoval \vmodel{}' }
  { (\crCtx, \inInEqL{\zipper}{\eIL}, s) \choiceRemoval \vmodel{}' }

\inferrule*[right=\crInEqR]
  { (\aStore{}, s_1 \inequalities{} s_2) \accumulation (\aStore{}', s_3) \\
    (\crCtx, \zipper, s_3) \choiceRemoval \vmodel{}' }
  { (\crCtx, \inInEqR{s_1}{\zipper}, s_2) \choiceRemoval \vmodel{}' }

\inferrule*[right=\crArith]
  { (\crCtx, \inArithL{z}{\eIL*{2}}, \eIL*{1}) \choiceRemoval \vmodel{}' }
  { (\crCtx, \zipper, \eIL*{1} \integerFuncs{} \eIL*{2}) \choiceRemoval \vmodel{}' }

\inferrule*[right=\crArithL]
  { (\crCtx, \inArithR{s}{\zipper}, \eIL) \choiceRemoval \vmodel{}' }
  { (\crCtx, \inArithL{\zipper}{\eIL}, s) \choiceRemoval \vmodel{}' }

\inferrule*[right=\crArithR]
  { (\aStore{}, s_1 \integerFuncs{} s_2) \accumulation (\aStore{}', s_3) \\
    (\crCtx, \zipper, s_3) \choiceRemoval \vmodel{}' }
  { (\crCtx, \inArithR{s_1}{\zipper}, s_2) \choiceRemoval \vmodel{}' }
\end{mathpar}

  \caption{Variational \ac{smt} choice removal inference rules}
  \label{fig:vsmt:chc-cont}
\end{figure}
%
With accumulation and evaluation complete we turn to choice removal.
%
Our strategy is similar to accumulation; we generalize the zipper context over
Boolean, arithmetic and inequality relations using the syntactic categories
defined in \autoref{fig:vsmt:categories}.
%
Conceptually, choice removal remains a variational left fold that builds the
zipper context until a symbolic value is the focus. at which point rules of the
form \rn{C-*-InL} \emph{switch} the fold to process the right child of the
relations, and rules of the form \rn{C-*-InR} call accumulation to reduce the
relation over symbolic values. We formally specify generalized choice removal in
\autoref{fig:vsmt:chc}. The heart of choice removal remains the same, the rules
\crEval{}, \crChc{}, \crChcT{}, and \crChcF{} are reproduced for the new zipper
context \zipper{} but are semantically identical to the specialized versions. The
remaining rules are generalized versions of the \ac{sat} rules to handle each
syntactic category the variational solver can process.

For each syntactic category we define three kinds of rules which form a template
to extend choice removal to new background theories: First, we have rules which
determine what to do when encountering a binary or unary relation for the first
time. As a design choice this is defined to proceed into the left child. For
example \crNot{}, \crBool{}, and \crInEq{} initiate the left fold by storing the
relation in the zipper and focusing the left child $\eIL*{1}$.
%
Second, we define rules which recur down the left child of the relations until a
symbolic value results from accumulation. For example, \crInEqL{}, \crBoolL{},
and \crArithL{} all move the focused symbol value $s$ to the zipper context
allowing choice removal to proceed to the right children of the same relation.
%
Lastly, we have computation rules which perform the fold on the relation by
calling accumulation. For example, \crUnaryIn{}, \crArithR{}, and \crInEqR{}
call accumulation to process the symbolic value and reduce the given relation.
In effect, accumulation \emph{encapsulates} the semantics of the relations,
evaluation propagates accumulation and performs code generation in the base
solver, and choice removal alters the configuration, maintains evaluation
contexts, and removes choices introducing new plain terms to the formula.

%%% Local Variables:
%%% mode: latex
%%% TeX-master: "../../thesis"
%%% End:%

\section{Variational \ac{smt} models}
~\label{section:vsmt:models}
% %
\begin{figure}[h]
    \centering
    \begin{subfigure}[t]{\textwidth}
  \begin{tabbing}
    \qquad \quad \= $\iV{} \rightarrow$ -1 \\
    \> $\cV{} \rightarrow$ 0 \\
    \\
    \\
    \quad $C_{FF}$ = \{(\AV{}, \false{}), (\BV{}, \false{})\} \\
  \end{tabbing}
\end{subfigure}%
\begin{subfigure}[t]{\textwidth}
  \begin{tabbing}
    \\
    \\
    \qquad \quad \= $\aV{} \rightarrow$ \tru{} \\
    \\
    \quad $C_{FF}$ = \{(\AV{}, \fls{}), (\BV{}, \tru{})\} \\
  \end{tabbing}
\end{subfigure}%
\newline
\begin{subfigure}[t]{\textwidth}
  \begin{tabbing}
    \qquad \quad \= $\iV{} \rightarrow$ 0 \\
    \> $\cV{} \rightarrow$ 1 \\
    \\
    \\
    \quad $C_{FT}$ = \{(\AV{}, \tru{}), (\BV{}, \fls{})\} \\
  \end{tabbing}
\end{subfigure}%
\begin{subfigure}[t]{\textwidth}
  \begin{tabbing}
    \qquad \quad \= $\iV{} \rightarrow$ 0 \\
    \> $\cV{} \rightarrow$ 0 \\
    \\
    \> $\bV{} \rightarrow$ -10 \\
    \quad $C_{TT}$ = \{(\AV{}, \tru{}), (\BV{}, \tru{})\} \\
  \end{tabbing}
\end{subfigure}%%
    \caption{Possible plain models for variants of $\kf{f}$.}%
    \label{fig:vsmt:models:plain}
\end{figure}
% %
We have thus far covered accumulation, evaluation, and choice removal. However,
to support \ac{smt} theories, variational models must be abstract enough to
handle values other than Booleans. Functionally, variational \ac{smt} models
must satisfy several constraints: the variational \ac{smt} model must be more
memory efficient than storing all models returned by the solver naively. The
variational model must allow users to find satisfying values for a variant. The
model must allow users to find all variants in which a variable has a particular
value or range of values.

Furthermore, several useful properties of variational models should be
maintained: The model is non-variational; the user should not need to understand
the choice calculus to understand their results. The model produces results that
can be fed into a plain \ac{sat} or \ac{smt} solver. The model can be built
incrementally and without regard to the ordering of results. Variational
\ac{sat} models guaranteed this last constraint by forming commutative monoid
under $\vee$; a technique which we cannot replicate for variational \ac{smt}
models.

To maintain these properties and satisfy the functional requirements, our
strategy for variational \ac{smt} models is to create a mapping of variables to
\ac{smt} expressions. By virtue of this strategy, variables are disallowed from
changing types across variants and hence disallowed from changing type as the
result of a choice. For any variable in the model, we assume the type returned
by the base solver is correct, and store the satisfying value in a linked list
constructed \emph{if-statements}\footnote{Also called a church-encoded list}.
Specifically, we utilize the function $ite : \mathbb{B} \rightarrow T
\rightarrow T$ from the SMTLIB2 standard to construct the list. All variables
are initialized as undefined (\undefined) until a value is returned from the
base solver for a variant. To ensure the correct value of a variable corresponds
to the appropriate variant, we translate the configuration which determines the
variant to a variation context, and place the appropriate value in the
\emph{then} branch.

Consider the following variational \ac{smt} problem extended with an integer
arithmetic theory: $f = ((\chc[A]{\iV{}, 13} - \cV{}) < (\bV + 10)) \rightarrow
\chc[B]{\aV{}, \cV{} > \iV{}}$. \fV{} contains two unique choices, $\kf{A}$,
$\kf{B}$, and thus represents four variants. In this case, the expression is
under-constrained and so each variant will be found satisfiable.
% 
\begin{figure}[h]
  \centering
  \begin{subfigure}[t]{\textwidth}
  \begin{tabbing}
    \qquad \quad \= $\iV{} \rightarrow$ -1 \\
    \> $\cV{} \rightarrow$ 0 \\
    \\
    \\
    \quad $C_{FF}$ = \{(\AV{}, \fls{}), (\BV{}, \fls{})\} \\
  \end{tabbing}
\end{subfigure}%
\begin{subfigure}[t]{\textwidth}
  \begin{tabbing}
    \\
    \\
    \qquad \quad \= $\aV{} \rightarrow$ \tru{} \\
    \\
    \quad $C_{FF}$ = \{(\AV{}, \fls{}), (\BV{}, \tru{})\} \\
  \end{tabbing}
\end{subfigure}%
\begin{subfigure}[t]{\textwidth}
  \begin{tabbing}
    \qquad \quad \= $\iV{} \rightarrow$ 0 \\
    \> $\cV{} \rightarrow$ 1 \\
    \\
    \\
    \quad $C_{FT}$ = \{(\AV{}, \tru{}), (\BV{}, \fls{})\} \\
  \end{tabbing}
\end{subfigure}%
\begin{subfigure}[t]{\textwidth}
  \begin{tabbing}
    \qquad \quad \= $\iV{} \rightarrow$ 0 \\
    \> $\cV{} \rightarrow$ 0 \\
    \\
    \> $\bV{} \rightarrow$ -10 \\
    \quad $C_{TT}$ = \{(\AV{}, \tru{}), (\BV{}, \tru{})\} \\
  \end{tabbing}
\end{subfigure}%
  \caption{Possible plain models for variants of $\kf{f}$.}%
  \label{fig:vsmt:models:plain}
\end{figure}
\begin{figure}[h]
  \centering
  \begin{subfigure}[t]{\textwidth}
  \begin{tabbing}
  \qquad \qquad \= $\_Sat \rightarrow (\neg \AV{} \wedge \neg \BV{}) \vee (\neg \AV{} \wedge \BV{}) \vee (\AV{} \wedge \neg \BV{}) \vee (\AV{} \wedge \BV)$ \\
  \> \iV{}\quad\hspace{1.7ex}\=$\rightarrow$ ($\kf{ite}$ ($\AV{} \wedge \BV{}$)\; \=0 \\
  \> \> \> ($\kf{ite}$\; ($\AV{} \wedge \neg \BV{}$)\; \=0 \\
  \> \> \> \> ($\kf{ite}$\; ($\neg \AV{} \wedge \neg \BV{}$)\; -1 $\kf{Undefined}$))) \\

  \> \cV{}\quad\hspace{1.7ex}\=$\rightarrow$ ($\kf{ite}$ ($\AV{} \wedge \BV{}$)\; \=0 \\
  \> \> \> ($\kf{ite}$\; ($\AV{} \wedge \neg \BV{}$)\; \=1 \\
  \> \> \> \> ($\kf{ite}$\; ($\neg \AV{} \wedge \neg \BV{}$)\; 0 $\kf{Undefined}$))) \\

  \> \aV{}\quad\hspace{1.7ex}\=$\rightarrow$ ($\kf{ite}$ ($\neg \AV{} \wedge \BV{}$)\; \tru{} $\kf{Undefined}$)\\
  \> \bV{}\quad\hspace{1.7ex}\=$\rightarrow$ ($\kf{ite}$ ($\AV{} \wedge \BV{}$)\; -10 $\kf{Undefined}$)
\end{tabbing}
\end{subfigure}

  \caption{Variational model corresponding to the plain models in
    \autoref{fig:vsmt:models:plain}.}%
  \label{fig:vsmt:models:var}
\end{figure}

\autoref{fig:vsmt:models:plain} show possible plain models for $\kf{f}$ with the
corresponding variational \ac{smt} model presented in
\autoref{fig:vsmt:models:var}. We've added line breaks to emphasize the
$\kf{then}$ and $\kf{else}$ branches of the $\kf{ite}$ SMTLIB2 primitive. 

This formulation maintains the functional requirements and desirable properties
of the variational \ac{sat} models. The variable $\_Sat$ is used to track the
variants that were found satisfiable, just as in the variational \ac{sat}
solver. In this case, all variants are satisfiable and thus we have four clauses
over dimensions in disjunctive normal form. If a user has a configuration then
they only need to perform substitution to determine the value of a variable
under that configuration. For example, if the user were interested in the value
of \iV{} in the $\{(\AV{}, \tru{}), (\BV{}, \tru{})\}$ variant they would
substitute the configuration into the result for \iV{} and recover $2$ from the
first $\kf{ite}$ case. To find the variants at which a variable has a value, a
user may employ a \ac{smt} solver, add the entry for \iV{} as a constraint, and
query for a model.
%
This specification of variational \ac{smt} models does not require knowledge of
choice calculus or variation, it is still monoidal---although not a commutative
monoid---and can be built in any order as long as there are no duplicate
variants; a scenario that is impossible by the property of synchronization on
choices.


However, there are some notable differences. Where variational \ac{sat} models
clearly compressed results by preventing duplicate values with constant
variables, the variational \ac{smt} model allows for duplicate values, if those
values are produced our of order. For example, both models for \iV{} and \cV{}
contain duplicate values. The \iV{} model has duplicate $-1$ and the \cV{} model
contains duplicate $0$. However only one: \cV{} is easy to check in \bigOof{1}
time; each call to \rn{Combine} could check the last immediate value to prevent
duplicate branches. In contrast, the duplicate $-1$'s for \iV{} occur in
variants that are likely to occur with several other plain models between them,
namely the models for the $C_{\tru\fls}$ and $C_{\fls\fls}$ variants. Hence, a
check during \rn{Combine} would require \bigOof{n} time, where $n$ is the number
of satisfiable variants that \iV{} occurs in. While such a case is easily
avoided in an implementation by tracking the values a variable has been
previously assigned, we provide only a minimum specification and thus leave the
details to an implementation.
%
Lastly, the use of \undefined{} may seem unattractive. While all bindings in the
model end with an \undefined{}, a binding cannot result in an \undefined{} as
that would imply a variant that was found to be satisfiable but was not
satisfiable, and hence would be indicative of a bug in the variational solver
implementation. Thus mathematically inclined readers may observe that the monoid
variational \ac{smt} models form corresponds to the free monoid formed by
church-encoded lists.

%%% Local Variables:
%%% mode: latex
%%% TeX-master: "../../thesis"
%%% End:%

\section{A complete Variational \ac{smt} example}
~\label{section:vsmt:example}
%
With variational \ac{smt} solving formally specified. We present a complete
example of solving a variational \ac{smt} problem. Consider the query formula
\newline$\kf{\hV = ((1 + 2 < (i - \chc[A]{k,l})) \wedge a) \wedge
  (\chc[B]{c,\neg b} \vee b)}$ with two choices parameterized by the dimensions
$A$ and $B$. Derivation of the variational core for $\kf{h}$ begins with all
evaluation and all stores \aStore{}, \eStore{} initialized to $\emptyset$.

The root of \hV{} is $\wedge$ and thus \evAnd{} is the only applicable rule.
From \evAnd{} we have $\eIL*{1} = \kf{((1 + 2 < (i - \chc[A]{k,l})) \wedge a)}$,
and $\eIL*{2} = (\chc[B]{c,\neg b} \vee b)$. We traverse \eIL*{1} first, leading
to a recursive application of \evAnd{}. We denote recursive levels with a tick
mark: $'$, thus $\kf{\eIL*{1}' = (1 + 2 < (i - \chc[A]{k,l}))}$ is the left
child of \eIL*{1}, with $\kf{\eIL*{2}' = a}$ as the right child.

The root of $\eIL*{1}'$ is an inequality, so the only way to progress is to try
to accumulate $\eIL*{1}'$. The accumulation will succeed; in accumulation, only
\acInEqV{} can apply as accumulation will be unable to transform the right child
of $\eIL*{1}'$ to a symbolic value due to the presence of a choice. \acInEqV{}
will further destruct $\eIL*{1}'$ to $\eAR*{1} = 1 + 2$ and $\eAR*{2} = i -
\chc[A]{k,l}$. $\eAR*{1}$ will be accumulated to a single symbolic value by
application of \acArithS{} and \acRef{} on the literals $1$ and $2$ yielding
$\eAR*{1} = s_{12}$, with store $\aStore{}=\set{(s_{1} + s_{2}, s_{12}),
  (2,s_{2}), (1,s_{1})}$.

Using the resultant store from accumulating $\eAR*{1}$, accumulation on
$\eAR*{2}$ will yield the term $s_{i} - \chc[A]{k,l}$. The variable $i$ will be
accumulated to a symbolic value with \acRef{} and the choice will be passed over
by \acChc{}. Thus we have the accumulated result for $\eIL*{1}'$ as the
intermediate term $\eIL*{1}_{acc} = s_{12} < (s_{i} - \chc[A]{k,l})$ with store
$\aStore{}=\set{(i, s_{i}),(s_{1} + s_{2}, s_{12}), (2,s_{2}), (1,s_{1})}$.

With the left child of $\eIL*{1}'$ accumulated, \evAnd{} attempts to continue
evaluation on the right child and will succeed. Notice that this is a special
case as the root of \eIL*{1} is $\wedge$ and so is the root of \hV{}. Thus,
$\eIL*{2}$ will transform $a$ to a symbolic value through accumulation using the
previous store and assert the symbolic value in the base solver with \evSym{}.
The resulting intermediate term will be $s_{12} < (s_{i} - \chc[A]{k,l}) \wedge
\unit{}$, with stores $\eStore{}=\set{s_{a}}$, $\aStore{}=\set{(a,s_{a}),(i,
  s_{i}),(s_{1} + s_{2}, s_{12}), (2,s_{2}), (1,s_{1})}$ and will be reduce to
the final intermediate result $\eIL*{1}_{core} = s_{12} < (s_{i} -
\chc[A]{k,l})$ with the same stores via application of \evAndR{}.

We have now returned back to the top level call to \evAnd{} with a result for
the left child and populated stores. Evaluation will proceed on the right child
$\eIL*{2}$. \eIL*{2}'s root is a disjunction, and thus to proceed evaluation
switched to accumulation by applying \evAcc{}. The accumulation is
straightforward; the left child is the choice $\chc[B]{c, \neg b}$ and is
returned by \acChc{}. The right child is a single variable, and thus is
translated to the symbolic value $s_{b}$. Thus we have the final result for
$\eIL*{2}$, $\eIL*{2}_{core} = \chc[B]{c,\neg b} \vee{} s_{b}$ and the
variational core of \hV{}, $\hV{}_{core} = s_{12} < (s_{i} - \chc[A]{k,l})
\wedge{} (\chc[B]{c,\neg b} \vee{} s_{b})$ with stores $\eStore{}=\set{s_{a}}$,
$\aStore{}=\set{(b, s_{b}),(a,s_{a}),(i, s_{i}),(s_{1} + s_{2}, s_{12}),
  (2,s_{2}), (1,s_{1})}$.

With the variational core derived we can begin choice removal. We assume an
empty configuration for the remainder of the example. The exact semantics of a
\vc{} is implementation specific. For example, our prototype variational
\ac{sat} solver pre-populates the configuration with a generated configuration
based on the user \vc{}. In contrast, the prototype variational \ac{smt} solver
checks the dimensions assignments of \true{} or \false{} in \crChc{} are valid
with respect to the \vc{}, if not then the variant is skipped.

Choice removal begins with the variational core in the focus and an evaluation
context \zipper{} = \inRoot{}, because $\hV_{core}$'s root is $\wedge$ only
\crBool{} applies moving $s_{12} < (s_{i} - \chc[A]{k,l})$ into the focus and
storing the right child in the context: $\zipper =
\inBoolL*{\inRoot{}}{(\chc[B]{c,\neg b} \vee{} s_{b})}{\wedge}$. With $s_{12} <
(s_{i} - \chc[A]{k,l})$ as the focus, the only applicable rule is \crInEq{} due
to $<$ at the root of the focus. \crInEq{} again recurs left, focusing on the
sub-term $s_{12}$ with context $\zipper =
\inInEqL*{\inBoolL*{\inRoot{}}{(\chc[B]{c,\neg b} \vee{} s_{b})}{\wedge}}{(s_{i}
  - \chc[A]{k,l})}{<}$ which states that $s_{12}$ exists in the left child of an
inequality which also exists in the left child of a conjunction.

We have arrived at the base case with a symbolic value is in focus, and the
immediate parent in the evaluation context is an inequality. To proceed we need
to \emph{switch} to begin processing the right child of the inequality; thus we
must apply \crInEqL{}. \crInEqL{} swaps the symbolic with the un-processed right
child held in the context, hence we have $(s_{i} - \chc[A]{k,l})$ in focus with
context $\zipper = \inInEqR*{s_{12}}{\inBoolL*{\inRoot{}}{(\chc[B]{c,\neg b}
    \vee{} s_{b})}{\wedge}}{<}$. $-$ is a previously unseen relation, when an
new relation is found choice removal will recur into the left child. In this
case $- \in \integerFuncs{}$ and so \crArith{} applies. \crArith{} moves $s_{i}$
into the focus and extend the evaluation context to $\zipper =
\inArithL*{\inInEqR*{s_{12}}{\inBoolL*{\inRoot{}}{(\chc[B]{c,\neg b} \vee{}
      s_{b})}{\wedge}}{<}}{\chc[A]{k,l}}{-}$.

With $s_{i}$ in the focus we've arrived at another base case, only this time when
the switch occurs a choice will be in focus. The switch is performed by
\crArithL{} and yields $\chc[A]{k, l}$ as the focus with context $\zipper =
\inArithR*{s_{i}}{\inInEqR*{s_{12}}{\inBoolL*{\inRoot{}}{(\chc[B]{c,\neg b}
      \vee{} s_{b})}{\wedge}}{<}}{-}$. Now we the heart of choice removal
applies, because we have $\configuration{} = \emptyset$, the only applicable
rule with a choice in the focus is \crChc. \crChc{} creates two recursive calls,
one for each alternative using \emph{the same context}, thus we'll have
$\configuration= \set{(A,\true)}$, with focus $k$, and context $\zipper =
\inArithR*{s_{i}}{\inInEqR*{s_{12}}{\inBoolL*{\inRoot{}}{(\chc[B]{c,\neg b}
      \vee{} s_{b})}{\wedge}}{<}}{-}$.

For the remainder of the example we'll continue with only true alternatives; the
other variants follow similar paths. \todo{make sure this isn't a type
  error}Accumulation is called on the introduced plain terms, converting $k$ to
$s_{k}$ and extending the accumulation store to
\newline$\aStore{}=\set{(k,s_{k}),(b, s_{b}),(a,s_{a}),(i, s_{i}),(s_{1} +
  s_{2}, s_{12}), (2,s_{2}), (1,s_{1})}$.

With a symbolic value in the focus, and with the context already switched to the
right child we have come to a sequence of base cases which perform the folds, in
this case \crArithR{} applies. \crArithR{} calls accumulation on $s_{i} -
s_{k}$. $s_{i} - s_{k}$ has not yet been observed in accumulation and thus the
new symbolic value $s_{ik}$ will be generated. This yields $s_{ik}$ in the
focus, with $\aStore{}=\set{(s_{i} - s_{k}, s_{ik}), (k,s_{k}),(b,
  s_{b}),(a,s_{a}),(i, s_{i}),(s_{1} + s_{2}, s_{12}), (2,s_{2}), (1,s_{1})}$,
and $\zipper = \inInEqR*{s_{12}}{\inBoolL*{\inRoot{}}{(\chc[B]{c,\neg b} \vee{}
    s_{b})}{\wedge}}{<}$.

With $s_{ik}$ in the focus we have yet another base case which consumes some
context, only this time we consume the inequality using \crInEqR{}. \crInEqR{}
calls accumulation on $s_{12} < s_{ik}$, similar to the previous call over $-$
this call produces a new symbolic value and extends the accumulation store.
Hence, we'll have $s_{12ik}$ in the focus, with $\aStore{}=\set{(s_{12} <
  s_{ik}, s_{12ik}),(s_{i} - s_{k}, s_{ik}), (k,s_{k}),(b, s_{b}),(a,s_{a}),(i,
  s_{i}),(s_{1} + s_{2}, s_{12}), (2,s_{2}), (1,s_{1})}$, and $\zipper =
\inBoolL*{\inRoot{}}{(\chc[B]{c,\neg b} \vee{} s_{b})}{\wedge}$ as the
evaluation context. With a symbolic value in the focus, and a context indicating
the left child of a relation, choice removal switches to process the right
alternative. The relation in this case is $\wedge$ and so \crBoolL{} applies to
execute the switch yielding $\chc[B]{c,\neg b} \vee{} s_{b}$ in the focus, and
$\zipper = \inBoolR*{s_{12ik}}{\inRoot{}}{\wedge}$. $\vee$ is a relation that is
previously unseen, and thus \crBool{} recurs into its left child yielding the
choice $\chc[B]{c, \neg b}$ in the focus and \newline$\zipper =
\inBoolL*{\inBoolR*{s_{12ik}}{\inRoot{}}{\wedge}}{s_{b}}{\vee}$ as the context.

We have arrived at the second choice. $B \notin \dom{C}$ and thus only \crChc
applies. Following the true alternative for $B$, accumulation is called on $c$
yielding $s_{c}$ in the focus, with $\configuration = \set{(B,\true),
  (A,\true)}$, $\aStore{}=\{(c,s_{c}),(s_{12} < s_{ik}, s_{12ik}),(s_{i} -
s_{k}, s_{ik}),(k,s_{k})$ $ ,(b, s_{b}), (a,s_{a}),(i, s_{i}),(s_{1} + s_{2},
s_{12}), (2,s_{2}), (1,s_{1})\}$, and $\zipper =
\inBoolL*{\inBoolR*{s_{12ik}}{\inRoot{}}{\wedge}}{s_{b}}{\vee}$. All that is
left is a switch and then to complete the fold with the symbolic values.
\crBoolL{} switches the context placing $s_{b}$ in the focus and yielding
$\zipper = \inBoolR*{s_{c}}{\inBoolR*{s_{12ik}}{\inRoot{}}{\wedge}}{\vee}$,
which will be followed by \crBoolR{} to disjunct $s_{c}$ and $s_{b}$ using
accumulation. The resulting term will have $s_{bc}$ in the focus,
$\aStore{}=\{(s_{c} \vee s_{b}, s_{cb}),(c,s_{c}),(s_{12} < s_{ik},
s_{12ik}),(s_{i} - s_{k}, s_{ik}),(k,s_{k})$ $ ,(b, s_{b}), (a,s_{a}),(i,
s_{i}),(s_{1} + s_{2}, s_{12}), (2,s_{2}), (1,s_{1})\}$, and $\zipper =
\inBoolR*{s_{12ik}}{\inRoot{}}{\wedge}$, which leaves only one more reduction
until model generation. \crBoolR{} applies again to conjunct the last two
symbolic values, yielding $\zipper = \inRoot{}$, $s_{12ikbc}$ in the focus and
$\aStore{}=\{(s_{12ik} \wedge s_{bc}, s_{12ikbc}), (s_{c} \vee s_{b},
s_{cb}),(c,s_{c}),(s_{12} < s_{ik}, s_{12ik}),(s_{i} - s_{k}, s_{ik}),(k,s_{k})$
$ ,(b, s_{b}), (a,s_{a}),(i, s_{i}),(s_{1} + s_{2}, s_{12}), (2,s_{2}),
(1,s_{1})\}$.

Thus we have reached the variant parameterized by
$\configuration=\set{(B,\true),(A,\true)}$ \crEval{} applies due to $\zipper =
\inRoot{}$ and the symbolic value in the focus, \evSym{} will yields \unit{}
with $\zipper = \inRoot{}$, indicating that it is safe to query a model for this
variant from the base solver. Due to the two application of \crChc three other
variants will be found during back tracking beginning with the dimension used in
the most recent application. In this case that is the dimension $B$, and thus
the next variant that will be found is parameterized by
$\configuration=\set{(B,\false),(A, \true)}$ with context $\zipper =
\inBoolL*{\inBoolR*{s_{12ik}}{\inRoot{}}{\wedge}}{s_{b}}{\vee}$ and
$\aStore{}=\{(c,s_{c}),(s_{12} < s_{ik}, s_{12ik}),(s_{i} - s_{k},
s_{ik}),(k,s_{k})$ $,(b, s_{b}), (a,s_{a}),(i, s_{i}),(s_{1} + s_{2}, s_{12}),
(2,s_{2}), (1,s_{1})\}$.

%%% Local Variables:
%%% mode: latex
%%% TeX-master: "../../thesis"
%%% End:

\section{Variational \ac{smt} Arrays}
\label{section:vsmt:arrays}
%
With variational \ac{smt} solving fully specified we can reflect on the
generalization recipe from the previous sections. Say we desired to add the
\ac{smt} background for \rn{Reals}. Doing so would follow the straightforward
recipe demonstrated with \rn{Ints}: From the \acl{smtlib} standard we have a set
of primitive operators, we would define wrapped primitive versions for each
operator. Using these wrapped operators we would define new cases for
accumulation and a base case for evaluation indicating that the new operator
requires accumulation. Then we would add the new domain to the syntactic
categories in \autoref{fig:vsmt:categories}. Choice removal would be extended
with three new rules, a rule to begin the processing of the left child of the
relation, a rule to switch from the left child to the right only when a symbolic
value is in the focus, and a rule that performs the fold by combining two
symbolic values and thus consuming some of the context.

In essence, we have a recipe for a generalized variational folding algorithm
over binary relations, that forces reuse of shared terms and is applied to the
domain of \ac{sat} and \ac{smt} solvers. Recall that a symbolic value is a
sequence of statements in the base solver. Thus, another way to view our
generalized folding algorithm is as a compiler from the language of variational
\ac{sat} or \ac{smt} to plain \acl{smtlib} script. The stages of the compiler in
this interpretation are straightforward: We parse a variational \ac{sat} or
\ac{smt} problem to a abstract syntax tree in an intermediate language. The
intermediate language enables optimization passes and is easier to work with
than the object language. Accumulation and evaluation produce a variational core
which can be seen as another, further reduced core language, or as a syntax
object which encapsulates the variational aspects of the input. The core
language is then operated upon by choice removal which deterministically
produces the variant syntax objects. Code generation is spread across generation
of symbolic values in accumulation, assertion of constraints in evaluation, and
calls to \rn{push} or \rn{pop} during choice removal, specifically during
\crChc.

The exact ordering of the operations in the base solver, or the ordering of code
generation, is implementation specific. In the prototype solvers that we have
produced and will discuss further in the next chapter, code generation that
corresponds to generating symbolic values occurs when the symbolic value becomes
known to \aStore. When a configuration occurs the \rn{push}/\rn{pop} calls
encapsulate the operator the choice was nested in and any new symbolic values
which result from the configuration. This ensures sharing of terms that are in
the same assertion levels. For example, consider the case of $s_{12}$ from
\autoref{section:vsmt:example}, $s_{12}$ will be shared once for each variant
because it is plain, thus the code which defines it must occur \emph{before} a
\rn{push} and \rn{pop} call:
%
\begin{lstlisting}[columns=flexible,keepspaces=true,language=SMTLIB]
(declare-const $s_{1}$ Int)           ;; literal declarations
(declare-const $s_{2}$ Int)
(declare-fun $s_{12}$ () Int
  (+ $s_{1}$ $s_{2}$))
(push)                         ;; push for true alternative of $A$
$\vdots{}$
\end{lstlisting}
%
Similarly for terms such as $s_{ik}$ which will be shared twice but are not
plain; because $i$ was transformed into a symbolic before the choice was
configured its declaration still occurs outside the \rn{push}/\rn{pop} block. In
contrast, $k$ is parameterized by a choice and thus its declaration occurs
inside a \rn{push}/\rn{pop} block:
%
\begin{lstlisting}[columns=flexible,keepspaces=true,language=SMTLIB,breaklines=true]
(declare-const $s_{1}$ Int)         ;; literal declarations
(declare-const $s_{2}$ Int)
(declare-fun $s_{12}$ () Int
  (+ $s_{1}$ $s_{2}$))
$\vdots{}$
(declare-const $s_{i}$ Int)           ;; i is declared
(push)                         ;; push for true alternative of A
(declare-const $s_{k}$ Int)
(declare-fun $s_{ik}$ () Int
  (< $s_{i}$ $s_{k}$))
$\vdots{}$
\end{lstlisting}
%
In this case the ordering of symbolic value generation forced $s_{k}$ to be
inside the \rn{push} call so that it is removed from the local scope of the
solver after a \rn{pop} and thus the boundaries between alternatives do not leak
information. Notice that the inference rules in
\autoref{section:vsmt:accumulation}, \autoref{section:vsmt:evaluation} and
\autoref{section:vsmt:choice-removal} guarantee this behavior because symbolic
value creation is ordered according to levels of variation. For example, plain
terms are level 0 as no configuration has happened. In a sense they are globally
scoped and thus become symbolic values or \unit{}'s first. It is only after a
configuration occurs from \crChc{} that more plain terms are introduced. When a
configuration occurs a new \rn{push}/\rn{pop} block is entered, and thus any
calls to accumulation which occur inside it occur inside that block in the base
solver and correspond to level 1. Furthermore, the level is propagated by the $D
\in \configuration$ check in \crChcT{} and \crChcF{}.

To demonstrate the generality of this design we now consider the case of adding
\ac{smt} arrays. To add \ac{smt} arrays we treat arrays as a new kind of
relation. By treating them like any other relation, we take advantage of the
aforementioned ordering behavior and offload the hard work to choice removal.
\ac{smt} arrays are defined by two operations \store*{a}{i}{e} and
\select*{a}{i}, where $a$ is a variable representing the array, $i$ is an index
into the array, and $e$ is an element of the array. Each operation must exist
inside a boolean constraint to propagate information about the array, for
example $\store*{a}{2}{b}$ leaves $a$ unconstrained, while $a \equiv
\store*{a}{2}{b}$ adds a constraint that forces position 2 to store $b$ in $a$.
Similarly, \select{} must exist in a constraint, \eg{}, $x \equiv
\select*{a}{2}$ will add a constraint which sets $x$ to $b$.

Assume that we restrict $i$ to only \rn{Int}, using the recipe above we would
wrap these operations, add new rules to accumulation to accumulate anything in
the $a$, $i$, or $e$ positions and then extend \zipper{} such that a context
could be captured wherever choices may occur. For example we might have:
%
\begin{figure}
  \centering
  \begin{syntax}
    \zipper & \Coloneqq{} & \inRoot  \\
    & | & \inNot{\zipper}            \\
    & | & \inUnary{\zipper}          \\
    & | & \inBoolL{\zipper}{\,\eIL}   \\
    & | & \inBoolR{s}{\zipper}       \\
    & | & \inArithL{\zipper}{\,\eIL}  \\
    & | & \inArithR{s}{\zipper}      \\
    & | & \inInEqL{\zipper}{\,\eIL}   \\
    & | & \inInEqR{s}{\zipper}       \\
    & | & \inSelectL{s}{s}{\zipper} \\
    & | & \inSelectM{s}{s}{\zipper} \\
    & | & \inSelectR{s}{s}{\zipper} \\
    & | & \inStoreL{s}{\zipper} \\
    & | & \inStoreR{s}{\zipper} \\
  \end{syntax}
\end{figure}
%
Which is verbose but would work. Now consider a formula which contains a nested
choice in an arithmetic expression in the element slot of \select{} but not
store: $\kf{f} = (a \equiv \store*{a}{2}{(i - \chc[A]{k,l})}) \wedge (i \equiv
\select{a}{2}) \wedge (i \equiv l)$. The conjunctions indicate separate
statements in \acl{smtlib} due to the behavior of the assertion stack. The
formula is noteworthy because both the \select{} and $i \equiv l$ call are plain
and thus will be processed by evaluation/accumulation \emph{before} choice
removal via the \evAndL{} and \evAndR. This may seem problematic as calling a
\select{} before a \store{} in other paradigms would lead to an error. However
this will not be the case, consider the compiled \acl{smtlib} script for $f$:
%
\begin{lstlisting}[columns=flexible,keepspaces=true,language=SMTLIB,breaklines=true]
(declare-const $s_{a}$ (Array Int Int))         ;; variable declarations
(declare-const $s_{i}$ Int)
(declare-const $s_{l}$ Int)
(declare-const $s_{2}$ Int)
(declare-fun $s_{sel}$ () Int                    ;; select
  (= $s_{i}$ (select $s_{a}$ $s_{2}$)))
(declare-fun $s_{il}$ () Int                    ;; equivalency constraint
  (= $s_{i}$ $s_{l}$))
(assert $s_{sel}$)
(assert $s_{il}$)
(push)                                  ;; push for true alternative of A
(declare-const $s_{k}$ Int)
(declare-fun $s_{ik}$ () Int
  (- $s_{i}$ $s_{k}$))
(assert (= $s_{a}$ (store $s_{a}$ ($s_{2}$) ($s_{ik}$)))
(check-sat)
(get-model)                             ;; plain model for true alternative
(pop)
(push)                                  ;; push for false alternative of A
(declare-fun $s_{il}$ () Int
  (- $s_{i}$ $s_{l}$))
(assert (= $s_{a}$ (store $s_{a}$ ($s_{2}$) ($s_{il}$)))
(check-sat)
(get-model)                             ;; plain model for false alternative
(pop)
\end{lstlisting}
%
We see that evaluation/accumulation did find the plain \select{} and $i \equiv
l$ constraints and assert them before the choice is processed. However, because
constraints in the base solver can be unordered---due to the implicit
conjunction of all assertions in an assertion level---the out of order \select{}
and constraint $i \equiv l$ are not problematic. Furthermore, we see that the
\acl{smtlib} snippet has desirable properties: every plain or variational term
that can be shared, such as $l$ and $i$, is shared. When a \rn{push}/\rn{pop}
block is entered the block is as small as possible, thus sharing is maximized as
much as possible. Due to the symbolic values generated by
accumulation/evaluation, variation does not spread past the immediate relation
and thus other relations do not suffer from \todo{find a
  reference or cite kleene}\emph{variational infection}.

We have demonstrated the generality of our approach by extensions with the
differing domains of arithmetic over integers and arrays. Key to the approach is
the indirection with symbolic values, and the use of a zipper to construct a
generalized variational folding algorithm over any unary, binary, or ternary
relation. Thus, with just what has been presented here we can support the rest
of the core theories in \acl{smtlib} using the aforementioned extension recipe.
We return to this point in \autoref{chapter:related-work} when we discuss the
implications for a variational logic programming language.

%%% Local Variables:
%%% mode: latex
%%% TeX-master: "../../thesis"
%%% End:

%%% Local Variables:
%%% mode: latex
%%% TeX-master: "../../thesis"
%%% End:
