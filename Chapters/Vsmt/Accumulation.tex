~\label{section:vsmt:accumulation}
%
\begin{figure}
  \begin{mathpar}
\inferrule*[right=\acRef]
  { \pwrap{\pspawn}(\aStore{},r) = (\aStore{}', s) }
  { (\aStore{}, r) \accumulation (\aStore{}', s) }

  \inferrule*[right=\acRefI]
  { \pwrap{\pspawni}(\aStore{},r_{i}) = (\aStore{}', s) }
  { (\aStore{}, r_{i}) \accumulation (\aStore{}', s) }

\inferrule*[right=\acNotS]
  { (\aStore{}, \eIL) \accumulation (\aStore{}', s) \\
    \pwrap{\pnot}(\aStore{}', s) = (\aStore{}'', s') }
  { (\aStore{}, \neg{} \eIL) \accumulation (\aStore{}'', s') }

  \inferrule*[right=\acUnaryS]
  { \op \in \integerUnary{} \\
    \lookup{\op}{\integerUnary{}} \coloneqq \pwrap{\op} \\
    (\aStore{}, \eAR) \accumulation (\aStore{}', s) \\
    \pwrap{\op}(\aStore{}', s) = (\aStore{}'', s') }
  { (\aStore{}, \op\; \eAR) \accumulation (\aStore{}'', s') }

\inferrule*[right=\acBoolS]
 { \op \in \boolFuncs{} \\
   \lookup{\op}{\boolFuncs{}} \coloneqq \pwrap{\op} \\
  (\aStore{}, \eIL*{1}) \accumulation (\aStore{1}, s_1) \\
    (\aStore{1}, \eIL*{2}) \accumulation (\aStore{2}, s_2) \\
    \pwrap{\op}(\aStore{2}, s_1, s_2) = (\aStore{3}, s_3) }
  { (\aStore{}, \eIL*{1}\; \op\; \eIL*{2}) \accumulation{} (\aStore{3}, s_3) }

  \inferrule*[right=\acArithS]
  { \op \in \integerFuncs{} \\
    \lookup{\op}{\integerFuncs{}} \coloneqq \pwrap{\op} \\
    (\aStore{}, \eAR*{1}) \accumulation (\aStore{1}, s_1) \\
    (\aStore{1}, \eAR*{2}) \accumulation (\aStore{2}, s_2) \\
    \pwrap{\op}(\aStore{2}, s_1, s_2) = (\aStore{3}, s_3) }
  { (\aStore{}, \eAR*{1}\; \op\; \eAR*{2}) \accumulation{} (\aStore{3}, s_3) }

  \inferrule*[right=\acInEqS]
  { \op \in\ \inequalities{} \\
    \lookup{\op}{\inequalities} \coloneqq \pwrap{\op} \\
    (\aStore{}, \eAR*{1}) \accumulation (\aStore{1}, s_1) \\
    (\aStore{1}, \eAR*{2}) \accumulation (\aStore{2}, s_2) \\
    \pwrap{\op}(\aStore{2}, s_1, s_2) = (\aStore{3}, s_3) }
  { (\aStore{}, \eAR*{1}\; \op\; \eAR*{2}) \accumulation{} (\aStore{3}, s_3) }

\inferrule*[right=\acChc]
  { }
  {(\aStore,\ \chc[D]{e_1,e_2}) \accumulation (\aStore{},\chc[D]{e_1,e_2})}

\inferrule*[right=\acChcI]
  { }
  {(\aStore,\ \chc[D]{ar_{1},ar_{2}}) \accumulation (\aStore{},\chc[D]{ar_{1},ar_{2}})}

\inferrule*[right=\acNotV]
  { (\aStore{}, \eIL) \accumulation (\aStore{}', \eIL') }
  { (\aStore{}, \neg{} \eIL) \accumulation (\aStore{}', \neg \eIL') }

  \inferrule*[right=\acUnaryV]
  { (\aStore{}, \eIL) \accumulation (\aStore{}', \eIL') }
  { (\aStore{}, \integerUnary \eIL) \accumulation (\aStore{}', \integerUnary \eIL') }

\inferrule*[right=\acBoolV]
  { (\aStore{}, \eIL_1) \accumulation (\aStore{1}, \eIL*{1}') \\
    (\aStore{1}, \eIL_2) \accumulation (\aStore{2}, \eIL*{2}') }
  { (\aStore{}, \eIL_1 \boolFuncs{} \eIL_2) \accumulation{} (\aStore{2}, \eIL*{1}' \boolFuncs{} \eIL*{2}') }

  \inferrule*[right=\acArithV]
  { (\aStore{}, \eAR*{1}) \accumulation (\aStore{1}, \eAR*{1}') \\
    (\aStore{1}, \eAR*{2}) \accumulation (\aStore{2}, \eAR*{2}') }
  { (\aStore{}, \eAR*{1} \integerFuncs{} \eAR*{2}) \accumulation{} (\aStore{2}, \eAR*{1}' \integerFuncs{} \eAR*{2}') }

  \inferrule*[right=\acInEqV]
  { (\aStore{}, \eAR*{1}) \accumulation (\aStore{1}, \eAR*{1}') \\
    (\aStore{1}, \eAR*{2}) \accumulation (\aStore{2}, \eAR*{2}') }
  { (\aStore{}, \eAR*{1} \inequalities{} \eAR*{2}) \accumulation{} (\aStore{2}, \eAR*{1}' \inequalities{} \eAR*{2}') }

\end{mathpar}

  \caption{Accumulation inference rules}%
  \label{fig:vsmt:inf:acc}
\end{figure}
%
%
With primitive operations and metavariables defined we specify accumulation in
\autoref{fig:vsmt:inf:acc}. Since the metavariable $\kf{s}$ has two meanings:
sub-trees of \rn{IL} in the variational \ac{smt} domain and sequences of
clauses in the base solver, we treat it as overloaded.
%
\eric{I only see one definition of $s$ (in Sec.~4). Can we refactor to avoid
this problem?}
%
An implementation of the rules would require a store which maps symbolics at
the variational level to terms or sequences at the base solver level.

Accumulation is represented as a binary relation with \accumulation{}. The rules
follow a simple pattern: \rn{Ac-Chc} skips any choices, \rn{Ac-Gen} and
\rn{Ac-Geni} provide a method to inject references into the symbolic domain,
\rn{Ac-Ref} and \rn{Ac-Refi} cache references to ensure the same
reference is mapped to the same symbolic, and the rest of the rules provide
operations on symbolic terms, \eg{}, \rn{Ac-SBinB}, or are congruence rules
such as \rn{Ac-BinI}. We elide rules which process formulas composed of
constants such $\kf{(\tru{} \wedge \fls{})}$ or $(1 + 2 + 3)$. In cases such as
$1 + 2 < \kf{i}$, constants are reduced and treated as references, thus this
formula becomes $3 < \kf{i}$ and is accumulated to $s_{3} < \kf{i}$.

Accumulation maintains a store, \aStore{}, to track and cache symbolic terms.
For example, given formula such as: $g = a \wedge (a \wedge b)$, \rn{Ac-Gen}
will spawn only two new references, one for $\kf{a}$ and one for $\kf{b}$, and
\rn{Ac-Ref} ensures the same symbolic will represent the $a$ reference. This
will produce $g = s_{a} \wedge (s_{a} \wedge s_{b})$, because we $g$ contains
two boolean connective \rn{Ac-BinB} will be called twice beginning with the
inner conjunction. \rn{Ac-BinB} will combine $s_{a}$ and $s_{b}$ into a new
symbolic $s_{ab}$, update the store to \aStore{}'. The new store will include
entries for both references \textit{and} symbolic references, thus, in this
example \aStore{}' contains $\kf{a} \rightarrow s_{a}$, $\kf{b} \rightarrow
s_{b}$, and $\kf{s_{ab}} \rightarrow (s_{a} \wedge s_{b})$. Finally
\rn{Ac-BinB} will repeat the last procedure on the outermost conjunction
adding a new entry to the symbolic store.


%%% Local Variables:
%%% mode: latex
%%% TeX-master: "../../thesis"
%%% End: