\label{section:vsmt:accumulation}
%
\begin{figure}
  \begin{mathpar}
\inferrule*[right=\acRef]
  { \pwrap{\pspawn}(\aStore{},r) = (\aStore{}', s) }
  { (\aStore{}, r) \accumulation (\aStore{}', s) }

  \inferrule*[right=\acRefI]
  { \pwrap{\pspawni}(\aStore{},r_{i}) = (\aStore{}', s) }
  { (\aStore{}, r_{i}) \accumulation (\aStore{}', s) }

\inferrule*[right=\acNotS]
  { (\aStore{}, \eIL) \accumulation (\aStore{}', s) \\
    \pwrap{\pnot}(\aStore{}', s) = (\aStore{}'', s') }
  { (\aStore{}, \neg{} \eIL) \accumulation (\aStore{}'', s') }

  \inferrule*[right=\acUnaryS]
  { \op \in \integerUnary{} \\
    \lookup{\op}{\integerUnary{}} \coloneqq \pwrap{\op} \\
    (\aStore{}, \eAR) \accumulation (\aStore{}', s) \\
    \pwrap{\op}(\aStore{}', s) = (\aStore{}'', s') }
  { (\aStore{}, \op\; \eAR) \accumulation (\aStore{}'', s') }

\inferrule*[right=\acBoolS]
 { \op \in \boolFuncs{} \\
   \lookup{\op}{\boolFuncs{}} \coloneqq \pwrap{\op} \\
  (\aStore{}, \eIL*{1}) \accumulation (\aStore{1}, s_1) \\
    (\aStore{1}, \eIL*{2}) \accumulation (\aStore{2}, s_2) \\
    \pwrap{\op}(\aStore{2}, s_1, s_2) = (\aStore{3}, s_3) }
  { (\aStore{}, \eIL*{1}\; \op\; \eIL*{2}) \accumulation{} (\aStore{3}, s_3) }

  \inferrule*[right=\acArithS]
  { \op \in \integerFuncs{} \\
    \lookup{\op}{\integerFuncs{}} \coloneqq \pwrap{\op} \\
    (\aStore{}, \eAR*{1}) \accumulation (\aStore{1}, s_1) \\
    (\aStore{1}, \eAR*{2}) \accumulation (\aStore{2}, s_2) \\
    \pwrap{\op}(\aStore{2}, s_1, s_2) = (\aStore{3}, s_3) }
  { (\aStore{}, \eAR*{1}\; \op\; \eAR*{2}) \accumulation{} (\aStore{3}, s_3) }

  \inferrule*[right=\acInEqS]
  { \op \in\ \inequalities{} \\
    \lookup{\op}{\inequalities} \coloneqq \pwrap{\op} \\
    (\aStore{}, \eAR*{1}) \accumulation (\aStore{1}, s_1) \\
    (\aStore{1}, \eAR*{2}) \accumulation (\aStore{2}, s_2) \\
    \pwrap{\op}(\aStore{2}, s_1, s_2) = (\aStore{3}, s_3) }
  { (\aStore{}, \eAR*{1}\; \op\; \eAR*{2}) \accumulation{} (\aStore{3}, s_3) }

\inferrule*[right=\acChc]
  { }
  {(\aStore,\ \chc[D]{e_1,e_2}) \accumulation (\aStore{},\chc[D]{e_1,e_2})}

\inferrule*[right=\acChcI]
  { }
  {(\aStore,\ \chc[D]{ar_{1},ar_{2}}) \accumulation (\aStore{},\chc[D]{ar_{1},ar_{2}})}

\inferrule*[right=\acNotV]
  { (\aStore{}, \eIL) \accumulation (\aStore{}', \eIL') }
  { (\aStore{}, \neg{} \eIL) \accumulation (\aStore{}', \neg \eIL') }

  \inferrule*[right=\acUnaryV]
  { (\aStore{}, \eIL) \accumulation (\aStore{}', \eIL') }
  { (\aStore{}, \integerUnary \eIL) \accumulation (\aStore{}', \integerUnary \eIL') }

\inferrule*[right=\acBoolV]
  { (\aStore{}, \eIL_1) \accumulation (\aStore{1}, \eIL*{1}') \\
    (\aStore{1}, \eIL_2) \accumulation (\aStore{2}, \eIL*{2}') }
  { (\aStore{}, \eIL_1 \boolFuncs{} \eIL_2) \accumulation{} (\aStore{2}, \eIL*{1}' \boolFuncs{} \eIL*{2}') }

  \inferrule*[right=\acArithV]
  { (\aStore{}, \eAR*{1}) \accumulation (\aStore{1}, \eAR*{1}') \\
    (\aStore{1}, \eAR*{2}) \accumulation (\aStore{2}, \eAR*{2}') }
  { (\aStore{}, \eAR*{1} \integerFuncs{} \eAR*{2}) \accumulation{} (\aStore{2}, \eAR*{1}' \integerFuncs{} \eAR*{2}') }

  \inferrule*[right=\acInEqV]
  { (\aStore{}, \eAR*{1}) \accumulation (\aStore{1}, \eAR*{1}') \\
    (\aStore{1}, \eAR*{2}) \accumulation (\aStore{2}, \eAR*{2}') }
  { (\aStore{}, \eAR*{1} \inequalities{} \eAR*{2}) \accumulation{} (\aStore{2}, \eAR*{1}' \inequalities{} \eAR*{2}') }

\end{mathpar}

  \caption{Accumulation inference rules}%
  \label{fig:vsmt:inf:acc}
\end{figure}
%
%
The variational \ac{smt} version of accumulation is specified in
\autoref{fig:vsmt:inf:acc} and is a generalized variational fold over the
abstract syntax tree of \eIL{}. Just as before, accumulation is split into
congruence rules over the intermediate language, computation rules over symbolic
values and computation rules for references and choices.

The variational \ac{sat} version of accumulation is a specialized form of this
version of accumulation. The only semantic difference between operators is the
code emitted to the base solver, hence we generalize the previous version by
performing a lookup to retrieve the appropriate wrapped primitive. The primitive
is indicated with an \pwrap{underline}. For example, if $\boolFuncs{} = \wedge$
then \acBoolS{} specializes to \acAndS{} where $\pwrap{\boolFuncs{}} =
\pwrap{\pand}$, and thus the resulting call becomes $\pwrap{\pand}(\aStore{2},
s_1, s_2)$. Hence, the rules \acAndS{}, and \acOrS{} are specialized forms of
the general rule \acBoolS.

Similarly, we collapse the arithmetic and inequality computation rules to
\acArithS{} and \acInEqS{}. The semantics of each rule, besides the operator
lookup, remains unchanged; the congruence rules recur into the abstract syntax
tree to convert references to symbolic values, choices are skipped over due to
\acChc{} and \acChcI{}, and plain values are combined with the computation rules
\acBoolS, \acArithS, and \acInEqS.
%
The only other substantial difference is two new computation rules to handle
arithmetic choices and variables, \acChcI{}, and \acRefI{}. Both serve the same
function as their boolean counterparts \acChc{} and \acRef{}.

In this form it should be plain to see the recipe to further extend accumulation
to another background theory. One would add a new computation rules for the new
kinds of references and choices, a new computation rule for symbolic references
in the theory, and a new congruence rule over the new abstract syntax trees.
Extending accumulation with new operators is similarly trivial. Recall the
modulus example, to extend accumulation with a modulus operator, assuming the
wrapped primitive has been defined, we would only need to add the operator to
\integerFuncs{} syntactic category and create a case such that $\kf{mod} \in
\integerFuncs{}$ succeeds.



%%% Local Variables:
%%% mode: latex
%%% TeX-master: "../../thesis"
%%% End: