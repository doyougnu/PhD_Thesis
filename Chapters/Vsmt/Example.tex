~\label{section:vsmt:example}
%
With variational \ac{smt} solving defined. We conclude the chapter with a
complete example of solving a variational \ac{smt} problem. Consider the query
formula $\kf{h = ((1 + 2 < (i - \chc[A]{k,l})) \wedge a) \wedge (\chc[A]{c,\neg
    b} \vee b)}$; derivation of the variational core $\kf{h}$ begins with
evaluation and all stores \aStore{}, \eStore{} initialized to $\emptyset$. When
a sure inputs a \vc{} the configuration, \configuration{}, is initialized to it,
otherwise \configuration{} is initialized to empty.

\rn{Ev-And} is the only applicable rule, matching \boolFuncs{} with $\wedge$
at the root of $\kf{h}$. Thus, $v_{1} = \kf{((1 + 2 < (i -
  \chc[A]{k,l})) \wedge a)}$, and $v_{2} = (\chc[A]{c,\neg b} \vee b)$. We
traverse $v_{1}$ first, leading to a recursive application of \rn{Ev-And}.
We denote the recursive levels with a tick mark $'$, thus \newline{} $\kf{v_{1}' = (1 + 2 <
  (i - \chc[A]{k,l}))}$ is the recursive left child and the right child is
$\kf{v_{2}' = a}$.

\rn{Ev-AccIB} matches \inequalities{}with the $<$ at the root of $v_{1}'$ and
switches to accumulation. $v_{2}'$ is a terminal, will match \rn{Ev-Tm}, be
sent to the base solver, and replaced with \unit{}. \rn{Ev-Tm} updates
\eStore{}, recording the interaction and yields $(\eStore{v_{2}'},\aStore{},
\unit{})$, where $\eStore{v_{2}'} = \{\kf{a}\} \cup \eStore{}$ as the result for
$v_{2}'$.

Accumulation on $v_{1}'$ matches \inequalities{} to $<$, applying
\rn{Ev-AccIB} yields two recursive cases: $v_{1}'' = 1 + 2$; and $v_{2}'' =
\kf{i - \chc[A]{k,l}}$. \itodo{v1'' will be turned into a constant but we don't
  show rules for constants because they aren't interesting, should we? We could
  also transform Ac-Ref and Ac-Refi to work on t and $t_{i}$. Thoughts?}.
%
$v_{1}''$ will be preprocessed to the value 3, and accumulated to a symbolic with
\rn{Ac-Refi} yielding $(\aStore{v_{1}''}, s_{3})$ where $\aStore{v_{1}''} =
\{(3, \kf{s_{3}})\} \cup \aStore{}$. $v_{2}''$ is the interesting case.
\rn{Ac-BinI} will match $-$ at the root node, $\kf{i}$ will be accumulated
to $\kf{s_{i}}$ with \rn{Ac-Refi} and the choice is skipped with
\rn{Ac-Chc}. Hence we have $(\aStore{v_{2}''}, s_{i} - \chc[A]{k,l}$), where
$\aStore{v_{2''}} = (\{\kf{i, s_{i}}\} \cup \aStore{v_{1}''})$ as the result for
$v_{2}''$. Note that the stores, \aStore{}, \eStore{}, are threaded through from the left
child to the right child and thus can only monotonically increase until the
query formula is processed.

With results for $v_{1}''$, $v_{2}''$, and $v_{2}'$ the recursive calls can finally
resolve. $v_{1}'$ yields $(\aStore{v_{1}'}, \kf{s_{3}} < s_{i} - \chc[A]{k,l})$,
where \aStore{v_{1}'} $= \{(\kf{i}, s_{i}), (3, s_{3})\}$, $v_{2}'$'s result
only manipulated \eStore{} and thus $v_{1}$'s result is $(\eStore{v_{2}'},
\aStore{v_{1}'}, (\kf{s_{3}} < s_{i} - \chc[A]{k,l}) \wedge \unit{})$, which can
be further reduced by \rn{Ev-UR} to \newline{} $(\eStore{v_{2}'},
\aStore{v_{1}'}, (\kf{s_{3}} < s_{i} - \chc[A]{k,l}))$.

This process is repeated for $v_{2} = (\chc[A]{c,\neg b} \vee b)$ with the final
stores from processing $v_{1}$. The only rule that matches $\vee$ is
\rn{Ev-AccB}, thus $v_{2}$ is processed in accumulation. Accumulation
matches on the disjunction and applies \rn{Acc-BinB} with $v_{1}' =
\chc[A]{c,\neg b}$ and $v_{2}' = \kf{b}$. The choice, by \rn{Ac-Chc}, is
skipped over; $\kf{b}$, by \rn{Ac-Gen} will be converted to a symbolic
$s_{b}$ yielding $(\eStore{v_{2}}, \aStore{v_{2}}, \chc[A]{c,\neg b} \vee
s_{b})$, where $(\aStore{v_{2}} = \{(\kf{b}, \kf{s_{b}})\} \cup
\aStore{v_{1}'})$, and $\eStore{v_{2}} = \{\kf{a}\}$ as the result for $v_{2}$.
With both $v_{1}$ and $v_{2}$ processed the variational core for $\kf{h}$ is
found to be $\kf{h_{core}} = (s_{3} < (s_{i} - \chc[A]{k,l})) \wedge
(\chc[A]{c,\neg b} \vee s_{b})$ with stores $\eStore{h_{core}} = \{(a)\}$,
$\aStore{h_{core}} = \{\kf{(b,s_{b}), (i,s{i}), (3,s_{3})}\}$.%

\paragraph{Solving the variational core}
Solving the variational core begins with choice removal and proceeds with
recursive calls to evaluation and consequently accumulation. We assume an empty
configuration for the remainder of the example because the \vc{} case is a
sub-case. The computation rules which remove choices, such as \rn{Cr-LB},
and \rn{Cr-IB-ChcR}, require a choice in the child node of a binary
relation, however $h_{core}$'s immediate child nodes are binary relations
themselves, $<$ on the left, and $\vee{}$ on the right. We use a zipper to
manipulate the core such that a choice is in position for removal, while the
remainder of the core is held in a context, a variational \ac{sat} solving may
instead choose to migrate choices according to Boolean equivalency laws.

Assuming $\kf{\chc[A]{k,l}}$ is found to be the focus, then the left version of
\rn{Cr-IB-ChcR}, \rn{Cr-IB-ChcL} would apply. Clearly $D \notin
\configuration{}$, thus a recursive case for each alternative, beginning with
the left alternative $e_{1}$, is performed. Several changes occur: the assertion
stack is incremented; indicating a push for the next call to evaluation, the
configuration mutates to account for the selection, and $e_{1}$ is translated
into IL and replaces the choice, thereby introducing a \textit{new} plain term:
$\kf{l}$. Thus, the recursive call for the left alternative is $(s_{3} < (s_{i}
- k)) \wedge (\chc[A]{c,\neg b} \vee s_{b})$ where $\configuration{\kf{L}} =
\{(\kf{A}, \true{})\}$, and $\kf{i_{\kf{L}} = 1}$. Similarly the right
alternative is $(s_{3} < (s_{i} - l)) \wedge (\chc[A]{c,\neg b} \vee s_{b})$
with $\configuration{\kf{R}} = \{(\kf{A}, \false{})\}$, and $\kf{i_{R} = 1}$.

With the choice removed the rules are no longer stuck. \rn{Cr-BinB} will
apply to both alternatives because their root node, $\wedge$ matches
\boolFuncs{}. We walk through the processing of the left alternative in detail,
the right alternative follows the same procedure. \rn{Cr-BinB} produces two
calls to accumulation with $v_{1} = (s_{3} < (s_{i} - k))$, and $v_{2} =
\chc[A]{c,\neg b} \vee s_{b}$, $v_{2}$ is still stuck and will thus be returned,
$v_{1}$ is no longer stuck will be fully reduced to a symbolic term.

Accumulation will apply \rn{Ac-BinIB} with $v_{1}' = s_{3}$ and $v_{2}' = s_{i}
- k$. $v_{1}'$ is already accumulated and will be returned, \rn{Ac-BinI}
will be applied to $v_{2}'$, will translate $\kf{k}$ to a symbolic $s_{k}$ via
\rn{Ac-GenI}, and update \aStore{h_{core}} to $\aStore{h_{L}} = \{(k, s_{k})
\cup \aStore{h_{core}}\}$. Thus we have $v_{2}'$ accumulated to $v_{2}' = s_{i}
- s_{k}$ which allows an application of the computation rule \rn{Ac-SBinI}
to produce a single symbolic, $v_{2}' = s_{i-k}$ with $\aStore{h_{L}} =
\kf{\{(s_{i} - s_{k}, s_{i-k}), (k,s_{k})\} \cup \aStore{h_{core}}}$. The
recursion continues to unwind with the result of \rn{Ac-BinIB} as $v_{1}' =
s_{3} < s_{i-k}$, the rule \rn{Ac-SBinIB} can be applied yielding the result
for $v_{1}$ as $v_{1} = s_{s_{3}<s_{i-k}}$ with store $\aStore{h_{L}} =
\kf{\{(s_{3} < s_{i-k}, s_{s_{3}<s_{i-k}}), (s_{i} - s_{k}, s_{i-k}),
  (k,s_{k})\} \cup \aStore{h_{core}}}$.

With $v_{1}$ accumulated we have a new variational core $s_{s_{3}<s_{i-k}} \wedge
(\chc[A]{c, \neg b} \vee s_{b})$, only this time, depending on the alternative,
\configuration{} has enough information to configure $\kf{A}$. Again, we must
find a choice in the focus in order to proceed, once \chc[A]{c, \neg b} is in
focus \rn{Cr-RB} (the right version of \rn{Cr-LB}) will be applied.
$\kf{A} \in \configuration{L}$ and so the left alternative $\kf{c}$ will replace
the choice for $s_{s_{3}<s_{i-k}} \wedge (c \vee s_{b})$. This formula will
switch into accumulation due to \rn{Cr-BinB} and be processed to a single
symbolic similarly to $s_{s_{3}<s_{i-k}}$. Once the symbolic has been created,
the \rn{Sym} rule calls evaluation which performs the assertion stack
manipulation, writes the symbolic to the base solver. A model is generated with
the \rn{Gen} rule and combined with an empty variational model. With the
model for the \true{} variant of $\kf{A}$ the process backtracks to compute the
false variant.

%%% Local Variables:
%%% mode: latex
%%% TeX-master: "../../thesis"
%%% End: