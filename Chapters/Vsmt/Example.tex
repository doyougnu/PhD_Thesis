\label{section:vsmt:example}
%
With variational \ac{smt} solving formally specified. We present a complete
example of solving a variational \ac{smt} problem. Consider the query formula
\newline$\kf{\hV = ((1 + 2 < (i - \chc[A]{k,l})) \wedge a) \wedge
  (\chc[B]{c,\neg b} \vee b)}$ with two choices parameterized by the dimensions
$A$ and $B$. Derivation of the variational core for $\kf{h}$ begins with all
evaluation contexts and all stores \aStore{}, \eStore{} initialized to
$\varnothing$.

The root of \hV{} is $\wedge$ and thus \evAnd{} is the only applicable rule.
From \evAnd{} we have $\eIL*{1} = \kf{((1 + 2 < (i - \chc[A]{k,l})) \wedge a)}$,
and $\eIL*{2} = (\chc[B]{c,\neg b} \vee b)$. We traverse \eIL*{1} first, leading
to a recursive application of \evAnd{}. We denote recursive levels with a tick
mark: $'$, thus $\kf{\eIL*{1}' = (1 + 2 < (i - \chc[A]{k,l}))}$ is the left
child of \eIL*{1}, with $\kf{\eIL*{2}' = a}$ as the right child.

The root of $\eIL*{1}'$ is an inequality, so the only way to progress is to try
to accumulate $\eIL*{1}'$. The accumulation will succeed; in accumulation, only
\acInEqV{} can apply as accumulation will be unable to transform the right child
of $\eIL*{1}'$ to a symbolic value due to the presence of a choice. \acInEqV{}
will further destruct $\eIL*{1}'$ to $\eAR*{1} = 1 + 2$ and $\eAR*{2} = i -
\chc[A]{k,l}$. $\eAR*{1}$ will be accumulated to a single symbolic value by
application of \acArithS{} and \acRef{} on the literals $1$ and $2$ yielding
$\eAR*{1} = s_{12}$, with store $\aStore{}=\set{(s_{1} + s_{2}, s_{12}),
  (2,s_{2}), (1,s_{1})}$.

Using the resultant store from accumulating $\eAR*{1}$, accumulation on
$\eAR*{2}$ will yield the term $s_{i} - \chc[A]{k,l}$. The variable $i$ will be
accumulated to a symbolic value with \acRef{} and the choice will be passed over
by \acChc{}. Thus we have the accumulated result for $\eIL*{1}'$ as the
intermediate term $\eIL*{1}_{acc} = s_{12} < (s_{i} - \chc[A]{k,l})$ with store
$\aStore{}=\set{(i, s_{i}),(s_{1} + s_{2}, s_{12}), (2,s_{2}), (1,s_{1})}$.

With the left child of $\eIL*{1}'$ accumulated, \evAnd{} attempts to continue
evaluation on the right child and will succeed. Notice that this is a special
case as the root of \eIL*{1} is $\wedge$ and so is the root of \hV{}. Thus,
$\eIL*{2}$ will transform $a$ to a symbolic value through accumulation using the
previous store and assert the symbolic value in the base solver with \evSym{}.
The resulting intermediate term will be $s_{12} < (s_{i} - \chc[A]{k,l}) \wedge
\unit{}$, with stores $\eStore{}=\set{s_{a}}$, $\aStore{}=\set{(a,s_{a}),(i,
  s_{i}),(s_{1} + s_{2}, s_{12}), (2,s_{2}), (1,s_{1})}$ and will be reduce to
the intermediate result $\eIL*{1}_{core} = s_{12} < (s_{i} - \chc[A]{k,l})$ with
the same stores via application of \evAndR{}.

We have now returned back to the top level call to \evAnd{} with a result for
the left child and populated stores. Evaluation will proceed on the right child
$\eIL*{2}$. \eIL*{2}'s root is a disjunction, and thus to proceed evaluation
switched to accumulation by applying \evAcc{}. The accumulation is
straightforward; the left child is the choice $\chc[B]{c, \neg b}$ and is
returned by \acChc{}. The right child is a single variable, and thus is
translated to the symbolic value $s_{b}$. Thus we have the final result for
$\eIL*{2}$, $\eIL*{2}_{core} = \chc[B]{c,\neg b} \vee{} s_{b}$ and the
variational core of \hV{}, $\hV{}_{core} = s_{12} < (s_{i} - \chc[A]{k,l})
\wedge{} (\chc[B]{c,\neg b} \vee{} s_{b})$ with stores $\eStore{}=\set{s_{a}}$,
$\aStore{}=\set{(b, s_{b}),(a,s_{a}),(i, s_{i}),(s_{1} + s_{2}, s_{12}),
  (2,s_{2}), (1,s_{1})}$.

With the variational core derived we can begin choice removal. We assume an
empty configuration for the remainder of the example. The exact semantics of a
\vc{} is implementation specific. For example, our prototype variational
\ac{sat} solver pre-populates the configuration with a generated configuration
based on the user \vc{}. In contrast, the prototype variational \ac{smt} solver
checks the dimensions assignments of \true{} or \false{} in \crChc{} are valid
with respect to the \vc{}, if not then the variant is skipped.

Choice removal begins with the variational core in the focus and an evaluation
context \zipper{} = \inRoot{}, because $\hV_{core}$'s root is $\wedge$ only
\crBool{} applies moving $s_{12} < (s_{i} - \chc[A]{k,l})$ into the focus and
storing the right child in the context: $\zipper =
\inBoolL*{\inRoot{}}{(\chc[B]{c,\neg b} \vee{} s_{b})}{\wedge}$. With $s_{12} <
(s_{i} - \chc[A]{k,l})$ as the focus, the only applicable rule is \crInEq{} due
to $<$ at the root of the focus. \crInEq{} again recurs left, focusing on the
sub-term $s_{12}$ with context $\zipper =
\inInEqL*{\inBoolL*{\inRoot{}}{(\chc[B]{c,\neg b} \vee{} s_{b})}{\wedge{}}}{(s_{i}
  - \chc[A]{k,l})}{<}$ which states that $s_{12}$ exists in the left child of an
inequality which also exists in the left child of a conjunction.

We have arrived at the base case with a symbolic value in focus, and the
immediate parent in the evaluation context is an inequality. To proceed we need
to \emph{switch} to begin processing the right child of the inequality; thus we
must apply \crInEqL{}. \crInEqL{} swaps the symbolic with the un-processed right
child held in the context, hence we have $(s_{i} - \chc[A]{k,l})$ in focus with
context $\zipper = \inInEqR*{s_{12}}{\inBoolL*{\inRoot{}}{(\chc[B]{c,\neg b}
    \vee{} s_{b})}{\wedge}}{<}$. Subtraction, $-$, is a previously unseen
relation. When a new relation is found, choice removal will recur into the left
child. In this case $- \in \integerFuncs{}$ and so \crArith{} applies.
\crArith{} moves $s_{i}$ into the focus and extend the evaluation context to
$\zipper = \inArithL*{\inInEqR*{s_{12}}{\inBoolL*{\inRoot{}}{(\chc[B]{c,\neg b}
      \vee{} s_{b})}{\wedge}}{<}}{\chc[A]{k,l}}{-}$.

With $s_{i}$ in the focus, we've arrived at another base case, only this time
when the switch occurs a choice will be in focus. The switch is performed by
\crArithL{} and yields $\chc[A]{k, l}$ as the focus with context $\zipper =
\inArithR*{s_{i}}{\inInEqR*{s_{12}}{\inBoolL*{\inRoot{}}{(\chc[B]{c,\neg b}
      \vee{} s_{b})}{\wedge}}{<}}{-}$. Now the heart of choice removal applies,
because we have $\configuration{} = \varnothing$, the only applicable rule with a
choice in the focus is \crChc. \crChc{} creates two recursive calls, one for
each alternative using \emph{the same context}, thus we'll have $\configuration=
\set{(A,\true)}$, with focus $k$, and context $\zipper =
\inArithR*{s_{i}}{\inInEqR*{s_{12}}{\inBoolL*{\inRoot{}}{(\chc[B]{c,\neg b}
      \vee{} s_{b})}{\wedge}}{<}}{-}$.

For the remainder of the example we'll continue with only true alternatives; the
other variants follow similar paths. Accumulation is called on the introduced
plain terms, converting $k$ to $s_{k}$ and extending the accumulation store to
\newline$\aStore{}=\set{(k,s_{k}),(b, s_{b}),(a,s_{a}),(i, s_{i}),(s_{1} +
  s_{2}, s_{12}), (2,s_{2}), (1,s_{1})}$.

With a symbolic value in focus, and with the context already switched to the
right child, we have come to a sequence of base cases which perform the folds,
in this case \crArithR{} applies. \crArithR{} calls accumulation on $s_{i} -
s_{k}$. $s_{i} - s_{k}$ has not yet been observed in accumulation and thus the
new symbolic value $s_{ik}$ will be generated. This yields $s_{ik}$ in the
focus, with $\aStore{}=\set{(s_{i} - s_{k}, s_{ik}), (k,s_{k}),(b,
  s_{b}),(a,s_{a}),(i, s_{i}),(s_{1} + s_{2}, s_{12}), (2,s_{2}), (1,s_{1})}$,
and $\zipper = \inInEqR*{s_{12}}{\inBoolL*{\inRoot{}}{(\chc[B]{c,\neg b} \vee{}
    s_{b})}{\wedge}}{<}$.

With $s_{ik}$ in focus, we have yet another base case which consumes some
context, only this time we consume the inequality using \crInEqR{}. \crInEqR{}
calls accumulation on $s_{12} < s_{ik}$, similar to the previous call over $-$,
this call produces a new symbolic value and extends the accumulation store.
Hence, we'll have $s_{12ik}$ in the focus, with $\aStore{}=\set{(s_{12} <
  s_{ik}, s_{12ik}),(s_{i} - s_{k}, s_{ik}), (k,s_{k}),(b, s_{b}),(a,s_{a}),(i,
  s_{i}),(s_{1} + s_{2}, s_{12}), (2,s_{2}), (1,s_{1})}$, and $\zipper =
\inBoolL*{\inRoot{}}{(\chc[B]{c,\neg b} \vee{} s_{b})}{\wedge}$ as the
evaluation context. With a symbolic value in the focus, and a context indicating
the left child of a relation, choice removal switches to process the right
alternative. The relation in this case is $\wedge$ and so \crBoolL{} applies to
execute the switch yielding $\chc[B]{c,\neg b} \vee{} s_{b}$ in the focus, and
$\zipper = \inBoolR*{s_{12ik}}{\inRoot{}}{\wedge}$. $\vee$ is a relation that is
previously unseen, and thus \crBool{} recurs into its left child yielding the
choice $\chc[B]{c, \neg b}$ in the focus and \newline$\zipper =
\inBoolL*{\inBoolR*{s_{12ik}}{\inRoot{}}{\wedge}}{s_{b}}{\vee}$ as the context.

We have arrived at the second choice. $B \notin \dom{C}$ and thus only \crChc
applies. Following the true alternative for $B$, accumulation is called on $c$
yielding $s_{c}$ in the focus, with $\configuration = \set{(B,\true),
  (A,\true)}$, $\aStore{}=\{(c,s_{c}),(s_{12} < s_{ik}, s_{12ik}),(s_{i} -
s_{k}, s_{ik}),(k,s_{k})$ $ ,(b, s_{b}), (a,s_{a}),(i, s_{i}),(s_{1} + s_{2},
s_{12}), (2,s_{2}), (1,s_{1})\}$, and $\zipper =
\inBoolL*{\inBoolR*{s_{12ik}}{\inRoot{}}{\wedge}}{s_{b}}{\vee}$. All that is
left is a switch and then to complete the fold with the symbolic values.
\crBoolL{} switches the context placing $s_{b}$ in the focus and yielding
$\zipper = \inBoolR*{s_{c}}{\inBoolR*{s_{12ik}}{\inRoot{}}{\wedge}}{\vee}$,
which will be followed by \crBoolR{} to disjunct $s_{c}$ and $s_{b}$ using
accumulation. The resulting term will have $s_{bc}$ in the focus,
$\aStore{}=\{(s_{c} \vee s_{b}, s_{cb}),(c,s_{c}),(s_{12} < s_{ik},
s_{12ik}),(s_{i} - s_{k}, s_{ik}),(k,s_{k})$ $ ,(b, s_{b}), (a,s_{a}),(i,
s_{i}),(s_{1} + s_{2}, s_{12}), (2,s_{2}), (1,s_{1})\}$, and $\zipper =
\inBoolR*{s_{12ik}}{\inRoot{}}{\wedge}$, which leaves only one more reduction
until model generation. \crBoolR{} applies again to conjunct the last two
symbolic values, yielding $\zipper = \inRoot{}$, $s_{12ikbc}$ in the focus and
$\aStore{}=\{(s_{12ik} \wedge s_{bc}, s_{12ikbc}), (s_{c} \vee s_{b},
s_{cb}),(c,s_{c}),(s_{12} < s_{ik}, s_{12ik}),(s_{i} - s_{k}, s_{ik}),(k,s_{k})$
$ ,(b, s_{b}), (a,s_{a}),(i, s_{i}),(s_{1} + s_{2}, s_{12}), (2,s_{2}),
(1,s_{1})\}$.

Thus we have reached the variant parameterized by
$\configuration=\set{(B,\true),(A,\true)}$ \crEval{} applies due to $\zipper =
\inRoot{}$ and the symbolic value in the focus, \evSym{} will yields \unit{}
with $\zipper = \inRoot{}$, indicating that it is safe to query a model for this
variant from the base solver. Due to the two application of \crChc{} three other
variants will be found during backtracking beginning with the dimension used in
the most recent application. In this case that is the dimension $B$, and thus
the next variant that will be found is parameterized by
$\configuration=\set{(B,\false),(A, \true)}$ with context $\zipper =
\inBoolL*{\inBoolR*{s_{12ik}}{\inRoot{}}{\wedge}}{s_{b}}{\vee}$ and
$\aStore{}=\{(c,s_{c}),(s_{12} < s_{ik}, s_{12ik}),(s_{i} - s_{k},
s_{ik}),(k,s_{k})$ $,(b, s_{b}), (a,s_{a}),(i, s_{i}),(s_{1} + s_{2}, s_{12}),
(2,s_{2}), (1,s_{1})\}$.

%%% Local Variables:
%%% mode: latex
%%% TeX-master: "../../thesis"
%%% End: