~\label{section:vsmt:primitives}
%
\autoref{fig:arith:stx} defines the syntax of the integer arithmetic extension,
which consists of integer variables, integer literals, a set of standard
operators, and choices.
%
The sets of Boolean and arithmetic variables are disjoint, thus an expression
such as $(\kf{s < 10) \wedge (s \vee p})$, where $s$ occurs as both an integer
and Boolean variable is disallowed.
%
The syntax of the language prevents type errors and expressions that do not
yield Boolean values. For example, $\chc[D]{1,2} \wedge p$ is syntactically
invalid.
%
Thus, the language is purposefully limited to arithmetic expressions that
have an inequality at the root of the expression, such as: $\kf{g} =
(\chc[A]{1, 2} + j \geq 2) \vee a \wedge \chc[A]{c, d}$.
%
Choices in the same dimension are synchronized across Boolean and arithmetic
sub-expressions, for example, the expression
%
$\kf{g} = (\chc[A]{1, 2} + j \geq 2) \vee (a \wedge \chc[A]{c,d})$
% %
represents two variants:
%
$\sem[A_{\tru{}}]{g} = (1 + j \geq 2) \vee (a \wedge c)$ and
$\sem[A_{\fls{}}]{g} = (2 + j \geq 2) \vee (a \wedge d)$.

\begin{figure}
  \centering
\begin{tabular}{r@{~~:~~}l@{~~\OB{\to}~~}ll}
\pnot
  & $(\aStore{},s)$
  & $(\aStore{},s)$
  & \emph{Negate a symbolic value} \\
\pand
  & $(\aStore{},s,s)$
  & $(\aStore{},s)$
  & \emph{Conjunction of symbolic values} \\
\por
  & $(\aStore{},s,s)$
  & $(\aStore{},s)$
  & \emph{Disjunction of symbolic values} \\
\pspawn
  & $(\aStore{},r)$
  & $(\aStore{},s)$
  & \emph{Create symbolic value based on a variable} \\
\passert
  & $(\eStore{},\aStore{},s)$
  & $\eStore{}$
  & \emph{Assert a symbolic value to the solver} \\
\pmodel
  & $(\eStore{},\aStore{})$
  & $m$
  & \emph{Get a model for the current solver state}
\end{tabular}

  \caption{Assumed base solver primitive operations}%
  \label{fig:vsmt:inf:prim}
\end{figure}
%
\begin{figure}
  \begin{mathpar}
\inferrule*[right=\acRef]
  { \pwrap{\pspawn}(\aStore{},r) = (\aStore{}', s) }
  { (\aStore{}, r) \accumulation (\aStore{}', s) }

  \inferrule*[right=\acRefI]
  { \pwrap{\pspawni}(\aStore{},r_{i}) = (\aStore{}', s) }
  { (\aStore{}, r_{i}) \accumulation (\aStore{}', s) }

\inferrule*[right=\acNotS]
  { (\aStore{}, \eIL) \accumulation (\aStore{}', s) \\
    \pwrap{\pnot}(\aStore{}', s) = (\aStore{}'', s') }
  { (\aStore{}, \neg{} \eIL) \accumulation (\aStore{}'', s') }

  \inferrule*[right=\acNegS]
  { \op \in \integerUnary{} \\
    \lookup{\op}{\integerUnary{}} \coloneqq \pwrap{\op} \\
    (\aStore{}, \eAR) \accumulation (\aStore{}', s) \\
    \pwrap{\op}(\aStore{}', s) = (\aStore{}'', s') }
  { (\aStore{}, \op\; \eAR) \accumulation (\aStore{}'', s') }

\inferrule*[right=\acBoolS]
 { \op \in \boolFuncs{} \\
   \lookup{\op}{\boolFuncs{}} \coloneqq \pwrap{\op} \\
  (\aStore{}, \eIL*{1}) \accumulation (\aStore{1}, s_1) \\
    (\aStore{1}, \eIL*{2}) \accumulation (\aStore{2}, s_2) \\
    \pwrap{\op}(\aStore{2}, s_1, s_2) = (\aStore{3}, s_3) }
  { (\aStore{}, \eIL*{1}\; \op\; \eIL*{2}) \accumulation{} (\aStore{3}, s_3) }

  \inferrule*[right=\acArithS]
  { \op \in \integerFuncs{} \\
    \lookup{\op}{\integerFuncs{}} \coloneqq \pwrap{\op} \\
    (\aStore{}, \eAR*{1}) \accumulation (\aStore{1}, s_1) \\
    (\aStore{1}, \eAR*{2}) \accumulation (\aStore{2}, s_2) \\
    \pwrap{\op}(\aStore{2}, s_1, s_2) = (\aStore{3}, s_3) }
  { (\aStore{}, \eAR*{1}\; \op\; \eAR*{2}) \accumulation{} (\aStore{3}, s_3) }

  \inferrule*[right=\acInEqS]
  { \op \in\ \inequalities{} \\
    \lookup{\op}{\inequalities} \coloneqq \pwrap{\op} \\
    (\aStore{}, \eAR*{1}) \accumulation (\aStore{1}, s_1) \\
    (\aStore{1}, \eAR*{2}) \accumulation (\aStore{2}, s_2) \\
    \pwrap{\op}(\aStore{2}, s_1, s_2) = (\aStore{3}, s_3) }
  { (\aStore{}, \eAR*{1}\; \op\; \eAR*{2}) \accumulation{} (\aStore{3}, s_3) }

\inferrule*[right=\acChc]
  { }
  {(\aStore,\chc[D]{e_1,e_2}) \accumulation (\aStore{},\chc[D]{e_1,e_2})}

\inferrule*[right=\acChcI]
  { }
  {(\aStore,\chc[D]{ar_{1},ar_{2}}) \accumulation (\aStore{},\chc[D]{ar_{1},ar_{2}})}

\inferrule*[right=\acNotV]
  { (\aStore{}, \eIL) \accumulation (\aStore{}', \eIL') }
  { (\aStore{}, \neg{} \eIL) \accumulation (\aStore{}', \neg \eIL') }

  \inferrule*[right=\acNegV]
  { (\aStore{}, \eIL) \accumulation (\aStore{}', \eIL') }
  { (\aStore{}, \integerUnary \eIL) \accumulation (\aStore{}', \integerUnary \eIL') }

\inferrule*[right=\acBoolV]
  { (\aStore{}, \eIL_1) \accumulation (\aStore{1}, \eIL*{1}') \\
    (\aStore{1}, \eIL_2) \accumulation (\aStore{2}, \eIL*{2}') }
  { (\aStore{}, \eIL_1 \boolFuncs{} \eIL_2) \accumulation{} (\aStore{2}, \eIL*{1}' \boolFuncs{} \eIL*{2}') }

  \inferrule*[right=\acArithV]
  { (\aStore{}, \eAR*{1}) \accumulation (\aStore{1}, \eAR*{1}') \\
    (\aStore{1}, \eAR*{2}) \accumulation (\aStore{2}, \eAR*{2}') }
  { (\aStore{}, \eAR*{1} \integerFuncs{} \eAR*{2}) \accumulation{} (\aStore{2}, \eAR*{1}' \integerFuncs{} \eAR*{2}') }

  \inferrule*[right=\acInEqV]
  { (\aStore{}, \eAR*{1}) \accumulation (\aStore{1}, \eAR*{1}') \\
    (\aStore{1}, \eAR*{2}) \accumulation (\aStore{2}, \eAR*{2}') }
  { (\aStore{}, \eAR*{1} \inequalities{} \eAR*{2}) \accumulation{} (\aStore{2}, \eAR*{1}' \inequalities{} \eAR*{2}') }

\end{mathpar}

  \caption{Accumulation inference rules}%
  \label{fig:vsmt:inf:acc}
\end{figure}
%
\begin{figure}
  \begin{mathpar}
\inferrule*[right=\evAcc]
  { (\aStore{},\eIL) \accumulation (\aStore{}',\eIL') \\
    (\eStore{},\aStore{}',\eIL') \evaluation (\eStore{}',\aStore{}'',\eIL'') }
  { (\eStore{},\aStore{},\eIL) \evaluation (\eStore{}',\aStore{}'',\eIL'') }
% \qquad
\\
\inferrule*[right=\evSym]
  { \passert(\eStore{},\aStore{},s) = \eStore{}' }
  { (\eStore{},\aStore{},s) \evaluation (\eStore{}',\aStore{},\unit) }

\inferrule*[right=\evChc]
  { }
  { (\eStore{},\aStore{},\chc[D]{e_1,e_2}) \evaluation
    (\eStore{},\aStore{},\chc[D]{e_1,e_2}) }

\inferrule*[right=\evOr]
  { }
  { (\eStore{},\aStore{}, \eIL*{1} \vee \eIL*{2}) \evaluation
    (\eStore{},\aStore{}, \eIL*{1} \vee \eIL*{2}) }

  \inferrule*[right=\evInEq]
  { }
  { (\eStore{},\aStore{}, \eIL*{1} \inequalities \eIL*{2}) \evaluation
    (\eStore{},\aStore{}, \eIL*{1} \inequalities \eIL*{2}) }

\inferrule*[right=\evAndL]
  { (\eStore{},\aStore{},\eIL*{1}) \evaluation (\eStore{1},\aStore{1},\unit) \\
    (\eStore{1},\aStore{1},\eIL*{2}) \evaluation (\eStore{2},\aStore{2},\eIL*{2}') }
  { (\eStore{},\aStore{}, \eIL*{1} \wedge \eIL*{2}) \evaluation
    (\eStore{2},\aStore{2},\eIL*{2}') }

\inferrule*[right=\evAndR]
  { (\eStore{},\aStore{},\eIL*{1}) \evaluation (\eStore{1},\aStore{1},\eIL*{1}') \\
    (\eStore{1},\aStore{1},\eIL*{2}) \evaluation (\eStore{2},\aStore{2},\unit) }
  { (\eStore{},\aStore{}, \eIL*{1} \wedge \eIL*{2}) \evaluation
    (\eStore{2},\aStore{2},\eIL*{1}') }

\inferrule*[right=\evAnd]
  { (\eStore{},\aStore{},\eIL*{1}) \evaluation (\eStore{1},\aStore{1},\eIL*{1}') \\
    (\eStore{1},\aStore{1},\eIL*{1}) \evaluation (\eStore{2},\aStore{2},\eIL*{2}') }
  { (\eStore{},\aStore{}, \eIL*{1} \wedge \eIL*{2}) \evaluation
    (\eStore{2},\aStore{2}, \eIL*{1}' \wedge \eIL*{2}') }
\end{mathpar}%

  \caption{Evaluation inference rules}%
  \label{fig:vsmt:inf:eval}
\end{figure}
%
% \begin{figure}
  % \begin{mathpar}
\inferrule*[right=\crEval]
  { (\eStore{},\aStore{},v) \evaluation (\eStore{}',\aStore{}',\unit) \\
    \texttt{Combine}(\vmodel{},\pmodel(\aStore{},\eStore{})) = \vmodel{}' }
  { (\crCtx, \inRoot, v) \choiceRemoval \vmodel{}' }

\inferrule*[right=\crChcT]
  { (D,\true)\in C \\
    (\crCtx, z, e_1) \choiceRemoval \vmodel{}' }
  { (\crCtx, z, \chc[D]{e_1,e_2} \choiceRemoval \vmodel{}' }

\inferrule*[right=\crChcF]
  { (D,\false)\in C \\
    (\crCtx, z, e_2) \choiceRemoval \vmodel{}' }
  { (\crCtx, z, \chc[D]{e_1,e_2} \choiceRemoval \vmodel{}' }

\inferrule*[right=\crChc]
  { D\notin\dom{C} \\
    (C\cup(D,\true),\eStore{},\aStore{},\vmodel{}, z, e_1)
      \choiceRemoval \vmodel{1} \\
    (C\cup(D,\false),\eStore{},\aStore{},\vmodel{}', z, e_2)
      \choiceRemoval \vmodel{2} }
  { (\crCtx, z, \chc[D]{e_1,e_2} \choiceRemoval \vmodel{2} }

\inferrule*[right=\crNot]
  { (\crCtx, \inNot{z}, v) \choiceRemoval \vmodel{}' }
  { (\crCtx, z, \neg v) \choiceRemoval \vmodel{}' }

\inferrule*[right=\crNotIn]
  { (\aStore{}, \neg s) \accumulation (\aStore{}', s') \\
    (\crCtx, z, s') \choiceRemoval \vmodel{}' }
  { (\crCtx, \inNot{z}, s) \choiceRemoval \vmodel{}' }

\inferrule*[right=\crAnd]
  { (\crCtx, \inAndL{z}{v_2}, v_2) \choiceRemoval \vmodel{}' }
  { (\crCtx, z, v_1 \wedge v_2) \choiceRemoval \vmodel{}' }

\inferrule*[right=\crAndL]
  { (\crCtx, \inAndR{s}{z}, v) \choiceRemoval \vmodel{}' }
  { (\crCtx, \inAndL{z}{v}, s) \choiceRemoval \vmodel{}' }

\inferrule*[right=\crAndR]
  { (\aStore{}, s_1 \wedge s_2) \accumulation (\aStore{}', s_3) \\
    (\crCtx, z, s_3) \choiceRemoval \vmodel{}' }
  { (\crCtx, \inAndR{s_1}{z}, s_2) \choiceRemoval \vmodel{}' }

\inferrule*[right=\crOr]
  { (\crCtx, \inOrL{z}{v_2}, v_2) \choiceRemoval \vmodel{}' }
  { (\crCtx, z, v_1 \vee v_2) \choiceRemoval \vmodel{}' }

\inferrule*[right=\crOrL]
  { (\crCtx, \inOrR{s}{z}, v) \choiceRemoval \vmodel{}' }
  { (\crCtx, \inOrL{z}{v}, s) \choiceRemoval \vmodel{}' }

\inferrule*[right=\crOrR]
  { (\aStore{}, s_1 \vee s_2) \accumulation (\aStore{}', s_3) \\
    (\crCtx, z, s_3) \choiceRemoval \vmodel{}' }
  { (\crCtx, \inOrR{s_1}{z}, s_2) \choiceRemoval \vmodel{}' }
\end{mathpar}

  % \caption{Choice removal inference rules}%
  % \label{fig:vsmt:inf:chc}
% \end{figure}
