~\label{section:vsmt:primitives}
%
\begin{figure}
  \centering
\begin{tabular}{r@{~~:~~}l@{~~\OB{\to}~~}ll}
\pnot
  & $(\aStore{},s)$
  & $(\aStore{},s)$
  & \emph{Negate a symbolic value} \\
\pand
  & $(\aStore{},s,s)$
  & $(\aStore{},s)$
  & \emph{Conjunction of symbolic values} \\
\por
  & $(\aStore{},s,s)$
  & $(\aStore{},s)$
  & \emph{Disjunction of symbolic values} \\
\pneg
  & $(\aStore{},s)$
  & $(\aStore{},s)$
  & \emph{Negate an arithmetic symbolic value} \\
\padd
  & $(\aStore{},s,s)$
  & $(\aStore{},s)$
  & \emph{Add symbolic values} \\
\psub
  & $(\aStore{},s,s)$
  & $(\aStore{},s)$
  & \emph{Subtract symbolic values} \\
\pdiv
  & $(\aStore{},s,s)$
  & $(\aStore{},s)$
  & \emph{Divide symbolic values} \\
\pmult
  & $(\aStore{},s,s)$
  & $(\aStore{},s)$
  & \emph{Multiply symbolic values} \\
\plt
  & $(\aStore{},s,s)$
  & $(\aStore{},s)$
  & \emph{Less than over symbolic values} \\
\plte
  & $(\aStore{},s,s)$
  & $(\aStore{},s)$
  & \emph{Less than equals over symbolic values} \\
\pgt
  & $(\aStore{},s,s)$
  & $(\aStore{},s)$
  & \emph{Greater than over symbolic values} \\
\pgte
  & $(\aStore{},s,s)$
  & $(\aStore{},s)$
  & \emph{Greater than equals over symbolic values} \\
\peq
  & $(\aStore{},s,s)$
  & $(\aStore{},s)$
  & \emph{Arithmetic equivalence over symbolic values} \\
\pspawn
  & $(\aStore{},r)$
  & $(\aStore{},s)$
  & \emph{Create symbolic value based on a boolean variable} \\
\pspawni
  & $(\aStore{},r_{i})$
  & $(\aStore{},s)$
  & \emph{Create symbolic value based on a arithmetic variable} \\
\passert
  & $(\eStore{},\aStore{},s)$
  & $\eStore{}$
  & \emph{Assert a symbolic value to the solver} \\
\pmodel
  & $(\eStore{},\aStore{})$
  & $m$
  & \emph{Get a model for the current solver state}
\end{tabular}

  \caption{Syntactic categories of primitive operations}%
  \label{fig:vsmt:inf:prim}
\end{figure}
%
In order to construct a variational \ac{smt} solver we must first extend
\ac{vpl} to include non-Boolean values. \ac{vpl} included two kinds of
relations: relations such as $\neg$ and $\vee$ which required accumulation in
the presence of variation, and relations such as $\wedge$ which required no
special handling. Unfortunately, in the presence of variation there are no
relations such as $\wedge$ for the \ac{smt} theories. Thus we add support for
each theory except arrays through accumulation. Our strategy to extend \ac{vpl}
to \evpl{} is to add the appropriate cases to the syntax of \ac{vpl}, add the
requisite primitive operations, and then extend the inference rules of
accumulation and choice removal.

The \evpl{} syntax is presented in \autoref{fig:arith:stx}. \evpl{} includes
syntax of the integer arithmetic extension, which consists of integer variables,
integer literals, a set of standard operators, and choices.
%
The sets of Boolean and arithmetic variables are disjoint, thus an expression
such as $(\kf{s < 10) \wedge (s \vee p})$, where $s$ occurs as both an integer
and Boolean variable is disallowed.
%
The syntax of the language prevents type errors and expressions that do not
yield Boolean values. For example, $\chc[D]{1,2} \wedge p$ is syntactically
invalid.
%
Thus, the language is purposefully limited to arithmetic expressions that
have an inequality at the root of the expression, such as: $\kf{g} =
(\chc[A]{1, 2} + j \geq 2) \vee a \wedge \chc[A]{c, d}$.
%
Choices in the same dimension are synchronized across Boolean and arithmetic
sub-expressions, for example, the expression
%
$\kf{g} = (\chc[A]{1, 2} + j \geq 2) \vee (a \wedge \chc[A]{c,d})$
% %
represents two variants:
%
$\sem[\{(A, \true)\}]{g} = (1 + j \geq 2) \vee (a \wedge c)$ and
$\sem[\{(A, \false)\}]{g} = (2 + j \geq 2) \vee (a \wedge d)$.


%%% Local Variables:
%%% mode: latex
%%% TeX-master: "../../thesis"
%%% End: