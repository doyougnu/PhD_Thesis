\label{section:vsmt:primitives}%
%
\begin{figure}
  \centering
\begin{tabular}{r@{~~:~~}l@{~~\OB{\to}~~}ll}
\pnot
  & $(\aStore{},s)$
  & $(\aStore{},s)$
  & \emph{Negate a symbolic value} \\
\pand
  & $(\aStore{},s,s)$
  & $(\aStore{},s)$
  & \emph{Conjunction of symbolic values} \\
\por
  & $(\aStore{},s,s)$
  & $(\aStore{},s)$
  & \emph{Disjunction of symbolic values} \\
\pneg
  & $(\aStore{},s)$
  & $(\aStore{},s)$
  & \emph{Negate an arithmetic symbolic value} \\
\padd
  & $(\aStore{},s,s)$
  & $(\aStore{},s)$
  & \emph{Add symbolic values} \\
\psub
  & $(\aStore{},s,s)$
  & $(\aStore{},s)$
  & \emph{Subtract symbolic values} \\
\pdiv
  & $(\aStore{},s,s)$
  & $(\aStore{},s)$
  & \emph{Divide symbolic values} \\
\pmult
  & $(\aStore{},s,s)$
  & $(\aStore{},s)$
  & \emph{Multiply symbolic values} \\
\plt
  & $(\aStore{},s,s)$
  & $(\aStore{},s)$
  & \emph{Less than over symbolic values} \\
\plte
  & $(\aStore{},s,s)$
  & $(\aStore{},s)$
  & \emph{Less than equals over symbolic values} \\
\pgt
  & $(\aStore{},s,s)$
  & $(\aStore{},s)$
  & \emph{Greater than over symbolic values} \\
\pgte
  & $(\aStore{},s,s)$
  & $(\aStore{},s)$
  & \emph{Greater than equals over symbolic values} \\
\peq
  & $(\aStore{},s,s)$
  & $(\aStore{},s)$
  & \emph{Arithmetic equivalence over symbolic values} \\
\pspawn
  & $(\aStore{},r)$
  & $(\aStore{},s)$
  & \emph{Create symbolic value based on a boolean variable} \\
\pspawni
  & $(\aStore{},r_{i})$
  & $(\aStore{},s)$
  & \emph{Create symbolic value based on a arithmetic variable} \\
\passert
  & $(\eStore{},\aStore{},s)$
  & $\eStore{}$
  & \emph{Assert a symbolic value to the solver} \\
\pmodel
  & $(\eStore{},\aStore{})$
  & $m$
  & \emph{Get a model for the current solver state}
\end{tabular}

  \caption{Assumed base solver primitive operations for \evpl{}}%
  \label{fig:vsmt:primops}
\end{figure}
%
\begin{figure}
  \centering
  \begin{subfigure}[t]{\linewidth}
    \newcommand{\wrappedprimspacer}{\mbox{\hspace{1.8cm}} & \\[-3ex]}
\centering
\begin{align*}
\pwrap{\pspawni}(\aStore{},r_{i}) &=
                               \begin{cases}
                                 \wrappedprimspacer
                                 (\aStore{},s) & (r_{i},s) \in \aStore{} \\
                                 \pspawni(\aStore{},r_{i}) & \mathit{otherwise}
                               \end{cases} \\
\pwrap{\pneg}(\aStore{},s) &=
                               \begin{cases}
                                 \wrappedprimspacer
                                 (\aStore{},s') & (-s, s') \in \aStore{} \\
                                 \pneg(\aStore{},s) & \mathit{otherwise}
                               \end{cases} \\
\pwrap{\padd}(\aStore{},s_1,s_2) &=
                                 \begin{cases}
                                   \wrappedprimspacer
                                   (\aStore{},s_3) & (s_1 + s_2, s_3) \in \aStore{} \\
                                   \padd(\aStore{},s_1,s_2) & \mathit{otherwise}
                                 \end{cases} \\
\pwrap{\psub}(\aStore{},s_1,s_2) &=
                                \begin{cases}
                                  \wrappedprimspacer
                                  (\aStore{},s_3) & (s_1 - s_2, s_3) \in \aStore{} \\
                                  \psub(\aStore{},s_1,s_2) & \mathit{otherwise}
                                \end{cases} \\
\pwrap{\pdiv}(\aStore{},s_1,s_2) &=
                                 \begin{cases}
                                   \wrappedprimspacer
                                   (\aStore{},s_3) & (s_1 \div s_2, s_3) \in \aStore{} \\
                                   \pdiv(\aStore{},s_1,s_2) & \mathit{otherwise}
                                 \end{cases} \\
  \pwrap{\pmult}(\aStore{},s_1,s_2) &=
                                     \begin{cases}
                                       \wrappedprimspacer
                                       (\aStore{},s_3) & (s_1 * s_2, s_3) \in \aStore{} \\
                                       \pmult(\aStore{},s_1,s_2) & \mathit{otherwise}
                                     \end{cases}
\end{align*}

    \caption{Wrapped arithmetic primitives.}%
    \label{fig:vsmt:primops:arithmetic}
  \end{subfigure}
  \vfill
  \begin{subfigure}[t]{\linewidth}
    \newcommand{\wrappedprimspacer}{\mbox{\hspace{1.8cm}} & \\[-3ex]}
\centering
\begin{align*}
  \pwrap{\plt}(\aStore{},s_1,s_2) &=
                                     \begin{cases}
                                       \wrappedprimspacer
                                       (\aStore{},s_3) & (s_1 < s_2, s_3) \in \aStore{} \\
                                       \plt(\aStore{},s_1,s_2) & \mathit{otherwise}
                                     \end{cases} \\
  \pwrap{\plte}(\aStore{},s_1,s_2) &=
                                     \begin{cases}
                                       \wrappedprimspacer
                                       (\aStore{},s_3) & (s_1 \leq s_2, s_3) \in \aStore{} \\
                                       \plte(\aStore{},s_1,s_2) & \mathit{otherwise}
                                     \end{cases} \\
  \pwrap{\pgt}(\aStore{},s_1,s_2) &=
                                     \begin{cases}
                                       \wrappedprimspacer
                                       (\aStore{},s_3) & (s_1 > s_2, s_3) \in \aStore{} \\
                                       \pgt(\aStore{},s_1,s_2) & \mathit{otherwise}
                                     \end{cases} \\
  \pwrap{\pgte}(\aStore{},s_1,s_2) &=
                                    \begin{cases}
                                      \wrappedprimspacer
                                      (\aStore{},s_3) & (s_1 \geq s_2, s_3) \in \aStore{} \\
                                      \pgte(\aStore{},s_1,s_2) & \mathit{otherwise}
                                    \end{cases} \\
  \pwrap{\peq}(\aStore{},s_1,s_2) &=
                                     \begin{cases}
                                       \wrappedprimspacer
                                       (\aStore{},s_3) & (s_1 \equiv s_2, s_3) \in \aStore{} \\
                                       \peq(\aStore{},s_1,s_2) & \mathit{otherwise}
                                     \end{cases} \\
\end{align*}

    \caption{Wrapped inequality  primitives.}%
    \label{fig:vsmt:primops:inequality}
  \end{subfigure}
  \caption{Wrapped \ac{smt} primitives.}
\end{figure}
%
\begin{figure}
  \begin{syntax}
  % U_{\booleans{}} & : \booleans{} \rightarrow{} \booleans{} & \textit{Integers} \\
  \neg & \Coloneqq{} & \texttt{Not}    & \emph{Boolean negation} \\
  [1.5ex]
  % _{\booleans{}} & : \booleans{} \rightarrow{} \booleans{} & \textit{Integers} \\
  \integerUnary & \Coloneqq{} & \texttt{Negate} & \emph{Negation} \\
  [1.5ex]

  \boolFuncs & \Coloneqq{} & \texttt{And} & \textit{Conjunction} \\
                      &      |      & \texttt{Or}  & \textit{Disjunction} \\
  [1.5ex]

  \inequalities & \Coloneqq{} & \texttt{LT}   & \textit{Less than} \\
                      &     |       & \texttt{GT}   & \textit{Greater than} \\
                      &     |       & \texttt{LTE}  & \textit{Less than Equal} \\
                      &     |       & \texttt{GTE}  & \textit{Greater than Equal} \\
                      &     |       & \texttt{Eqv}  & \textit{Equivalency} \\
  [1.5ex]

  \integerFuncs & \Coloneqq{} & \texttt{Add}   & \textit{Addition} \\
                      &     |       & \texttt{Sub}   & \textit{Subtraction} \\
                      &     |       & \texttt{Mult}  & \textit{Multiplication} \\
                      &     |       & \texttt{Div}  & \textit{Division} \\
                      &     |       & \texttt{Mod}  & \textit{Modulus} \\
\end{syntax}

  \caption{Syntactic categories of primitive operations}%
  \label{fig:vsmt:categories}
\end{figure}
%
In order to construct a variational \ac{smt} solver we must first extend
\ac{vpl} to include non-Boolean values. \ac{vpl} included two kinds of
relations: relations such as $\neg$ and $\vee$ which required accumulation in
the presence of variation, and relations such as $\wedge$ which required no
special handling. Unfortunately, in the presence of variation there are no
relations such as $\wedge$ for the \ac{smt} theories. Thus we add support for
each theory except arrays through accumulation. Our strategy to extend \ac{vpl}
to \evpl{} is to add the appropriate cases to the syntax of \ac{vpl}, extend the
intermediate language, add the requisite primitive operations, and then extend
the inference rules of accumulation and choice removal.

The \evpl{} syntax is presented in \autoref{fig:arith:stx}. \evpl{} includes
syntax of the integer arithmetic extension, which consists of integer variables,
integer literals, a set of standard operators, and choices.
%
The sets of Boolean and arithmetic variables are disjoint, thus an expression
such as $(\kf{s < 10) \wedge (s \vee p})$, where $s$ occurs as both an integer
and Boolean variable is disallowed.
%
The syntax of the language prevents type errors and expressions that do not
yield Boolean values. For example, $\chc[D]{1,2} \wedge p$ is syntactically
invalid.
%
Thus, the language is purposefully limited to arithmetic expressions that
have an inequality at the root of the expression, such as: $\kf{g} =
(\chc[A]{1, 2} + j \geq 2) \vee a \wedge \chc[A]{c, d}$.
%
Choices in the same dimension are synchronized across Boolean and arithmetic
sub-expressions, for example, the expression
%
$\kf{g} = (\chc[A]{1, 2} + j \geq 2) \vee (a \wedge \chc[A]{c,d})$
% %
represents two variants:
%
$\sem[\{(A, \true)\}]{g} = (1 + j \geq 2) \vee (a \wedge c)$ and
$\sem[\{(A, \false)\}]{g} = (2 + j \geq 2) \vee (a \wedge d)$.

Similarly to \autoref{chapter:vsat}, we define the assumed primitive operations
of the base solver in \autoref{fig:vsmt:primops}, and wrapped versions for new
operators in \autoref{fig:vsmt:primops:arithmetic} and
\autoref{fig:vsmt:primops:inequality}. The wrapped versions are defined
identically as the wrapped primitives in \autoref{fig:vsat:primwrapped} and
serve the same purpose.
%
From the perspective of the variational solver, operations such as addition,
division, and subtraction only differ in the primitive operation emitted to the
base solver. Thus, we define syntactic categories over like operations in
\autoref{fig:vsmt:categories}. Notice that the categories correspond to the
respective type of each operation. For example, the boolean categories
encapsulate operations which take two boolean expressions and return a boolean
expression, similarly the inequality category encapsulate operators which take
numeric expressions and return boolean expressions. Further \ac{smt} extensions
would directly copy this pattern, that is, defining a syntactic category of
\rn{FixedSizeBitVector} or \rn{Reals} operators. Similarly, while we present
only a single arithmetic unary function $-$, other arithmetic unary functions
would be straightforward to add. For example, to include a modulus operator
$\kf{mod}$, one would define the wrapped primitive, and add the operator to the
appropriate syntactic category without requiring any modification to the
inference rules or intermediate languages.

Just as \ac{vpl} was extended the intermediate language must be extended. First
we must add cases for inequality operations, and second we must define an
intermediate language for the arithmetic domain.
%
\autoref{fig:vsmt:il} defines the intermediate arithmetic language \eAR, and the
extended intermediate language \eIL. The syntax of both intermediate languages
follow directly from \evpl and should be unsurprising. The only important
difference from IL is that \eAR cannot express a \unit{} value. This is a
purposeful design decision; recall that a \unit{} represents a term that has
been sent to the base solver. Thus if \unit{} were in \eAR then expressions such
as $\unit{} + 2$ would be expressible in \eAR, however because all arithmetic
formula's require accumulation the only possible result of
evaluation/accumulation on arithmetic expressions is either a choice or a
symbolic term, not a \unit{}. Hence, we syntactically avoid classes bugs by
eliding the \unit{} value in \eAR.
%
\begin{figure}
  \[
  \eIL \hquad \Coloneqq\hquad \unit{}
  \hquad|\hquad t
  \hquad|\hquad r
  \hquad|\hquad s
  \hquad|\hquad \neg \eIL
  \hquad|\hquad \eIL \boolFuncs{} \eIL
  \hquad|\hquad \eAR \inequalities{} \eAR 
  \hquad|\hquad \chc[D]{e,e}
\]
\[
  \eAR \hquad \Coloneqq\hquad i
  \hquad|\hquad r_{i}
  \hquad|\hquad s
  \hquad|\hquad \integerUnary \eAR
  \hquad|\hquad \eAR \integerFuncs \eAR
  \hquad|\hquad \chc[D]{ar,ar}
\]



% \begin{syntax}
%   \eIL & \Coloneqq{} & \unit{} & | & t   & | & r  & | & s      & | & \neg{} \eIL
%   & | & \eIL \boolFuncs{} \eIL & | & ar \inequalities ar & | & \chc[D]{f,f} \\
% \end{syntax}
% [1.5ex]
% \eAR & \Coloneqq{} & i       & | & r_{i} & | & s  & | & - \eAR & | & \eAR \integerFuncs{} \eAR & | & \chc[D]{f,f} 
  \caption{Extended intermediate language syntax}%
  \label{fig:vsmt:il}
\end{figure}

%%% Local Variables:
%%% mode: latex
%%% TeX-master: "../../thesis"
%%% End: