~\label{section:vsmt:primitives}
%
\begin{figure}
  \centering
\begin{tabular}{r@{~~:~~}l@{~~\OB{\to}~~}ll}
\pnot
  & $(\aStore{},s)$
  & $(\aStore{},s)$
  & \emph{Negate a symbolic value} \\
\pand
  & $(\aStore{},s,s)$
  & $(\aStore{},s)$
  & \emph{Conjunction of symbolic values} \\
\por
  & $(\aStore{},s,s)$
  & $(\aStore{},s)$
  & \emph{Disjunction of symbolic values} \\
\pneg
  & $(\aStore{},s)$
  & $(\aStore{},s)$
  & \emph{Negate an arithmetic symbolic value} \\
\padd
  & $(\aStore{},s,s)$
  & $(\aStore{},s)$
  & \emph{Add symbolic values} \\
\psub
  & $(\aStore{},s,s)$
  & $(\aStore{},s)$
  & \emph{Subtract symbolic values} \\
\pdiv
  & $(\aStore{},s,s)$
  & $(\aStore{},s)$
  & \emph{Divide symbolic values} \\
\pmult
  & $(\aStore{},s,s)$
  & $(\aStore{},s)$
  & \emph{Multiply symbolic values} \\
\plt
  & $(\aStore{},s,s)$
  & $(\aStore{},s)$
  & \emph{Less than over symbolic values} \\
\plte
  & $(\aStore{},s,s)$
  & $(\aStore{},s)$
  & \emph{Less than equals over symbolic values} \\
\pgt
  & $(\aStore{},s,s)$
  & $(\aStore{},s)$
  & \emph{Greater than over symbolic values} \\
\pgte
  & $(\aStore{},s,s)$
  & $(\aStore{},s)$
  & \emph{Greater than equals over symbolic values} \\
\peq
  & $(\aStore{},s,s)$
  & $(\aStore{},s)$
  & \emph{Arithmetic equivalence over symbolic values} \\
\pspawn
  & $(\aStore{},r)$
  & $(\aStore{},s)$
  & \emph{Create symbolic value based on a boolean variable} \\
\pspawni
  & $(\aStore{},r_{i})$
  & $(\aStore{},s)$
  & \emph{Create symbolic value based on a arithmetic variable} \\
\passert
  & $(\eStore{},\aStore{},s)$
  & $\eStore{}$
  & \emph{Assert a symbolic value to the solver} \\
\pmodel
  & $(\eStore{},\aStore{})$
  & $m$
  & \emph{Get a model for the current solver state}
\end{tabular}

  \caption{Assumed base solver primitive operations for \evpl{}}%
  \label{fig:vsmt:primops}
\end{figure}
%
\begin{figure}
  \centering
  \begin{subfigure}[t]{\linewidth}
    \newcommand{\wrappedprimspacer}{\mbox{\hspace{1.8cm}} & \\[-3ex]}
\centering
\begin{align*}
\pwrap{\pspawni}(\aStore{},r_{i}) &=
                               \begin{cases}
                                 \wrappedprimspacer
                                 (\aStore{},s) & (r_{i},s) \in \aStore{} \\
                                 \pspawni(\aStore{},r_{i}) & \mathit{otherwise}
                               \end{cases} \\
\pwrap{\pneg}(\aStore{},s) &=
                               \begin{cases}
                                 \wrappedprimspacer
                                 (\aStore{},s') & (-s, s') \in \aStore{} \\
                                 \pneg(\aStore{},s) & \mathit{otherwise}
                               \end{cases} \\
\pwrap{\padd}(\aStore{},s_1,s_2) &=
                                 \begin{cases}
                                   \wrappedprimspacer
                                   (\aStore{},s_3) & (s_1 + s_2, s_3) \in \aStore{} \\
                                   \padd(\aStore{},s_1,s_2) & \mathit{otherwise}
                                 \end{cases} \\
\pwrap{\psub}(\aStore{},s_1,s_2) &=
                                \begin{cases}
                                  \wrappedprimspacer
                                  (\aStore{},s_3) & (s_1 - s_2, s_3) \in \aStore{} \\
                                  \psub(\aStore{},s_1,s_2) & \mathit{otherwise}
                                \end{cases} \\
\pwrap{\pdiv}(\aStore{},s_1,s_2) &=
                                 \begin{cases}
                                   \wrappedprimspacer
                                   (\aStore{},s_3) & (s_1 \div s_2, s_3) \in \aStore{} \\
                                   \pdiv(\aStore{},s_1,s_2) & \mathit{otherwise}
                                 \end{cases} \\
  \pwrap{\pmult}(\aStore{},s_1,s_2) &=
                                     \begin{cases}
                                       \wrappedprimspacer
                                       (\aStore{},s_3) & (s_1 * s_2, s_3) \in \aStore{} \\
                                       \pmult(\aStore{},s_1,s_2) & \mathit{otherwise}
                                     \end{cases}
\end{align*}

    \caption{Wrapped arithmetic primitive.}%
    \label{fig:vsmt:primops:arithmetic}
  \end{subfigure}
  \vfill
  \begin{subfigure}[t]{\linewidth}
    \newcommand{\wrappedprimspacer}{\mbox{\hspace{1.8cm}} & \\[-3ex]}
\centering
\begin{align*}
  \pwrap{\plt}(\aStore{},s_1,s_2) &=
                                     \begin{cases}
                                       \wrappedprimspacer
                                       (\aStore{},s_3) & (s_1 < s_2, s_3) \in \aStore{} \\
                                       \plt(\aStore{},s_1,s_2) & \mathit{otherwise}
                                     \end{cases} \\
  \pwrap{\plte}(\aStore{},s_1,s_2) &=
                                     \begin{cases}
                                       \wrappedprimspacer
                                       (\aStore{},s_3) & (s_1 \leq s_2, s_3) \in \aStore{} \\
                                       \plte(\aStore{},s_1,s_2) & \mathit{otherwise}
                                     \end{cases} \\
  \pwrap{\pgt}(\aStore{},s_1,s_2) &=
                                     \begin{cases}
                                       \wrappedprimspacer
                                       (\aStore{},s_3) & (s_1 > s_2, s_3) \in \aStore{} \\
                                       \pgt(\aStore{},s_1,s_2) & \mathit{otherwise}
                                     \end{cases} \\
  \pwrap{\pgte}(\aStore{},s_1,s_2) &=
                                    \begin{cases}
                                      \wrappedprimspacer
                                      (\aStore{},s_3) & (s_1 \geq s_2, s_3) \in \aStore{} \\
                                      \pgte(\aStore{},s_1,s_2) & \mathit{otherwise}
                                    \end{cases} \\
  \pwrap{\peq}(\aStore{},s_1,s_2) &=
                                     \begin{cases}
                                       \wrappedprimspacer
                                       (\aStore{},s_3) & (s_1 \equiv s_2, s_3) \in \aStore{} \\
                                       \peq(\aStore{},s_1,s_2) & \mathit{otherwise}
                                     \end{cases} \\
\end{align*}

    \caption{Wrapped inequality  primitives.}%
    \label{fig:vsmt:primops:inequality}
  \end{subfigure}
  \caption{Wrapped \ac{smt} primitives.}
\end{figure}
%
\begin{figure}
  \begin{syntax}
  % U_{\booleans{}} & : \booleans{} \rightarrow{} \booleans{} & \textit{Integers} \\
  \neg & \Coloneqq{} & \texttt{Not}    & \emph{Boolean negation} \\
  [1.5ex]
  % _{\booleans{}} & : \booleans{} \rightarrow{} \booleans{} & \textit{Integers} \\
  \integerUnary & \Coloneqq{} & \texttt{Negate} & \emph{Negation} \\
  [1.5ex]

  \boolFuncs & \Coloneqq{} & \texttt{And} & \textit{Conjunction} \\
                      &      |      & \texttt{Or}  & \textit{Disjunction} \\
  [1.5ex]

  \inequalities & \Coloneqq{} & \texttt{LT}   & \textit{Less than} \\
                      &     |       & \texttt{GT}   & \textit{Greater than} \\
                      &     |       & \texttt{LTE}  & \textit{Less than Equal} \\
                      &     |       & \texttt{GTE}  & \textit{Greater than Equal} \\
                      &     |       & \texttt{Eqv}  & \textit{Equivalency} \\
  [1.5ex]

  \integerFuncs & \Coloneqq{} & \texttt{Add}   & \textit{Addition} \\
                      &     |       & \texttt{Sub}   & \textit{Subtraction} \\
                      &     |       & \texttt{Mult}  & \textit{Multiplication} \\
                      &     |       & \texttt{Div}  & \textit{Division} \\
                      &     |       & \texttt{Mod}  & \textit{Modulus} \\
\end{syntax}

  \caption{Syntactic categories of primitive operations}%
  \label{fig:vsmt:primops}
\end{figure}
%

In order to construct a variational \ac{smt} solver we must first extend
\ac{vpl} to include non-Boolean values. \ac{vpl} included two kinds of
relations: relations such as $\neg$ and $\vee$ which required accumulation in
the presence of variation, and relations such as $\wedge$ which required no
special handling. Unfortunately, in the presence of variation there are no
relations such as $\wedge$ for the \ac{smt} theories. Thus we add support for
each theory except arrays through accumulation. Our strategy to extend \ac{vpl}
to \evpl{} is to add the appropriate cases to the syntax of \ac{vpl}, add the
requisite primitive operations, and then extend the inference rules of
accumulation and choice removal.

The \evpl{} syntax is presented in \autoref{fig:arith:stx}. \evpl{} includes
syntax of the integer arithmetic extension, which consists of integer variables,
integer literals, a set of standard operators, and choices.
%
The sets of Boolean and arithmetic variables are disjoint, thus an expression
such as $(\kf{s < 10) \wedge (s \vee p})$, where $s$ occurs as both an integer
and Boolean variable is disallowed.
%
The syntax of the language prevents type errors and expressions that do not
yield Boolean values. For example, $\chc[D]{1,2} \wedge p$ is syntactically
invalid.
%
Thus, the language is purposefully limited to arithmetic expressions that
have an inequality at the root of the expression, such as: $\kf{g} =
(\chc[A]{1, 2} + j \geq 2) \vee a \wedge \chc[A]{c, d}$.
%
Choices in the same dimension are synchronized across Boolean and arithmetic
sub-expressions, for example, the expression
%
$\kf{g} = (\chc[A]{1, 2} + j \geq 2) \vee (a \wedge \chc[A]{c,d})$
% %
represents two variants:
%
$\sem[\{(A, \true)\}]{g} = (1 + j \geq 2) \vee (a \wedge c)$ and
$\sem[\{(A, \false)\}]{g} = (2 + j \geq 2) \vee (a \wedge d)$.

Similarly to \autoref{chapter:vsat}, we define the assumed primitive operations
of the base solver in \autoref{fig:vsmt:primops}, and wrapped versions for new
operators in \autoref{fig:vsmt:primops:arithmetic} and
\autoref{fig:vsmt:primops:inequality}. The wrapped versions are defined
identically as the wrapped primitives in \autoref{fig:vsat:primwrapped} and
serve the same purpose. Finally, from the perspective of the variational solver,
operations such as addition, division, and subtraction only differ in the
primitive operation emitted to the base solver. Thus, we define syntactic
categories over like operations to simplify the inference rules of accumulation,
choice removal, and evaluation in \autoref{fig:vsmt:primcategories}.

%%% Local Variables:
%%% mode: latex
%%% TeX-master: "../../thesis"
%%% End: