~\label{section:vsmt:evaluation}
%
\begin{figure}
  \begin{mathpar}
  %%% Computation rules
  \inferrule*[Right=Ev-Tm]
  { \texttt{Assert}((\eStore{}, \aStore{}), $t$) = \eStore{}' }
  %%% --------------------------------------------------------------
  {(\eStore{}, \aStore{}, t) \evaluation{} (\eStore{}', \aStore{}, \unit{}) }


  \inferrule*[Right=Ev-Sym]
  { \texttt{Assert}((\eStore{}, \aStore{}), s) = \eStore{}' }
  %%% --------------------------------------------------------------
  { (\eStore{}, \aStore{}, s) \evaluation{} (\eStore{}, \Delta, \unit{}) }
\\

\inferrule*[Right=Ev-Model]
{ \texttt{GetModel}(\aStore{}, \eStore{}) = m }
%%% --------------------------------------------------------------
{ (\aStore{}, \eStore{}, \unit{}) \evaluation{} m }
\\

\inferrule*[Right=Ev-Chc]
{ }
%%% --------------------------------------------------------------
{(\aStore{}, \eStore{}, \chc[D]{e_{1},e_{2}}) \evaluation{} (\aStore{}, \eStore{}, \chc[D]{e_{1},e_{2}})}
\end{mathpar}

\begin{mathpar}
  %%%\unit{} elimination rules
  \inferrule*[Right=Ev-UL]
  { \boolFuncs{} = \texttt{And}  }
  %%% --------------------------------------------------------------
  { (\Theta,\unit{} \boolFuncs{} v) \evaluation{} (\Theta, v) }


  \inferrule*[Right=Ev-UR]
  { \boolFuncs{} = \texttt{And} }
  %%% --------------------------------------------------------------
  { (\Theta, v \boolFuncs{} \unit{}) \evaluation{} (\Theta, v)}
\end{mathpar}

\begin{mathpar}
  \inferrule*[Right=Ev-BU]
  { (\aStore{},\neg{} v) \accumulation{} (\aStore{}', v') }
  %%% --------------------------------------------------------------
  { (\eStore{},\aStore{}, \neg{} v) \evaluation{} (\eStore{},\aStore{}', v') }


  \inferrule*[Right=Ev-IU]
  { (\aStore{}, \integerUnary{} v) \accumulation{} (\aStore{}', v') }
  %%% --------------------------------------------------------------
  { (\eStore{},\aStore{}, \integerUnary{} v) \evaluation{} (\eStore{},\aStore{}', v') }
\end{mathpar}

\begin{mathpar}
  \inferrule*[Right=Ev-And]
  { \boolFuncs{} = \texttt{And} \\
    (\Theta, v_{1}) \evaluation{} (\Theta', v_{1}') \\
    (\Theta', v_{2}) \evaluation{} (\Theta'', v_{2}')
  }
  %%% --------------------------------------------------------------
  { (\Theta, v_{1} \boolFuncs{} v_{2}) \evaluation{} (\Theta'', v_{1}' \boolFuncs{} v_{2}') }


    \inferrule*[Right=Ev-AccB]
    { \boolFuncs \neq \texttt{And} \\
      (\aStore{}, v_{1}) \accumulation{} (\aStore{}', v_{1}') \\
      (\aStore{}',v_{2}) \accumulation{} (\aStore{}'', v_{2}')
    }
%%% --------------------------------------------------------------
    { (\Gamma, \Delta, v_{1} \boolFuncs{} v_{2}) \evaluation{} (\Gamma, \Delta'', v_{1}'
      \boolFuncs{} v_{2}')}
\end{mathpar}%

\begin{mathpar}
  \inferrule*[Right=Ev-AccIB]
  {
    (\aStore{}, v_{1}) \accumulation{} (\aStore{}', v_{1}') \\
    (\aStore{}',v_{2}) \accumulation{} (\aStore{}'', v_{2}')
  }
  %%% --------------------------------------------------------------
  { (\Gamma, \Delta, v_{1} \inequalities{} v_{2}) \evaluation{} (\Gamma, \Delta'', v_{1}' \inequalities{} v_{2}')}
\end{mathpar}%
  \caption{Evaluation inference rules}%
  \label{fig:vsmt:inf:eval}
\end{figure}
%
\todo{make into subsection or paragraph? Only 1 page long...}Evaluation's
purpose is to assert symbolic terms in the base solver if it is safe to do so.
Thus, the extensions to evaluation are minimal as accumulation is performing the
majority of the work in creating the symbolic terms.

Variational \ac{smt} evaluation is defined in \autoref{fig:vsmt:inf:eval}. The
only change is the addition of \evInEq{} corresponding to the addition of
inequalities to \ac{vpl}. Just as \evOr{} skips over un-accumulated
disjunctions, \evInEq{} skips over un-accumulated inequalities as evaluating
inside an inequality is unsound in the base solver. Since evaluation calls
\evAcc{} to accumulate relations if \evAndL, \evAndR, and \evAnd don't apply,
variational \ac{smt} evaluation simply relies on accumulation to progress. The
special cases for conjunctions are maintained in order to sequence evaluation
from left to right to take advantage of the behavior of the assertion stack and
to propagate accumulation across conjunctions, just as in variational
satisfiability evaluation.


%%% Local Variables:
%%% mode: latex
%%% TeX-master: "../../thesis"
%%% End: