~\label{section:vsmt:evaluation}
%
\begin{figure}
  \begin{mathpar}
  %%% Computation rules
  \inferrule*[Right=Ev-Tm]
  { \texttt{Assert}((\eStore{}, \aStore{}), $t$) = \eStore{}' }
  %%% --------------------------------------------------------------
  {(\eStore{}, \aStore{}, t) \evaluation{} (\eStore{}', \aStore{}, \unit{}) }


  \inferrule*[Right=Ev-Sym]
  { \texttt{Assert}((\eStore{}, \aStore{}), s) = \eStore{}' }
  %%% --------------------------------------------------------------
  { (\eStore{}, \aStore{}, s) \evaluation{} (\eStore{}, \Delta, \unit{}) }
\\

\inferrule*[Right=Ev-Model]
{ \texttt{GetModel}(\aStore{}, \eStore{}) = m }
%%% --------------------------------------------------------------
{ (\aStore{}, \eStore{}, \unit{}) \evaluation{} m }
\\

\inferrule*[Right=Ev-Chc]
{ }
%%% --------------------------------------------------------------
{(\aStore{}, \eStore{}, \chc[D]{e_{1},e_{2}}) \evaluation{} (\aStore{}, \eStore{}, \chc[D]{e_{1},e_{2}})}
\end{mathpar}

\begin{mathpar}
  %%%\unit{} elimination rules
  \inferrule*[Right=Ev-UL]
  { \boolFuncs{} = \texttt{And}  }
  %%% --------------------------------------------------------------
  { (\Theta,\unit{} \boolFuncs{} v) \evaluation{} (\Theta, v) }


  \inferrule*[Right=Ev-UR]
  { \boolFuncs{} = \texttt{And} }
  %%% --------------------------------------------------------------
  { (\Theta, v \boolFuncs{} \unit{}) \evaluation{} (\Theta, v)}
\end{mathpar}

\begin{mathpar}
  \inferrule*[Right=Ev-BU]
  { (\aStore{},\neg{} v) \accumulation{} (\aStore{}', v') }
  %%% --------------------------------------------------------------
  { (\eStore{},\aStore{}, \neg{} v) \evaluation{} (\eStore{},\aStore{}', v') }


  \inferrule*[Right=Ev-IU]
  { (\aStore{}, \integerUnary{} v) \accumulation{} (\aStore{}', v') }
  %%% --------------------------------------------------------------
  { (\eStore{},\aStore{}, \integerUnary{} v) \evaluation{} (\eStore{},\aStore{}', v') }
\end{mathpar}

\begin{mathpar}
  \inferrule*[Right=Ev-And]
  { \boolFuncs{} = \texttt{And} \\
    (\Theta, v_{1}) \evaluation{} (\Theta', v_{1}') \\
    (\Theta', v_{2}) \evaluation{} (\Theta'', v_{2}')
  }
  %%% --------------------------------------------------------------
  { (\Theta, v_{1} \boolFuncs{} v_{2}) \evaluation{} (\Theta'', v_{1}' \boolFuncs{} v_{2}') }


    \inferrule*[Right=Ev-AccB]
    { \boolFuncs \neq \texttt{And} \\
      (\aStore{}, v_{1}) \accumulation{} (\aStore{}', v_{1}') \\
      (\aStore{}',v_{2}) \accumulation{} (\aStore{}'', v_{2}')
    }
%%% --------------------------------------------------------------
    { (\Gamma, \Delta, v_{1} \boolFuncs{} v_{2}) \evaluation{} (\Gamma, \Delta'', v_{1}'
      \boolFuncs{} v_{2}')}
\end{mathpar}%

\begin{mathpar}
  \inferrule*[Right=Ev-AccIB]
  {
    (\aStore{}, v_{1}) \accumulation{} (\aStore{}', v_{1}') \\
    (\aStore{}',v_{2}) \accumulation{} (\aStore{}'', v_{2}')
  }
  %%% --------------------------------------------------------------
  { (\Gamma, \Delta, v_{1} \inequalities{} v_{2}) \evaluation{} (\Gamma, \Delta'', v_{1}' \inequalities{} v_{2}')}
\end{mathpar}%
  \caption{Evaluation inference rules}%
  \label{fig:vsmt:inf:eval}
\end{figure}
%
Evaluation is defined in \autoref{fig:vsmt:inf:eval} as a relation of the form
(\aStore{}, \eStore{}, v) \evaluation{} (\aStore{}, \eStore{}, v), where
\eStore{} represents the base solver state. The rules \rn{EV-TM} and
\rn{Ev-Sym} push new clauses to the base solver using the primitive assert
operation. \rn{Ev-Model} calls for a plain model from the base solver,
only once a variant is fully reduced to \unit{}. \rn{Ev-Chc} skips
choices, \rn{Ev-UL} and \rn{Ev-UR} implement left and right unit,
reducing conjunctions where one side has been processed by the base solver. Of
special note is the difference between the \rn{Ev-AccB} and
\rn{Ev-And} rules. While \rn{Ev-And} is a straightforward congruence
rule, \rn{Ev-AccB} instead processes its arguments using accumulation
(\accumulation{}). Disjunctions are a source of back-tracking in variational
solving, and thus the solver cannot evaluate the left-hand side without
evaluating the right, both of which may contain choices, hence evaluation must
switch to accumulation, as we informally described in the previous subsection.
This problem is repeated for inequalities as well. \rn{Ev-AccIB} switches
to accumulation as one side of an inequality cannot be processed without
knowledge of the adjacent side. Thus, evaluation contains no rules for
arithmetic.


%%% Local Variables:
%%% mode: latex
%%% TeX-master: "../../thesis"
%%% End: