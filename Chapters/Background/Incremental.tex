~\label{section:background-incremental}
%
%
\begin{figure}[h]
  \begin{subfigure}[t]{.45\textwidth}
    \tikzstyle{block} = [draw,fill=blue!20,minimum size=2em, node distance=1cm]
% diameter of semicircle used to indicate that two lines are not connected
\def\radius{.7mm}
\tikzstyle{branch}=[fill,shape=circle,minimum size=3pt,inner sep=0pt]

\begin{tikzpicture}[>=latex']

        % blocks
        \node[block] at (2,-1) (prod) {$SAT$};
        \node[block, name=stging, below=3.75cm of prod] {$SAT$};
        \node[block, name=devel, below=2.0cm of stging] {$SAT$};

        % input nodes
        \node[left of = devel, xshift=-10pt] (input1) {\rV{}};
        \node[left of = stging, xshift=-10pt] (input2) {\qV{}};
        \node[left of = prod, xshift=-10pt] (input3) {\pV{}};

        % inputs
        \draw[->] (input3) -- (prod);
        \draw[->] (input2) -- (stging);
        \draw[->] (input1) -- (devel);

        % outputs
        \draw[->] (prod.east) -- +(1.0,0);
        \draw[->] (stging.east) -- +(1.0,0);
        \draw[->] (devel.east) -- +(1.0,0);


        % \node at (3.5,-3) (result1) {$result$};
        \node[right of = stging, xshift=38pt] (input2) {$result_{\qV{}}$};
        \node[right of = devel, xshift=38pt] (input2) {$result_{\rV{}}$};
        \node[right of = prod, xshift=38pt] (input3) {$result_{\pV{}}$};
\end{tikzpicture}
    \vspace{1.2em}
    \caption{Brute force procedure, no reuse between solver calls.}%
    \label{fig:bkg:bf}
  \end{subfigure}%
  \hfill
  \begin{subfigure}[t]{.45\textwidth}
    \tikzstyle{block} = [draw,fill=blue!20,node distance = 3.2cm, minimum size=2em]

% diameter of semicircle used to indicate that two lines are not connected
\tikzstyle{branch}=[fill,shape=circle,minimum size=3pt,inner sep=0pt]

\begin{tikzpicture}[>=latex']

    % Draw blocks, inputs and outputs

    % \foreach \y in {1,2,3,4,5} {
        % blocks
        \node[block] at (2,-1) (prod) {$SAT$};
        \node[block, name=stging, below=3.3cm of prod] {$SAT$};
        \node[block, name=devel, below=1.3cm of stging] {$SAT$};

        % input nodes
        % \node[left of = devel, xshift=-15pt] (input1) {$development$};
        % \node[left of = stging] (input2) {$\turnBlue{staging}$};
        \node[left of =prod,xshift=-10pt] (input3) {\pV{}};

        % inputs
        \draw[->] (input3) -- (prod);
        \draw[->] (prod) -- node[xshift = 60pt, align=left] {%
          $ \begin{aligned}
            \texttt{pop}  &\quad \eV{}\\
            \texttt{pop}  &\quad \cV{}\\
            \texttt{pop}  &\quad \bV{}\\
            \texttt{push} &\quad (\bV{} \vee \neg \iV{})  \\
            \texttt{push} &\quad \cV{} \\
            \texttt{push} &\quad (\gV{} \rightarrow \cV{}) \\
          \end{aligned}
          $
        } (stging);
\draw[->] (stging) -- node[xshift = 85pt, align=left] {%
  $\begin{aligned}
    & \texttt{resetAssertionStack}\\
    &\texttt{push} \quad \zV{} \leftrightarrow (a \wedge b \wedge c \wedge e)
    % \text{pop} &\quad \kf{pti} \rightarrow c_{i.1}\\
    % \text{pop} &\quad (\kf{spectre\_v2} \vee \kf{l1tf}) \leftrightarrow \\&\quad (c_{0} \wedge (\kf{nospec\_store\_bypass\text{-}disable} \rightarrow f_{j})\\
    % \text{pop} &\quad (c_{0.0} \wedge c_{1} \wedge \ldots c_{n})\\
  \end{aligned}
  $
} (devel);

        % outputs
        \draw[->] (prod.east) -- +(0.5,0);
        \draw[->] (stging.east) -- +(0.5,0);
        \draw[->] (devel.east) -- +(0.5,0);


        % \node at (3.5,-3) (result1) {$result$};
        % \node[right of = stging, xshift=15pt] (input2) {$result$};
        % \node[right of = devel, xshift=15pt] (input2) {$result$};
        % \node at (3.5,-1) (input3) {$result$};

        \node[right of = stging, xshift=18pt] (input2) {$result_{\qV{}}$};
        \node[right of = devel, xshift=18pt] (input2) {$result_{\rV{}}$};
        \node at (3.8,-1) (input3) {$result_{\pV{}}$};
    % }
    % \node[block] at (2,-6) (block6) {$f_6$};
    % \draw[->] (block6.east) -- +(0.5,0);

    % % Calculate branch point coordinate
    % \path (input1) -- coordinate (branch) (block1);

    % % Define a style for shifting a coordinate upwards
    % % Note the curly brackets around the coordinate.
    % \tikzstyle{s}=[shift={(0mm,\radius)}]
    % % It would be natural to use the yshift or xshift option, but that does
    % % not seem to work when shifting coordinates.

    % \draw[->] (branch) node[branch] {}{ % draw branch junction
    %         \foreach \c in {2,3,4,5} {
    %             % Draw semicircle junction to indicate that the lines are
    %             % not connected. The intersection between the lines are
    %             % calculated using the convenient -| syntax. Since we want
    %             % the semicircle to have its center where the lines intersect,
    %             % we have to shift the intersection coordinate using the 's'
    %             % style to account for this.
    %             [shift only] -- ([s]input\c -| branch) arc(90:-90:\radius)
    %             % Note the use of the [shift only] option. It is not necessary,
    %             % but I have used it to ensure that the semicircles have the
    %             % same size regardless of scaling.
    %         }
    %     } |- (block6);
\end{tikzpicture}
    \caption{Incremental procedure, reuse defined by \rn{pop} and \rn{push}.}%
      % sat calls share state that is determined by }
    \label{fig:bkg:inc}
  \end{subfigure}
  \caption{}%
  \label{fig:bkg}
\end{figure}
%
%
Suppose, we have three related propositional formulas that we desire to solve.
%
\begin{align*}
  p =\ a \wedge b \wedge c \wedge e && q=\ a \wedge (b \vee \neg \iV{}) \wedge c \wedge (\gV{} \rightarrow c) && r =~z \leftrightarrow (a \wedge b \wedge c \wedge e)
\end{align*}
%
\pV{} is simply a conjunction of variables. In \qV{}, relative to \pV{}, we see
that two variables are added, \iV{}, \gV{}, \eV{} is removed, and there are two
new clauses: $(b \vee \neg \iV{})$ and $(\gV{} \rightarrow c)$, both of which
possibly affect the values of \bV{} and \cV{}. In \rV{}, the variables and
constraints introduced in \pV{} are further constrained to a new variable,
\zV{}.

Suppose one wants to find a model for each formula. Using a non-incremental
\ac{sat} solver results in the procedure illustrated in \autoref{fig:bkg:bf};
where \ac{sat} solving is a batch process and no information is reused.
Alternatively, a procedure using an incremental \ac{sat} solver is illustrated
in \autoref{fig:bkg:inc}; in this scenario, all of the formulas are solved by
single solver instance where terms are programmatically added or removed from
the solver.

The ability to add and remove terms is enabled by manipulating the
\textit{assertion stack}. The previous example demonstrated a single level on
the stack, this example demonstrates three. The incremental interface provides
two new commands: \rn{push} to create a new variable \emph{scope} and add a
level to the stack and \rn{pop} to remove the level. The following \acl{smtlib}
program follows the procedure outlined in \autoref{fig:bkg:inc} and solves \pV,
\qV{} and \rV{}:

\begin{lstlisting}[columns=flexible,keepspaces=true,language=SMTLIB]
(declare-const a Bool)         ;; variable declarations for p
(declare-const b Bool)
(declare-const c Bool)
(declare-const e Bool)
(assert a)                     ;; a is shared between p and q
(push)                         ;; solve p
(assert e)
(assert c)
(assert b)
(check-sat)                    ;; check-sat on p
(pop)                          ;; remove e, c, and b assertions
(push)                         ;; solve for q
(declare-const i Bool)         ;; new variables
(declare-const g Bool)
(assert (or b (not i)))        ;; new clause
(assert c)                     ;; re-add c
(assert (=> g c))              ;; new clause
(check-sat)                    ;; check sat of q
(pop)                          ;; i and g out of scope
(reset)                        ;; reset the assertion stack
(declare-const a Bool)         ;; variable declarations for r
(declare-const b Bool)
(declare-const c Bool)
(declare-const e Bool)
(declare-const z Bool)
(assert (= z (and a (and b (and c (and e ))))))
(check-sat)                    ;; check-sat on r
\end{lstlisting}

In this example we begin by defining \pV{}, we assert $a$ outside of a new scope
so that it can be reused for \qV{}. We reuse $a$ by exploiting the conjunction
of assertions per level on the assertion stack during a \rn{check-sat} call. Had
we asserted \lstinline{(and a (and b (and c (and e))))} then we would not be
able to reuse the assertion on $a$. The first \rn{push} command enters a new
level on the assertion stack, the remaining variables are asserted and we issue
a check-sat call. After the \rn{pop} command, all assertions and declarations
from the previous level are removed. Thus, after we solve \qV{} the variables
$i$ and $g$ cannot be referenced as they are no longer in scope.

In an efficient process one would initially add as many \emph{shared} terms as
possible, such as $a$ from \pV{} and then reuse that term as many times as
needed. Thus, an efficient process should perform only enough manipulation of
the assertion stack as required to reach the next \ac{sat} problem of interest
from the current one. However, notice that doing so is not entirely straight
forward; we were only able to reuse $a$ from \pV{} in \qV{} because we defined
\pV{} in a non-intuitive way by utilizing the internal behavior of the assertion
stack. This situation is exacerbated by \ac{sat} problems such as \rV{} where we
were forced by the problem structure to completely remove everything on the
stack in order to construct \rV{}. \todo{this is motivation but seems like a
  good time to make this point?}Thus incremental \ac{sat} solvers provide the
primitive operations required to solve related \ac{sat} problems efficiently,
yet writing the \acl{smtlib} program to solve the set efficiently is not
straightforward.

%%% Local Variables:
%%% mode: latex
%%% TeX-master: "../../thesis"
%%% End: