\label{section:conclusion:cont-summary}

The main contribution of this work is the formalization of a method of
variational satisfiability solving using non-variational incremental \ac{sat}
solvers.
%
In \autoref{chapter:vpl} we formalized a many-valued logic to express
variational \ac{sat} problems, and demonstrated an application of the choice
calculus to propositional logic as the object language. We defined the
denotational semantics of the logic via configuration, and defined fundamental
concepts such as variants and synchronization.

In \autoref{chapter:vsat} we formalized our approach to variational
satisfiability solving based on this logic. Our approach is to variationalize
non-variational solvers by constructing a compiler to a standardized input
format. We saw that this approach has many desirable properties: First, the
stages of accumulation, evaluation, and choice removal cleanly separate
concerns. Second, sharing of plain terms is guaranteed between variants because
we use a zipper to capture evaluation contexts. Third, since our design
integrates plain base solvers, our variational solver can take advantage of
advances made by the \ac{sat} and \ac{smt} communities.
%
% Lastly, we proved that our design is variation preserving,
% thereby showing that the variational solver is sound up to the soundness of the
% base solver.

In \autoref{chapter:vsmt} we extended the architecture to handle non-Boolean
constraints. We saw that extensions over the term language follow a pattern: One
wraps the primitive base solver operations to handle symbolic values, then
defines a congruence rule to process the recur on the left child of the
relation, and finally defines a computation rule that calls the wrapped
primitive to combine two symbolic values, thereby producing a fold over the
relation. We presented two extensions, one over integer constraints, and one
over array based constraints. Since symbolic values are untyped, we carefully
constructed the extended logic to make type errors inexpressible. Lastly, we saw
that this extension pattern works even for background theories that seem
difficult such as arrays, because our architecture processes plain terms before
variational terms due to the ordering between evaluation, accumulation, and
choice removal.

In \autoref{chapter:case-studies}, we built two prototype variational solvers
called \vsat{} and \vsmt{}. We evaluated the solvers over two real-world
datasets. We observed that variational solving does produce speedups over
standard use of an incremental solver when solving many variants for these
datasets. The variational solvers produce this speedup by reusing shared terms
and avoiding redundant computation. Furthermore, we observed that the base
solver does have an impact on runtime performance. Therefore, an advantage of
our architecture is that it is base solver agnostic, and implementations may
choose whichever solver is performant for its problem domain as long as the
solver accepts the \acl{smtlib} standard. However, we found that when solving
only a single variant, variational solving does show a performance overhead that
was statistically significant for one dataset. Lastly, our finding that the
sharing ratio is positively correlated to runtime performance repeats similar
findings in the variational literature as described in
\autoref{chapter:related-work}.
