\label{chapter:conclusion}
%
This thesis has presented variational satisfiability and satisfiability-modulo
theory solving. In \autoref{chapter:introduction} we defined the success of this
thesis as applying the concept of variation in the domain of satisfiability
solving to create a variational satisfiability solver. The solver must
explicitly express the concept of variation in a user-facing language and must
be performent with respect to the performance of plain satisfiability solvers.
By this definition we have succeeded in demonstrating these ideas work in
practice in the domain of satisfiability solving. We have not only shown that
through the application of the choice calculus variation can be directly
expressed by the user, but also performance can be improved if local points of
variation are made explicit, at least for the two datasets we've assessed in
\autoref{chapter:case-studies}.
%
To conclude the thesis we review the important contributions in
\autoref{section:conclusion:cont-summary}.
\autoref{section:conclusion:future-work} provides immediate directions for
future work.

\section{Summary of Contributions}
\label{section:conclusion:cont-summary}

The main contribution of this work is the formalization of a method of
variational satisfiability solving using non-variational incremental \ac{sat}
solvers.
%
In \autoref{chapter:vpl} we formalized a many-valued logic to express
variational \ac{sat} problems, and demonstrated an application of the choice
calculus with propositional logic as the object language. We defined the
denotational semantics of configuration over the logic, and fundamental concepts
such as variants and synchronization.

In \autoref{chapter:vsat} we formalized our approach to variational
satisfiability solving based on this logic. Our approach is to variationalize
non-variational solvers by constructing a compiler to a standardized input
format. We saw that this approach has many desirable properties: The stages of
accumulation, evaluation, and choice removal cleanly separate concerns. The
sharing of plain terms is guaranteed between variants because we use a zipper to
capture evaluation contexts. Since our design integrates plain base solvers, our
variational solver can take advantage of advances made by the \ac{sat} and
\ac{smt} communities.
%
% Lastly, we proved that our design is variation preserving,
% thereby showing that the variational solver is sound up to the soundness of the
% base solver.

In \autoref{chapter:vsmt} we extended the architecture to handle non-Boolean
constraints. We saw that extensions over the term language follow a pattern: One
wraps the primitive base solver operations to handle symbolic values, then
defines a congruence rule to process the recur on the left child of the
relation, and finally defines a computation rule that calls the wrapped
primitive to combine two symbolic values, thereby producing a fold over the
relation. We presented two extensions; one over integer constraints, and one
over array based constraints. Since, symbolic values are untyped, we carefully
constructed the extended logic to make type errors inexpressible, we could have
otherwise chose to employ a simple type system as the \acl{smtlib} standard
does. Lastly, we saw that this extension pattern works even for background
theories that seem difficult, because our architecture processes plain terms
before variational terms due to the ordering between evaluation, accumulation
and choice removal.

In \autoref{chapter:case-studies}, we built two prototype variational solvers
called \vsat{} and \vsmt{}. We evaluated the solvers over two real-world
datasets. We observed that variational solving does produce speedups over
standard use of an incremental solver when solving many variants for these
datasets. The variational solvers produce this speedup by reusing shared terms
and avoiding redundant computation. Furthermore, we observed that the base
solver does have an impact on runtime performance. Therefore an advantage of our
architecture is that it is base solver agnostic, and implementations may choose
whichever solver is performent for its problem domain as long as the solver
accepts the \acl{smtlib} standard. However, we found that outside of its use
case---when solving only a single variant---variational solving does show a
performance overhead that was statistically significant for one dataset. Lastly,
our finding that the sharing ratio is positively correlated to runtime
performance repeats similar findings in the variational literature as described
in \autoref{chapter:related-work}.

%
\section{Future Work}
\label{section:conclusion:future-work}

There are numerous avenues of future work ranging from novel applications, to
refining the implementations, to extended the solvers with new features. In this
section we collect and discuss the most promising future work.

\subsection{Utilization of variational cores}
Variational cores are an important and foundational concept for the variational
solver's and consequently for the variationalization recipe. Recall that the
purpose of variational cores was twofold: First, to condense the query formula
such that the variational terms were the majority of terms core. Second, to
simplify the choice removal process by reducing the amount of traversal required
to process the choices. Third, to enforce sharing between variants as the
contexts captured by the core were are reused during choice removal.

This last point is key, because variational cores in combination with the
accumulation and evaluation stores, completely capture the context of a formula
they can be reused in novel ways. For example, one might serialize a variational
core and associated stores to disk, thus effectively caching the core for future
use. Such a feature would enable desirable user facing effects: the solver could
restart without losing information and thus might be useful for debugging or
exploration, if the variational cores require a lot of processing time to
generate this time be amortized, or if the application domain only builds on
previous versions of the same formulas, then the variational core could be
reused.

For example, consider the case of a feature model which evolves every month for
several months, similarly to the \fin{} and \auto{} datasets. Since the feature
model, and consequently the \ac{vpl} formula evolves over time, the previous
variational core could be modified to reflect the changes for the new formula.
Adding new constraints is straightforward; one would simply nest the previous
variational core in a conjunction context with the new core and reuse the
previous stores when generating the new core to ensure sharing. A more difficult
problem is removing constraints or variables in the previous core. Both removing
constraints and removing variables is problematic as the variable or constraint
could have been accumulated into a symbol value or several symbolic values. One
could traverse a dependency graph to find all references of the variable and
symbolic value, and then seek to replace those references with a unit value,
such as \tru{} for $\wedge$ or \fls{} for $\vee$. However this immediately leads
to the problematic case where the variable or symbolic to be removed is in a
$\neg$ context. There is no unit value where $\neg$ does not have meaning and
thus we cannot remove arbitrary variables from a variational core.

In addition to manipulating or storing variational cores, future variational
solvers might utilize them as a convenient messaging format. Throughout this
thesis, we have assumed, and only considered execution of all variants occurring
in a single base solver instance, however this need not be the case.

\subsection{Further \ac{smt} background theories and tool extensions}

\subsection{Automatic construction of a \ac{vpl} formula}

\subsection{Abstracting the variationalization recipe to other domains}


%%% Local Variables:
%%% mode: latex
%%% TeX-master: "../../thesis"
%%% End:
%

% \ac{sat} and \ac{smt} solvers are ubiquitous and powerful tools in computer
% science and software engineering. Incremental \ac{sat} and \ac{smt} provide an
% interface that supports solving many related problems efficiently. However, the
% interface could be automated and improved.

% The goal of this thesis is to explore the design and architecture of a
% variational satisfiability solver that automates and improves on the incremental
% \ac{sat} interface. Through the application of the choice calculus the interface
% can be automated for satisfiability problems, the solver interaction can be
% formalized and made asynchronous, and the solver is able to directly express
% variation in a problem domain.

% The thesis will present a complete approach to variational satisfiability and
% satisfiability modulo theory solving based on incremental solving. It will
% include a method to automatically encode a set of Boolean formulae into a
% variational propositional formula. A method for detecting the difficulty of
% solving such a variational propositional formula. A data set suitable for future
% research in the \ac{sat}, \ac{spl} and variation research communities. A
% variational satisfiability solver, an asynchronous variational satisfiable
% modulo theory solver and a proof of variational preservation.

%%% Local Variables:
%%% mode: latex
%%% TeX-master: "../../thesis"
%%% End: