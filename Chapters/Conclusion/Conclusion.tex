~\label{chapter:conclusion}
%
\section{Applications}
~\label{section:conclusion:applications}
%
Software variability, as explored in this paper, is a natural application domain
for our work. The variability of SPLs or configurable software is often reduced
to propositional logic~\citep{B05,CW07,MWCC08} for analysis
purposes~\citep{BSRC10,TAK+:CSUR14,GBT+19}. Many analyses have been implemented
using \ac{sat} solving such as~\cite{TAK+:CSUR14}, including feature-model
analysis~\citep{BSRC10,GBT+19}, parsing~\citep{KGR+:OOPSLA11}, dead-code
analysis~\citep{TLSS:EuroSys11}, code simplification~\citep{RGA+:ICSE15}, type
checking~\citep{TBKC07}, consistency checking~\citep{CP06}, dataflow
analysis~\citep{LKA+:ESECFSE13}, model checking~\citep{CCS+13},
variability-aware execution~\citep{NKN:ICSE14}, testing~\citep{MMCA:IST14},
product sampling~\citep{MKR+:ICSE16,VAT+:SPLC18}, product
configuration~\citep{SIMA:ASE13}, optimization of non-functional
properties~\citep{SRK+:SQJ12}, and variant-preserving
refactoring~\citep{FMS+:SANER17}. While each of these analyses gives rise to
multiple \ac{sat} problems for even a single analysis run, the authors typically
do not discuss how they are solved. We argue that many could benefit from
variational solving.

More generally, any scenario that involves solving many related \ac{sat}
problems, and where all of these problems are known or can be generated in
advance, is a potential application for variational \ac{sat} solving.
%
Such situations arise in program analysis~\citep{VGD:FSE12}, and especially in
\emph{speculative} program analyses that involve generating and exploring huge
numbers of variations of a program, for example, as in
counterfactual~\citep{CE14popl} and migrational~\citep{CCW18icfp,CCEW18popl}
typing. Furthermore, we believe that variational solving could provide a basis
for similar speculative analyses on feature models.


%%% Local Variables:
%%% mode: latex
%%% TeX-master: "../../thesis"
%%% End:
%
\section{Future Work}
\label{section:conclusion:future-work}

There are numerous avenues of future work ranging from novel applications, to
refining the implementations, to extended the solvers with new features. In this
section we collect and discuss the most promising future work beginning with
tool extensions and ending with abstracting this work to domains other than
satisfiability solving.

\subsection{Utilization of Variational Cores}
Variational cores are an important and foundational concept for the variational
solver and consequently for the variationalization recipe. Recall that the
purpose of variational cores was threefold: First, to condense the query formula
such that the variational terms were the majority of terms in the core. Second,
to simplify the choice removal process by reducing the amount of traversal
required to process the choices. Third, to enforce sharing between variants as
the contexts captured by the core were are reused during choice removal.

This last point is key, because variational cores in combination with the
accumulation and evaluation stores, completely capture the context of a formula
they can be reused in novel ways. For example, one might serialize a variational
core and associated stores to disk, effectively caching the core for future use.
Such a feature would enable desirable user facing features: the solver could
restart without losing information and thus might be useful for debugging or
exploration, if the variational cores require a lot of processing time to
generate this time would be amortized, or if the application domain only builds
on previous versions of the same formulas, then the variational core could be
consistently reused for every new version.

For example, consider the case of a feature model which evolves every month for
several months, similarly to the \fin{} and \auto{} datasets. Since the feature
model, and consequently the \ac{vpl} formula evolves over time, the previous
variational core could be modified to reflect the changes for the new formula.
Adding new constraints is straightforward; one would simply nest the previous
variational core in a conjunction context (\inAndL{\kf{core}}{\kf{~new}}) with
the new core and reuse the previous stores when generating the new core to
ensure sharing. A more difficult problem is removing constraints or variables in
the previous core. Both removing constraints and removing variables is
problematic as the variable or constraint could have been accumulated into a
symbol value or several symbolic values. One could traverse a dependency graph
to find all references of the variable and symbolic value, and then seek to
replace those references with a unit value, such as \tru{} for $\wedge$ or
\fls{} for $\vee$. However this immediately leads to the problematic case where
the variable or symbolic to be removed is in a $\neg$ context. There is no unit
value where $\neg$ does not have meaning and thus we cannot remove arbitrary
variables from a variational core.

In addition to manipulating or storing variational cores, future variational
solvers might utilize them as a convenient messaging format. Throughout this
thesis, we have assumed and have only considered systems which process all
variants in a single base solver instance, however this need not be the case.
Instead, when a choice in focus during choice removal one might choose to solve
the true alternative variants in a different solver and all the false
alternatives in the same solver. For example, a user might know that all true
alternative variants have particularly good performance characteristics for
boolector, while all false variants have good characteristics for yices. Since
we compile to \acl{smtlib} script such a feature is possible with few changes to
our method of variational solving. To add such a feature, a future variational
solver would allow the user to select particular solvers over the input \vc{} or
the configuration for a query formula.

\subsection{Further \ac{smt} background Theories and Tool Extensions}
\ac{sat} and \ac{smt} solvers are attractive targets for research on variational
languages. As of this writing, designing a language with variational
side-effects is an open research problem. The essential problem is tracking
effects for particular variants across the interface between a variational-aware
system and a plain system. For example, imagine writing a file to disk in one
variant and deleting a different file in another variant. Since the file system
has no concept of variation or variant, the variational system is not able to
guarantee variants are isolated, and therefore variants may interact in
undesirable and difficult to predict ways. \ac{sat} and \ac{smt} solvers side
step this limitation as they are side-effect free systems. There is simply no
way to read a file from disc in an \acl{smtlib} script. Similarly, classes of
traditional run-time errors, such as dividing by zero, are not possible. If a
script divides by zero then the script will not simply not unify and an
\rn{unsat} will be returned.

Due to the attractive properties of \ac{sat} and \ac{smt} solvers for
variational research, a straightforward avenue of future work is to continue to
investigate efficient variational folds by further extending the variational
solvers. Modern \ac{sat} and \ac{smt} solvers allow quantified constraints
following first-order logic. In this thesis, we have only considered
unquantified constraints, and thus the interaction of between quantified
constraints and choices is an open research problem.

Similarly, we have demonstrated extensions for core background theories, but
there are many features of plain solvers that would be desirable additions to
variational solvers. Such features include generation of variational
unsatisfiable-cores. An unsatisfiable core is a subset of constraints that
prevent the \ac{sat} or \ac{smt} from unifying. Unsatisfiable cores are
desirable for many problems. For example, one might desire to find the clique in
a \ac{sat} encoded weighted graph which prevents a traversal under some cost
limit. Or one might desire to find the sub-set of features in a feature model
that prevent classes of products from being built.

Enabling variational unsatisfiable cores is possible with our approach of
accumulation, evaluation and choice removal. The key requirement would be to
ensure that the plain, \rn{get-unsat-core} command occurs inside the
\rn{push}/\rn{pop} block for a given variant. Thus far we have only seen the
\rn{get-model} command have this property. So a straightforward extension is to
create a syntactic category that contains useful plain commands in this context,
such as \rn{get-model} or \rn{get-unsat-core}, which would be issued to the base
solver once a variant has been reduced to \unit{}. Another approach is to create
full fledged variational \acl{smtlib} language, instead of expressions of
variational constraints as we have presented here. Constructing such a
variational \acl{smtlib} language is likely to save work for future extensions.
The language would be identical to \acl{smtlib} except that \rn{push}/\rn{pop}
would not be exposed to the user (or would only be enabled with an option), and
choices would be included in the language just as we have included the for
\ac{vpl} and \evpl{}.

Lastly, a promising area of future work is constructing an asynchronous
variational \ac{sat} and \ac{smt} solver. During our experience bench-marking
the variational prototype solvers we found that the majority of the time spent
in the base solver is spent querying for a model. Furthermore, each variant
waits until they can be processed by the base solver. For example, consider the
formula $\fV{} = \kf{\chc[A]{a,b} \wedge{} \chc[B]{c,d}}$, \fV{} has four
satisfiable variants. Our prototype solvers choose true alternatives first
(recurring down the left child of a relation), thus the order of the variants in
the base solver will be \sem[\set{(A,\tru{}), (B,\tru{})}]{f},
\sem[\set{(A,\tru{}), (B, \fls{})}]{f}, \sem[\set{(A,\fls{}), (B,\tru{})}]{f},
\sem[\set{(A, \fls{}), (B, \fls{})}]{f}. Notice that each false variant waits
for its true variant before being considered, for example every variant with
$\set{(A,\fls{})} \in C$ is processed \emph{after} variants where
$\set{(A,\tru{})} \in C$, and similarly so for the $\kf{B}$ dimension. Due to
this ordering, the runtime cost of solving false variants includes the cost of
solving the true variants, unless the variation context excludes true variants.
However the problem is tractable, instead of using \rn{push} and \rn{pop} to
represent variation, we could instead fork a new solver thread and solve all
$(A, \fls{})$ variants on that solver thread, or mix independent solver
instances and incremental solving.

We have created three versions of asynchronous prototype solvers but have not
succeeded in constructing a generalized sound asynchronous variational solver,
and thus do not provide a formalization. Constructing an asynchronous solver is
relatively straightforward. Since variational models form monoids, the order in
which plain models are added to the variational model isn't important.
Similarly, since variational cores capture the evaluation context at a given
time, transmitting variational cores to other solver instances is also
straightforward.

The problem for asynchronous solvers is ensuring that the ordering of
alternatives is maintained and consequently that variants remain isolated from
each other. For example, a simple model might be to have a pool of producer base
solver instances and a pool of consumers instances. The producer instances could
derive variational cores, and the consumers would take a variational core and a
configuration, and find the next choice that is not in the configuration or
generate a model. The two pool model's appeal is its simplicity, however a
subtle bugs are introduced due to the interaction between variation and
asynchronous workloads.

Assume we have a formula with three unique dimensions $A$, $B$, and $C$ which
will be processed in that order, \ie{} the same order as the variants of \fV{}
above. Since the order of alternatives is no longer deterministic we might
encounter a case where we are stuck or have mixed variants. Consider the case
where there are an unbalanced number of consumers and producers, with consumers
significantly outnumbering produces. Now consider a scenario where a consumer
thread has consumed the \set{(A, \tru{}), (B,\fls{})} core and then finds a
choice with a $C$ dimension. This thread must wait for a request from a producer
thread to mutate its local configuration, thereby configuring for an alternative
and continuing to solve. Suppose the consumer observes a request to consume
\set{(C,\tru{})}, does so, and produces a model for that variant. Now, the
consumer will backtrack with a \rn{pop} call and wait for another request from a
producer for $\kf{(C, \fls{})}$. However, this is an asynchronous environment
and so this thread may have out paced other threads. Thus the next request might
be to consume \set{(B,\tru{})}, and now we are stuck. If the consumer accepts
the request we will have mixed two variants, \set{(B,\fls{})} and \set(B,\tru{})
on this thread yielding incorrect results, if the consumer does not take the
request then we could end in a deadlock if the scenario is repeated for each
consumer.

Such an example is contrived but occurs with asynchronous communication and must
be accounted for. The fix is for each thread to track which variant it has
solved and maintain a stack to track the ordering of choices. We must ensure
that the choices are solved in order such that if a request comes to solve a
\set{(A, \tru{})} variant, and the thread has consumed the variational core with
\set{(A, \fls{})} then the thread must issue as many \rn{pop}s as needed to
backtrack. By tracking this information we can avoid deadlocks, and malformed
variants and still gain the benefits of concurrent solving which could be
substantial especially for large variational formulas. Whether the performance
gains outweigh the costs is an open research problem. It simply could be the
case that the runtime cost of forking, inter-process communication, and the cost
of avoiding poor performing scenarios, such as more than one pop, does not
outweigh the performance gains from asynchronously finding plain models.

\subsection{Automated \ac{vpl} Formulas}
Thus far we have only considered a \ac{vpl} or \evpl{} formula as input to a
variational solver. This format is likely to be inconvienient as end-users
consider sets of \ac{sat} problems. Thus, a useful extension for these users is
to change the input from a \ac{vpl} formula to a set of \ac{sat} problems the
user is interested in. With the set of \ac{sat} problems, one could synthesize a
\ac{vpl} formula with a sharing ratio that is \emph{good enough} and then run
the solver on that \ac{vpl} formula. For the rest of this section, we'll refer
to the problem of synthesizing a \emph{good} \ac{vpl} formula from a set of
\ac{sat} formulas the \emph{synthesis problem}.

There are several considerations to highlight. First, we found that the sharing
ratio of a formula positively correlates to run-time performance in
\autoref{chapter:case-studies}, echoing results from previous research on
variation. Therefore, the synthesis algorithm should try to maximize the sharing
ratio as it chooses which variants to combine in a choice. Second, minimizing
the number of choices is high priority for the algorithm. Our results indicate
that the run-time of the variational solver grows linearly in the number of
variants to solve (hence exponentially in the number of unique dimensions), thus
adding a single new choice doubles the number of variants and the expected
run-time. Rather than provide an algorithm that find the \emph{best} \ac{vpl}
formula, we instead describe a greedy algorithm that tries to find a reasonable
\ac{vpl} formula. An algorithm that finds the \emph{best} \ac{vpl} formula,
\eg{} one which maximizes the sharing ratio while minimizing the number of
choices is an open research problem. We suspect it is at least NP-hard (likely
by demonstrating that the Binary Decision Diagram variable ordering problem karp
reduces to the \ac{vpl} synthesis problem), although we have not begun to
investigate the problem space.

The synthesis problem is a search problem over a total undirected graph of
possible formula combinations. Each vertex in the graph is a \ac{sat} formula or
\ac{vpl} formula that can be combined and is connected to every other vertex.
Edges represent the possible combinations of two vertices and is weighted with a
\emph{fitness metric} indicating a good or bad match. Good or bad in this domain
indicates a high degree of sharing between two vertices. Our approach is to
traverse the graph and greedily select the best combination between two
vertices. Combinations mutate the graph. The old vertices are replaced with the
combined vertex, the old edges of the combined vertices are relaxed and new
edges connect the combined vertex to every other vertex in the graph. The
algorithm then repeats until only a single vertex remains in the graph.

There are two sub-procedures in the algorithm: A procedure to combine vertices,
and a procedure to calculate the fitness metric between two vertices. To combine
two vertices we generate a unique dimension, nest one vertex in the true
alternative, and the other in the false alternative. This simple combination
procedure results in poor sharing as the choice will always be at the root of
the abstract syntax tree of the resulting \ac{vpl} formula. Thus to increase the
sharing ratio, we drive the choice towards the leaves of the abstract syntax
tree of the \ac{vpl} formula using the equivalency laws in \autoref{fig:cc:eqv}.

Next we need a procedure that inputs two \ac{sat} or \ac{vpl} formulas, and
returns a fitness metric. There are several possible algorithms; ranging from
string edit distance, to a tree edit distance over the abstract syntax trees of
the \ac{sat} or \ac{vpl} formulas. String comparison algorithms such Levenshtein
distance\cite{Levenshtein_SPD66} or Hamming distance~\cite{H:BST50} are
promising as both have implementations which run in polynomial time, assuming an
encoding from the \ac{sat} problems to strings is computationally feasible.
Graph edit distance is a more direct approach but is NP-Complete with an
approximate solution that is APX-hard~\cite{hardnessOfGraphEditDistance}.
However, most edit distance algorithms work well in practice, and it is likely
that the graph comparisons in this domain are simpler than comparisons which
occur in the worst case, \eg{}, over enormous graphs such as those found in
social networks. Furthermore there are many heuristics such as longest-common
sub-string which might produce metrics that are good enough for reasonable
sharing ratios. The exact design of the informal algorithm described here is
left as an open research problem.

\subsection{Abstracting the Variationalization Recipe to Other Domains}
Our approach to creating a variation-aware system by using the plain version of
that system is not specific to satisfiability solvers. The only portion of our
work that is particular to satisfiability solvers is code generation in the base
solver. In essence, our method is a variational left-fold over a variational
language. Thus, one might reuse the ideas of accumulation, evaluation, choice
removal and variational cores in other domains. In particular, the recipe for
variationalization to other domains is clear: To variationalize a plain system
one needs to define the variational artifact for the domain, and a method to
express variation in that system. Our variational artifact was a \ac{vpl}
formula and we chose to use scopes from the \acl{smtlib} standard to express
variation in the plain \ac{sat} solver. Then, one needs a method to express
segments of plain terms and preserve sharing between variants in the plain
system, our approach was to define symbolic values and utilize the internal
cache of the plain solvers to preserve sharing. Lastly, one needs a way to
retrieve results and combine plain results in any order, just as we defined
monoidal variational models.

Using this recipe one can imagine a variational prolog which reuses the work
presented in this thesis. For such a language, the variational artifact would be
a prolog-like programming language with choices. Expressing segments of plain
terms with symbolic values could be directly reused from this thesis. Similarly,
the variational result would be nearly identical to the variational models
presented in \autoref{section:vsat:models}. Embedding variation in prolog is the
difficult part although there are several possibilities.
SWI-prolog~\cite{wielemaker:2011:tplp} defines a special kind of predicate
called \emph{dynamic predicates}. Dynamic predicates indicate to the prolog
interpreter that the predicate may change during execution. Changing the
predicate during execution is performed using two primitives, \emph{assertz} and
\emph{retract}. Thus prolog defines a way to assert a constraint in the
interpreter and then refine the constraint as needed, and so dynamic predicates
may serve as a viable primitive for variation in the prolog interpreter. Another
promising embedding is using delimited continuations. In \autoref{chapter:vsat}
we hypothesized that because a Heut zipper is used for choice removal, using
delimited continuations is also feasible as Huet zippers have been shown to be
isomorphic to delimited continuation~\cite{olegZippers}. Fortunately prolog has
first class support for delimited
continuations~\cite{DBLP:journals/tplp/SchrijversDDW13} and thus choice removal
could be done in the base prolog interpreter rather than at the variation-aware
level. Using delimited continuations could greatly reducing the complexity of
creating a variational prolog, so much so that it might be possible to define
variational prolog as a library rather than a separate entity. The exact details
for a variational implementation are not clear but creating a variational prolog
is a promising avenue of future work.

%%% Local Variables:
%%% mode: latex
%%% TeX-master: "../../thesis"
%%% End:
%
\section{Summary of Contributions}
\label{section:conclusion:cont-summary}

The main contribution of this work is the formalization of variational
satisfiability solver. In \autoref{chapter:introduction} we defined the success
of this thesis as applying the concept of variation in the domain of
satisfiability solving to create a variational satisfiability solver. The solver
must explicitly expresses the concept of variation in a user-facing language and
must be reasonably performent with respect to the performance of plain
satisfiability solvers. By this definition we have succeeded.\todo{make sure
  this was defined in intro}

In \autoref{chapter:vpl} we formalized a many-valued logic to express
variational \ac{sat} problems, demonstrated an application of the choice
calculus with propositional logic as the object language. We defined the
denotational semantics of configuration over the logic, and fundamental concepts
such as variants and synchronization.

In \autoref{chapter:vsat} we formalized our approach to variational
satisfiability solving based on this logic. Our approach is to variationalize
non-variational solvers by constructing a compiler to a standardized input
format. We saw that this approach has many desirable properties: The stages of
accumulation, evaluation, and choice removal cleanly separate concerns. The
sharing of plain terms is guaranteed between variants because we use a zipper to
capture evaluation contexts. Since our design integrates plain base solvers, our
variational solver can take advantage of advances made by the \ac{sat} and
\ac{smt} communities. Lastly, we proved that our design is confluent, thereby
showing that the variational solver is variation-preserving and thus sound.

In \autoref{chapter:vsmt} we extended the architecture to handle non-Boolean
constraints. We saw that extensions over the term language follow a pattern: One
wraps the primitive base solver operations to handle symbolic values, then
defines a congruence rule to process the recur on the left child of the
relation, and finally defines a computation rule that calls the wrapped
primitive to combine two symbolic values thereby producing a fold over the
relation. We presented two extensions; one over integer constraints, and one
over array based constraints. Since, symbolic values are untyped, we carefully
constructed the extended logic to make type errors inexpressible, we could have
otherwise chose to employ a simple type system as the \acl{smtlib} standard
does. Lastly, we saw that this extension pattern works even for background
theories that seem difficult, because our architecture processes plain terms
before variational terms due to the ordering between accumulation and choice
removal.

In \autoref{chapter:case-studies}, we built two prototype variational solvers
called \vsat{} and \vsmt{}. We evaluated the solvers over two real-world
datasets. We observed that variational solving does produce speedups over
standard use of an incremental solver when solving many variants. The
variational solvers produce this performance speedup by reusing shared terms and
avoiding redundant computation. We also observed that the base solver does have
an impact on runtime performance. Therefore an advantage of our method is that
it is base solver agnostic, and thus implementations can choose whichever
\acl{smtlib} solver is performent for its problem domain. However, we found that
outside of its use case---when solving only a single variant---variational
solving did show a performance overhead that was statistically significant for
one dataset. Lastly, our finding that the sharing ratio is positively correlated
to runtime performance repeats similar findings in the variational literature.


% \ac{sat} and \ac{smt} solvers are ubiquitous and powerful tools in computer
% science and software engineering. Incremental \ac{sat} and \ac{smt} provide an
% interface that supports solving many related problems efficiently. However, the
% interface could be automated and improved.

% The goal of this thesis is to explore the design and architecture of a
% variational satisfiability solver that automates and improves on the incremental
% \ac{sat} interface. Through the application of the choice calculus the interface
% can be automated for satisfiability problems, the solver interaction can be
% formalized and made asynchronous, and the solver is able to directly express
% variation in a problem domain.

% The thesis will present a complete approach to variational satisfiability and
% satisfiability modulo theory solving based on incremental solving. It will
% include a method to automatically encode a set of Boolean formulae into a
% variational propositional formula. A method for detecting the difficulty of
% solving such a variational propositional formula. A data set suitable for future
% research in the \ac{sat}, \ac{spl} and variation research communities. A
% variational satisfiability solver, an asynchronous variational satisfiable
% modulo theory solver and a proof of variational preservation.

%%% Local Variables:
%%% mode: latex
%%% TeX-master: "../../thesis"
%%% End: