\label{section:conclusion:future-work}

There are numerous avenues of future work ranging from novel applications, to
refining the implementations, to extended the solvers with new features. In this
section we collect and discuss the most promising future work beginning with
tool extensions and ending with abstracting this work to domains other than
satisfiability solving.

\subsection{Utilization of variational cores}
Variational cores are an important and foundational concept for the variational
solver's and consequently for the variationalization recipe. Recall that the
purpose of variational cores was twofold: First, to condense the query formula
such that the variational terms were the majority of terms core. Second, to
simplify the choice removal process by reducing the amount of traversal required
to process the choices. Third, to enforce sharing between variants as the
contexts captured by the core were are reused during choice removal.

This last point is key, because variational cores in combination with the
accumulation and evaluation stores, completely capture the context of a formula
they can be reused in novel ways. For example, one might serialize a variational
core and associated stores to disk, thus effectively caching the core for future
use. Such a feature would enable desirable user facing effects: the solver could
restart without losing information and thus might be useful for debugging or
exploration, if the variational cores require a lot of processing time to
generate this time be amortized, or if the application domain only builds on
previous versions of the same formulas, then the variational core could be
reused.

For example, consider the case of a feature model which evolves every month for
several months, similarly to the \fin{} and \auto{} datasets. Since the feature
model, and consequently the \ac{vpl} formula evolves over time, the previous
variational core could be modified to reflect the changes for the new formula.
Adding new constraints is straightforward; one would simply nest the previous
variational core in a conjunction context with the new core and reuse the
previous stores when generating the new core to ensure sharing. A more difficult
problem is removing constraints or variables in the previous core. Both removing
constraints and removing variables is problematic as the variable or constraint
could have been accumulated into a symbol value or several symbolic values. One
could traverse a dependency graph to find all references of the variable and
symbolic value, and then seek to replace those references with a unit value,
such as \tru{} for $\wedge$ or \fls{} for $\vee$. However this immediately leads
to the problematic case where the variable or symbolic to be removed is in a
$\neg$ context. There is no unit value where $\neg$ does not have meaning and
thus we cannot remove arbitrary variables from a variational core.

In addition to manipulating or storing variational cores, future variational
solvers might utilize them as a convenient messaging format. Throughout this
thesis, we have assumed and have only considered systems which process all
variants in a single base solver instance, however this need not be the case.
Instead, when a choice in is the focus of a the evaluation context (and thus the
variational core) one might choose to solve the true alternative variants in a
different solver and all the false alternatives in the same solver. For example,
a user might know that all true alternative variants have particularly good
performance characteristics for boolector, while all false variants have good
characteristics for yices. Since we compile to \acl{smtlib} script such a
feature is possible with few changes to our method of variational solving. To
add such a feature a future variational solver would allow the user to select
particular solvers over the input \vc{} or the configuration for a query
formula.

\subsection{Further \ac{smt} background theories and tool extensions}
\ac{sat} and \ac{smt} solvers are attractive targets for research on variational
languages. As of this writing, designing a language with variational
side-effects is an open research question. The essential problem is tracking
effects for particular variants across the interface between a variational-aware
system and a plain system. For example, imagine writing a file to disk in one
variant and deleting a different file in another variant. Since the file system
has no concept of variation or variant the variational system is not able to
guarantee variants are isolated and therefore variants may interact in
undesirable and difficult to predict ways. \ac{sat} and \ac{smt} solvers side
step this limitation as they are side-effect free systems. There is simply no
way to read a file disc in an \acl{smtlib} script. Similarly, classes of
run-time errors are not possible. For example, consider an \acl{smtlib} script
which divides by zero, in this case the script simplify will not unify and an
\rn{unsat} will be returned.

Due to the attractive properties of \ac{sat} and \ac{smt} solvers for
variational research, a straightforward avenue of future work is to continue to
investigate efficient variational folds by further extending the variational
solvers. Modern \ac{sat} and \ac{smt} solvers allow quantified constraints
following first-order logic. In this thesis, we have only considered
unquantified constraints, and thus the interaction of between quantified
constraints and choices is an open research problem.

Similarly, we have demonstrated extensions for core background theories, but
there are many features of plain solvers that would be desirable additions to
variational solvers. Such features include generation of variational
unsatisfiable-cores. An unsatisfiable core is a subset of constraints that
prevent the \ac{sat} or \ac{smt} from unifying. Unsatisfiable cores are thus
desirable for many problems where discovering this information is desirable. For
example, one might desire to find the clique in a \ac{sat} encoded weighted
graph which prevents a traversal under some cost limit. Or one might desire to
find the sub-set of features in a feature model that prevent classes of products
from being built.

Enabling variational unsatisfiable cores is possible with our approach of
accumulation, evaluation and choice removal. The key requirement would be to
ensure that the plain, \rn{get-unsat-core} command occurs inside the
\rn{push}/\rn{pop} block for a given variant. Thus far we have only seen the
\rn{get-model} command have this property. Thus a straightforward approach would
be to create a syntactic category that contains arbitrary plain commands, such
as \rn{get-model} or \rn{get-unsat-core} to be sent to the base solver once a
variant has been reduced to \unit{}. Another approach is to create full fledged
variational \acl{smtlib} language instead of expressions of constraints as we
have presented here. Constructing such a variational \acl{smtlib} language is
likely to save work for future extensions. The language would be identical to
\acl{smtlib} except that \rn{push}/\rn{pop} would not be exposed to the user (or
would only be enabled with an option), and choices would be included in the
language just as we have included the for \ac{vpl} and \evpl{}.

Lastly, a promising area of future work is constructing an asynchronous
variational \ac{sat} and \ac{smt} solver. During our experience bench marking
the variational prototype solvers we found that the majority of the time spent
in the base solver is spent querying for a model. Furthermore, each variant
waits until they can be processed by the base solver. For example, consider the
formula $\chc[A]{a,b} \wedge{} \chc[B]{c,d}$, solving this formula results in
solving all four variants. Our prototype solvers choose true alternatives first,
thus the ordering of the variants in the base solver will be \set{(A,\tru{}),
  B,\tru{}}, \set{(A,\tru{}), (B, \fls{})}, \set{(A,\fls{}), (B,\tru{})},
\set{(A, \fls{}), (B, \fls{})}. Notice that all $(A, \fls{})$ variants wait for
$(A, \tru{})$ variants before being considered. Thus, instead of using \rn{push}
and \rn{pop} to represent variation we could instead fork a new solver thread
and solve all $(A, \fls{})$ variants on that solver thread.

We have created three versions of asynchronous prototype solvers but have not
succeeded in constructing a general sound asynchronous variational solver.
Constructing an asynchronous solver is relatively straightforward. Since
variational models form monoids, the order in which models are added to the
variational model isn't important. Similarly, since variational cores capture
the evaluation context at a given time transmitting variational cores to other
solver instances is also straightforward. The problem for asynchronous solvers
is ensuring that the ordering of alternatives is maintained. For example, a
simple model might be to have a pool of producers, which derive variational
cores, and a pool of consumers, which take a variational core and a
configuration and find the next choice that is not in the configuration or
generate a model. The two pool model is sufficient in most cases, however a
subtle bug is now possible. Assume we have a formula with three unique
dimensions $A$, $B$, and $C$ which will be processed in that order. Since the
order of alternatives is no longer deterministic we might encounter a case where
we are either stuck or have mixed variants. For example, assume we have an
unbalanced number of consumer's and producers. Now consider the case where a
consumer thread has consumed \set{(A, \tru{}), (B,\fls{})}, then finds a choice
with a $C$ dimension and waits for a request from a producer thread. The
consumer observes a request to consume \set{(C,\tru{})}, does so, and produces a
model for that variant. Now, the consumer will backtrack with a \rn{pop} call
and wait for another request from a producer for another $\kf{C}$. However, this
thread may have out paced other threads, and so the only request from the
producer pool is to consume \set{(B,\tru{})}, and now we are stuck. If the
consumer accepts the request we will have mixed two variants on this thread
yielding incorrect results, if the consumer does not take the request then we
may end in a deadlock. Such an example is contrived but occurs with asynchronous
communication and must be properly handled. The correct method is for each
thread to track which variant it has solved and track the ordering of choices.
We must ensure that the choices are solved in order such that if a request comes
to solve a \set{(A, \tru{})} variant, and the thread has consumed the
variational core with \set{(A, \fls{})} then the thread must issue as many
\rn{pop}s as needed to backtrack. By tracking this information we can avoid
deadlocks, and malformed variants and still gain the benefits of concurrent
solving which could be substantial especially for large variational formulas.


\subsection{Automated \ac{vpl} formulas}
Thus far we have only considered a \ac{vpl} or \evpl{} formula as input to a
variational solver. This format is likely to be inconvienient as end-users
consider sets of \ac{sat} problems. Thus, a useful extension for these users is
to change the input from a \ac{vpl} formula to a set of \ac{sat} problems the
user is interested in. With the set of \ac{sat} problems, one could synthesize a
\ac{vpl} formula with a sharing ratio that is \emph{good enough} and then run
the solver on that \ac{vpl} formula. For the rest of this section, we'll refer
to the problem of synthesizing a \emph{good} \ac{vpl} formula from a set of
\ac{sat} formulas the \emph{synthesis problem}.

There are several considerations to highlight. First, we found that the sharing
ratio of a formula positively correlates to run-time performance in
\autoref{chapter:case-studies}, echoing results from previous research on
variation. Therefore, the synthesis algorithm should try to maximize the sharing
ratio as it chooses which variants to combine in a choice. Second, minimizing
the number of choices is high priority for the algorithm. Our results indicate
that the run-time of the variational solver grows linearly in the number of
variants to solve (hence exponentially in the number of unique dimensions), thus
adding a single new choice doubles the number of variants and the expected
run-time. Rather than provide an algorithm that find the \emph{best} \ac{vpl}
formula, we instead describe a greedy algorithm that tries to find a reasonable
\ac{vpl} formula. An algorithm that finds the \emph{best} \ac{vpl} formula,
\eg{} one which maximizes the sharing ratio while minimizing the number of
choices is an open research problem. We suspect it is at least NP-hard (likely
by demonstrating that the Binary Decision Diagram variable ordering problem karp
reduces to the \ac{vpl} synthesis problem), although we have not begun to
investigate the problem space.

The synthesis problem is a search problem over a total un-directed graph of
possible formula combinations. Each node in the graph is a \ac{sat} formula or
\ac{vpl} formula that can be combined and is connected to every other node.
Edges represent the possible combinations of two nodes and is weighted with a
\emph{fitness metric} indicating a good match (and thus high amount of sharing
between two nodes) or a bad match. Our approach is to traverse the graph and
greedily select the best combination between two nodes. Combinations mutate the
graph. The old nodes are replaced with the combined node, all old edges are
relaxed and new edges connect the combined node to every other node in the
graph. The algorithm then repeats until only a single node remains in the graph.

There are two sub-procedures in the algorithm: A procedure to combine nodes, and
a procedure to calculate the fitness metric between two nodes. To combine two
nodes we generate a unique dimension, nest one node in the true alternative, and
the other in the false alternative. This simple combination procedure results in
poor sharing as the choice is always be at the root of the abstract syntax tree.
Thus to increase the sharing ratio, we attempt to drive the choice towards the
leaves of the abstract syntax tree of the \ac{vpl} formula using the equivalency
laws in in \autoref{fig:cc:eqv}.

Next we need a procedure that inputs two \ac{sat} or \ac{vpl} formulas, and
returns a fitness metric. There are several possible algorithms; ranging from
string edit distance, to a tree edit distance over the abstract syntax trees of
the \ac{sat} or \ac{vpl} formulas. String comparison algorithms such Levenshtein
distance\cite{Levenshtein_SPD66} or Hamming distance~\cite{H:BST50} are
promising as both have implementations which run in polynomial time, assuming an
encoding from the \ac{sat} problems to strings is computationally feasible.
Graph edit distance is a more direct approach but is NP-Complete with an
approximate solution that is APX-hard~\cite{hardnessOfGraphEditDistance}.
However, most edit distance algorithms work well in practice, and it is likely
that the graph comparisons in this domain are simpler than comparisons which
occur in the worst case, \eg{}, over enormous graphs such as those found in
social networks. Furthermore there are many heuristics such as longest-common
sub-string which might produce metrics that are good enough for reasonable
sharing ratios.

\subsection{Abstracting the variationalization recipe to other domains}
Our approach to creating a variation-aware system by using the plain version of
that system is not specific to satisfiability solvers. The only portion of our
work that is particular to satisfiability solvers is code generation in the base
solver. In essence, our method is a variational left-fold over a variational
language. Thus, one might reuse the ideas of accumulation, evaluation, choice
removal and variational cores in other domains. In particular, the recipe for
variationalization of other domains is clear: To variationalize a plain system
one needs to define the variational artifact for the domain and a method to
express variation in that system; our variational artifact was \ac{vpl} and we
chose to use scopes from the \acl{smtlib} standard, although forking solver
instances is also feasible. One needs a method to express segments of plain
terms and preserve sharing between variants in the plain system, our approach
was to define symbolic values and utilize the internal cache of the plain
solvers to preserve sharing. Lastly, one needs a way to retrieve results from
the plain system and combine those results in any order, hence we defined
monoidal variational models.

Using this recipe one can might imagine a variational prolog which reuses the
work presented in this thesis. For such a language, the variational artifact
would be a prolog-like programming language with choices. Expressing segments of
plain terms with symbolic values could be directly reused from this thesis.
Similarly, the variational result would be nearly identical to the variational
models presented in \autoref{section:vsat:models}. Embedding variation in prolog
is the difficult part although there are several possibilities.
SWI-prolog~\cite{wielemaker:2011:tplp} defines a special kind of predicate
called \emph{dynamic predicates}. Dynamic predicates indicate to the prolog
interpreter that the predicate may change during execution. Changing the
predicate during execution is performed using two primitives, \emph{assertz} and
\emph{retract}. Thus prolog defines a way to assert a constraint in the
interpreter and then refine the constraint as needed, and so dynamic predicates
may serve as a viable primitive for variation in the prolog interpreter. Another
promising embedding is using delimited continuations. In \autoref{chapter:vsat}
we hypothesized that because a Heut zipper is used for choice removal, using
delimited continuations is also feasible as Huet zippers have been shown to be
isomorphic to delimited continuation~\cite{olegZippers}. Fortunately prolog has
first class support for delimited
continuations~\cite{DBLP:journals/tplp/SchrijversDDW13} and thus choice removal
could be done in the base prolog interpreter rather than at the variational
level. Using delimited continuations could greatly reducing the complexity of
creating a variational prolog, and it might be possible to define variational
prolog as a library rather than a separate entity. The exact details for a
variational implementation are not clear but creating a variational prolog is
certainly feasible.

%%% Local Variables:
%%% mode: latex
%%% TeX-master: "../../thesis"
%%% End: