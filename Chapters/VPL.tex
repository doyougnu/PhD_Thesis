~\label{sec:vpl} In this section, I present the syntax and semantics of
variational propositional logic. While this section fulfills the majority of
\autoref{vpl-deliverable} I restate it here to serve as background for
\autoref{sec:vsat}, \autoref{sec:vsmt} and the choice calculus. I conclude the
section by summarizing work left to do.
% \revised{and provide background to the choice calculus as the logic is
% formalized.}
%
The logic is a conservative extension of classic two-valued logic
(\pl{})\footnote{Notation for propositional logic comes from work on many-valued
  logic, see~\cite{Rescher1969-RESML}.} with a \emph{choice} construct from the
choice calculus~\cite{EW11tosem,Walk13thesis}, a formal language for describing
variation.

% We define the syntax and semantics of \vpl{} in Section~\ref{sec:logic:vpl} and
% demonstrate its use by encoding the example from Section~\ref{sec:bkgrnd} into
% \vpl{} in Section~\ref{sec:logic:enc}.

\begin{figure}[h!]
  \begin{subfigure}[t]{\textwidth}
    \centering
    \begin{syntax}
  % D & \Coloneqq & \text{(any dimension name)} & \textit{Dimension} \\ \\
  % t & \Coloneqq & r & \textit{Variable reference} \\
  % & | & \mathit{T} & \textit{True} \\
  % & | & \mathit{F} & \textit{False} \\ \\

  t & \Coloneqq & r \quad|\quad \tru \quad|\quad \fls
    & \textit{Variables and Boolean literals} \\[1.5ex]

  f & \Coloneqq & t    & \textit{Terminal} \\
    & | & \neg f       & \textit{Negate} \\
    & | & f \vee f     & \textit{Or} \\
    & | & f \wedge f   & \textit{And} \\
    & | & \chc[D]{f,f} & \textit{Choice} \\
\end{syntax}

    \caption{Syntax of \vpl{}.}%
    \label{fig:cc:stx}
  \end{subfigure}
  \begin{subfigure}[t]{\textwidth}
    \begin{align*}
  C : D \rightarrow \mathbb{B_{\bot}} & \textit{Configuration} \\
  \sem{\cdot} &: f\rightarrow C \rightarrow f
    \qquad\qquad \text{where } C = D\rightarrow\mathbb{B}_\bot \\
  \sem[C]{t}             &= t \\
  \sem[C]{\neg f}        &= \neg \sem[C]{f} \\
  \sem[C]{f_1\wedge f_2} &= \sem[C]{f_1}\wedge\sem[C]{f_2}\\
  \sem[C]{f_1\vee f_2}   &= \sem[C]{f_1}\vee\sem[C]{f_2}\\
  \sem[C]{\chc[D]{f_1,f_2}} &=
    \begin{cases}
      \sem[C]{f_1}                       & C(D) = \true \\
      \sem[C]{f_2}                       & C(D) = \false \\
      \chc[D]{\sem[C]{f_1},\sem[C]{f_2}} & C(D) = \bot \\
    \end{cases}
\end{align*}

    \centering
    \caption{Configuration semantics of \vpl{}.}%
    \label{fig:cc:cfg}
  \end{subfigure}
  \begin{subfigure}[t]{\textwidth}
    \begin{align*}
  \chc[D]{f,f}
    & \equiv f
    & \rn{Idemp} \\
  \chc[D]{\chc[D]{f_1,f_2},f_3}
    & \equiv \chc[D]{f_1,f_3}
    & \rn{Dom-L} \\
  \chc[D]{f_1,\chc[D]{f_2,f_3}}
    & \equiv \chc[D]{f_1,f_3}
    & \rn{Dom-R} \\
  \chc[D_1]{\chc[D_2]{f_1,f_2},\chc[D_2]{f_3,f_4}}
    & \equiv \chc[D_2]{\chc[D_1]{f_1,f_3},\chc[D_1]{f_2,f_4}}
    & \rn{Swap} \\
  \chc[D]{\neg f_1,\neg f_2}
    & \equiv \neg\chc[D]{f_1,f_2}
    & \rn{Neg} \\
  \chc[D]{f_1\vee f_3,\;f_2\vee f_4}
    & \equiv \chc[D]{f_1,f_2}\vee\chc[D]{f_3,f_4}
    & \rn{Or} \\
  \chc[D]{f_1\wedge f_3,\;f_2\wedge f_4}
    & \equiv \chc[D]{f_1,f_2}\wedge\chc[D]{f_3,f_4}
    & \rn{And} \\
  \chc[D]{f_1\wedge f_2,f_1}
    & \equiv f_1\wedge\chc[D]{f_2,\tru}
    & \rn{And-L} \\
  \chc[D]{f_1\vee f_2,f_1}
    & \equiv f_1\vee\chc[D]{f_2,\fls}
    & \rn{Or-L} \\
  \chc[D]{f_1,f_1\wedge f_2}
    & \equiv f_1\wedge \chc[D]{\tru,f_2}
    & \rn{And-R} \\
  \chc[D]{f_1,f_1\vee f_2}
    & \equiv f_1\vee\chc[D]{\fls,f_2}
    & \rn{Or-R}
\end{align*}

    \centering
    \caption{\vpl{} equivalence laws}%
    \label{fig:cc:eqv}
    \vspace{0.4cm}
  \end{subfigure}
\caption{Formal definition of \vpl{}.}%
\label{fig:cc}
\end{figure}
%
% The idea behind the choice calculus extended logic (\vpl{}) is to construct a
% logic that \TODO{is this a word?}deterministically expresses what is shared, and
% what is distinct, in a set of formulas in classic two valued logic (\pl{}).
% Semantically, one can think of \vpl{} as describing, in a deterministic, and
% constrained way, what could be and what must be, where a term in \pl{} must be
% but a term in \vpl{} \textit{could be} optional. Where as you might have encode
% a sentence such as: All men are mortal; Socrates is a man; therefore Socrates is
% mortal in \pl{}, a corresponding sentence in \vpl{} could be: it could be that
% either Socrates or Anaximander is a man; all men are mortal; thus it could be that
% Socrates or Anaximander is a man. Notice the ending of our \vpl{} statement:
% \textit{is a man}, this is purposeful and highlights that in \vpl{} a point of
% variation simultaneously represents two options.
%

\subsection{Syntax}
%
The syntax of variational propositional logic is given in \autoref{fig:cc:stx}.
It extends the propositional formula notation of \pl{} with a single new
connective called a \emph{choice} from the choice calculus.
%
A choice $\chc[D]{f_1,f_2}$ represents either $f_1$ or $f_2$ depending on the
Boolean value of its \emph{dimension} $D$. We call $f_1$ and $f_2$ the
\emph{alternatives} of the choice.
%
Although dimensions are Boolean variables, the set of dimensions is disjoint
from the set of variables from \pl{}, which may be referenced in the leaves of
a formula. We use lowercase letters to range over variables and uppercase
letters for dimensions.
%
The syntax of \vpl{} does not include derived logical connectives, such as
$\rightarrow$ and $\leftrightarrow$. However, such forms can be defined from
other primitives and are assumed throughout the rest of the proposal.

\subsection{Semantics}
%
Conceptually, a variational formula represents several propositional logic
formulas at once, which can be obtained by resolving all of the choices. For
researchers unfamiliar with work on variation, it is useful to think of \ac{vpl}
as analogous to \cpp{ifdef}-annotated \pl{}, where choices correspond to a
disciplined~\cite{LKA:AOSD11} application of \cpp{ifdef} annotations.
%
From a logical perspective, following the many-valued logic of
Kleene~\cite{Rescher1969-RESML}, the intuition behind \ac{vpl} is that a choice is
a placeholder for two equally possible alternatives that is deterministically
resolved by reference to an external environment.
%
In this sense, \ac{vpl} deviates from other many-valued logics, such as modal
logic~\cite{sep-logic-modal}, because a choice \emph{waits} until there is
enough information to choose an alternative (i.e., until the formula is
\emph{configured}).

The \emph{configuration semantics} of \ac{vpl} is given in \autoref{fig:cc:cfg}
and describes how choices are eliminated from a formula. The semantics are
parameterized by a \emph{configuration}\ $C$, which is a partial function from
dimensions to Boolean values.
%
The first four cases of the semantics simply propagate configuration down the
formula, terminating at the leaves. The case for choices is the interesting one:
if the dimension of the choice is defined in the configuration, then the choice
is replaced by its left or right alternative corresponding to the associated
value of the dimension in the configuration. If the dimension is undefined in
the configuration, then the choice is left intact and configuration propagates
into the choice's alternatives.


If a configuration $C$ eliminates all choices in a formula $f$, we call $C$
\emph{total} with respect to $f$. If $C$ does \emph{not} eliminate all choices
in $f$ (i.e., a dimension used in $f$ is undefined in $C$), we call $C$
\emph{partial} with respect to $f$.
%
We call a choice-free formula \emph{plain}, and call the set of all plain
formulas that can be obtained from $f$ (by configuring it with every possible
total configuration) the \emph{variants} of $f$.
%
% Every plain variant of a \vpl{} formula is a \pl{} formula, demonstrating the
% intuition that a \vpl{} formula conceptually represents many plain \pl{}
% formula.


To illustrate the semantics of \vpl{}, consider the formula
$p\wedge\chc[A]{q,r}$, which has two variants: $p\wedge q$ when $C(A)=\true$
and $p\wedge r$ when $C(A)=\false$.
%
From the semantics, it follows that choices in the same dimension are
\emph{synchronized} while choices in different dimensions are
\emph{independent}. For example, $\chc[A]{p,q}\wedge\chc[B]{r,s}$ has four
variants, while $\chc[A]{p,q}\wedge\chc[A]{r,s}$ has only two ($p\wedge r$ and
$q\wedge s$).
%
It also follows from the semantics that nested choices in the same dimension
contain redundant alternatives; that is, inner choices are \emph{dominated} by
outer choices in the same dimension. For example, $\chc[A]{p,\chc[A]{r,s}}$ is
equivalent to $\chc[A]{p,s}$ since the alternative $r$ cannot be reached by any
configuration.
% \begin{theorem}[\vpl{} reducible to \pl{}]
%   \label{thm:vplToPl}
%   For any configuration $C$ and any formula $e$, if $C$ is
%   valid and total with respect to $e$, then $\sem[C]{e} \in \pl{}$
% \end{theorem}
%
% \begin{proof}
%   This follows directly from the semantics of configuration in
%   Figure~\ref{fig:cc:cfg}, and Definition~\ref{tot:conf}. The proof is done
%   by structural induction and case analysis; because we have a total
%   configuration, and the configuration semantic function is a total function,
%   every choice and its' configured alternatives, will be recursively reified for
%   $e$. Then by the definition of \vpl{} a formula which lacks choices is by
%   definition in \pl{}.
% \end{proof}
%
% \begin{lemma}[Variants are plain]
%   By Theorem~\ref{thm:vplToPl} and the fact that variants are found via total
%   configurations
% \end{lemma}
%
As the previous example illustrates, the representation of a \ac{vpl} formula is
not unique; that is, the same set of variants may be encoded by different
formulas. \autoref{fig:cc:eqv} defines a set of equivalence laws for \ac{vpl}
formulas. These laws follow directly from the configuration semantics in
\autoref{fig:cc:cfg} and can be used to derive semantics-preserving
transformations of \ac{vpl} formulas.
%
For example, we can simplify the formula $\chc[A]{p\vee q, p\vee r}$ by first
applying the \rn{Or} law to obtain $\chc[A]{p,p}\vee\chc[A]{q,r}$, then applying
the \rn{Idemp} law to the first argument to obtain $p\vee\chc[A]{q,r}$ in which
the redundant $p$ has been factored out of the choice.

\subsection{Research Plan}

The previous sections describe \ac{vpl} but is missing an efficient strategy for
encoding sets of \pl{} formulas in a \ac{vpl} formula. The proposed thesis will
directly address this gap:

\paragraph{Encoding strategies} This item will produce an efficient algorithm
that combines a set of \pl{} formulas into a single \ac{vpl} formula. Efficient
has two meanings: It should produce a \ac{vpl} formula in reasonable time, and
it should produce a \ac{vpl} formula that has measurably less variation if
possible. Such an algorithm is desirable for two reasons. First, it is
practically important; a result of a previous study of variational
satisfiability solving~\cite{10.1145/3382025.3414965} was that the greater the
\emph{sharing ratio}, the ratio of plain to total terms in a variational
formula, the faster the \ac{vpl} formula was solved, on average. Thus, by
developing a more efficient encoding algorithm, the performance of the solver is
less volatile. Second, it lowers the barrier of use, with such an algorithm, the
a new user because the new user need not understand \ac{vpl}, rather they only
need to identify the problem set they desire to solve.

A naive encoding algorithm is easy to construct: one wraps all formulas in
unique choices and then uses equivalency rules to increase the sharing ratio.
However, it is likely that better algorithms exist. I envision an algorithm that
utilizes a strategy similar to Huffman coding~\cite{4051119} to find similar
formulas to merge.

This work is able to be done in parallel to much of the other deliverables but
intersects with two other items. First, a proof of variation preservation,
\autoref{proof-deliverable}, for a variational solver must include a similar
proof for the encoding strategy. Second, the encoding strategy will affect
solver performance and thus also affect the evaluation,
\autoref{eval-deliverable}. Lastly, the performance of the encoding strategy
itself will require a set of data to be evaluated on. I discuss harvesting such
a set of data from real world software product lines in \autoref{sec:vsmt},
although enough real world data is already available to begin on this item.

%%% Local Variables:
%%% mode: latex
%%% TeX-master: "../thesis"
%%% End: