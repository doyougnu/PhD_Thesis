~\label{sec:background}

This section provides necessary background on incremental \ac{sat} solving. All
descriptions follow the SMT-LIB2~\cite{BarFT-SMTLIB} standard and describe
incremental solvers as a black box eliding internal details of any specific
solver which adheres to the standard.

%
\begin{figure}[h]
  \begin{subfigure}[t]{.45\textwidth}
    \tikzstyle{block} = [draw,fill=blue!20,minimum size=2em, node distance=1cm]
% diameter of semicircle used to indicate that two lines are not connected
\def\radius{.7mm}
\tikzstyle{branch}=[fill,shape=circle,minimum size=3pt,inner sep=0pt]

\begin{tikzpicture}[>=latex']

    % Draw blocks, inputs and outputs

    % \foreach \y in {1,2,3,4,5} {

        % blocks
        \node[block] at (2,-1) (prod) {$SAT$};
        \node[block, name=stging, below of = prod] {$SAT$};
        \node[block, name=devel, below of = stging] {$SAT$};

        % input nodes
        \node[left of = devel, xshift=-10pt] (input1) {\rV{}};
        \node[left of = stging, xshift=-10pt] (input2) {\qV{}};
        \node[left of = prod, xshift=-10pt] (input3) {\pV{}};

        % inputs
        \draw[->] (input3) -- (prod);
        \draw[->] (input2) -- (stging);
        \draw[->] (input1) -- (devel);

        % outputs
        \draw[->] (prod.east) -- +(0.5,0);
        \draw[->] (stging.east) -- +(0.5,0);
        \draw[->] (devel.east) -- +(0.5,0);


        % \node at (3.5,-3) (result1) {$result$};
        \node[right of = stging, xshift=20pt] (input2) {$result_{\qV{}}$};
        \node[right of = devel, xshift=20pt] (input2) {$result_{\rV{}}$};
        \node[right of = prod, xshift=20pt] (input3) {$result_{\pV{}}$};
    % }
    % \node[block] at (2,-6) (block6) {$f_6$};
    % \draw[->] (block6.east) -- +(0.5,0);

    % % Calculate branch point coordinate
    % \path (input1) -- coordinate (branch) (block1);

    % % Define a style for shifting a coordinate upwards
    % % Note the curly brackets around the coordinate.
    % \tikzstyle{s}=[shift={(0mm,\radius)}]
    % % It would be natural to use the yshift or xshift option, but that does
    % % not seem to work when shifting coordinates.

    % \draw[->] (branch) node[branch] {}{ % draw branch junction
    %         \foreach \c in {2,3,4,5} {
    %             % Draw semicircle junction to indicate that the lines are
    %             % not connected. The intersection between the lines are
    %             % calculated using the convenient -| syntax. Since we want
    %             % the semicircle to have its center where the lines intersect,
    %             % we have to shift the intersection coordinate using the 's'
    %             % style to account for this.
    %             [shift only] -- ([s]input\c -| branch) arc(90:-90:\radius)
    %             % Note the use of the [shift only] option. It is not necessary,
    %             % but I have used it to ensure that the semicircles have the
    %             % same size regardless of scaling.
    %         }
    %     } |- (block6);
\end{tikzpicture}
    \caption{Brute force procedure, no reuse between solver calls.}%
    \label{fig:bkg:bf}
  \end{subfigure}%
  \vfill
  \begin{subfigure}[t]{.45\textwidth}
    \tikzstyle{block} = [draw,fill=blue!20,node distance = 3.2cm, minimum size=2em]

% diameter of semicircle used to indicate that two lines are not connected
\tikzstyle{branch}=[fill,shape=circle,minimum size=3pt,inner sep=0pt]

\begin{tikzpicture}[>=latex']

    % Draw blocks, inputs and outputs

    % \foreach \y in {1,2,3,4,5} {
        % blocks
        \node[block] at (2,-1) (prod) {$SAT$};
        \node[block, name=stging, below=3.3cm of prod] {$SAT$};
        \node[block, name=devel, below=1.3cm of stging] {$SAT$};

        % input nodes
        % \node[left of = devel, xshift=-15pt] (input1) {$development$};
        % \node[left of = stging] (input2) {$\turnBlue{staging}$};
        \node[left of =prod,xshift=-10pt] (input3) {\pV{}};

        % inputs
        \draw[->] (input3) -- (prod);
        \draw[->] (prod) -- node[xshift = 60pt, align=left] {%
          $ \begin{aligned}
            \texttt{pop}  &\quad \eV{}\\
            \texttt{pop}  &\quad \cV{}\\
            \texttt{pop}  &\quad \bV{}\\
            \texttt{push} &\quad (\bV{} \vee \neg \iV{})  \\
            \texttt{push} &\quad \cV{} \\
            \texttt{push} &\quad (\gV{} \rightarrow \cV{}) \\
          \end{aligned}
          $
        } (stging);
\draw[->] (stging) -- node[xshift = 85pt, align=left] {%
  $\begin{aligned}
    & \texttt{resetAssertionStack}\\
    &\texttt{push} \quad \zV{} \leftrightarrow (a \wedge b \wedge c \wedge e)
    % \text{pop} &\quad \kf{pti} \rightarrow c_{i.1}\\
    % \text{pop} &\quad (\kf{spectre\_v2} \vee \kf{l1tf}) \leftrightarrow \\&\quad (c_{0} \wedge (\kf{nospec\_store\_bypass\text{-}disable} \rightarrow f_{j})\\
    % \text{pop} &\quad (c_{0.0} \wedge c_{1} \wedge \ldots c_{n})\\
  \end{aligned}
  $
} (devel);

        % outputs
        \draw[->] (prod.east) -- +(0.5,0);
        \draw[->] (stging.east) -- +(0.5,0);
        \draw[->] (devel.east) -- +(0.5,0);


        % \node at (3.5,-3) (result1) {$result$};
        % \node[right of = stging, xshift=15pt] (input2) {$result$};
        % \node[right of = devel, xshift=15pt] (input2) {$result$};
        % \node at (3.5,-1) (input3) {$result$};

        \node[right of = stging, xshift=18pt] (input2) {$result_{\qV{}}$};
        \node[right of = devel, xshift=18pt] (input2) {$result_{\rV{}}$};
        \node at (3.8,-1) (input3) {$result_{\pV{}}$};
    % }
    % \node[block] at (2,-6) (block6) {$f_6$};
    % \draw[->] (block6.east) -- +(0.5,0);

    % % Calculate branch point coordinate
    % \path (input1) -- coordinate (branch) (block1);

    % % Define a style for shifting a coordinate upwards
    % % Note the curly brackets around the coordinate.
    % \tikzstyle{s}=[shift={(0mm,\radius)}]
    % % It would be natural to use the yshift or xshift option, but that does
    % % not seem to work when shifting coordinates.

    % \draw[->] (branch) node[branch] {}{ % draw branch junction
    %         \foreach \c in {2,3,4,5} {
    %             % Draw semicircle junction to indicate that the lines are
    %             % not connected. The intersection between the lines are
    %             % calculated using the convenient -| syntax. Since we want
    %             % the semicircle to have its center where the lines intersect,
    %             % we have to shift the intersection coordinate using the 's'
    %             % style to account for this.
    %             [shift only] -- ([s]input\c -| branch) arc(90:-90:\radius)
    %             % Note the use of the [shift only] option. It is not necessary,
    %             % but I have used it to ensure that the semicircles have the
    %             % same size regardless of scaling.
    %         }
    %     } |- (block6);
\end{tikzpicture}
    \caption{Incremental procedure, reuse defined by pop and push calls.}%
      % sat calls share state that is determined by }
    \label{fig:bkg:inc}
  \end{subfigure}
  \caption{}%
  \label{fig:bkg}
\end{figure}
%

Suppose, we have three related propositional formulas that we want to solve.
%
\begin{align*}
  p =\ a \wedge b \wedge c \wedge e && q=\ a \wedge (b \vee \neg \iV{}) \wedge c \wedge (\gV{} \rightarrow c) && r =~z \leftrightarrow (a \wedge b \wedge c \wedge e)
\end{align*}
%
\pV{} is simply a conjunction of variables. In \qV{}, relative to \pV{}, we can
see that two variables are added, \iV{}, \gV{}, one variable is removed \eV{},
and and there are two new clauses: $(b \vee \neg \iV{})$ and $(\gV{} \rightarrow
c)$, both of which possibly affect the values of \bV{} and \cV{}. In \rV{}, the
variables and constraints introduced in \pV{} are further constrained to a new
variable, \zV{}.

Suppose one wants to find a satisfying assignment for each formula. Using a
classic \ac{sat} solver results in the procedure illustrated in
\autoref{fig:bkg:bf}; where \sat{} solving is a batch process and no information
is reused. Alternatively, a procedure using an incremental \sat{} solver is
illustrated in \autoref{fig:bkg:inc}; in this scenario, all of the formulas are
solved by single solver instance where terms are programmatically added and
removed from the solver throughout the process. The ability to add and remove
terms from the solvers is enabled by a data structure within the incremental
\sat{} solver called an \textit{assertion stack}. The assertion stack is a stack
of declarations, definitions, or formulas that determine the \textit{context} of
the solver. A solver context is the union of all global variable definitions and
everything on the assertion stack. A program may add an assertion to the stack
via the \texttt{push} operation and remove from the top via a \texttt{pop}
operation~\cite{10.1007/978-3-319-09284-3_16}. A call to
\texttt{resetAssertionStack} pops everything on the stack.

In an efficient process one would initially add as many \emph{shared} terms as
possible; \pV{} in this example. Then check for satisfiability, request a model,
and manipulate the assertion stack to reach the next problem of interest; \qV{}
in this case. Notice that to reach the next problem, \qV{}, from \pV{}, several
operations are required: \eV{} and \cV{} must be removed, \bV{} must be updated,
and the new clauses must be introduced. To reach \rV{} from \qV{} all assertions
would need to be popped to add \zV{}, then re-pushed.

%%% Local Variables:
%%% mode: latex
%%% TeX-master: "../main"
%%% End: