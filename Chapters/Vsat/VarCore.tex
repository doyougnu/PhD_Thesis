\label{section:vsat:vcore}
%
A variational core is an IL formula that captures the variational structure of
a query formula. Plain terms will either be placed on the assertion stack or
will be symbolically reduced, leaving only logical connectives, symbolic
references, and choices.

\begin{figure}
  \centering
  % \begin{subfigure}[t]{0.5\textwidth}
    \tikzstyle{block}    = [draw,fill=white!20,node distance = 1.75cm,align=center]
\tikzstyle{inEdge}   = [fill=white, text width=1cm]
\tikzstyle{overEdge} = [midway,above]
\tikzstyle{input}    = [fill=white!20,node distance = 2.2cm,align=center, text width=1cm]
\tikzstyle{double} = [draw, anchor=text, rectangle split,rectangle split parts=2]
% diameter of semicircle used to indicate that two lines are not connected
\tikzstyle{branch}=[fill,shape=circle,minimum size=3pt,inner sep=0pt]
\tikzstyle{pinstyle} = [pin edge={to-,thin,black}]
\begin{center}
\begin{tikzpicture}[>=latex]

  % input nodes
  \node[input] (input) {$\kf{Query}$ $\kf{formula}$};
  \node[block, right= 0.55cm of input] (ilInput) {to IL};

  % The loop
  \node[block, right=0.55cm of ilInput] (eval) {Evaluation};
  \node[block, below right=1.5cm of eval] (acc) {Accumulation};
  \node[block, below left=1.5cm of eval] (base) {Base\\Solver};

  % output
  \node[input, above left=0.75cm and 0.05 of eval] (vcore) {};

  % nowhere node
  \node[input, above right=0.75cm and 0.05cm of eval] (nowhere) {};

  % escape edges
  \draw
  %% inputs
  (input) edge[->] node[overEdge] {} (ilInput)
  (ilInput) edge[->] node[overEdge] {} (eval)

  %% Eval
  (eval) edge[->,sloped,bend right=10]  node[overEdge] {$r$, $s$, $t$} (base)
  (eval) edge[->,sloped,bend left=10]  node[overEdge,xshift=0.1cm] {$\neg v$, $v_{1} \vee v_{2}$} (acc)
  (eval) edge[->, loop above]  node[overEdge, text width=1cm] {$v_{1} \wedge v_{2}$} (eval)
  (eval) edge[<-,sloped,bend right=25] node[overEdge,right,rotate=-57,text width=1.5cm,xshift=0.5em] {$s$, $v_{1} \vee v_{2}$, $v_{1} \wedge v_{2}$} (nowhere)

  %% acc
  (acc) edge[->, loop below, transform canvas={xshift=6mm}] node[overEdge,below, text width = 1cm] {$\neg v$, $v_{1} \vee v_{2}$, $v_{1} \wedge v_{2}$} (acc)
  (acc) edge[->, loop below, transform canvas={xshift=-6mm}] node[overEdge,below, text width = 1cm] {$r$, $\neg s$, $s_{1} \vee s_{2}$, $s_{1} \wedge s_{2}$} (acc)
  (acc) edge[->,sloped, bend left=25] node[overEdge,below, text width = 1.3cm] {$s$, $v_{1} \wedge v_{2}$, $\ v_{1}\vee v_{2}$} (eval)
  (base) edge[->,sloped, bend right=25] node[overEdge,below] {$\unit{}$} (eval)

  %% output
  (eval) edge[->,sloped, bend left = 25] node[overEdge,left,rotate=54] {$\kf{VCore}$} (vcore) ;

\end{tikzpicture}%
\end{center}

    \caption{Overview of the reduction engine.}%
    \label{impl:vcore}
  % \end{subfigure}
\end{figure}


The variational core for a \ac{vpl} formula is computed by a reduction engine
illustrated in \autoref{impl:vcore}. The reduction engine has two states:
\emph{evaluation}, which communicates to the base solver to process plain terms,
and \emph{accumulation}, which is called by evaluation to create symbolic
references and reduce plain formulas.


To illustrate how the reduction engine computes a variational core, consider the
query formula $f = ((a \wedge b) \wedge \chc[A]{e_{1}, e_{2}}) \wedge ((p \wedge
\neg q) \vee \chc[B]{e_{3}, e_{4}})$. Translated to an IL formula, $f$ has four
references ($a$, $b$, $p$, $q$) and two choices. The reduction engine will
ultimately produce a variational core that asserts $(a \wedge b)$ in the base
solver, thus pushing it onto the assertion stack, and create a symbolic
reference for $(p \wedge \neg q)$.

Generating the core begins with evaluation. Evaluation matches on the root
$\wedge$ node of $f$ and recurs following the $v_1 \wedge v_2$ edge, where
%
$v_1=(a\wedge b)\wedge\chc[A]{e_1,e_2}$ and
$v_2=(p\wedge \neg q) \vee \chc[B]{e_3, e_4}$.
%
The recursion processes the left child first. Thus, evaluation again matches on
$\wedge$ of $v_{1}$ creating another recursive call with $v_{1}' = (a\wedge b)$
and $v_{2}' = \chc[A]{e_1,e_2}$. Finally, the base case is reached with a final
recursive call where $v_{1}'' = a$, and $v_{2}'' = b$. At the base case, both
$a$ and $b$ are references, so evaluation sends $a$ to the base solver
following the $\kf{r, s, t}$ edge, which returns $\unit{}$ for the left child.
The right child follows the same process yielding $\unit{} \wedge \unit{}$.
Since the assertion stack implicitly conjuncts all assertions, $\unit{} \wedge
\unit{}$ will be further reduced to $\unit{}$ and returned as the result of
$v_{1}'$, indicating that both children have been pushed to the base solver.
This leaves $v_{1}' = \unit{}$ and $v_{2}' = \chc[A]{e_1,e_2}$. $v_{2}'$ is a
base case for choices and cannot be reduced in evaluation, so $\unit{} \wedge
\chc[A]{e_1,e_2}$ will be reduced to just $\chc[A]{e_1,e_2}$ as the result for
$v_{1}$.

In evaluation, conjunctions can be split because of the behavior of the
assertion stack and the and-elimination property of $\wedge$. Disjunctions and
negations cannot be split in this way because both cannot be performed if a
child node has been lost to the solver, e.g., $\neg \unit{}$. Thus, in
accumulation, we construct symbolic terms to represent entire subtrees, which
ensures information is not lost while still allowing for the subtree to be
evaluated if it is sound to do so.

The right child, $v_2=(p\wedge \neg q) \vee \chc[B]{e_3, e_4}$ requires
accumulation. Evaluation will match on the root $\vee$ and send $(p\wedge \neg
q) \vee \chc[B]{e_3, e_4}$ to accumulation via the $v_{1} \vee v_{2}$ edge.
Accumulation has two self-loops, one to create symbolic references (with labels
$r, s, \hdots$), and one to recur to values. Accumulation matches the root
$\vee$ and recurs on the self-loop with edge $v_{1} \vee v_{2}$, where $v_{1} =
(p\wedge \neg q)$ and $v_{2} = \chc[B]{e_3, e_4}$. Processing the left child
first, accumulation will recur again with $v_{1}' = p$ and $v_{2}' = \neg q$.
$v_{1}' = p$ is a base case for references, so a unique symbolic reference
$s_{p}$ is generated for $p$ following the self-loop with label $r$ and
returned as the result for $v_{1}'$. $v_{2}'$ will follow the self-loop with
label $\neg v$ to recur through $\neg$ to $q$, where a symbolic term $s_{q}$
will be generated and returned. This yields $\neg s_{q}$, which follows the
$\neg s$ edge to be processed into a new symbolic term, yielding the result for
$v_{2}'$ as $s_{\neg q}$. With both results $v_{1} = s_{p}\wedge s_{\neg q}$,
accumulation will match on $\wedge$ \emph{and} both $s_{p}$ and $s_{\neg q}$ to
accumulate the entire subtree to a single symbolic term, $s_{pq}$, which will
be returned as the result for $v_{1}$. $v_{2}$ is a base case, so accumulation
will return $s_{pq} \vee \chc[B]{e_3, e_4}$ to evaluation. Evaluation will
conclude with $\chc[A]{e_1,e_2}$ as the result for the left child of $\wedge$
and $s_{pq} \vee \chc[B]{e_3, e_4}$ for the right child, yielding
$\chc[A]{e_1,e_2} \wedge s_{pq} \vee \chc[B]{e_3, e_4}$ as the variational core
of $\kf{f}$.

A variational core is derived to save redundant work.
%
If solved naively, plain sub-formulas of $f$, such as $a \wedge b$ and $p
\wedge \neg q$, would be processed once for each variant even though they are
unchanged. Evaluation moves sub-formulas into the solver state to be reused
among different variants. Accumulation caches sub-formulas that cannot be
immediately evaluated to be evaluated later.


Symbolic references are variables in the reduction engine's memory that
represent a set of statements in the base solver.%
%
\footnote{Note that while we use \acl{smtlib} as an implementation target, any
  solver that exposes an incremental API as defined by
  minisat~\cite{10.1007/978-3-319-09284-3_16} can be used to implement
  variational satisfiability solving.}
%
For example, $s_{pq}$ represents three declarations in the base solver:
%
\begin{lstlisting}[columns=flexible,keepspaces=true,language=SMTLIB]
(declare-const p Bool)
(declare-const q Bool)
(declare-fun $s_{pq}$ () Bool (and p (not q)))
\end{lstlisting}

Similarly a variational core is a sequence of statements in the base solver with
holes $\Diamond$. For example, the variational core of $f$ would be encoded as:
%
\begin{lstlisting}[columns=flexible,keepspaces=true,language=SMTLIB]
(assert (and a b))                 ;; add $a \wedge b$ to the assertion stack
(declare-const $\Diamond$)                   ;; choice A
  (*@\lstvdots@*)                                 ;; potentially many declarations and assertions
(declare-fun $s_{pq}$ () Bool (and p q))  ;;  get symbolic reference for $s_{pq}$
(declare-const $\Diamond$)                   ;; choice B
  (*@\lstvdots@*)                                 ;; potentially many declarations and assertions
(assert (or $s_{ab}$ $\Diamond$))                    ;; assert waiting on $\sem[C]{\chc[B]{e_{3},e_{4}}}$
\end{lstlisting}
%
Each hole is filled by configuring a choice and may require multiple
statements to process the alternative.

%%% Local Variables:
%%% mode: latex
%%% TeX-master: "../../thesis"
%%% End: