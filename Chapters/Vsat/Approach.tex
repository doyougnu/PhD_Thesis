~\label{sec:approach}
%
\begin{figure}
  \centering
    \tikzstyle{block}    = [draw,fill=white!20,node distance = 3.5cm,align=center]
\tikzstyle{inEdge}   = [fill=white, text width=1cm]
\tikzstyle{overEdge} = [midway,above]
\tikzstyle{input}    = [fill=white!20,node distance = 2.2cm,align=center, text width=1cm]
\tikzstyle{double} = [draw, anchor=text, rectangle split,rectangle split parts=2]
% diameter of semicircle used to indicate that two lines are not connected
\tikzstyle{branch}=[fill,shape=circle,minimum size=3pt,inner sep=0pt]
\tikzstyle{pinstyle} = [pin edge={to-,thin,black}]
\begin{center}
\begin{tikzpicture}[>=latex]

  % The loop
  \node[block]  (solve) {Reification\\Engine};
  \node[block, below right=2.05cm and 3.125cm of solve] (vmodel) {VModel\\Constructor};
  \node[block, below=2.05cm of solve] (vcore) {Reduction\\Engine};
  \node[block, right=of solve] (base) {Base\\Solver};

  % input nodes
  \node (input) [input, left=0.7cm of vcore] {$\kf{Query}$ $\kf{formula}$};
  \node (vf)    [input, left=0.7cm of solve] {$\kf{variant}$ $\kf{formula}$};


  % outputs
  \node[input, right=0.75cm of vmodel] (output) {$\kf{VModel}$};

  \draw
  %% Inputs
  (input) edge[->] node[overEdge] {} (vcore)
  (vf)    edge[dotted,->] node[overEdge] {} (solve)

  %% outputs
  (vmodel) edge[->] node[overEdge] {} (output)
  (base)   edge[->,sloped] node[overEdge,left,rotate=89,xshift=3.0cm]
  {$\kf{plain~models}$} (vmodel)
  (base)   edge[->,sloped,bend right=10] node[overEdge,right,rotate=-90] {} (vmodel)
  (base)   edge[->,sloped,bend right=20] node[overEdge,right,rotate=-90] {} (vmodel)
  (base)   edge[->,sloped,bend left=10] node[overEdge,right,rotate=-90] {} (vmodel)
  (base)   edge[->,sloped,bend left=20] node[overEdge,right,rotate=-90] {} (vmodel)

  %% Loop
  (solve) edge[->,sloped, bend left=25] node[overEdge,right,text width=5.5cm,rotate=90] {$s$, $v_{1} \vee v_{2}$, $v_{1} \wedge v_{2}$} (vcore)
  (vcore) edge[->,sloped, bend left=25] node[overEdge,left,rotate=-90] {$\kf{VCore}$, $\unit{}$} (solve)
  (solve) edge[->,in=100,out=160,loop, min distance=20mm] node[overEdge,xshift=-2em,yshift=0.2em] {$v \wedge \sem[C]{\chc[D]{e_{1}, e_{2}}}$} (solve)
  (solve) edge[dashed,->,in=70,out=15,loop, min distance=20mm] node[overEdge,xshift=4.75em,yshift=0.2em,text width = 5.5cm] {$v \wedge \sem[C\ \cup\ \{(D, \true)\}]{\chc[D]{e_{1}, e_{2}}}$, $v \wedge \sem[C\ \cup\ \{(D, \false)\}]{\chc[D]{e_{1}, e_{2}}}$} (solve)

  %% Loop escape
  (solve) edge[->] node[overEdge] {} (base)
  (solve) edge[->,sloped,bend right=5] node[overEdge] {} (base)
  (solve) edge[->,sloped,bend right=10] node[overEdge] {} (base)
  (solve) edge[->,sloped,bend left=5] node[overEdge] {} (base)
  (solve) edge[->,sloped,bend left=10] node[overEdge] {\unit{}} (base)
  % (vmodel) edge[->,sloped,bend right=10]  node[overEdge] {$r$, $s$, $t$} (base)
  % (vmodel) edge[->,sloped,bend left=10]  node[overEdge,xshift=0.1cm] {$\neg v$, $v_{1} \vee v_{2}$} (acc)
  % (vmodel) edge[->, loop above]  node[overEdge] {$\unit{} \wedge v$, $v \wedge \unit{}$} (vmodel)
  % (acc) edge[->, loop below] node[overEdge,below] {$r$, $\neg s$, $s_{1} \wedge s_{2}$, $s_{1} \vee s_{2}$} (acc)
  % (acc) edge[->,sloped, bend left=25] node[overEdge,below] {$s$} (vmodel)
  % (base) edge[->,sloped, bend right=25] node[overEdge,below] {$\unit{}$} (vmodel)
  % (vmodel) edge[->] node[overEdge] {} (vcore)
  ;

\end{tikzpicture}%
\end{center}

    \caption{System overview of the variational solver.}%
    \label{impl:overview}
\end{figure}

\ac{vpl} formulas are solved recursively; decoupling the handling of plain terms
from the handling of variational terms.
%
The intuition behind the algorithm is to first process as many plain terms as
possible (e.g.\ by pushing those terms to the underlying solver) while skipping
choices, yielding a \emph{variational core} that represents only the variational
parts of the original formula. We then alternate between configuring choices in
the variational core and processing the new plain terms produced by
configuration until the entire term has been consumed.
%
A variant in of the original \ac{vpl} formula is found every time the entire
term is consumed, since all choices will have been configured. Once a variant
has been found the algorithm queries the underlying solver to obtain a model,
then backtracks to solve a different variant by configuring the same choices
differently. The models for each variant are combined into a single
\emph{variational model} that captures the result of solving all variants of the
original \ac{vpl} formula. Crucially building a variational model is
associative, and thus the order the variants' models are found is not important
to the correctness of the final model.

We present an overview of the variational solver as a state diagram in
\autoref{impl:overview} that operates on the input's abstract syntax tree.
Labels on incoming edges denote inputs to a state and labels on outgoing edges
denote return values; we show only inputs for recursive edges; labels separated
by a comma share the edge. We omit labels that can be derived from the logical
properties of connectives, such as commutativity of $\vee$ and $\wedge$.
Similarly, we omit base case edge labels for choices and describe these cases
in the text.

The solver has four subsystems: The \emph{reduction engine} processes plain
terms and generates the variational core, which is ready for reification.
The \emph{reification engine} configures choices in a variational core. The
\textit{base solver} is the incremental solver used to produce plain models.
Finally, the \emph{variational model constructor} synthesizes a single
variational model from the set of plain models returned by the base solver.

The solver takes a \ac{vpl} formula called a \emph{query formula} and an
optional input called a \emph{variation context} (\vc{}). A \vc{} is a
propositional formula of dimensions that restricts the solver to a subset of
variants, thus prevents computation on extra variants.
%
The variational solver translates the query formula to a formula in an
intermediate language (IL) that the reduction and reification engines operate
over. The syntax of the IL is given below.
%
\[
  v \hquad \Coloneqq\hquad \unit{}
  \hquad|\hquad t
  \hquad|\hquad r
  \hquad|\hquad s
  \hquad|\hquad \neg v
  \hquad|\hquad (v \wedge v)
  \hquad|\hquad v \vee v
  \hquad|\hquad \chc[D]{e,e}
\]
%

The IL includes two kinds of terminals not present in the input query formulas:
plain subterms that can be reduced symbolically will be replaced by a reference
to a \emph{symbolic value} $s$, and subterms that have been sent to the base
solver will be represented by the unit value \unit{}.
%
Note that choices contain unprocessed expressions ($e$) as alternatives.



%%% Local Variables:
%%% mode: latex
%%% TeX-master: "../../thesis"
%%% End: