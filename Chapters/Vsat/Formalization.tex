~\label{section:vsat:formalization}
%
In this section we formalize variational SAT solving by specifying the
semantics of the \emph{accumulation} and \emph{evaluation} phases of the
variational solver, as well as the semantics of processing the variational
core, which we call \emph{choice removal}.
%
Variational SAT solving assumes the existence of an underlying incremental SAT
solver, which we refer to as the \emph{base solver}.


The variational solver interacts with the base solver via several primitive
operations. In our semantics, we simulate the effects of these primitive
operations by tracking their effects on two stores.
%
The \emph{accumulation store} \aStore{} tracks values cached during
accumulation by mapping IL terms to symbolic references. The \emph{evaluation
store} \eStore{} tracks the symbolic references that have been sent to the base
solver during evaluation.

\subsection{Primitives}
%
\autoref{fig:vsat:prim} lists a minimal set of primitive operations that
the base solver is assumed to support. These primitive operations define the
interface between the base solver and the variational solver.


\begin{figure}
  \centering
\begin{tabular}{r@{~~:~~}l@{~~\OB{\to}~~}ll}
\pnot
  & $(\aStore{},s)$
  & $(\aStore{},s)$
  & \emph{Negate a symbolic value} \\
\pand
  & $(\aStore{},s,s)$
  & $(\aStore{},s)$
  & \emph{Conjunction of symbolic values} \\
\por
  & $(\aStore{},s,s)$
  & $(\aStore{},s)$
  & \emph{Disjunction of symbolic values} \\
\pspawn
  & $(\aStore{},r)$
  & $(\aStore{},s)$
  & \emph{Create symbolic value based on a variable} \\
\passert
  & $(\eStore{},\aStore{},s)$
  & $\eStore{}$
  & \emph{Assert a symbolic value to the solver} \\
\pmodel
  & $(\eStore{},\aStore{})$
  & $m$
  & \emph{Get a model for the current solver state}
\end{tabular}

  \caption{Assumed base solver primitive operations.}%
  \label{fig:vsat:prim}
\end{figure}


The primitive operations can be roughly grouped into two categories:
%
The first four operations, consisting of the logical operations \pnot, \pand,
and \por, plus the \pspawn\ operation, are used in the accumulation phase and
are concerned with creating and maintaining symbolic references that may stand
for arbitrarily complex subtrees of the original formula. These operations
simulate caching information in the base solver.
%
The final two operations, \passert\ and \pmodel, are used in the evaluation
phase and simulate pushing new assertions to the base solver and obtaining a
satisfiability model based on the current solver state, respectively.


It's important to note that our primitive operations are pure functions and do
not simulate interacting with the base solver via side effects. The effect of a
primitive operation can be determined by observing its type. For example, the
\passert\ operation pushes new assertions to the base solver. This is reflected
in its type, which includes an evaluation store as input and produces a new
evaluation store (with the assertion included) as output.
%
Since evaluation stores are immutable, we do not need a primitive operation to
simulate popping assertions from the base solver. Instead, we simulate this by
directly reusing old evaluation stores.


Many of the primitive operations operate on references to symbolic values. Such
symbolic references may stand for arbitrarily complex subtrees of the original
formula, built up through successive calls to the corresponding primitive
operations.
%
For example, recall the example formula $p\wedge\neg q$ from
\autoref{section:vsat:approach}, which was replaced by the symbolic value $s_{pq}$ after
the following sequence of \acs{smtlib} declarations.
%
\begin{lstlisting}[columns=flexible,keepspaces=true,language=SMTLIB]
(declare-const p Bool)
(declare-const q Bool)
(declare-fun $s_{pq}$ () Bool (and p (not q)))
\end{lstlisting}
%
In our formalization, we would represent this same transformation of the
formula $p\wedge\neg q$ into a symbolic reference $s_{pq}$ using the
following sequence of primitive operations:

\vspace{-2ex}
{\footnotesize
\begin{align*}
\pspawn(\aStore{0}, p)       &= (\aStore{1}, s_p) \\
\pspawn(\aStore{1}, q)       &= (\aStore{2}, s_q) \\
\pnot(\aStore{2}, s_q)       &= (\aStore{3}, s_q') \\
\pand(\aStore{3}, s_p, s_q') &= (\aStore{4}, s_{pq})
\end{align*}}%

\noindent
%
The accumulation store tracks what information is associated with each symbolic
reference. The store must therefore be threaded through the calls to each
primitive operation so that subsequent operations have access to existing
definitions and can produce a new, updated store.
%
For example, the final store produced by the above example contains the
following mappings from IL terms to symbolic references,
$\aStore{4}=\set{(p,s_p),(q,s_q),(\neg s_q,s_q'),(s_p\wedge s_q',s_{pq})}$.


When comparing the \acs{smtlib} notation to our formalization, observe that
each use of \lstinline{declare-const} corresponds to a use of the \pspawn{}
primitive, while the \lstinline{declare-fun} line in \acs{smtlib} may
potentially expand into several primitive operations in our formalization.
%
For the evaluation primitives, the \passert{} operation corresponds to an
\acs{smtlib} \lstinline{assert} call, while the \pmodel{} operation corresponds
roughly to an \acs{smtlib} \lstinline{check-sat} call, which retrieves a model
for the current set of assertions on the stack.
%
However, the exact semantics of \lstinline{check-sat} depends on the base
solver in use. For example, given the plain formula $p = a \vee b \vee c$, z3
returns only a minimal satisifiable model, such as $\{b = \tru{}\}$, providing
no values for the other variables in the formula.
%
To normalize this behavior across solvers, we instead consider \pmodel{}
equivalent to \lstinline{check-sat} followed by a \lstinline{get-value} call
call for every variable in the query formula, yielding a complete model. For
example, a complete model for $p$ would be $\set{a=\fls{}, b=\tru{},
c=\fls{}}$.


\begin{figure}
  \newcommand{\wrappedprimspacer}{\mbox{\hspace{1.8cm}} & \\[-3ex]}
\centering
\begin{align*}
\pwrap{\pspawn}(\aStore{},r) &=
  \begin{cases}
    \wrappedprimspacer
    (\aStore{},s) & (r,s) \in \aStore{} \\
    \pspawn(\aStore{},r) & \mathit{otherwise}
  \end{cases} \\
\pwrap{\pnot}(\aStore{},s) &=
  \begin{cases}
    \wrappedprimspacer
    (\aStore{},s') & (\neg s, s') \in \aStore{} \\
    \pnot(\aStore{},s) & \mathit{otherwise}
  \end{cases} \\
\pwrap{\pand}(\aStore{},s_1,s_2) &=
  \begin{cases}
    \wrappedprimspacer
    (\aStore{},s_3) & (s_1 \wedge s_2, s_3) \in \aStore{} \\
    \pand(\aStore{},s_1,s_2) & \mathit{otherwise}
  \end{cases} \\
\pwrap{\por}(\aStore{},s_1,s_2) &=
  \begin{cases}
    \wrappedprimspacer
    (\aStore{},s_3) & (s_1 \vee s_2, s_3) \in \aStore{} \\
    \por(\aStore{},s_1,s_2) & \mathit{otherwise}
  \end{cases}
\end{align*}

  \caption{Wrapped accumulation primitive operations.}
  \label{fig:vsat:primwrapped}
\end{figure}


Finally, in \autoref{fig:vsat:primwrapped} we define wrapped versions of the
primitive operations used in accumulation. These wrapper functions first check
to see whether a symbolic reference for the given IL term exists already in the
accumulation store, and if so, returns it without changing the store.
Otherwise, it invokes the corresponding primitive operation to generate and
return the new symbolic reference and updated store.


\subsection{Accumulation}
%
The accumulation phase is formally specified in \autoref{fig:vsat:acc} as a
relation of the form $(\aStore{},v)\accumulation(\aStore{}',v')$.
%
Accumulation replaces plain subterms of the formula with references to symbolic
values, wherever possible. The replacement of subterms by symbolic references
is achieved by the first four rules in the figure. In the \acRef\ rule, a
variable reference is replaced by a symbolic reference by invoking the wrapped
version of the \pspawn\ primitive, which returns the corresponding symbolic
reference or generates a new one, if needed.
%
The \acNotS, \acAndS, and \acOrS\ rules all similarly replace an IL term by a
symbolic reference by invoking the corresponding wrapped primitive operation.
These rules all require that their subterms completely reduce to symbolic
references, as illustrated by the premise
$(\aStore{},v)\accumulation(\aStore{}',s)$ in the \acNotS\ rule, otherwise the
primitive operation cannot be invoked.


\begin{figure}
  \begin{mathpar}
\inferrule*[right=\acRef]
  { \pwrap{\pspawn}(\aStore{},r) = (\aStore{}', s) }
  { (\aStore{}, r) \accumulation (\aStore{}', s) }

\inferrule*[right=\acNotS]
  { (\aStore{}, v) \accumulation (\aStore{}', s) \\
    \pwrap{\pnot}(\aStore{}', s) = (\aStore{}'', s') }
  { (\aStore{}, \neg{} v) \accumulation (\aStore{}'', s') }

\inferrule*[right=\acAndS]
  { (\aStore{}, v_1) \accumulation (\aStore{1}, s_1) \\
    (\aStore{1}, v_2) \accumulation (\aStore{2}, s_2) \\
    \pwrap{\pand}(\aStore{2}, s_1, s_2) = (\aStore{3}, s_3) }
  { (\aStore{}, v_1 \wedge v_2) \accumulation{} (\aStore{3}, s_3) }

\inferrule*[right=\acOrS]
  { (\aStore{}, v_1) \accumulation (\aStore{1}, s_1) \\
    (\aStore{1}, v_2) \accumulation (\aStore{2}, s_2) \\
    \pwrap{\por}(\aStore{2}, s_1, s_2) = (\aStore{3}, s_3) }
  { (\aStore{}, v_1 \vee v_2) \accumulation{} (\aStore{3}, s_3) }

\inferrule*[right=\acChc]
  { }
  {(\aStore,\chc[D]{e_1,e_2}) \accumulation (\aStore{},\chc[D]{e_1,e_2})}

\inferrule*[right=\acNotV]
  { (\aStore{}, v) \accumulation (\aStore{}', v') }
  { (\aStore{}, \neg{} v) \accumulation (\aStore{}', \neg v') }

\inferrule*[right=\acAndV]
  { (\aStore{}, v_1) \accumulation (\aStore{1}, v_1') \\
    (\aStore{1}, v_2) \accumulation (\aStore{2}, v_2') }
  { (\aStore{}, v_1 \wedge v_2) \accumulation{} (\aStore{2}, v_1' \wedge v_2') }

\inferrule*[right=\acOrV]
  { (\aStore{}, v_1) \accumulation (\aStore{1}, v_1') \\
    (\aStore{1}, v_2) \accumulation (\aStore{2}, v_2') }
  { (\aStore{}, v_1 \vee v_2) \accumulation{} (\aStore{2}, v_1' \vee v_2') }
\end{mathpar}

  \caption{Accumulation inference rules.}
  \label{fig:vsat:acc}
\end{figure}


However, not all subterms can be completely reduced to symbolic references. In
particular, variational subterms---subterms that contain one or more choices
within them---cannot be accumulated to a symbolic reference.
%
The \acChc\ rule prevents accumulation under a choice.
%
The \acNotV, \acAndV, and \acOrV\ rules are congruence rules that recursively
apply accumulation to subterms. Although not explicitly stated in the premises,
it is assumed that these \rn{A-*-V} rules apply only if the corresponding
\rn{A-*-S} rule does not apply, that is, when at least one of the subterms does
not reduce completely to a symbolic reference.


We have omitted rules for processing the constant values \tru\ and \fls. These
rules correspond closely to the \acRef\ rule, but use a predefined variable
reference for the true and false constants.


To illustrate the semantics of accumulation, consider the plain formula
%
$g = a \vee (a \wedge b)$ with an initial accumulation store
$\aStore{}=\varnothing$. The \acOrS\ rule matches the root $\vee$ connective
with $v_1=a$ and $v_2 = a \wedge b$.
%
Since $v_1$ is a reference, the \acRef\ rule applies, generating a new symbolic
reference $s_a$ and returning the store $\aStore{1}=\set{(a,s_a)}$.
% relating the variable reference $a$ to the symbolic reference $s_a$.
%
Processing $v_2$ requires an application of the \acAndS\ rule with $v_1'=a$ and
$v_2'=b$, both of which require another application of the \acRef\ rule. For
$v_1'$, the variable $a$ is found in the store returning $s_a$, while for
$v_2'$, a new symbolic reference $s_b$ is generated and added to the resulting
store $\aStore{2}=\set{(a,s_a),(b,s_b)}$.
%
Since both the left and right sides of $v_2$ reduce to a symbolic reference,
the \pand\ primitive is invoked, yielding a new symbolic reference $s_{ab}$ and
the store $\aStore{3}=\set{(a,s_a),(b,s_b),(a \wedge b, s_{ab})}$.
%
Finally, since both the left and right sides of the original formula $g$ reduce
to symbolic references, the \por\ primitive is invoked yielding the final
symbolic reference $s_g$ and the final accumulation store
\( \aStore{4} =
  \set{
    (a,s_a), (b,s_b),
    (s_a \wedge s_b, s_{ab})
    (s_a \vee s_{ab}, s_g) }
\).


When a formula contains choices, all of the plain subterms surrounding the
choices are accumulated to symbolic references, but choices remain in place
and their alternatives are not accumulated. For example, consider the
variational formula
%
\( g' =
  (a \vee (a \wedge b)) \vee
  \chc[D]{a, a \wedge b} \wedge
  (a \vee (a \wedge b))
\)
%
which contains two instances of $g$ as subterms. The formula $g'$ accumulates
to the variational core
%
\( s_g \vee \chc[D]{a, a \wedge b} \wedge s_g \) with the same final store
$\aStore{4}$ produced when accumulating $g$ alone.
%
Note that the each instance of $g$ in $g'$ was reduced to the same symbolic
reference $s_g$ and the alternatives of the choice were not reduced.

\begin{figure}
  \begin{mathpar}
\inferrule*[right=\evAcc]
  { (\aStore{},v) \accumulation (\aStore{}',v') \\
    (\eStore{},\aStore{}',v') \evaluation (\eStore{}',\aStore{}'',v'') }
  { (\eStore{},\aStore{},v) \evaluation (\eStore{}',\aStore{}'',v'') }
\qquad
\inferrule*[right=\evSym]
  { \passert(\eStore{},\aStore{},s) = \eStore{}' }
  { (\eStore{},\aStore{},s) \evaluation (\eStore{}',\aStore{},\unit) }

\inferrule*[right=\evChc]
  { }
  { (\eStore{},\aStore{},\chc[D]{e_1,e_2}) \evaluation
    (\eStore{},\aStore{},\chc[D]{e_1,e_2}) }
\quad
\inferrule*[right=\evOr]
  { }
  { (\eStore{},\aStore{}, v_1 \vee v_2) \evaluation
    (\eStore{},\aStore{}, v_1 \vee v_2) }

\inferrule*[right=\evAndL]
  { (\eStore{},\aStore{},v_1) \evaluation (\eStore{1},\aStore{1},\unit) \\
    (\eStore{1},\aStore{1},v_2) \evaluation (\eStore{2},\aStore{2},v_2') }
  { (\eStore{},\aStore{}, v_1 \wedge v_2) \evaluation
    (\eStore{2},\aStore{2},v_2') }

\inferrule*[right=\evAndR]
  { (\eStore{},\aStore{},v_1) \evaluation (\eStore{1},\aStore{1},v_1') \\
    (\eStore{1},\aStore{1},v_2) \evaluation (\eStore{2},\aStore{2},\unit) }
  { (\eStore{},\aStore{}, v_1 \wedge v_2) \evaluation
    (\eStore{2},\aStore{2},v_1') }

\inferrule*[right=\evAnd]
  { (\eStore{},\aStore{},v_1) \evaluation (\eStore{1},\aStore{1},v_1') \\
    (\eStore{1},\aStore{1},v_2) \evaluation (\eStore{2},\aStore{2},v_2') }
  { (\eStore{},\aStore{}, v_1 \wedge v_2) \evaluation
    (\eStore{2},\aStore{2}, v_1' \wedge v_2') }
\end{mathpar}%

  \caption{Evaluation inference rules.}%
  \label{fig:vsat:eval}
\end{figure}

\subsection{Evaluation}
%
The evaluation phase is formally specified in \autoref{fig:vsat:eval} as a
relation of the form
$(\eStore{},\aStore{},v)\evaluation(\eStore{}',\aStore{}',v')$, where an
evaluation store \eStore{} represents the base solver's state.
%
The \evAcc\ and \evSym\ rules are the heart of evaluation: the \evAcc\ rule
enables accumulating subterms during evaluation, while the \evSym\ rule sends a
fully accumulated subterm to the base solver. Evaluation cannot occur under
choices or un-accumulated disjunctions (i.e.\ disjunctions that contain
choices), as seen in the \evChc\ and \evOr\ rules, but can occur under
un-accumulated conjunctions, as reflected by the three \rn{E-And*} rules. This
will be explained in more detail below.


When a subterm is sent to the base solver by \evSym, it is replaced by the unit
value \unit\ and the evaluation store \eStore{} is updated accordingly.
%
Conceptually, the evaluation store represents the internal state of the
underlying solver (e.g.\ z3's internal state), but we model it formally as the
set of assertions that have been sent to the solver. For example, given the
accumulation store $\aStore{}=\set{(a,s_a),(b,s_b),(s_a \wedge s_b, s_{ab})}$,
the assertion $\passert(\set{},\aStore{},s_{a})$ yields $\set{s_a}$ and
subsequent assertions add more elements to this set, for example,
$\passert(\set{s_a},\aStore{},s_{ab})=\set{s_a,s_{ab}}$.
%
The assertions sent to a SAT solver are implicitly conjuncted together, which
is why partially accumulated conjunctions may still be evaluated, but partially
accumulated disjunctions may not. Such disjunctions are instead handled during
choice removal using back-tracking.


The three \rn{E-And*} rules propagate accumulation over conjunctions. In all
three rules, the subterms are evaluated left-to-right, propagating the
resulting stores accordingly.
%
The \evAndL\ rule states that if the left side of a conjunction can be fully
evaluated to \unit, then the expression can be evaluated to the result of the
right side; likewise, \evAndR\ states that if the right side fully evaluates,
the result of evaluating the expression is the result of the left side. If
neither side fully evaluates to \unit\ (i.e.\ because both contain choices or
disjunctions), then \evAnd\ applies, which leaves the conjunction in place
(with evaluated subterms) to be handled during choice removal.


Consider evaluating the formula $g=(a\vee b)\wedge\chc[D]{a,c}$ with initially
empty stores. We start by applying accumulation using the \evAcc\ rule,
yielding the intermediate term $g'=s_{ab}\wedge\chc[D]{a,c}$ with the
accumulation store $\aStore{}=\set{(a,s_a),(b,s_b),(s_a \vee s_b,s_{ab})}$. We
then apply \evAndL\ to $g'$, which sends the left subterm $s_{ab}$ to the base
solver via the \evSym\ rule, and the right side will be unevaluated via the
\evChc\ rule.
%
Ultimately, evaluation yields the expression $\chc[D]{a,c}$ with accumulation
store \aStore{} and evaluation store $\set{s_{ab}}$.


\subsection{Choice removal}
%
The main driver of variational solving is the choice removal phase, which is
formally specificed in \autoref{fig:vsmt:inf:chc} as a relation of the form
$(\crCtx,z,v)\choiceRemoval\vmodel{}'$.
%
The main role of choice removal is to relate an IL term $v$ to a variational
model $\vmodel{}'$. However, to do this requires several pieces of context
including a configuration $C$, an evaluation store \eStore{}, an accumulation
store \aStore{}, an initial variational model \vmodel{}, and an evaluation
context $z$. The two stores have been explained earlier in this chapter, and
variational models are explained at the end of \autoref{section:vsat:approach}.
We explain configurations and evaluation contexts in the context of the relevant
rules below.


\begin{figure}
  \begin{mathpar}
\inferrule*[right=\crEval]
  { (\eStore{},\aStore{},v) \evaluation (\eStore{}',\aStore{}',\unit) \\
    \texttt{Combine}(\vmodel{},\pmodel(\aStore{},\eStore{})) = \vmodel{}' }
  { (\crCtx, \inRoot, v) \choiceRemoval \vmodel{}' }

\inferrule*[right=\crChcT]
  { (D,\true)\in C \\
    (\crCtx, z, e_1) \choiceRemoval \vmodel{}' }
  { (\crCtx, z, \chc[D]{e_1,e_2} \choiceRemoval \vmodel{}' }

\inferrule*[right=\crChcF]
  { (D,\false)\in C \\
    (\crCtx, z, e_2) \choiceRemoval \vmodel{}' }
  { (\crCtx, z, \chc[D]{e_1,e_2} \choiceRemoval \vmodel{}' }

\inferrule*[right=\crChc]
  { D\notin\dom{C} \\
    (C\cup(D,\true),\eStore{},\aStore{},\vmodel{}, z, e_1)
      \choiceRemoval \vmodel{1} \\
    (C\cup(D,\false),\eStore{},\aStore{},\vmodel{}', z, e_2)
      \choiceRemoval \vmodel{2} }
  { (\crCtx, z, \chc[D]{e_1,e_2} \choiceRemoval \vmodel{2} }

\inferrule*[right=\crNot]
  { (\crCtx, \inNot{z}, v) \choiceRemoval \vmodel{}' }
  { (\crCtx, z, \neg v) \choiceRemoval \vmodel{}' }

\inferrule*[right=\crNotIn]
  { (\aStore{}, \neg s) \accumulation (\aStore{}', s') \\
    (\crCtx, z, s') \choiceRemoval \vmodel{}' }
  { (\crCtx, \inNot{z}, s) \choiceRemoval \vmodel{}' }

\inferrule*[right=\crAnd]
  { (\crCtx, \inAndL{z}{v_2}, v_2) \choiceRemoval \vmodel{}' }
  { (\crCtx, z, v_1 \wedge v_2) \choiceRemoval \vmodel{}' }

\inferrule*[right=\crAndL]
  { (\crCtx, \inAndR{s}{z}, v) \choiceRemoval \vmodel{}' }
  { (\crCtx, \inAndL{z}{v}, s) \choiceRemoval \vmodel{}' }

\inferrule*[right=\crAndR]
  { (\aStore{}, s_1 \wedge s_2) \accumulation (\aStore{}', s_3) \\
    (\crCtx, z, s_3) \choiceRemoval \vmodel{}' }
  { (\crCtx, \inAndR{s_1}{z}, s_2) \choiceRemoval \vmodel{}' }

\inferrule*[right=\crOr]
  { (\crCtx, \inOrL{z}{v_2}, v_2) \choiceRemoval \vmodel{}' }
  { (\crCtx, z, v_1 \vee v_2) \choiceRemoval \vmodel{}' }

\inferrule*[right=\crOrL]
  { (\crCtx, \inOrR{s}{z}, v) \choiceRemoval \vmodel{}' }
  { (\crCtx, \inOrL{z}{v}, s) \choiceRemoval \vmodel{}' }

\inferrule*[right=\crOrR]
  { (\aStore{}, s_1 \vee s_2) \accumulation (\aStore{}', s_3) \\
    (\crCtx, z, s_3) \choiceRemoval \vmodel{}' }
  { (\crCtx, \inOrR{s_1}{z}, s_2) \choiceRemoval \vmodel{}' }
\end{mathpar}

  \caption{Choice removal inference rules}%
  \label{fig:vsmt:inf:chc}
\end{figure}


\todo{are the C-AND like rules wrong? Should the $v_{1}$ be in the focus not
  $v_{2}$}The \crEval\ rule states that $v$ fully evaluates to \unit, then we can
get the current model from the base solver using the \pmodel\ primitive and
update our variational model. We use the operation \texttt{Combine} to perform
the variational model update operation described in
\autoref{section:vsat:approach}.
%
\todo{is there a missing rule to convert references to symbolic with
  accumulation?}The rest of the choice removal rules are structured so that
\crEval\ will be invoked once for every variant of the variational core so that
the final output will be a variational model that encodes the solutions to every
variant of the original formula.


The next three rules concern choices and are the heart of choice removal.
%
These rules make use of a \emph{configuration} $C$, which maps dimensions to
Boolean values (encoded as a set of pairs). The configuration tracks which
dimensions have been selected and how to ensure that all choices in the same
dimension are synchronized.
%
Whenever a choice \chc[D]{e_1,e_2} is encountered during choice removal, we
check $C$ to determine what to do.
%
In \crChcT, if $(D,\true)\in C$, then the first alternative of the dimension
has already been selected, so choice removal proceeds on $e_1$. Similarly, in
\crChcF, if $(D,\false)\in C$, the right alternative has been selected, so
choice removal proceeds on $e_2$.
%
In \crChc, if $D\notin\dom{C}$, then the dimension has not yet been selected,
so we recursively apply choice removal to both $e_1$ and $e_2$, updating $C$
accordingly in each case. Observe that we use the same accumulation store,
evaluation store, and evaluation context for each alternative. This simulates a
backtracking point in the solver, where we first solve $e_1$, then reset the
state of the solver to the point where we encountered the choice and solve
$e_2$. Only the variational model, which is threaded through the solution of
both $e_1$ and $e_2$, is maintained to accumulate the results of solving each
alternative.


The final eight rules apply choice removal to the logical operations.
%
These rules make heavy use of an \emph{evaluation context} $z$ that keeps track
of where we are in a partially evaluated IL term during choice removal.
Evaluation contexts are defined as a zipper data structure~\cite{huet_1997}
over IL terms, given by the following grammar.
%
\[
  z \hquad\Coloneqq\hquad \inRoot
  \hquad|\hquad \inNot{z}
  \hquad|\hquad \inAndL{z}{\,v}
  \hquad|\hquad \inAndR{s}{z}
  \hquad|\hquad \inOrL{z}{\,v}
  \hquad|\hquad \inOrR{s}{z}
\]
%
An evaluation context $z$ is like a breadcrumb trail that enables focusing on a
subterm within a partially evaluated IL term while also keeping track of work
left to do.
%
The empty context \inRoot\ indicates the root of the term. The other cases in
the grammar prepend a ``crumb'' to the trail. The crumb $\cdot\,\neg$ focuses
on the subterm within a negation, $\cdot\wedge v$ focuses on the left subterm
within a conjunction whose right subterm is $v$, and $v\wedge\cdot$ focuses on
the right subterm of a conjunction whose left subterm has already been reduced
to $s$. The cases for disjunction are similar to conjunction.


As an example, consider the IL term $\neg(a\vee b) \wedge c$.
%
When evaluation is focused on $a$, the evaluation context will be
$\inOrL{\inNot{\inAndL{\inRoot}{c}}}{b}$, which states that $a$ exists as the
left child of a disjunction whose right child is $b$, which is inside a
negation, which is the left child of a conjunction whose right child is $c$.
The $b$ and $c$ terms captured in the context are subterms of the original term
that must still be evaluated.
%
During choice removal, IL terms are evaluated according to a left-to-right,
post-order traversal; as IL subterms are evaluated they are replaced by
symbolic references via accumulation.
%
When evaluation is focused on $b$, the context will be
$\inOrR{s_a}{\inNot{\inAndL{\inRoot}{c}}}$, where $s_a$ is the symbolic
reference produced by accumulating the variable $a$.
%
When evaluation is eventually focused on $c$, the evaluation context will be
simply $\inAndR{s_{ab}}{\inRoot}$ since the entire subtree $\neg (a \vee b)$ on
the left side of the conjunction will have been accumulated to the symbolic
reference $s_{ab}$.


The \crNot, \crAnd, and \crOr\ rules define what to do when encountering a
logical operation for the first time. In \crNot, we focus on the subterm of the
negation, while in \crAnd\ and \crOr, we focus on the left child while saving
the right child in the context.
%
The \crAndL\ and \crOrL\ rules define what to do when \emph{finished}
processing the left child of the corresponding operation. A fully processed
child have been accumulated to a symbolic reference $s$. At this point, we move
the $s$ into the evaluation context and shift focus to the previously saved
right child of the logical operation.
%
Finally, the \crNotIn, \crAndR, and \crOrR\ rules define what to do when
finished processing all children of a logical operation. At this point, all
children will have been reduced to symbolic references so we can accumulate the
entire subterm and apply choice removal to the result. For example, in \crAndR,
we have just finished processing the right child to $s_2$ and we previously
reduced the left child to $s_1$, so we now accumulate $s_1\vee s_2$ to $s_3$
and proceed from there.

Evaluation contexts support a simple recursive approach to solving variational
formulas by adding to the context as we move down the term and removing from
the context as we move back up.
%
The extra effort over a more direct recursive strategy is necessary to support
the backtracking pattern implemented by the \crChc\ rule. Whenever we encounter
a choice in a new dimension, we can simply split the state of the solver to
explore each alternative. Without evaluation contexts, this would be extremely
difficult since choices may be deeply nested within a variational formula. We
would have to somehow remember all of the locations in the term that we must
backtrack to later and the state of the solver at each of those locations.

%% Local Variables:
%% mode: latex
%% TeX-master: "../../thesis"
%% End:
