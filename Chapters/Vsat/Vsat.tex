~\label{chapter:vsat}

This chapter presents the variational satisfiability solving algorithm.
\autoref{sec:approach} provides an overview of the algorithm and introduces the
notion of \emph{variational models} as solutions to variational satisfiability
problems. \autoref{sec:vsat:formalization} provides the formal specification.
\todo{add proofs for confluence and variation preseration}We conclude the
chapter by proving that the variational solving algorithm is confluent and
variation preservation.


\section{General Approach}
\label{sec:approach}

\begin{figure}
  \centering
    \tikzstyle{block}    = [draw,fill=white!20,node distance = 3.5cm,align=center]
\tikzstyle{inEdge}   = [fill=white, text width=1cm]
\tikzstyle{overEdge} = [midway,above]
\tikzstyle{input}    = [fill=white!20,node distance = 2.2cm,align=center, text width=1cm]
\tikzstyle{double} = [draw, anchor=text, rectangle split,rectangle split parts=2]
% diameter of semicircle used to indicate that two lines are not connected
\tikzstyle{branch}=[fill,shape=circle,minimum size=3pt,inner sep=0pt]
\tikzstyle{pinstyle} = [pin edge={to-,thin,black}]
\begin{center}
\begin{tikzpicture}[>=latex]

  % The loop
  \node[block]  (solve) {Reification\\Engine};
  \node[block, below right=2.05cm and 3.125cm of solve] (vmodel) {VModel\\Constructor};
  \node[block, below=2.05cm of solve] (vcore) {Reduction\\Engine};
  \node[block, right=of solve] (base) {Base\\Solver};

  % input nodes
  \node (input) [input, left=0.7cm of vcore] {$\kf{Query}$ $\kf{formula}$};
  \node (vf)    [input, left=0.7cm of solve] {$\kf{variant}$ $\kf{formula}$};


  % outputs
  \node[input, right=0.75cm of vmodel] (output) {$\kf{VModel}$};

  \draw
  %% Inputs
  (input) edge[->] node[overEdge] {} (vcore)
  (vf)    edge[dotted,->] node[overEdge] {} (solve)

  %% outputs
  (vmodel) edge[->] node[overEdge] {} (output)
  (base)   edge[->,sloped] node[overEdge,left,rotate=89,xshift=3.0cm]
  {$\kf{plain~models}$} (vmodel)
  (base)   edge[->,sloped,bend right=10] node[overEdge,right,rotate=-90] {} (vmodel)
  (base)   edge[->,sloped,bend right=20] node[overEdge,right,rotate=-90] {} (vmodel)
  (base)   edge[->,sloped,bend left=10] node[overEdge,right,rotate=-90] {} (vmodel)
  (base)   edge[->,sloped,bend left=20] node[overEdge,right,rotate=-90] {} (vmodel)

  %% Loop
  (solve) edge[->,sloped, bend left=25] node[overEdge,right,text width=5.5cm,rotate=90] {$s$, $v_{1} \vee v_{2}$, $v_{1} \wedge v_{2}$} (vcore)
  (vcore) edge[->,sloped, bend left=25] node[overEdge,left,rotate=-90] {$\kf{VCore}$, $\unit{}$} (solve)
  (solve) edge[->,in=100,out=160,loop, min distance=20mm] node[overEdge,xshift=-2em,yshift=0.2em] {$v \wedge \sem[C]{\chc[D]{e_{1}, e_{2}}}$} (solve)
  (solve) edge[dashed,->,in=70,out=15,loop, min distance=20mm] node[overEdge,xshift=4.75em,yshift=0.2em,text width = 5.5cm] {$v \wedge \sem[C\ \cup\ \{(D, \true)\}]{\chc[D]{e_{1}, e_{2}}}$, $v \wedge \sem[C\ \cup\ \{(D, \false)\}]{\chc[D]{e_{1}, e_{2}}}$} (solve)

  %% Loop escape
  (solve) edge[->] node[overEdge] {} (base)
  (solve) edge[->,sloped,bend right=5] node[overEdge] {} (base)
  (solve) edge[->,sloped,bend right=10] node[overEdge] {} (base)
  (solve) edge[->,sloped,bend left=5] node[overEdge] {} (base)
  (solve) edge[->,sloped,bend left=10] node[overEdge] {\unit{}} (base)
  % (vmodel) edge[->,sloped,bend right=10]  node[overEdge] {$r$, $s$, $t$} (base)
  % (vmodel) edge[->,sloped,bend left=10]  node[overEdge,xshift=0.1cm] {$\neg v$, $v_{1} \vee v_{2}$} (acc)
  % (vmodel) edge[->, loop above]  node[overEdge] {$\unit{} \wedge v$, $v \wedge \unit{}$} (vmodel)
  % (acc) edge[->, loop below] node[overEdge,below] {$r$, $\neg s$, $s_{1} \wedge s_{2}$, $s_{1} \vee s_{2}$} (acc)
  % (acc) edge[->,sloped, bend left=25] node[overEdge,below] {$s$} (vmodel)
  % (base) edge[->,sloped, bend right=25] node[overEdge,below] {$\unit{}$} (vmodel)
  % (vmodel) edge[->] node[overEdge] {} (vcore)
  ;

\end{tikzpicture}%
\end{center}

    \caption{System overview of the variational solver.}%
    \label{impl:overview}
\end{figure}

\ac{vpl} formulas are solved recursively; decoupling the handling of plain terms
from the handling of variational terms.
%
The intuition behind the algorithm is to first process as many plain terms as
possible (e.g.\ by pushing those terms to the underlying solver) while skipping
choices, yielding a \emph{variational core} that represents only the variational
parts of the original formula. We then alternate between configuring choices in
the variational core and processing the new plain terms produced by
configuration until the entire term has been consumed.
%
A variant in of the original \ac{vpl} formula is found every time the entire
term is consumed, since all choices will have been configured. Once a variant
has been found the algorithm queries the underlying solver to obtain a model,
then backtracks to solve a different variant by configuring the same choices
differently. The models for each variant are combined into a single
\emph{variational model} that captures the result of solving all variants of the
original \ac{vpl} formula. Crucially building a variational model is
associative, and thus the order the variants' models are found is not important
to the correctness of the final model.

We present an overview of the variational solver as a state diagram in
\autoref{impl:overview} that operates on the input's abstract syntax tree.
Labels on incoming edges denote inputs to a state and labels on outgoing edges
denote return values; we show only inputs for recursive edges; labels separated
by a comma share the edge. We omit labels that can be derived from the logical
properties of connectives, such as commutativity of $\vee$ and $\wedge$.
Similarly, we omit base case edge labels for choices and describe these cases
in the text.

The solver has four subsystems: The \emph{reduction engine} processes plain
terms and generates the variational core, which is ready for reification.
The \emph{reification engine} configures choices in a variational core. The
\textit{base solver} is the incremental solver used to produce plain models.
Finally, the \emph{variational model constructor} synthesizes a single
variational model from the set of plain models returned by the base solver.

The solver takes a \ac{vpl} formula called a \emph{query formula} and an
optional input called a \emph{variation context} (\vc{}). A \vc{} is a
propositional formula of dimensions that restricts the solver to a subset of
variants, thus prevents computation on extra variants.
%
The variational solver translates the query formula to a formula in an
intermediate language (IL) that the reduction and reification engines operate
over. The syntax of the IL is given below.
%
\[
  v \hquad \Coloneqq\hquad \unit{}
  \hquad|\hquad t
  \hquad|\hquad r
  \hquad|\hquad s
  \hquad|\hquad \neg v
  \hquad|\hquad (v \wedge v)
  \hquad|\hquad v \vee v
  \hquad|\hquad \chc[D]{e,e}
\]
%

The IL includes two kinds of terminals not present in the input query formulas:
plain subterms that can be reduced symbolically will be replaced by a reference
to a \emph{symbolic value} $s$, and subterms that have been sent to the base
solver will be represented by the unit value \unit{}.
%
Note that choices contain unprocessed expressions ($e$) as alternatives.


\section{Derivation of a Variational Core}
\label{section:vsat:vcore}

A variational core is an IL formula that captures the variational structure of
a query formula. Plain terms will either be placed on the assertion stack or
will be symbolically reduced, leaving only logical connectives, symbolic
references, and choices.

\begin{figure}
  \centering
  % \begin{subfigure}[t]{0.5\textwidth}
    \tikzstyle{block}    = [draw,fill=white!20,node distance = 1.75cm,align=center]
\tikzstyle{inEdge}   = [fill=white, text width=1cm]
\tikzstyle{overEdge} = [midway,above]
\tikzstyle{input}    = [fill=white!20,node distance = 2.2cm,align=center, text width=1cm]
\tikzstyle{double} = [draw, anchor=text, rectangle split,rectangle split parts=2]
% diameter of semicircle used to indicate that two lines are not connected
\tikzstyle{branch}=[fill,shape=circle,minimum size=3pt,inner sep=0pt]
\tikzstyle{pinstyle} = [pin edge={to-,thin,black}]
\begin{center}
\begin{tikzpicture}[>=latex]

  % input nodes
  \node[input] (input) {$\kf{Query}$ $\kf{formula}$};
  \node[block, right= 0.55cm of input] (ilInput) {to IL};

  % The loop
  \node[block, right=0.55cm of ilInput] (eval) {Evaluation};
  \node[block, below right=1.5cm of eval] (acc) {Accumulation};
  \node[block, below left=1.5cm of eval] (base) {Base\\Solver};

  % output
  \node[input, above left=0.75cm and 0.05 of eval] (vcore) {};

  % nowhere node
  \node[input, above right=0.75cm and 0.05cm of eval] (nowhere) {};

  % escape edges
  \draw
  %% inputs
  (input) edge[->] node[overEdge] {} (ilInput)
  (ilInput) edge[->] node[overEdge] {} (eval)

  %% Eval
  (eval) edge[->,sloped,bend right=10]  node[overEdge] {$r$, $s$, $t$} (base)
  (eval) edge[->,sloped,bend left=10]  node[overEdge,xshift=0.1cm] {$\neg v$, $v_{1} \vee v_{2}$} (acc)
  (eval) edge[->, loop above]  node[overEdge, text width=1cm] {$v_{1} \wedge v_{2}$} (eval)
  (eval) edge[<-,sloped,bend right=25] node[overEdge,right,rotate=-57,text width=1.5cm,xshift=0.5em] {$s$, $v_{1} \vee v_{2}$, $v_{1} \wedge v_{2}$} (nowhere)

  %% acc
  (acc) edge[->, loop below, transform canvas={xshift=6mm}] node[overEdge,below, text width = 1cm] {$\neg v$, $v_{1} \vee v_{2}$, $v_{1} \wedge v_{2}$} (acc)
  (acc) edge[->, loop below, transform canvas={xshift=-6mm}] node[overEdge,below, text width = 1cm] {$r$, $\neg s$, $s_{1} \vee s_{2}$, $s_{1} \wedge s_{2}$} (acc)
  (acc) edge[->,sloped, bend left=25] node[overEdge,below, text width = 1.3cm] {$s$, $v_{1} \wedge v_{2}$, $\ v_{1}\vee v_{2}$} (eval)
  (base) edge[->,sloped, bend right=25] node[overEdge,below] {$\unit{}$} (eval)

  %% output
  (eval) edge[->,sloped, bend left = 25] node[overEdge,left,rotate=54] {$\kf{VCore}$} (vcore) ;

\end{tikzpicture}%
\end{center}

    \caption{Overview of the reduction engine.}%
    \label{impl:vcore}
  % \end{subfigure}
\end{figure}


The variational core for a \ac{vpl} formula is computed by a reduction engine
illustrated in \autoref{impl:vcore}. The reduction engine has two states:
\emph{evaluation}, which communicates to the base solver to process plain terms,
and \emph{accumulation}, which is called by evaluation to create symbolic
references and reduce plain formulas.


To illustrate how the reduction engine computes a variational core, consider the
query formula $f = ((a \wedge b) \wedge \chc[A]{e_{1}, e_{2}}) \wedge ((p \wedge
\neg q) \vee \chc[B]{e_{3}, e_{4}})$. Translated to an IL formula, $f$ has four
references ($a$, $b$, $p$, $q$) and two choices. The reduction engine will
ultimately produce a variational core that asserts $(a \wedge b)$ in the base
solver, thus pushing it onto the assertion stack, and create a symbolic
reference for $(p \wedge \neg q)$.

Generating the core begins with evaluation. Evaluation matches on the root
$\wedge$ node of $f$ and recurs following the $v_1 \wedge v_2$ edge, where
%
$v_1=(a\wedge b)\wedge\chc[A]{e_1,e_2}$ and
$v_2=(p\wedge \neg q) \vee \chc[B]{e_3, e_4}$.
%
The recursion processes the left child first. Thus, evaluation again matches on
$\wedge$ of $v_{1}$ creating another recursive call with $v_{1}' = (a\wedge b)$
and $v_{2}' = \chc[A]{e_1,e_2}$. Finally, the base case is reached with a final
recursive call where $v_{1}'' = a$, and $v_{2}'' = b$. At the base case, both
$a$ and $b$ are references, so evaluation sends $a$ to the base solver
following the $\kf{r, s, t}$ edge, which returns $\unit{}$ for the left child.
The right child follows the same process yielding $\unit{} \wedge \unit{}$.
Since the assertion stack implicitly conjuncts all assertions, $\unit{} \wedge
\unit{}$ will be further reduced to $\unit{}$ and returned as the result of
$v_{1}'$, indicating that both children have been pushed to the base solver.
This leaves $v_{1}' = \unit{}$ and $v_{2}' = \chc[A]{e_1,e_2}$. $v_{2}'$ is a
base case for choices and cannot be reduced in evaluation, so $\unit{} \wedge
\chc[A]{e_1,e_2}$ will be reduced to just $\chc[A]{e_1,e_2}$ as the result for
$v_{1}$.

In evaluation, conjunctions can be split because of the behavior of the
assertion stack and the and-elimination property of $\wedge$. Disjunctions and
negations cannot be split in this way because both cannot be performed if a
child node has been lost to the solver, e.g., $\neg \unit{}$. Thus, in
accumulation, we construct symbolic terms to represent entire subtrees, which
ensures information is not lost while still allowing for the subtree to be
evaluated if it is sound to do so.

The right child, $v_2=(p\wedge \neg q) \vee \chc[B]{e_3, e_4}$ requires
accumulation. Evaluation will match on the root $\vee$ and send $(p\wedge \neg
q) \vee \chc[B]{e_3, e_4}$ to accumulation via the $v_{1} \vee v_{2}$ edge.
Accumulation has two self-loops, one to create symbolic references (with labels
$r, s, \hdots$), and one to recur to values. Accumulation matches the root
$\vee$ and recurs on the self-loop with edge $v_{1} \vee v_{2}$, where $v_{1} =
(p\wedge \neg q)$ and $v_{2} = \chc[B]{e_3, e_4}$. Processing the left child
first, accumulation will recur again with $v_{1}' = p$ and $v_{2}' = \neg q$.
$v_{1}' = p$ is a base case for references, so a unique symbolic reference
$s_{p}$ is generated for $p$ following the self-loop with label $r$ and
returned as the result for $v_{1}'$. $v_{2}'$ will follow the self-loop with
label $\neg v$ to recur through $\neg$ to $q$, where a symbolic term $s_{q}$
will be generated and returned. This yields $\neg s_{q}$, which follows the
$\neg s$ edge to be processed into a new symbolic term, yielding the result for
$v_{2}'$ as $s_{\neg q}$. With both results $v_{1} = s_{p}\wedge s_{\neg q}$,
accumulation will match on $\wedge$ \emph{and} both $s_{p}$ and $s_{\neg q}$ to
accumulate the entire subtree to a single symbolic term, $s_{pq}$, which will
be returned as the result for $v_{1}$. $v_{2}$ is a base case, so accumulation
will return $s_{pq} \vee \chc[B]{e_3, e_4}$ to evaluation. Evaluation will
conclude with $\chc[A]{e_1,e_2}$ as the result for the left child of $\wedge$
and $s_{pq} \vee \chc[B]{e_3, e_4}$ for the right child, yielding
$\chc[A]{e_1,e_2} \wedge s_{pq} \vee \chc[B]{e_3, e_4}$ as the variational core
of $\kf{f}$.

A variational core is derived to save redundant work.
%
If solved naively, plain sub-formulas of $f$, such as $a \wedge b$ and $p
\wedge \neg q$, would be processed once for each variant even though they are
unchanged. Evaluation moves sub-formulas into the solver state to be reused
among different variants. Accumulation caches sub-formulas that cannot be
immediately evaluated to be evaluated later.


Symbolic references are variables in the reduction engine's memory that
represent a set of statements in the base solver.%
%
\footnote{Note that while we use \acl{smtlib} as an implementation target, any
  solver that exposes an incremental API as defined by
  minisat~\cite{10.1007/978-3-319-09284-3_16} can be used to implement
  variational satisfiability solving.}
%
For example, $s_{pq}$ represents three declarations in the base solver:
%
\begin{lstlisting}[columns=flexible,keepspaces=true,language=SMTLIB]
(declare-const p Bool)
(declare-const q Bool)
(declare-fun $s_{pq}$ () Bool (and p (not q)))
\end{lstlisting}

Similarly a variational core is a sequence of statements in the base solver with
holes $\Diamond$. For example, the variational core of $f$ would be encoded as:
%
\begin{lstlisting}[columns=flexible,keepspaces=true,language=SMTLIB]
(assert (and a b))                 ;; add $a \wedge b$ to the assertion stack
(declare-const $\Diamond$)                   ;; choice A
  (*@\lstvdots@*)                                 ;; potentially many declarations and assertions
(declare-fun $s_{pq}$ () Bool (and p q))  ;;  get symbolic reference for $s_{pq}$
(declare-const $\Diamond$)                   ;; choice B
  (*@\lstvdots@*)                                 ;; potentially many declarations and assertions
(assert (or $s_{ab}$ $\Diamond$))                    ;; assert waiting on $\sem[C]{\chc[B]{e_{3},e_{4}}}$
\end{lstlisting}
%
Each hole is filled by configuring a choice and may require multiple
statements to process the alternative.

\section{Solving the Variational Core}

The reduction engine performs the work at each recursive step whereas the
reification engine defines transitions between the recursive steps by
manipulating the configuration. In \autoref{chapter:vpl}, we formalized
a configuration as a function $D\to\booleans{}$, which we encode in the solver
as a set of tuples $\{\kf{D \times \mathbb{B}}\}$.
%
\autoref{impl:overview} shows two loops for the reification engine corresponding
to the reification of choices. The edges from the reification engine to the
reduction engine are transitions taken after a choice is removed, where new
plain terms have been introduced and thus a new core is derived. If the user
supplied a variation context, then it is used to check that the binding of a
Boolean value to a dimension is valid in the variation context. For example,
$\vc{} = \neg A$ would prevent any configurations where $\kf{(A,\true)} \in C$.
Finally, a model is retrieved from the base solver when the reduction engine
returns \unit{}, indicating that a variant has been reached.

We show the edges of the reification engine relating to the $\wedge$ connective;
the edges for the $\vee$ connective are similar. The left edge is taken when a
choice is observed in the variational core: $v \wedge \sem[C]{\chc[D]{e_{1},
    e_{2}}}$ and $D \in C$. This edge reduces choices with dimension $D$ to an
alternative, which is then translated to IL. The right edge is dashed to
indicate assertion stack manipulation and is taken when $D \notin C$. For this
edge, the configuration is mutated for both alternatives: $C \cup \{(D,
\true)\}$ and $C \cup \{(D, \false)\}$, and the recursive call is wrapped with a
\rn{push} and \rn{pop} command. To the base solver, this branching appears as a
linear sequence of assertion stack manipulations that performs backtracking
behavior. For example, the representation of $\kf{f}$ is:
%
\begin{lstlisting}[columns=flexible,keepspaces=true,language=SMTLIB]
  (*@\lstvdots@*)         ;; declarations and assertions from variational core
(push 1)   ;; a configuration on B has occurred
  (*@\lstvdots@*)         ;; new declarations for left alternative
(declare-fun $s$ () Bool (or $s_{pq}$ $\Diamond[\Diamond \rightarrow s_{B_{T}}]$))  ;; fill
(assert $s$)
  (*@\lstvdots@*)         ;; recursive processing
(pop 1)    ;; return for the right alternative
(push 1)   ;; repeat for right alternative
\end{lstlisting}
%
Where the hole $\Diamond$, will be filled with a newly defined variable
$s_{D_{T}}$ that represents the left alternative's formula.


\section{Variational Models}%
\label{section:vmodels}
%
Classic \ac{sat} models map variables to Boolean values; variational models map
variables to variation contexts that record the variants where the variable was
assigned \tru{}. The variational context for a variable $r$ is denoted as
\vc{r}, and a variational model reserves a special variable called \SatVar{} to
track the configurations that were found satisfiable.
%
\begin{figure}[h]
  \centering
  \begin{subfigure}[t]{\textwidth}
  \begin{tabbing}
    \qquad \quad \= $\aV{} \rightarrow$ \tru{} \\
    \> $\bV{} \rightarrow$ \fls{} \\
    \> $\cV{} \rightarrow$ \tru{} \\
    \> $\pV{} \rightarrow$ \tru{} \\
    \> $\qV{} \rightarrow$ \fls{} \\
    \quad $C_{FF}$ = \{(\AV{}, \fls{}), (\BV{}, \fls{})\} \\
  \end{tabbing}
\end{subfigure}%
\begin{subfigure}[t]{\textwidth}
  \begin{tabbing}
    \qquad \quad \= $\aV{} \rightarrow$ \tru{} \\
    \> $\bV{} \rightarrow$ \fls{} \\
    \> $\cV{} \rightarrow$ \tru{} \\
    \> $\pV{} \rightarrow$ \tru{} \\
    \> $\qV{} \rightarrow$ \fls{} \\
    \quad $C_{FT}$ = \{(\AV{}, \fls{}), (\BV{}, \tru{})\} \\
  \end{tabbing}
\end{subfigure}%
\begin{subfigure}[t]{\textwidth}
  \begin{tabbing}
    \qquad \quad \= $\aV{} \rightarrow$ \tru{} \\
    \> $\bV{} \rightarrow$ \fls{} \\
    \\
    \> $\pV{} \rightarrow$ \fls{} \\
    \> $\qV{} \rightarrow$ \tru{} \\
    \quad $C_{TT}$ = \{(\AV{}, \tru{}), (\BV{}, \tru{})\} \\
  \end{tabbing}
\end{subfigure}%
  \caption{Possible plain models for variants of $\kf{f}$.}%
  \label{fig:models:plain}
\end{figure}
\begin{figure}[h]
  \centering
  \begin{subfigure}[t]{\textwidth}
  \begin{tabbing}
  \qquad \qquad \= $\_Sat \rightarrow\ (\neg \AV{} \wedge \neg \BV{}) \vee (\neg \AV{} \wedge \BV{}) \vee (\AV{} \wedge \BV)$ \\
  \> $\aV{} \rightarrow\ (\neg \AV{} \wedge \neg \BV{}) \vee (\neg \AV{} \wedge \BV{}) \vee (\AV{} \wedge \BV)$ \\
  \> $\bV{} \rightarrow\ \fls{}$ \\
  \> $\cV{} \rightarrow\ (\neg{} \AV{} \wedge{} \neg{} \BV{}) \vee{} (\neg{} \AV{} \wedge{} \BV{})$ \\
  \> $\pV{} \rightarrow\ (\neg \AV{} \wedge \neg \BV{}) \vee (\neg \AV{} \wedge \BV{})$ \\
  \> $\qV{} \rightarrow\ (\AV{} \wedge \BV)$
\end{tabbing}
\end{subfigure}

  \caption{Variational model corresponding to the plain models in
    \autoref{fig:models:plain}.}%
  \label{fig:models:var}
\end{figure}
%
As an example, consider an altered version of the query formula from the
previous section $f = ((\aV{} \wedge \neg \bV{}) \wedge \chc[A]{\aV{}
  \rightarrow \neg \pV{}, \cV{}}) \wedge ((\pV{} \wedge \neg \qV{}) \vee
\chc[B]{\qV{}, \pV{}})$. We can easily see that one variant, with configuration
$\{(\AV{},\tru{}), (\BV{},\fls{})\}$ is unsatisfiable. If the remaining variants
are satisfiable, then three models would result, as illustrated in
\autoref{fig:models:plain}; with the corresponding variational model shown in
\autoref{fig:models:var}.

We see that \Satvc{} consists of three disjuncted terms, one for each
satisfiable variant. Variational models are flexible; a satisfiable assignment
of the query formula can be found by calling \ac{sat} on \Satvc{}. Assuming the
model $C_{FT} = \{(\AV{}, \fls{}), (\BV{}, \tru{})\}$ is returned, one can find
a variable's value through substitution with the configuration; for example,
substituting the returned model on \vc{c} yields:
%
\begin{align*}
  \cV{} \rightarrow\ & (\neg \AV{} \wedge \neg \BV{}) \vee (\neg \AV{} \wedge \BV{}) & \text{\vc{} for \cV{}} \\
  \cV{} \rightarrow\ & (\neg \fls{} \wedge \neg \tru{}) \vee (\neg \fls{} \wedge \tru{}) & \text{Substitute \fls{} for \AV{}, \tru{} for \BV{}} \\
  \cV{} \rightarrow\ & \tru{} & \text{Result}
\end{align*}%
%
Furthermore, finding variants where a variable such as \cV{} is satisfiable is
reduced to $\kf{SAT(\vc{\cV{}})}$

Variational models are constructed incrementally by merging each new plain model
returned by the solver into the variational model. A merge requires the current
configuration, the plain model, and current \vc{} of a variable. Variables are
initialized to \fls{}. For each variable $i$ in the model, if $i$'s assignment
is \tru{} in the plain model, then the configuration is translated to a
variation context and disjuncted with \vc{i}. For example, to merge the
$C_{FT}$'s plain model to the variational model in \autoref{fig:models:var},
$C_{FT}$'s configuration is converted to $\neg \AV{} \wedge \BV{}$. This clause
is disjuncted for variables assigned \tru{} in the plain model: \vc{\aV{}},
\vc{\cV{}}, and \vc{\pV{}}, even if they are new (e.g., \cV{}). Variables
assigned \fls{} are skipped, thus \vc{\qV{}} remains \fls{}. For example, in the
next model $C_{TT}$, \cV{} is \fls{} thus \vc{\cV{}} remains unaltered, while
\vc{\qV{}} flips to \tru{} hence \vc{\qV{}} records $\AV{} \wedge \BV{}$.
Variables such as \bV{}, whose \vc{}'s stay \fls{} are called \textit{constant}.


Variational models are constructed in \ac{dnf}, and form a monoid with $\vee$ as
the semigroup operation, and \fls{} as the unit value. We note this for
mathematically inclined readers and those looking to implement their own
variational solver because it has important ramifications for the asynchronous
version of variational satisfiability solvers.
%
\section{Formalization}
~\label{sec:formalization}


In this subsection we formalize variational SAT solving by specifying the
semantics of the \emph{accumulation} and \emph{evaluation} phases of the
variational solver, as well as the semantics of processing the variational
core, which we call \emph{choice removal}.
%
Variational SAT solving assumes the existence of an underlying incremental SAT
solver, which we refer to as the \emph{base solver}.


The variational solver interacts with the base solver via several primitive
operations. In our semantics, we simulate the effects of these primitive
operations by tracking their effects on two stores.
%
The \emph{accumulation store} \aStore{} tracks values cached during
accumulation by mapping IL terms to symbolic references. The \emph{evaluation
store} \eStore{} tracks the symbolic references that have been sent to the base
solver during evaluation.


% We generalize variational models to handle any type of value and discuss
% further extensions to other background theories, such as an array theory, in
% \autoref{sec:future-work}.


\paragraph{Primitives}
%
\autoref{fig:vsmt:inf:prim} lists a minimal set of primitive operations that
the base solver is assumed to support. These primitive operations define the
interface between the base solver and the variational solver.


\begin{figure}
  \centering
\begin{tabular}{r@{~~:~~}l@{~~\OB{\to}~~}ll}
\pnot
  & $(\aStore{},s)$
  & $(\aStore{},s)$
  & \emph{Negate a symbolic value} \\
\pand
  & $(\aStore{},s,s)$
  & $(\aStore{},s)$
  & \emph{Conjunction of symbolic values} \\
\por
  & $(\aStore{},s,s)$
  & $(\aStore{},s)$
  & \emph{Disjunction of symbolic values} \\
\pspawn
  & $(\aStore{},r)$
  & $(\aStore{},s)$
  & \emph{Create symbolic value based on a variable} \\
\passert
  & $(\eStore{},\aStore{},s)$
  & $\eStore{}$
  & \emph{Assert a symbolic value to the solver} \\
\pmodel
  & $(\eStore{},\aStore{})$
  & $m$
  & \emph{Get a model for the current solver state}
\end{tabular}

  \caption{Assumed base solver primitive operations.}%
  \label{fig:vsmt:inf:prim}
\end{figure}


The primitive operations can be roughly grouped into two categories:
%
The first four operations, consisting of the logical operations \pnot, \pand,
and \por, plus the \pspawn\ operation, are used in the accumulation phase and
are concerned with creating and maintaining symbolic references that may stand
for arbitrarily complex subtrees of the original formula. These operations
simulate caching information in the base solver.
%
The final two operations, \passert\ and \pmodel, are used in the evaluation
phase and simulate pushing new assertions to the base solver and obtaining a
satisfiability model based on the current solver state, respectively.


It's important to note that our primitive operations are pure functions and do
not simulate interacting with the base solver via side effects. The effect of a
primitive operation can be determined by observing its type. For example, the
\passert\ operation pushes new assertions to the base solver. This is reflected
in its type, which includes an evaluation store as input and produces a new
evaluation store (with the assertion included) as output.
%
Since evaluation stores are immutable, we do not need a primitive operation to
simulate popping assertions from the base solver. Instead, we simulate this by
directly reusing old evaluation stores.


Many of the primitive operations operate on references to symbolic values. Such
symbolic references may stand for arbitrarily complex subtrees of the original
formula, built up through successive calls to the corresponding primitive
operations.
%
For example, recall the example formula $p\wedge\neg q$ from
\autoref{sec:approach}, which was replaced by the symbolic value $s_{pq}$ after
the following sequence of \acs{smtlib} declarations.
%
\begin{lstlisting}[columns=flexible,keepspaces=true]
(declare-const p Bool)
(declare-const q Bool)
(declare-fun $s_{pq}$ () Bool (and p (not q)))
\end{lstlisting}
%
In our formalization, we would represent this same transformation of the
formula $p\wedge\neg q$ into a symbolic reference $s_{pq}$ using the
following sequence of primitive operations:

\vspace{-2ex}
{\footnotesize
\begin{align*}
\pspawn(\aStore{0}, p)       &= (\aStore{1}, s_p) \\
\pspawn(\aStore{1}, q)       &= (\aStore{2}, s_q) \\
\pnot(\aStore{2}, s_q)       &= (\aStore{3}, s_q') \\
\pand(\aStore{3}, s_p, s_q') &= (\aStore{4}, s_{pq})
\end{align*}}%

\noindent
%
The accumulation store tracks what information is associated with each symbolic
reference. The store must therefore be threaded through the calls to each
primitive operation so that subsequent operations have access to existing
definitions and can produce a new, updated store.
%
For example, the final store produced by the above example contains the
following mappings from IL terms to symbolic references,
$\aStore{4}=\set{(p,s_p),(q,s_q),(\neg s_q,s_q'),(s_p\wedge s_q',s_{pq})}$.


When comparing the \acs{smtlib} notation to our formalization, observe that
each use of \lstinline{declare-const} corresponds to a use of the \pspawn{}
primitive, while the \lstinline{declare-fun} line in \acs{smtlib} may
potentially expand into several primitive operations in our formalization.
%
For the evaluation primitives, the \passert{} operation corresponds to an
\acs{smtlib} \lstinline{assert} call, while the \pmodel{} operation corresponds
roughly to an \acs{smtlib} \lstinline{check-sat} call, which retrieves a model
for the current set of assertions on the stack.
%
However, the exact semantics of \lstinline{check-sat} depends on the base
solver in use. For example, given the plain formula $p = a \vee b \vee c$, z3
returns only a minimal satisifiable model, such as $\{b = \tru{}\}$, providing
no values for the other variables in the formula.
%
To normalize this behavior across solvers, we instead consider \pmodel{}
equivalent to \lstinline{check-sat} followed by a \lstinline{get-value} call
call for every variable in the query formula, yielding a complete model. For
example, a complete model for $p$ would be $\set{a=\fls{}, b=\tru{},
c=\fls{}}$.


\begin{figure}
  \newcommand{\wrappedprimspacer}{\mbox{\hspace{1.8cm}} & \\[-3ex]}
\centering
\begin{align*}
\pwrap{\pspawn}(\aStore{},r) &=
  \begin{cases}
    \wrappedprimspacer
    (\aStore{},s) & (r,s) \in \aStore{} \\
    \pspawn(\aStore{},r) & \mathit{otherwise}
  \end{cases} \\
\pwrap{\pnot}(\aStore{},s) &=
  \begin{cases}
    \wrappedprimspacer
    (\aStore{},s') & (\neg s, s') \in \aStore{} \\
    \pnot(\aStore{},s) & \mathit{otherwise}
  \end{cases} \\
\pwrap{\pand}(\aStore{},s_1,s_2) &=
  \begin{cases}
    \wrappedprimspacer
    (\aStore{},s_3) & (s_1 \wedge s_2, s_3) \in \aStore{} \\
    \pand(\aStore{},s_1,s_2) & \mathit{otherwise}
  \end{cases} \\
\pwrap{\por}(\aStore{},s_1,s_2) &=
  \begin{cases}
    \wrappedprimspacer
    (\aStore{},s_3) & (s_1 \vee s_2, s_3) \in \aStore{} \\
    \por(\aStore{},s_1,s_2) & \mathit{otherwise}
  \end{cases}
\end{align*}

  \caption{Wrapped accumulation primitive operations.}
  \label{fig:vsmt:inf:primwrapped}
\end{figure}


Finally, in \autoref{fig:vsmt:inf:primwrapped} we define wrapped versions of the
primitive operations used in accumulation. These wrapper functions first check
to see whether a symbolic reference for the given IL term exists already in the
accumulation store, and if so, returns it without changing the store.
Otherwise, it invokes the corresponding primitive operation to generate and
return the new symbolic reference and updated store.


\paragraph{Accumulation}
%
The accumulation phase is formally specified in \autoref{fig:vsmt:inf:acc} as a
relation of the form $(\aStore{},v)\accumulation(\aStore{}',v')$.
%
Accumulation replaces plain subterms of the formula with references to symbolic
values, wherever possible. The replacement of subterms by symbolic references
is achieved by the first four rules in the figure. In the \acRef\ rule, a
variable reference is replaced by a symbolic reference by invoking the wrapped
version of the \pspawn\ primitive, which returns the corresponding symbolic
reference or generates a new one, if needed.
%
The \acNotS, \acAndS, and \acOrS\ rules all similarly replace an IL term by a
symbolic reference by invoking the corresponding wrapped primitive operation.
These rules all require that their subterms completely reduce to symbolic
references, as illustrated by the premise
$(\aStore{},v)\accumulation(\aStore{}',s)$ in the \acNotS\ rule, otherwise the
primitive operation cannot be invoked.


\begin{figure}
  \begin{mathpar}
\inferrule*[right=\acRef]
  { \pwrap{\pspawn}(\aStore{},r) = (\aStore{}', s) }
  { (\aStore{}, r) \accumulation (\aStore{}', s) }

\inferrule*[right=\acNotS]
  { (\aStore{}, v) \accumulation (\aStore{}', s) \\
    \pwrap{\pnot}(\aStore{}', s) = (\aStore{}'', s') }
  { (\aStore{}, \neg{} v) \accumulation (\aStore{}'', s') }

\inferrule*[right=\acAndS]
  { (\aStore{}, v_1) \accumulation (\aStore{1}, s_1) \\
    (\aStore{1}, v_2) \accumulation (\aStore{2}, s_2) \\
    \pwrap{\pand}(\aStore{2}, s_1, s_2) = (\aStore{3}, s_3) }
  { (\aStore{}, v_1 \wedge v_2) \accumulation{} (\aStore{3}, s_3) }

\inferrule*[right=\acOrS]
  { (\aStore{}, v_1) \accumulation (\aStore{1}, s_1) \\
    (\aStore{1}, v_2) \accumulation (\aStore{2}, s_2) \\
    \pwrap{\por}(\aStore{2}, s_1, s_2) = (\aStore{3}, s_3) }
  { (\aStore{}, v_1 \vee v_2) \accumulation{} (\aStore{3}, s_3) }

\inferrule*[right=\acChc]
  { }
  {(\aStore,\chc[D]{e_1,e_2}) \accumulation (\aStore{},\chc[D]{e_1,e_2})}

\inferrule*[right=\acNotV]
  { (\aStore{}, v) \accumulation (\aStore{}', v') }
  { (\aStore{}, \neg{} v) \accumulation (\aStore{}', \neg v') }

\inferrule*[right=\acAndV]
  { (\aStore{}, v_1) \accumulation (\aStore{1}, v_1') \\
    (\aStore{1}, v_2) \accumulation (\aStore{2}, v_2') }
  { (\aStore{}, v_1 \wedge v_2) \accumulation{} (\aStore{2}, v_1' \wedge v_2') }

\inferrule*[right=\acOrV]
  { (\aStore{}, v_1) \accumulation (\aStore{1}, v_1') \\
    (\aStore{1}, v_2) \accumulation (\aStore{2}, v_2') }
  { (\aStore{}, v_1 \vee v_2) \accumulation{} (\aStore{2}, v_1' \vee v_2') }
\end{mathpar}

  \caption{Accumulation inference rules.}
  \label{fig:vsmt:inf:acc}
\end{figure}


However, not all subterms can be completely reduced to symbolic references. In
particular, variational subterms---subterms that contain one or more choices
within them---cannot be accumulated to a symbolic reference.
%
The \acChc\ rule prevents accumulation under a choice.
%
The \acNotV, \acAndV, and \acOrV\ rules are congruence rules that recursively
apply accumulation to subterms. Although not explicitly stated in the premises,
it is assumed that these \rn{A-*-V} rules apply only if the corresponding
\rn{A-*-S} rule does not apply, that is, when at least one of the subterms does
not reduce completely to a symbolic reference.


We have omitted rules for processing the constant values \tru\ and \fls. These
rules correspond closely to the \acRef\ rule, but use a predefined variable
reference for the true and false constants.


To illustrate the semantics of accumulation, consider the plain formula
%
$g = a \vee (a \wedge b)$ with an initial accumulation store
$\aStore{}=\varnothing$. The \acOrS\ rule matches the root $\vee$ connective
with $v_1=a$ and $v_2 = a \wedge b$.
%
Since $v_1$ is a reference, the \acRef\ rule applies, generating a new symbolic
reference $s_a$ and returning the store $\aStore{1}=\set{(a,s_a)}$.
% relating the variable reference $a$ to the symbolic reference $s_a$.
%
Processing $v_2$ requires an application of the \acAndS\ rule with $v_1'=a$ and
$v_2'=b$, both of which require another application of the \acRef\ rule. For
$v_1'$, the variable $a$ is found in the store returning $s_a$, while for
$v_2'$, a new symbolic reference $s_b$ is generated and added to the resulting
store $\aStore{2}=\set{(a,s_a),(b,s_b)}$.
%
Since both the left and right sides of $v_2$ reduce to a symbolic reference,
the \pand\ primitive is invoked, yielding a new symbolic reference $s_{ab}$ and
the store $\aStore{3}=\set{(a,s_a),(b,s_b),(a \wedge b, s_{ab})}$.
%
Finally, since both the left and right sides of the original formula $g$ reduce
to symbolic references, the \por\ primitive is invoked yielding the final
symbolic reference $s_g$ and the final accumulation store
\( \aStore{4} =
  \set{
    (a,s_a), (b,s_b),
    (s_a \wedge s_b, s_{ab})
    (s_a \vee s_{ab}, s_g) }
\).


When a formula contains choices, all of the plain subterms surrounding the
choices are accumulated to symbolic references, but choices remain in place
and their alternatives are not accumulated. For example, consider the
variational formula
%
\( g' =
  (a \vee (a \wedge b)) \vee
  \chc[D]{a, a \wedge b} \wedge
  (a \vee (a \wedge b))
\)
%
which contains two instances of $g$ as subterms. The formula $g'$ accumulates
to the variational core
%
\( s_g \vee \chc[D]{a, a \wedge b} \wedge s_g \) with the same final store
$\aStore{4}$ produced when accumulating $g$ alone.
%
Note that the each instance of $g$ in $g'$ was reduced to the same symbolic
reference $s_g$ and the alternatives of the choice were not reduced.


\begin{figure}
  \begin{mathpar}
\inferrule*[right=\evAcc]
  { (\aStore{},v) \accumulation (\aStore{}',v') \\
    (\eStore{},\aStore{}',v') \evaluation (\eStore{}',\aStore{}'',v'') }
  { (\eStore{},\aStore{},v) \evaluation (\eStore{}',\aStore{}'',v'') }
\qquad
\inferrule*[right=\evSym]
  { \passert(\eStore{},\aStore{},s) = \eStore{}' }
  { (\eStore{},\aStore{},s) \evaluation (\eStore{}',\aStore{},\unit) }

\inferrule*[right=\evChc]
  { }
  { (\eStore{},\aStore{},\chc[D]{e_1,e_2}) \evaluation
    (\eStore{},\aStore{},\chc[D]{e_1,e_2}) }
\quad
\inferrule*[right=\evOr]
  { }
  { (\eStore{},\aStore{}, v_1 \vee v_2) \evaluation
    (\eStore{},\aStore{}, v_1 \vee v_2) }

\inferrule*[right=\evAndL]
  { (\eStore{},\aStore{},v_1) \evaluation (\eStore{1},\aStore{1},\unit) \\
    (\eStore{1},\aStore{1},v_2) \evaluation (\eStore{2},\aStore{2},v_2') }
  { (\eStore{},\aStore{}, v_1 \wedge v_2) \evaluation
    (\eStore{2},\aStore{2},v_2') }

\inferrule*[right=\evAndR]
  { (\eStore{},\aStore{},v_1) \evaluation (\eStore{1},\aStore{1},v_1') \\
    (\eStore{1},\aStore{1},v_2) \evaluation (\eStore{2},\aStore{2},\unit) }
  { (\eStore{},\aStore{}, v_1 \wedge v_2) \evaluation
    (\eStore{2},\aStore{2},v_1') }

\inferrule*[right=\evAnd]
  { (\eStore{},\aStore{},v_1) \evaluation (\eStore{1},\aStore{1},v_1') \\
    (\eStore{1},\aStore{1},v_2) \evaluation (\eStore{2},\aStore{2},v_2') }
  { (\eStore{},\aStore{}, v_1 \wedge v_2) \evaluation
    (\eStore{2},\aStore{2}, v_1' \wedge v_2') }
\end{mathpar}%

  \caption{Evaluation inference rules.}
  \label{fig:vsmt:inf:eval}
\end{figure}


\paragraph{Evaluation}
%
The evaluation phase is formally specified in \autoref{fig:vsmt:inf:eval} as a
relation of the form
$(\eStore{},\aStore{},v)\evaluation(\eStore{}',\aStore{}',v')$, where an
evaluation store \eStore{} represents the base solver's state.
%
The \evAcc\ and \evSym\ rules are the heart of evaluation: the \evAcc\ rule
enables accumulating subterms during evaluation, while the \evSym\ rule sends a
fully accumulated subterm to the base solver. Evaluation cannot occur under
choices or un-accumulated disjunctions (i.e.\ disjunctions that contain
choices), as seen in the \evChc\ and \evOr\ rules, but can occur under
un-accumulated conjunctions, as reflected by the three \rn{E-And*} rules. This
will be explained in more detail below.


When a subterm is sent to the base solver by \evSym, it is replaced by the unit
value \unit\ and the evaluation store \eStore{} is updated accordingly.
%
Conceptually, the evaluation store represents the internal state of the
underlying solver (e.g.\ z3's internal state), but we model it formally as the
set of assertions that have been sent to the solver. For example, given the
accumulation store $\aStore{}=\set{(a,s_a),(b,s_b),(s_a \wedge s_b, s_{ab})}$,
the assertion $\passert(\set{},\aStore{},s_{a})$ yields $\set{s_a}$ and
subsequent assertions add more elements to this set, for example,
$\passert(\set{s_a},\aStore{},s_{ab})=\set{s_a,s_{ab}}$.
%
The assertions sent to a SAT solver are implicitly conjuncted together, which
is why partially accumulated conjunctions may still be evaluated, but partially
accumulated disjunctions may not. Such disjunctions are instead handled during
choice removal using back-tracking.


The three \rn{E-And*} rules propagate accumulation over conjunctions. In all
three rules, the subterms are evaluated left-to-right, propagating the
resulting stores accordingly.
%
The \evAndL\ rule states that if the left side of a conjunction can be fully
evaluated to \unit, then the expression can be evaluated to the result of the
right side; likewise, \evAndR\ states that if the right side fully evaluates,
the result of evaluating the expression is the result of the left side. If
neither side fully evaluates to \unit\ (i.e.\ because both contain choices or
disjunctions), then \evAnd\ applies, which leaves the conjunction in place
(with evaluated subterms) to be handled during choice removal.


Consider evaluating the formula $g=(a\vee b)\wedge\chc[D]{a,c}$ with initially
empty stores. We start by applying accumulation using the \evAcc\ rule,
yielding the intermediate term $g'=s_{ab}\wedge\chc[D]{a,c}$ with the
accumulation store $\aStore{}=\set{(a,s_a),(b,s_b),(s_a \vee s_b,s_{ab})}$. We
then apply \evAndL\ to $g'$, which sends the left subterm $s_{ab}$ to the base
solver via the \evSym\ rule, and the right side will be unevaluated via the
\evChc\ rule.
%
Ultimately, evaluation yields the expression $\chc[D]{a,c}$ with accumulation
store \aStore{} and evaluation store $\set{s_{ab}}$.


\paragraph{Choice removal}
%
The main driver of variational solving is the choice removal phase, which is
formally specificed in \autoref{fig:vsmt:inf:chc} as a relation of the form
$(\crCtx,z,v)\choiceRemoval\vmodel{}'$.
%
The main role of choice removal is to relate an IL term $v$ to a variational
model $\vmodel{}'$. However, to do this requires several pieces of context
including a configuration $C$, an evaluation store \eStore{}, an accumulation
store \aStore{}, an initial variational model \vmodel{}, and an evaluation
context $z$. The two stores have been explained earlier in this subsection, and
variational models are explained at the end of \autoref{sec:approach}. We
explain configurations and evaluation contexts in the context of the relevant
rules below.


\begin{figure}
  \begin{mathpar}
\inferrule*[right=\crEval]
  { (\eStore{},\aStore{},v) \evaluation (\eStore{}',\aStore{}',\unit) \\
    \texttt{Combine}(\vmodel{},\pmodel(\aStore{},\eStore{})) = \vmodel{}' }
  { (\crCtx, \inRoot, v) \choiceRemoval \vmodel{}' }

\inferrule*[right=\crChcT]
  { (D,\true)\in C \\
    (\crCtx, z, e_1) \choiceRemoval \vmodel{}' }
  { (\crCtx, z, \chc[D]{e_1,e_2} \choiceRemoval \vmodel{}' }

\inferrule*[right=\crChcF]
  { (D,\false)\in C \\
    (\crCtx, z, e_2) \choiceRemoval \vmodel{}' }
  { (\crCtx, z, \chc[D]{e_1,e_2} \choiceRemoval \vmodel{}' }

\inferrule*[right=\crChc]
  { D\notin\dom{C} \\
    (C\cup(D,\true),\eStore{},\aStore{},\vmodel{}, z, e_1)
      \choiceRemoval \vmodel{1} \\
    (C\cup(D,\false),\eStore{},\aStore{},\vmodel{}', z, e_2)
      \choiceRemoval \vmodel{2} }
  { (\crCtx, z, \chc[D]{e_1,e_2} \choiceRemoval \vmodel{2} }

\inferrule*[right=\crNot]
  { (\crCtx, \inNot{z}, v) \choiceRemoval \vmodel{}' }
  { (\crCtx, z, \neg v) \choiceRemoval \vmodel{}' }

\inferrule*[right=\crNotIn]
  { (\aStore{}, \neg s) \accumulation (\aStore{}', s') \\
    (\crCtx, z, s') \choiceRemoval \vmodel{}' }
  { (\crCtx, \inNot{z}, s) \choiceRemoval \vmodel{}' }

\inferrule*[right=\crAnd]
  { (\crCtx, \inAndL{z}{v_2}, v_2) \choiceRemoval \vmodel{}' }
  { (\crCtx, z, v_1 \wedge v_2) \choiceRemoval \vmodel{}' }

\inferrule*[right=\crAndL]
  { (\crCtx, \inAndR{s}{z}, v) \choiceRemoval \vmodel{}' }
  { (\crCtx, \inAndL{z}{v}, s) \choiceRemoval \vmodel{}' }

\inferrule*[right=\crAndR]
  { (\aStore{}, s_1 \wedge s_2) \accumulation (\aStore{}', s_3) \\
    (\crCtx, z, s_3) \choiceRemoval \vmodel{}' }
  { (\crCtx, \inAndR{s_1}{z}, s_2) \choiceRemoval \vmodel{}' }

\inferrule*[right=\crOr]
  { (\crCtx, \inOrL{z}{v_2}, v_2) \choiceRemoval \vmodel{}' }
  { (\crCtx, z, v_1 \vee v_2) \choiceRemoval \vmodel{}' }

\inferrule*[right=\crOrL]
  { (\crCtx, \inOrR{s}{z}, v) \choiceRemoval \vmodel{}' }
  { (\crCtx, \inOrL{z}{v}, s) \choiceRemoval \vmodel{}' }

\inferrule*[right=\crOrR]
  { (\aStore{}, s_1 \vee s_2) \accumulation (\aStore{}', s_3) \\
    (\crCtx, z, s_3) \choiceRemoval \vmodel{}' }
  { (\crCtx, \inOrR{s_1}{z}, s_2) \choiceRemoval \vmodel{}' }
\end{mathpar}

  \caption{Choice removal inference rules}%
  \label{fig:vsmt:inf:chc}
\end{figure}


The \crEval\ rule states that $v$ fully evaluates to \unit, then we can get the
current model from the base solver using the \pmodel\ primitive and update our
variational model. We use the operation \texttt{Combine} to perform the
variational model update operation described in \autoref{sec:approach}.
%
The rest of the choice removal rules are structured so that \crEval\ will be
invoked once for every variant of the variational core so that the final output
will be a variational model that encodes the solutions to every variant of the
original formula.


The next three rules concern choices and are the heart of choice removal.
%
These rules make use of a \emph{configuration} $C$, which maps dimensions to
Boolean values (encoded as a set of pairs). The configuration tracks which
dimensions have been selected and how to ensure that all choices in the same
dimension are synchronized.
%
Whenever a choice \chc[D]{e_1,e_2} is encountered during choice removal, we
check $C$ to determine what to do.
%
In \crChcT, if $(D,\true)\in C$, then the first alternative of the dimension
has already been selected, so choice removal proceeds on $e_1$. Similarly, in
\crChcF, if $(D,\false)\in C$, the right alternative has been selected, so
choice removal proceeds on $e_2$.
%
In \crChc, if $D\notin\dom{C}$, then the dimension has not yet been selected,
so we recursively apply choice removal to both $e_1$ and $e_2$, updating $C$
accordingly in each case. Observe that we use the same accumulation store,
evaluation store, and evaluation context for each alternative. This simulates a
backtracking point in the solver, where we first solve $e_1$, then reset the
state of the solver to the point where we encountered the choice and solve
$e_2$. Only the variational model, which is threaded through the solution of
both $e_1$ and $e_2$, is maintained to accumulate the results of solving each
alternative.


The final eight rules apply choice removal to the logical operations.
%
These rules make heavy use of an \emph{evaluation context} $z$ that keeps track
of where we are in a partially evaluated IL term during choice removal.
Evaluation contexts are defined as a zipper data structure~\citep{huet_1997}
over IL terms, given by the following grammar.
%
\[
  z \hquad\Coloneqq\hquad \inRoot
  \hquad|\hquad \inNot{z}
  \hquad|\hquad \inAndL{z}{\,v}
  \hquad|\hquad \inAndR{s}{z}
  \hquad|\hquad \inOrL{z}{\,v}
  \hquad|\hquad \inOrR{s}{z}
\]
%
An evaluation context $z$ is like a breadcrumb trail that enables focusing on a
subterm within a partially evaluated IL term while also keeping track of work
left to do.
%
The empty context \inRoot\ indicates the root of the term. The other cases in
the grammar prepend a ``crumb'' to the trail. The crumb $\cdot\,\neg$ focuses
on the subterm within a negation, $\cdot\wedge v$ focuses on the left subterm
within a conjunction whose right subterm is $v$, and $v\wedge\cdot$ focuses on
the right subterm of a conjunction whose left subterm has already been reduced
to $s$. The cases for disjunction are similar to conjunction.


As an example, consider the IL term $\neg(a\vee b) \wedge c$.
%
When evaluation is focused on $a$, the evaluation context will be
$\inOrL{\inNot{\inAndL{\inRoot}{c}}}{b}$, which states that $a$ exists as the
left child of a disjunction whose right child is $b$, which is inside a
negation, which is the left child of a conjunction whose right child is $c$.
The $b$ and $c$ terms captured in the context are subterms of the original term
that must still be evaluated.
%
During choice removal, IL terms are evaluated according to a left-to-right,
post-order traversal; as IL subterms are evaluated they are replaced by
symbolic references via accumulation.
%
When evaluation is focused on $b$, the context will be
$\inOrR{s_a}{\inNot{\inAndL{\inRoot}{c}}}$, where $s_a$ is the symbolic
reference produced by accumulating the variable $a$.
%
When evaluation is eventually focused on $c$, the evaluation context will be
simply $\inAndR{s_{ab}}{\inRoot}$ since the entire subtree $\neg (a \vee b)$ on
the left side of the conjunction will have been accumulated to the symbolic
reference $s_{ab}$.


The \crNot, \crAnd, and \crOr\ rules define what to do when encountering a
logical operation for the first time. In \crNot, we focus on the subterm of the
negation, while in \crAnd\ and \crOr, we focus on the left child while saving
the right child in the context.
%
The \crAndL\ and \crOrL\ rules define what to do when \emph{finished}
processing the left child of the corresponding operation. A fully processed
child have been accumulated to a symbolic reference $s$. At this point, we move
the $s$ into the evaluation context and shift focus to the previously saved
right child of the logical operation.
%
Finally, the \crNotIn, \crAndR, and \crOrR\ rules define what to do when
finished processing all children of a logical operation. At this point, all
children will have been reduced to symbolic references so we can accumulate the
entire subterm and apply choice removal to the result. For example, in \crAndR,
we have just finished processing the right child to $s_2$ and we previously
reduced the left child to $s_1$, so we now accumulate $s_1\vee s_2$ to $s_3$
and proceed from there.


Evaluation contexts support a simple recursive approach to solving variational
formulas by adding to the context as we move down the term and removing from
the context as we move back up.
%
The extra effort over a more direct recursive strategy is necessary to support
the backtracking pattern implemented by the \crChc\ rule. Whenever we encounter
a choice in a new dimension, we can simply split the state of the solver to
explore each alternative. Without evaluation contexts, this would be extremely
difficult since choices may be deeply nested within a variational formula. We
would have to somehow remember all of the locations in the term that we must
backtrack to later and the state of the solver at each of those locations.



% \paragraph{Example: Derivation of a Variational Core}
% Consider the query formula \newline{}$\kf{h = (\chc[A]{k,l} \wedge a) \wedge
%   (b \vee \chc[A]{c,\neg b})}$; derivation of the variational core $\kf{h}$
% begins with evaluation and all stores \aStore{}, \eStore{} initialized to empty.
% When a user inputs a \vc{} the configuration, C, is initialized
% to it, otherwise C is initialized to empty.
% %
% \evAnd{} is the only applicable rule, with $\wedge$ at the root of $\kf{h}$.
% Thus, \newline{}$v_{1} = \kf{(\chc[A]{k,l} \wedge a)}$, and $v_{2} =
% (b \vee \chc[A]{c,\neg b})$.
%
% We traverse $v_{1}$ first, leading to a recursive application of \texttt{Ev-And}.
% We denote the recursive levels with a tick mark $'$, thus $\kf{v_{1}' =
%   \chc[A]{k,l}}$ is the recursive left child and the right child is $\kf{v_{2}'
%   = a}$.
% %
% \evChc{} matches $v_{1}'$ preventing further evaluation. \TODO{evRef} applies to
% $v_{2}'$ producing a symbolic, $s_{a}$ for $a$, which can immediately be sent to
% the base solver with \evSym{}. Thus the result of the recursive call on $v_{1}$
% is $\kf{\chc[A]{k,l}}$ with stores \aStore{1} = \set{(a,s_{a})}, \eStore{1}
% = \set{s_{a}}.
%
% Accumulation is required to evaluate $v_{2}$. \evOr{} performs the phase change
% by matching on the root $\vee{}$. \evOr{} produces two calls to accumulation
% with $v_{1}' = \chc[A]{c,\neg b}$, $v_{2}' = b$ and \aStore{1}. $v_{1}'$ is a
% choice and thus only \acChc{} applies. \acRef{} will match on $v_{2}'$ and will
% produce $s_{b}$ with store \aStore{2} = \set{(a,s_{a}), (b,s_{b})}. Thus
% accumulation yields $\chc[A]{c,\neg b} \vee{} s_{b}$ with \aStore{2} as the
% result of $v_{2}$. With results for both $v_{1}$ and $v_{2}$ we have derived the
% variational core of $\kf{h}$ as $\chc[A]{k,l} \wedge{} (s_{b} \vee{}
% \chc[A]{c,\neg{} b})$ with stores \aStore{2} and \eStore{1}.
%
% \paragraph{Example: Solving the variational core}
% Solving the variational core begins with choice removal and proceeds with
% recursive calls to accumulation and consequently evaluation. We assume an empty
% configuration for the remainder of the example as C = $\varnothing{}$
% is the most general case.
%
% $\kf{\chc[A]{k,l}}$ is the immediate left child of the root $\wedge{}$, thus the
% left version of \evAnd{} applies. Clearly $D \notin C$, so a
% recursive case for each alternative, beginning with the left alternative $e_{1}$
% is performed. Several changes occur: the assertion stack is incremented,
% indicating future processing occurs in a \primOp{push} context; the
% configuration mutates to account for the selection; $e_{1}$ is translated into
% IL and replaces the choice, thereby introducing a \textit{new} plain term:
% $k$. Thus, the recursive call for the left alternative is $k \wedge
% (s_{b} \vee \chc[A]{c,\neg b})$ where $C_L = \{(A,
% \true{})\}$. Similarly the right alternative is $l \wedge
% (s_{b} \vee \chc[A]{c,\neg b})$ with $C_R = \{(A,
% \false{})\}$.
%
% We walk through the processing of the left alternative in detail, the right
% alternative follows the same procedure. Following \crAnd{} we begin accumulating
% the left alternative with stores \aStore{} = \set{(a,s_{a}), (b,s_{b})}.
% The only applicable rule is \TODO{acAnd} producing two recursive calls where $v_{1}
% = k$ and $v_{2} = s_{b} \vee \chc[A]{c,\neg b}$. $v_{1}$ will yield $s_{k}$ and
% $v_{2}$ cannot be accumulated further. Thus we'll have $s_{k} \wedge{} (s_{b}
% \vee{} \chc[A]{c,\neg b})$ as the result of \crAnd{} after returning from an
% accumulation phase.
% %
% This result is stuck as each choice removal rule only removes a choice if it is
% the immediate child of the root node. To become unstuck we utilize the
% aforementioned zipper to suspend the \acs{ast} and place the choice in focus,
% other implementations may opt for different strategies, such as distributing the
% $\vee{}$ over $\wedge{}$. In either case we assume the choice is able to be
% operated upon.
%
% To finish processing the variant, the only applicable rule will be the $\vee{}$
% version of \crAndL{}. $D \in C$ thus $\chc[A]{c,\neg{} b}$ will
% be reified to just $\kf{c}$ yielding a recursive call to choice removal with
% $s_{k} \wedge (s_{b} \vee{} c)$, \aStore{} =\set{(a,s_{a}), (b,s_{b}),
%   (k,s_{k})}, \eStore{} = \set{s_{a}}, and C =
% \set{(\kf{A}, \true{})}. \crAnd{} applies, yielding to another accumulation
% phase. Since there are no longer any choices, the formula is plain and will be
% accumulated to a single symbolic $s_{L}$ where \newline{}\aStore{L} =
% \set{(a,s_{a},), (b,s_{b}), (k,s_{k}), (c,s_{c}), (s_{b} \vee{} s_{c},
%   s_{bc}), (s_{k} \wedge{} s_{ab}, s_{L})}, \eStore{L} = \set{s_{a}}, and
% $C = C_L$. Finally $s_{L}$ can be evaluated with
% \crSym{} reducing the left variant to \unit{} and returning a model. With the
% model for the \true{} variant of $\kf{A}$ the process backtracks to compute the
% false variant eventually combining the two resulting models as the final result.


% \color{\pubcolor}

%%% Local Variables:
%%% mode: latex
%%% TeX-master: "../emse"
%%% End:


%%% Local Variables:
%%% mode: latex
%%% TeX-master: "../../thesis"
%%% End:
