\label{section:vsat:variation-preservation}
%
\begin{figure}[h]
  \centering
    \begin{tikzcd}[column sep=8cm,row sep=3cm]
    \kf{f} \arrow[r, Rightarrow, "\setc{\prime*{f}}{\exists C.\ \sem[C]{f} = \prime*{f},\ \prime*{f}\ \in\ \pl}"] %\arrow[r, shift left] \arrow[r, shift right]
    \arrow[d, swap, "VSAT(f)"] &
    \set{\prime*{f}} \arrow[d, Rightarrow, "SAT(\prime*{f})"]  \\
    \vmodel*{f}  \arrow[r, Rightarrow, "\setc{\submodel{f}{C}}{\exists C.\ \sem[C]{f} = \prime*{f},\ \prime*{f}\ \in\ \pl}"]
                 % \arrow[r, shift left]
                 % \arrow[r, shift right]
    &  \set{\rn{\Large sat}}
  \end{tikzcd}
  % \matrix (m) [matrix of math nodes,row sep=6em,column sep=18em,minimum width=2em]
  % {
  %   \kf{f}  & \set{\prime*{f}} \\
  %   \vmodel*{f} & \set{\rn{sat}} \\};
  % \path[-stealth]
  % (m-1-1) edge node [left] {$\kf{VSAT(f)}$} (m-2-1)
  % edge [double] node [below] {\setc{\prime*{f}}{\exists C.\ \sem[C]{f} = \prime*{f}, \prime*{f} \in \pl}} (m-1-2)
  % (m-2-1.east|-m-2-2) edge node [below] {\setc{\submodel{f}{C}}{\exists C.\ \sem[C]{f} = \prime*{f}, \prime*{f} \in \pl}} (m-2-2)
  % (m-1-2) edge node [right] {$\kf{SAT(\prime*{f})}$} (m-2-2)
  % (m-1-1) edge[below] (m-1-2)
  % % edge [dashed,-] (m-2-1)
  % ;




% \begin{tikzpicture}
%   \begin{scope}[every node/.style={circle,thick,draw}]
%     \node (A) at (0,0) {A};
%     % \node (B) at (0,3) {B};
%     \node (B) [right =of A] {B};
%     \node (C) [below =of A] {C};
%     \node (D) [right =of C] {D};
%   \end{scope}

%   \begin{scope}[%>={Stealth[black]},
%     every node/.style={fill=white,circle},
%     every edge/.style={draw=black,thick}]
%     \path [->] (A) edge (B);
%     \path [->] (B) edge (D);
%     \path [->] (C) edge (D);
%     \path [->] (A) edge (C);
%   \end{scope}



% \tikzstyle{block}    = [draw,fill=white!20,node distance = 3.5cm,align=center]
% \tikzstyle{inEdge}   = [fill=white, text width=1cm]
% \tikzstyle{overEdge} = [midway,above]
% \tikzstyle{input}    = [fill=white!20,node distance = 2.2cm,align=center, text width=1cm]
% \tikzstyle{double} = [draw, anchor=text, rectangle split,rectangle split parts=2]
% % diameter of semicircle used to indicate that two lines are not connected
% \tikzstyle{branch}=[fill,shape=circle,minimum size=3pt,inner sep=0pt]
% \tikzstyle{pinstyle} = [pin edge={to-,thin,black}]
% \begin{center}
% \begin{tikzpicture}[>=latex]


% \end{tikzpicture}%
% \end{center}
  \caption{Commuting diagram showing variation preservation for a \ac{vpl}
    formula $\kf{f}$.}%
  \label{vsat:commuting-diagram}
\end{figure}
%
We have formalized variational satisfiability solving. In this section we prove
that our method is variation preserving. Variation preservation is a key
property for variational or variation-aware systems. A system is variation
preserving if and only if processing the variational artifact produces a
variational result which can be used to recover semantically identical plain
results for each variant. For this work, variation preservation means that the
variational solver should find the same results as a plain solver that is
solving every variant.

The property of variation preservation is presented in
\autoref{vsat:commuting-diagram}. The commuting diagram states that for any
\ac{vpl} formula $\kf{f}$, if we configure $\kf{f}$ to find all possible plain
variants, and then run a plain \ac{sat} solver on each variant, we should find
the same number of \rn{sat}/\rn{unsat} results we would find if we ran the
variational solver on $\kf{f}$, received a variational model, and then
substituted every total configuration into the model. Note that the models may
still differ, because a formula may have more than one satisfying model.

To show variation preservation we must show two properties: First, if a variant
exists then it is found by the variational solver. Second, that if the variant
is found by the variational solver it is communicated to the base solver without
\emph{any loss} and without \emph{anything extra}. In other words if a variant
is found then it is communicated to the base solver without losing any terms and
without gaining any terms during communication.

We begin by showing that accumulation does not lose any information. We use
mathematical sets as a formal representation of symbolic values.

\begin{lemma}[Accumulation progress]
  \label{lemma:acc:all-terms}
  Given a \ac{vpl} formula \fV{}, accumulation visits all sub-terms $\tV{} \in
  \fV{}$.
\end{lemma}
%
\begin{proof}
  By construction, for any formula \fV{} all terms in \tV{} are reachable by one
  or repeated applications of the congruence rules \acOrV{}, \acAndV{} and
  \acNotV{}.
\end{proof}

\begin{lemma}[Accumulation preserves choices]
  \label{lemma:acc:preserving:choices}
  For any \ac{vpl} formula $\kf{f}$, and any store \aStore{}, accumulation on
  \fV{} with \aStore{} preserves choices.
\end{lemma}
%
\begin{proof}
  By construction, for any formula $\kf{f}$ and any store, \aStore{},
  accumulation can only process choices with the \acChc{} rule which preserves
  choices.
\end{proof}

\begin{lemma}[Accumulation preserves plain terms]
  \label{lemma:acc:preserving:plain}
  For any \ac{vpl} formula \fV, and any store \aStore{}, if a term $\tV{} \in
  \fV$ and \tV{} is not a choice, then $\exists\sV{}\prime*{\aStore{}}\ \forall\prime*{f}.\ \tV
  \in \sV{} \in \prime*{f}$ where \sV{} is a symbolic value, and \prime*{f} is
  the accumulated result of \fV{}.
\end{lemma}
%
\begin{proof}
  By structural induction on \tV{}. The base case is \tV{} is a terminal. If
  \tV{} is terminal than a computation rule such as \acRef{} will convert it to
  \sV{} and store it in \aStore{} yielding \prime*{\aStore{}}. If \tV{} is not a
  terminal then it is a logical connective with one or two children and a root
  operator. There are three cases for the operator: $\neg$, $\vee$, and
  $\wedge$, each of which will have a similar proof. We show the remainder of
  the proof assuming the operator is $\vee$ because it is a more general case
  than $\neg$ and identical to $\wedge$ in the proof. The induction hypothesis
  is given the immediate sub-terms $v_{1}$ and $v_{2}$ of \tV{}, accumulation
  over the sub-terms preserves plain terms. There are two rules to consider:
  \acOrS{} or \acOrV{}. By assumption \tV{} is plain, and therefore so are
  $v_{1}$ and $v_{2}$. Thus by \acOrS{} and the induction hypotheses we know
  that (\aStore{}, $v_1$) \accumulation (\aStore{1}, $s_1$) and (\aStore{1},
  $v_2$) \accumulation (\aStore{2}, $s_2$) are plain term preserving. Then
  application of \por{} we have $\sV$ and $\prime*{\aStore{}}$ under the
  assumption the solver is term preserving.
\end{proof}
%
\begin{theorem}[Accumulation is term preserving]
  \label{thm:accumulation:term-preserving}
  No terms are lost during accumulation of a formula \fV{}.
\end{theorem}
%
\begin{proof}
  By \autoref{lemma:acc:all-terms} all terms in a \ac{vpl} formula are visited,
  by \autoref{lemma:acc:preserving:plain} all plain terms are preserved sure
  conversion to symbolic values, and by \autoref{lemma:acc:preserving:choices}
  all choices and thus variants are preserved.
\end{proof}

Similarly we must show that terms are not lost during evaluation. To not lose
terms in evaluation means that any term reduced to a \unit{} is considered for
future \rn{check-sat} calls as long as the assertions are not removed from the
assertion stack:

\begin{lemma}[\unit{} values are shared]
  During evaluation, if a term is evaluated to \unit{} then the term is shared
  among variants in future assertion levels or the variant considered in the
  current assertion level.
\end{lemma}
%
\begin{proof}
  By definition if a term is \emph{not} shared, then it is in some variants and
  not in others and therefore in a choice. By construction, evaluation can only
  produce \unit{} values from symbolic value, which can only be created by
  calling accumulation. There are two cases: either a symbolic value is returned
  by accumulation or an accumulated formula is returned. If a symbolic value is
  returned, then it can be evaluated to a \unit{} via \evSym{}; placing it on
  the assertion stack of the base solver and therefore sharing it for all
  subsequent variants as it is deeper in the stack than future assertions. If an
  accumulated formula is returned, then there are three cases for the logical
  connective at the root of the formula: $\neg$, $\vee$ and $\wedge$. Only
  $\wedge$ allows evaluation to continue, in which case \evSym{} and \evAnd{}
  proceed until a connective other than $\wedge$ is at the root of the sub-term.
  By \autoref{thm:accumulation:term-preserving} accumulation preserves terms.
  Thus for each sub-term which is a symbolic value \evSym{} applies and is
  therefore shared. For each sub-term which is not a symbolic value, the
  sub-term is either a choice or an accumulated formula which cannot be further
  reduced.
\end{proof}

\begin{lemma}[Evaluation progress]
  \label{lemma:ev:all-terms}
  Given a \ac{vpl} formula \fV{}, for any term $\tV{} \in \fV{}$ either
  evaluation processes \tV{} or calls accumulation on \tV{}.
\end{lemma}
%
\begin{proof}
  By structural induction on \tV{}. For every possible connective there is a
  case in the evaluation inference rules. From \evAnd{} evaluation is
  propagated over $\wedge$, for $\neg$ and $\vee$ must switch to accumulation to
  process the term.
\end{proof}

With these properties we can finally show that choice removal finds all
variants, and communicates the variants to the base solver without losing or
gaining terms which do not exist in the variant. For the proofs we do not assume
an input \vc{}.

\begin{theorem}[Choice removal progress]
  \label{thm:choice-removal:progress}
  Given a \ac{vpl} formula \fV{}, choice removal processes all variants $v \in
  \variants{\fV}$.
\end{theorem}
%
\begin{proof}
  Given \fV{}, $\exists \prime*{\fV}$ such that \prime*{\fV} is the variational
  core of \fV{}. Then by structural induction on \prime*{\fV{}}, there are four
  cases: The base case is \prime*{\fV} is \unit{} with evaluation context
  \inRoot{}, and thus \crEval{} applies. The inductive cases have a connectives
  at the root of \prime*{\fV}. If \prime*{\fV} is a choice with evaluation
  context \zV{} then by structural induction on $C$ either \crChc{}, \crChcT{}
  or \crChcF{} applies. \crChc{} is the important rule. If \crChc{} applies then
  $D \notin \domain{C}$ and both variants are recursively processed with the
  same evaluation context and $\set{D} \cup \domain(C)$. Therefore, the
  evaluation context is shared between variants and no terms are lost. The
  \crChcT{} and \crChcF{} rules ensure variants do not cross communicate. Thus,
  if $D \in \domain{C}$ then either \crChcT{} or \crChcF{} applies and the
  current variant is recursively processed. If $\neg$, $\vee$ or $\wedge$ are at
  the root either \crAnd{}, \crOr{} or \crNot{} applies, forcing choice removal
  to recur into the left child. Since, variational cores are only composed of
  symbolic values and choices either the left child is a choice or a symbolic.
  If the left child is a choice then it is processed as described above. If the
  left child is a symbolic value then only \crAndL{}, \crOrL{} and \crNotIn{}
  apply, then by structural induction on \zV{} choice removal processes the
  right child. Once in the right child only \crAndR{}, \crOrR{} apply thereby
  folding symbolic values using accumulation. Thus, every choice will be
  reached, by manipulation of the configuration every variant is processed with
  \crChc{}, and variants are never mixed due to \crChcT{} and \crChcF{}.
\end{proof}

\begin{theorem}[Choice removal preservation]
  Given a \ac{vpl} formula \fV{}, choice removal does not lose terms during
  processing \fV{} and processed variants do not gain terms.
\end{theorem}
%
\begin{proof}
  The proof is nearly identical in structure to
  \autoref{thm:choice-removal:progress}. By
  \autoref{thm:accumulation:term-preserving} and threading of \aStore{} through
  choice removal terms are not lost. Terms are not gained in variants due to
  \crChcT{} and \crChcF{} as described in \autoref{thm:choice-removal:progress}.
\end{proof}



%%% Local Variables:
%%% mode: latex
%%% TeX-master: "../../thesis"
%%% End: