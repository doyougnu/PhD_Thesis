\label{section:vsat:variation-preservation}
%
\begin{figure}[h]
  \centering
    \begin{tikzcd}[column sep=8cm,row sep=3cm]
    \kf{f} \arrow[r, Rightarrow, "\setc{\prime*{f}}{\exists C.\ \sem[C]{f} = \prime*{f},\ \prime*{f}\ \in\ \pl}"] %\arrow[r, shift left] \arrow[r, shift right]
    \arrow[d, swap, "VSAT(f)"] &
    \set{\prime*{f}} \arrow[d, Rightarrow, "SAT(\prime*{f})"]  \\
    \vmodel*{f}  \arrow[r, Rightarrow, "\setc{\submodel{f}{C}}{\exists C.\ \sem[C]{f} = \prime*{f},\ \prime*{f}\ \in\ \pl}"]
                 % \arrow[r, shift left]
                 % \arrow[r, shift right]
    &  \set{\rn{\Large sat}}
  \end{tikzcd}
  % \matrix (m) [matrix of math nodes,row sep=6em,column sep=18em,minimum width=2em]
  % {
  %   \kf{f}  & \set{\prime*{f}} \\
  %   \vmodel*{f} & \set{\rn{sat}} \\};
  % \path[-stealth]
  % (m-1-1) edge node [left] {$\kf{VSAT(f)}$} (m-2-1)
  % edge [double] node [below] {\setc{\prime*{f}}{\exists C.\ \sem[C]{f} = \prime*{f}, \prime*{f} \in \pl}} (m-1-2)
  % (m-2-1.east|-m-2-2) edge node [below] {\setc{\submodel{f}{C}}{\exists C.\ \sem[C]{f} = \prime*{f}, \prime*{f} \in \pl}} (m-2-2)
  % (m-1-2) edge node [right] {$\kf{SAT(\prime*{f})}$} (m-2-2)
  % (m-1-1) edge[below] (m-1-2)
  % % edge [dashed,-] (m-2-1)
  % ;




% \begin{tikzpicture}
%   \begin{scope}[every node/.style={circle,thick,draw}]
%     \node (A) at (0,0) {A};
%     % \node (B) at (0,3) {B};
%     \node (B) [right =of A] {B};
%     \node (C) [below =of A] {C};
%     \node (D) [right =of C] {D};
%   \end{scope}

%   \begin{scope}[%>={Stealth[black]},
%     every node/.style={fill=white,circle},
%     every edge/.style={draw=black,thick}]
%     \path [->] (A) edge (B);
%     \path [->] (B) edge (D);
%     \path [->] (C) edge (D);
%     \path [->] (A) edge (C);
%   \end{scope}



% \tikzstyle{block}    = [draw,fill=white!20,node distance = 3.5cm,align=center]
% \tikzstyle{inEdge}   = [fill=white, text width=1cm]
% \tikzstyle{overEdge} = [midway,above]
% \tikzstyle{input}    = [fill=white!20,node distance = 2.2cm,align=center, text width=1cm]
% \tikzstyle{double} = [draw, anchor=text, rectangle split,rectangle split parts=2]
% % diameter of semicircle used to indicate that two lines are not connected
% \tikzstyle{branch}=[fill,shape=circle,minimum size=3pt,inner sep=0pt]
% \tikzstyle{pinstyle} = [pin edge={to-,thin,black}]
% \begin{center}
% \begin{tikzpicture}[>=latex]


% \end{tikzpicture}%
% \end{center}
  \caption{Commuting diagram showing variation preservation for a \ac{vpl}
    formula $\kf{f}$.}%
  \label{vsat:commuting-diagram}
\end{figure}
%
We have formalized variational satisfiability solving. In this section we prove
that our method is variation preserving. Variation preservation is a key
property for variational or variation-aware systems. A system is variation
preserving if and only if processing the variational artifact produces a
variational result which can be used to recover semantically identical plain
results for each variant. For this work, variation preservation means that the
variational solver should find the same results as a plain solver that is
solving every variant.

The property of variation preservation is presented in
\autoref{vsat:commuting-diagram}. The commuting diagram states that for any
\ac{vpl} formula $\kf{f}$, if we configure $\kf{f}$ to find all possible plain
variants, and then run a plain \ac{sat} solver on each variant, we should find
the same number of \rn{sat}/\rn{unsat} results we would find if we ran the
variational solver on $\kf{f}$, received a variational model, and then
substituted every total configuration into the model. Note that the models may
still differ, because a formula may have more than one satisfying model.

To show variation preservation we must show two properties: First, if a variant
exists then it is found by the variational solver. Second, that if the variant
is found by the variational solver it is communicated to the base solver without
\emph{any loss} and without \emph{anything extra}. In other words if a variant
is found then it is communicated to the base solver without losing any terms and
without gaining any terms during communication.

We begin by showing that accumulation does not lose any information. We use
mathematical sets as a formal representation of symbolic values.

\begin{lemma}[Accumulation progress]
  \label{lemma:acc:all-terms}
  Given a \ac{vpl} formula \fV{}, accumulation visits all sub-terms $\tV{} \in
  \fV{}$.
\end{lemma}
%
\begin{proof}
  By construction, for any formula \fV{} all terms in \tV{} are reachable by one
  or repeated applications of the congruence rules \acOrV{}, \acAndV{} and
  \acNotV{}.
\end{proof}

\begin{lemma}[Accumulation preserves choices]
  \label{lemma:acc:preserving:choices}
  For any \ac{vpl} formula $\kf{f}$, and any store \aStore{}, accumulation on
  \fV{} with \aStore{} preserves choices.
\end{lemma}
%
\begin{proof}
  By construction, for any formula $\kf{f}$ and any store, \aStore{},
  accumulation can only process choices with the \acChc{} rule which preserves
  choices.
\end{proof}

\begin{lemma}[Accumulation preserves plain terms]
  \label{lemma:acc:preserving:plain}
  For any \ac{vpl} formula \fV, and any store \aStore{}, if a term $\tV{} \in
  \fV$ and \tV{} is not a choice, then $\exists\sV{}\prime*{\aStore{}}\ \forall\prime*{f}.\ \tV
  \in \sV{} \in \prime*{f}$ where \sV{} is a symbolic value, and \prime*{f} is
  the accumulated result of \fV{}.
\end{lemma}
%
\begin{proof}
  By structural induction on \tV{}. The base case is \tV{} is a terminal. If
  \tV{} is terminal than a computation rule such as \acRef{} will convert it to
  \sV{} and store it in \aStore{} yielding \prime*{\aStore{}}. If \tV{} is not a
  terminal then it is a logical connective with one or two children and a root
  operator. There are three cases for the operator: $\neg$, $\vee$, and
  $\wedge$, each of which will have a similar proof. We show the remainder of
  the proof assuming the operator is $\vee$ because it is a more general case
  than $\neg$ and identical to $\wedge$ in the proof. The induction hypothesis
  is given the immediate sub-terms $v_{1}$ and $v_{2}$ of \tV{}, accumulation
  over the sub-terms preserves plain terms. There are two rules to consider:
  \acOrS{} or \acOrV{}. By assumption \tV{} is plain, and therefore so are
  $v_{1}$ and $v_{2}$. Thus by \acOrS{} and the induction hypotheses we know
  that (\aStore{}, $v_1$) \accumulation (\aStore{1}, $s_1$) and (\aStore{1},
  $v_2$) \accumulation (\aStore{2}, $s_2$) are plain term preserving. Then
  application of \por{} we have $\sV$ and $\prime*{\aStore{}}$ under the
  assumption the solver is term preserving.
\end{proof}
%
\begin{theorem}[Accumulation is term preserving]
  By \autoref{lemma:acc:all-terms}, \autoref{lemma:acc:preserving:choices} and
  \autoref{lemma:acc:preserving:plain}.
\end{theorem}


%%% Local Variables:
%%% mode: latex
%%% TeX-master: "../../thesis"
%%% End: