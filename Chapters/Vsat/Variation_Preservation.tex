\label{section:vsat:variation-preservation}
%
\begin{figure}[h]
  \centering
  \begin{tikzpicture}
  \matrix (m) [matrix of math nodes,row sep=6em,column sep=12em,minimum width=2em]
  {
    \kf{f}  & \set{\prime*{f}} \\
    \vmodel*{f} & \set{\rn{sat}} \\};
  \path[-stealth]
  (m-1-1) edge node [left] {$\kf{VSAT(f)}$} (m-2-1)
  edge [double] node [below] {\setc{\prime*{f}}{\exists C.\ \sem[C]{f} = \prime*{f}, \prime*{f} \in \pl}} (m-1-2)
  (m-2-1.east|-m-2-2) edge node [below] {todo} (m-2-2)
  (m-1-2) edge node [right] {$\kf{SAT(\prime*{f})}$} (m-2-2)
  % edge [dashed,-] (m-2-1)
  ;
\end{tikzpicture}




% \begin{tikzpicture}
%   \begin{scope}[every node/.style={circle,thick,draw}]
%     \node (A) at (0,0) {A};
%     % \node (B) at (0,3) {B};
%     \node (B) [right =of A] {B};
%     \node (C) [below =of A] {C};
%     \node (D) [right =of C] {D};
%   \end{scope}

%   \begin{scope}[%>={Stealth[black]},
%     every node/.style={fill=white,circle},
%     every edge/.style={draw=black,thick}]
%     \path [->] (A) edge (B);
%     \path [->] (B) edge (D);
%     \path [->] (C) edge (D);
%     \path [->] (A) edge (C);
%   \end{scope}



% \tikzstyle{block}    = [draw,fill=white!20,node distance = 3.5cm,align=center]
% \tikzstyle{inEdge}   = [fill=white, text width=1cm]
% \tikzstyle{overEdge} = [midway,above]
% \tikzstyle{input}    = [fill=white!20,node distance = 2.2cm,align=center, text width=1cm]
% \tikzstyle{double} = [draw, anchor=text, rectangle split,rectangle split parts=2]
% % diameter of semicircle used to indicate that two lines are not connected
% \tikzstyle{branch}=[fill,shape=circle,minimum size=3pt,inner sep=0pt]
% \tikzstyle{pinstyle} = [pin edge={to-,thin,black}]
% \begin{center}
% \begin{tikzpicture}[>=latex]


% \end{tikzpicture}%
% \end{center}
  \caption{Commuting diagram showing variation preservation for a \ac{vpl}
    formula $\kf{f}$.}%
  \label{vsat:commuting-diagram}
\end{figure}
We have formalized variational satisfiability solving. In this section we prove
that our method is variation preserving. Variation preserving is a key property
for variational or variation-aware systems. A system is variation preserving if
and only if processing the variational artifact produces a variational result
which can be used to recover semantically identical plain results for each
variant. For this work, variation preservation means that the variational solver
should find the same results as a plain solver that is solving every variant.

The property of variation preservation is presented in
\autoref{vsat:commuting-diagram}. The commuting diagram states that for any
\ac{vpl} formula $\kf{f}$, if we configure $\kf{f}$ to find all possible plain
variants, and then run a plain \ac{sat} solver on each variant, we should find
the same number of \ac{sat} results we would find if we ran the variational
solver on $\kf{f}$, received a variational model, and then substituted every
total configuration into the model.