\label{section:vsat:models}
%
Classic \ac{sat} models map variables to Boolean values; variational models map
variables to variation contexts that record the variants where the variable was
assigned \tru{}. The variational context for a variable $r$ is denoted as
\vc{r}, and a variational model reserves a special variable called \SatVar{} to
track the configurations that were found satisfiable.
%
\begin{figure}[h]
  \centering
  \begin{subfigure}[t]{\textwidth}
  \begin{tabbing}
    \qquad \quad \= $\aV{} \rightarrow$ \tru{} \\
    \> $\bV{} \rightarrow$ \fls{} \\
    \> $\cV{} \rightarrow$ \tru{} \\
    \> $\pV{} \rightarrow$ \tru{} \\
    \> $\qV{} \rightarrow$ \fls{} \\
    \quad $C_{FF}$ = \{(\AV{}, \fls{}), (\BV{}, \fls{})\} \\
  \end{tabbing}
\end{subfigure}%
\begin{subfigure}[t]{\textwidth}
  \begin{tabbing}
    \qquad \quad \= $\aV{} \rightarrow$ \tru{} \\
    \> $\bV{} \rightarrow$ \fls{} \\
    \> $\cV{} \rightarrow$ \tru{} \\
    \> $\pV{} \rightarrow$ \tru{} \\
    \> $\qV{} \rightarrow$ \fls{} \\
    \quad $C_{FT}$ = \{(\AV{}, \fls{}), (\BV{}, \tru{})\} \\
  \end{tabbing}
\end{subfigure}%
\begin{subfigure}[t]{\textwidth}
  \begin{tabbing}
    \qquad \quad \= $\aV{} \rightarrow$ \tru{} \\
    \> $\bV{} \rightarrow$ \fls{} \\
    \\
    \> $\pV{} \rightarrow$ \fls{} \\
    \> $\qV{} \rightarrow$ \tru{} \\
    \quad $C_{TT}$ = \{(\AV{}, \tru{}), (\BV{}, \tru{})\} \\
  \end{tabbing}
\end{subfigure}%
  \caption{Possible plain models for variants of $\kf{f}$.}%
  \label{fig:models:plain}
\end{figure}
\begin{figure}[h]
  \centering
  \begin{subfigure}[t]{\textwidth}
  \begin{tabbing}
  \qquad \qquad \= $\_Sat \rightarrow\ (\neg \AV{} \wedge \neg \BV{}) \vee (\neg \AV{} \wedge \BV{}) \vee (\AV{} \wedge \BV)$ \\
  \> $\aV{} \rightarrow\ (\neg \AV{} \wedge \neg \BV{}) \vee (\neg \AV{} \wedge \BV{}) \vee (\AV{} \wedge \BV)$ \\
  \> $\bV{} \rightarrow\ \fls{}$ \\
  \> $\cV{} \rightarrow\ (\neg{} \AV{} \wedge{} \neg{} \BV{}) \vee{} (\neg{} \AV{} \wedge{} \BV{})$ \\
  \> $\pV{} \rightarrow\ (\neg \AV{} \wedge \neg \BV{}) \vee (\neg \AV{} \wedge \BV{})$ \\
  \> $\qV{} \rightarrow\ (\AV{} \wedge \BV)$
\end{tabbing}
\end{subfigure}

  \caption{Variational model corresponding to the plain models in
    \autoref{fig:models:plain}.}%
  \label{fig:models:var}
\end{figure}
%
As an example, consider an altered version of the query formula from the
previous section $f = ((\aV{} \wedge \neg \bV{}) \wedge \chc[A]{\aV{}
  \rightarrow \neg \pV{}, \cV{}}) \wedge ((\pV{} \wedge \neg \qV{}) \vee
\chc[B]{\qV{}, \pV{}})$. We can easily see that one variant, with configuration
$\{(\AV{},\tru{}), (\BV{},\fls{})\}$ is unsatisfiable. If the remaining variants
are satisfiable, then three models would result, as illustrated in
\autoref{fig:models:plain}; with the corresponding variational model shown in
\autoref{fig:models:var}.

We see that \Satvc{} consists of three disjuncted terms, one for each
satisfiable variant. Variational models are flexible; a satisfiable assignment
of the query formula can be found by calling \ac{sat} on \Satvc{}. Assuming the
model $C_{FT} = \{(\AV{}, \fls{}), (\BV{}, \tru{})\}$ is returned, one can find
a variable's value through substitution with the configuration; for example,
substituting the returned model on \vc{c} yields:
%
\begin{align*}
  \cV{} \rightarrow\ & (\neg \AV{} \wedge \neg \BV{}) \vee (\neg \AV{} \wedge \BV{}) & \text{\vc{} for \cV{}} \\
  \cV{} \rightarrow\ & (\neg \fls{} \wedge \neg \tru{}) \vee (\neg \fls{} \wedge \tru{}) & \text{Substitute \fls{} for \AV{}, \tru{} for \BV{}} \\
  \cV{} \rightarrow\ & \tru{} & \text{Result}
\end{align*}%
%
Furthermore, finding variants where a variable such as \cV{} is satisfiable is
reduced to $\kf{SAT(\vc{\cV{}})}$

Variational models are constructed incrementally by merging each new plain model
returned by the solver into the variational model. A merge requires the current
configuration, the plain model, and current \vc{} of a variable. Variables are
initialized to \fls{}. For each variable $i$ in the model, if $i$'s assignment
is \tru{} in the plain model, then the configuration is translated to a
variation context and disjuncted with \vc{i}. For example, to merge the
$C_{FT}$'s plain model to the variational model in \autoref{fig:models:var},
$C_{FT}$'s configuration is converted to $\neg \AV{} \wedge \BV{}$. This clause
is disjuncted for variables assigned \tru{} in the plain model: \vc{\aV{}},
\vc{\cV{}}, and \vc{\pV{}}, even if they are new (e.g., \cV{}). Variables
assigned \fls{} are skipped, thus \vc{\qV{}} remains \fls{}. For example, in the
next model $C_{TT}$, \cV{} is \fls{} thus \vc{\cV{}} remains unaltered, while
\vc{\qV{}} flips to \tru{} hence \vc{\qV{}} records $\AV{} \wedge \BV{}$.
Variables such as \bV{}, whose \vc{}'s stay \fls{} are called \textit{constant}.

Variational models are constructed in \ac{dnf}, and form a monoid with $\vee$ as
the semigroup operation, and \fls{} as the unit value. We note this for
mathematically inclined readers and those looking to implement their own
variational solver because it is an important property for asynchronous
implementations of variational satisfiability solvers.


%%% Local Variables:
%%% mode: latex
%%% TeX-master: "../../thesis"
%%% End:
