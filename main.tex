% -*- mode: latex; -*-
\documentclass[11pt]{article}

\topmargin      0.0in
\headheight     0.0in
\headsep        0.0in
\oddsidemargin  0.0in
\evensidemargin 0.0in
\textheight     9.0in
\textwidth      6.5in

%%%%%%%%%%%%%%%%%%%%%%%%%%%%%%% Packages %%%%%%%%%%%%%%%%%%%%%%%%%%%%%%%%%%%%%%%
\usepackage{graphics}
\usepackage{epsfig}
\usepackage{times}
\usepackage{amsmath}
\usepackage{amssymb}                 %% for more arrows like rightarrowtail
\usepackage{amsfonts}                %% \mathbb
\usepackage{mathtools}               %% for coloneqq and others
\usepackage{cases}                   %% better math brackets
\usepackage{mathpartir}              %%% for inference rules
\usepackage[nolist]{acronym}
\usepackage{hyperref}                %% for auto references
\usepackage{xpunctuate}              %% for punctuation's after macros
\usepackage{cite}                    %% for bibliography ranges
\usepackage{subcaption}              %% For complex figures with subfigures/subcaptions
\usepackage{wrapfig}                 %% wrapping figures around text
\usepackage{listings}                %% for source code lstlisting
\usepackage{xcolor}                  %% For listings
\usepackage{caption}
\usepackage{tikz}                    %% assertion stack visualization
\usetikzlibrary{matrix}
\usetikzlibrary{arrows}
\usetikzlibrary{positioning}
\usetikzlibrary{shapes.multipart}
\usetikzlibrary{automata}

\usepackage{lib/cc}
\usepackage{lib/lambda}
\usepackage{lib/paperCommands}
%%%%%%%%%%%%%%%%%%%%%%%%%%%%%%% Packages %%%%%%%%%%%%%%%%%%%%%%%%%%%%%%%%%%%%%%%

\lstset{%
  frame=top,frame=bottom,
  basicstyle=\small\normalfont\ttfamily,    % the size of the fonts that are used for the code
  stepnumber=1,                           % the step between two line-numbers. If it is 1 each line will be numbered
  numbersep=10pt,                         % how far the line-numbers are from the code
  tabsize=2,                              % tab size in blank spaces
  extendedchars=true,                     %
  breaklines=true,                        % sets automatic line breaking
  captionpos=t,                           % sets the caption-position to top
  mathescape=true,
  showstringspaces=false,
  escapeinside={(*@}{@*)},
}%

\title{{\bf Variational Satisfiability Solving} \\
\it PhD. Thesis}%
\author{{\bf Jeffrey M. Young}  \\
Department of Electrical Engineering and Computer Science \\
Oregon State University\\
{\small youngjef@oregonstate.edu}
}
\date{\today}

\begin{document}
\pagestyle{plain}
\pagenumbering{roman}
\maketitle

\begin{abstract}
  Over the last two decades, satisfiability and satisfiability-modulo theory
(SAT/SMT) solvers have grown powerful enough to be general purpose reasoning
engines employed in various areas of software engineering and computer science.
However, most practical use cases of SAT/SMT solvers require not just solving a
single SAT/SMT problem, but solving sets of related SAT/SMT problems. This
discrepancy was directly addressed by the SAT/SMT community with the invention
of incremental SAT/SMT solving. However, incremental SAT/SMT solvers require
end-users to hand write a program which dictates terms in a set that are shared
between problems and terms which are unique. By placing the onus on end-users to
write a program, incremental solvers couple the end-users' solution to the
end-users' exact sequence of SAT/SMT problems---making the solution overly
specific---and require the end-user to write extra infrastructure to coordinate
or handle the results. In this thesis, I apply results from research on
\emph{variational} programming languages to the domain of SAT/SMT solvers to
automate this interaction, creating the first variational SAT/SMT solver. I
demonstrate numerous benefits to this approach: End-users need only identify the
set of SAT/SMT problems to solve rather than identify the set \emph{and} provide
a program. Otherwise difficult optimizations can now be automatically detected
and applied. Through use of variational constructs, the variational SAT/SMT can
be made asynchronous and both single threaded and multi-threaded versions of
variational SAT/SMT solvers are more performent in their expected use case.

%%% Local Variables:
%%% mode: latex
%%% TeX-master: "../main"
%%% End:
\end{abstract}

\pagenumbering{arabic}

\section{Introduction}
One of the most important aspects of any programming language is the ability to
control complexity, especially as software written in that language grows. The
burgeoning field of \emph{variation theory} and \emph{variational
  programming}~\cite{EW11gttse,EW11tosem,HW16fosd,CEW16ecoop,Walk14onward}
attempt to control complexity which is induced into software when many
\emph{similar yet distinct} kinds of software must coexist. For example, the
same piece of software is often \emph{ported} to other platforms, creating
similar, yet distinct instances of that software which must be maintained. Such
instances of variation are ubiquitous: Web applications are tested on multiple
servers; programming languages maintain backwards compatibility and so do
software libraries; databases evolve over time, local and data; and device
drivers must work with varying processors and architectures. Variation theory
and variational programming has been successful in small systems\todo{cite}, yet
it has not been tested in a performance demanding practical domain. In the words
of Joe Armstrong\cite{armstrongThesis}, ``No theory is complete without proof
that the ideas work in practice''; this is the central project of this thesis,
to put the ideas of \emph{variation} and \emph{variational programming}to the
test in the practical domain of \ac{sat}.

The major contribution of the thesis is the formalization of a \emph{\ac{vpl}},
\emph{variational \acl{sat}}, and the construction of a \emph{variational
  \ac{sat} solver}. The thesis makes several other contributions. It extends
variational \acl{sat} to variational \ac{smt}. It demonstrates reusable
techniques and architecture for constructing \emph{variational or
  variation-aware} systems in other domains using the non-variational
counterparts of these systems. It shows that, with the concept of variation, the
variational \ac{smt} and \ac{sat} solvers can be trivially parallelized. Lastly,
the thesis provides a general algorithm to construct variational strings from a
set of non-variational strings and argues for the proliferation of variation
theory to other domains in computer science.

%%% Local Variables:
%%% mode: latex
%%% TeX-master: "../thesis"
%%% End:

\section{Background}
~\label{sec:background}


%%% Local Variables:
%%% mode: latex
%%% TeX-master: "../thesis"
%%% End:

\section{Proposal Contribution 1: VPL =\@ Variation + Propositional Logic}
~\label{sec:vpl} In this section, I present the syntax and semantics of
variational propositional logic. While this section fulfills the majority of
\autoref{vpl-deliverable} I restate it here to serve as background for
\autoref{sec:vsat}, \autoref{sec:vsmt} and the choice calculus. I conclude the
section by summarizing work left to do.
% \revised{and provide background to the choice calculus as the logic is
% formalized.}
%
The logic is a conservative extension of classic two-valued logic
(\pl{})\footnote{Notation for propositional logic comes from work on many-valued
  logic, see~\cite{Rescher1969-RESML}.} with a \emph{choice} construct from the
choice calculus~\cite{EW11tosem,Walk13thesis}, a formal language for describing
variation.

% We define the syntax and semantics of \vpl{} in Section~\ref{sec:logic:vpl} and
% demonstrate its use by encoding the example from Section~\ref{sec:bkgrnd} into
% \vpl{} in Section~\ref{sec:logic:enc}.

\begin{figure}[h!]
  \begin{subfigure}[t]{\textwidth}
    \centering
    \begin{syntax}
  % D & \Coloneqq & \text{(any dimension name)} & \textit{Dimension} \\ \\
  % t & \Coloneqq & r & \textit{Variable reference} \\
  % & | & \mathit{T} & \textit{True} \\
  % & | & \mathit{F} & \textit{False} \\ \\

  t & \Coloneqq & r \quad|\quad \tru \quad|\quad \fls
    & \textit{Variables and Boolean literals} \\[1.5ex]

  f & \Coloneqq & t    & \textit{Terminal} \\
    & | & \neg f       & \textit{Negate} \\
    & | & f \vee f     & \textit{Or} \\
    & | & f \wedge f   & \textit{And} \\
    & | & \chc[D]{f,f} & \textit{Choice} \\
\end{syntax}

    \caption{Syntax of \vpl{}.}%
    \label{fig:cc:stx}
  \end{subfigure}
  \begin{subfigure}[t]{\textwidth}
    \begin{align*}
  C : D \rightarrow \mathbb{B_{\bot}} & \textit{Configuration} \\
  \sem{\cdot} &: f\rightarrow C \rightarrow f
    \qquad\qquad \text{where } C = D\rightarrow\mathbb{B}_\bot \\
  \sem[C]{t}             &= t \\
  \sem[C]{\neg f}        &= \neg \sem[C]{f} \\
  \sem[C]{f_1\wedge f_2} &= \sem[C]{f_1}\wedge\sem[C]{f_2}\\
  \sem[C]{f_1\vee f_2}   &= \sem[C]{f_1}\vee\sem[C]{f_2}\\
  \sem[C]{\chc[D]{f_1,f_2}} &=
    \begin{cases}
      \sem[C]{f_1}                       & C(D) = \true \\
      \sem[C]{f_2}                       & C(D) = \false \\
      \chc[D]{\sem[C]{f_1},\sem[C]{f_2}} & C(D) = \bot \\
    \end{cases}
\end{align*}

    \centering
    \caption{Configuration semantics of \vpl{}.}%
    \label{fig:cc:cfg}
  \end{subfigure}
  \begin{subfigure}[t]{\textwidth}
    \begin{align*}
  \chc[D]{f,f}
    & \equiv f
    & \rn{Idemp} \\
  \chc[D]{\chc[D]{f_1,f_2},f_3}
    & \equiv \chc[D]{f_1,f_3}
    & \rn{Dom-L} \\
  \chc[D]{f_1,\chc[D]{f_2,f_3}}
    & \equiv \chc[D]{f_1,f_3}
    & \rn{Dom-R} \\
  \chc[D_1]{\chc[D_2]{f_1,f_2},\chc[D_2]{f_3,f_4}}
    & \equiv \chc[D_2]{\chc[D_1]{f_1,f_3},\chc[D_1]{f_2,f_4}}
    & \rn{Swap} \\
  \chc[D]{\neg f_1,\neg f_2}
    & \equiv \neg\chc[D]{f_1,f_2}
    & \rn{Neg} \\
  \chc[D]{f_1\vee f_3,\;f_2\vee f_4}
    & \equiv \chc[D]{f_1,f_2}\vee\chc[D]{f_3,f_4}
    & \rn{Or} \\
  \chc[D]{f_1\wedge f_3,\;f_2\wedge f_4}
    & \equiv \chc[D]{f_1,f_2}\wedge\chc[D]{f_3,f_4}
    & \rn{And} \\
  \chc[D]{f_1\wedge f_2,f_1}
    & \equiv f_1\wedge\chc[D]{f_2,\tru}
    & \rn{And-L} \\
  \chc[D]{f_1\vee f_2,f_1}
    & \equiv f_1\vee\chc[D]{f_2,\fls}
    & \rn{Or-L} \\
  \chc[D]{f_1,f_1\wedge f_2}
    & \equiv f_1\wedge \chc[D]{\tru,f_2}
    & \rn{And-R} \\
  \chc[D]{f_1,f_1\vee f_2}
    & \equiv f_1\vee\chc[D]{\fls,f_2}
    & \rn{Or-R}
\end{align*}

    \centering
    \caption{\vpl{} equivalence laws}%
    \label{fig:cc:eqv}
    \vspace{0.4cm}
  \end{subfigure}
\caption{Formal definition of \vpl{}.}%
\label{fig:cc}
\end{figure}
%
% The idea behind the choice calculus extended logic (\vpl{}) is to construct a
% logic that \TODO{is this a word?}deterministically expresses what is shared, and
% what is distinct, in a set of formulas in classic two valued logic (\pl{}).
% Semantically, one can think of \vpl{} as describing, in a deterministic, and
% constrained way, what could be and what must be, where a term in \pl{} must be
% but a term in \vpl{} \textit{could be} optional. Where as you might have encode
% a sentence such as: All men are mortal; Socrates is a man; therefore Socrates is
% mortal in \pl{}, a corresponding sentence in \vpl{} could be: it could be that
% either Socrates or Anaximander is a man; all men are mortal; thus it could be that
% Socrates or Anaximander is a man. Notice the ending of our \vpl{} statement:
% \textit{is a man}, this is purposeful and highlights that in \vpl{} a point of
% variation simultaneously represents two options.
%

\subsection{Syntax}
%
The syntax of variational propositional logic is given in \autoref{fig:cc:stx}.
It extends the propositional formula notation of \pl{} with a single new
connective called a \emph{choice} from the choice calculus.
%
A choice $\chc[D]{f_1,f_2}$ represents either $f_1$ or $f_2$ depending on the
Boolean value of its \emph{dimension} $D$. We call $f_1$ and $f_2$ the
\emph{alternatives} of the choice.
%
Although dimensions are Boolean variables, the set of dimensions is disjoint
from the set of variables from \pl{}, which may be referenced in the leaves of
a formula. We use lowercase letters to range over variables and uppercase
letters for dimensions.
%
The syntax of \vpl{} does not include derived logical connectives, such as
$\rightarrow$ and $\leftrightarrow$. However, such forms can be defined from
other primitives and are assumed throughout the rest of the proposal.

\subsection{Semantics}
%
Conceptually, a variational formula represents several propositional logic
formulas at once, which can be obtained by resolving all of the choices. For
researchers unfamiliar with work on variation, it is useful to think of \ac{vpl}
as analogous to \cpp{ifdef}-annotated \pl{}, where choices correspond to a
disciplined~\cite{LKA:AOSD11} application of \cpp{ifdef} annotations.
%
From a logical perspective, following the many-valued logic of
Kleene~\cite{Rescher1969-RESML}, the intuition behind \ac{vpl} is that a choice is
a placeholder for two equally possible alternatives that is deterministically
resolved by reference to an external environment.
%
In this sense, \ac{vpl} deviates from other many-valued logics, such as modal
logic~\cite{sep-logic-modal}, because a choice \emph{waits} until there is
enough information to choose an alternative (i.e., until the formula is
\emph{configured}).

The \emph{configuration semantics} of \ac{vpl} is given in \autoref{fig:cc:cfg}
and describes how choices are eliminated from a formula. The semantics are
parameterized by a \emph{configuration}\ $C$, which is a partial function from
dimensions to Boolean values.
%
The first four cases of the semantics simply propagate configuration down the
formula, terminating at the leaves. The case for choices is the interesting one:
if the dimension of the choice is defined in the configuration, then the choice
is replaced by its left or right alternative corresponding to the associated
value of the dimension in the configuration. If the dimension is undefined in
the configuration, then the choice is left intact and configuration propagates
into the choice's alternatives.


If a configuration $C$ eliminates all choices in a formula $f$, we call $C$
\emph{total} with respect to $f$. If $C$ does \emph{not} eliminate all choices
in $f$ (i.e., a dimension used in $f$ is undefined in $C$), we call $C$
\emph{partial} with respect to $f$.
%
We call a choice-free formula \emph{plain}, and call the set of all plain
formulas that can be obtained from $f$ (by configuring it with every possible
total configuration) the \emph{variants} of $f$.
%
% Every plain variant of a \vpl{} formula is a \pl{} formula, demonstrating the
% intuition that a \vpl{} formula conceptually represents many plain \pl{}
% formula.


To illustrate the semantics of \vpl{}, consider the formula
$p\wedge\chc[A]{q,r}$, which has two variants: $p\wedge q$ when $C(A)=\true$
and $p\wedge r$ when $C(A)=\false$.
%
From the semantics, it follows that choices in the same dimension are
\emph{synchronized} while choices in different dimensions are
\emph{independent}. For example, $\chc[A]{p,q}\wedge\chc[B]{r,s}$ has four
variants, while $\chc[A]{p,q}\wedge\chc[A]{r,s}$ has only two ($p\wedge r$ and
$q\wedge s$).
%
It also follows from the semantics that nested choices in the same dimension
contain redundant alternatives; that is, inner choices are \emph{dominated} by
outer choices in the same dimension. For example, $\chc[A]{p,\chc[A]{r,s}}$ is
equivalent to $\chc[A]{p,s}$ since the alternative $r$ cannot be reached by any
configuration.
% \begin{theorem}[\vpl{} reducible to \pl{}]
%   \label{thm:vplToPl}
%   For any configuration $C$ and any formula $e$, if $C$ is
%   valid and total with respect to $e$, then $\sem[C]{e} \in \pl{}$
% \end{theorem}
%
% \begin{proof}
%   This follows directly from the semantics of configuration in
%   Figure~\ref{fig:cc:cfg}, and Definition~\ref{tot:conf}. The proof is done
%   by structural induction and case analysis; because we have a total
%   configuration, and the configuration semantic function is a total function,
%   every choice and its' configured alternatives, will be recursively reified for
%   $e$. Then by the definition of \vpl{} a formula which lacks choices is by
%   definition in \pl{}.
% \end{proof}
%
% \begin{lemma}[Variants are plain]
%   By Theorem~\ref{thm:vplToPl} and the fact that variants are found via total
%   configurations
% \end{lemma}
%
As the previous example illustrates, the representation of a \ac{vpl} formula is
not unique; that is, the same set of variants may be encoded by different
formulas. \autoref{fig:cc:eqv} defines a set of equivalence laws for \ac{vpl}
formulas. These laws follow directly from the configuration semantics in
\autoref{fig:cc:cfg} and can be used to derive semantics-preserving
transformations of \ac{vpl} formulas.
%
For example, we can simplify the formula $\chc[A]{p\vee q, p\vee r}$ by first
applying the \rn{Or} law to obtain $\chc[A]{p,p}\vee\chc[A]{q,r}$, then applying
the \rn{Idemp} law to the first argument to obtain $p\vee\chc[A]{q,r}$ in which
the redundant $p$ has been factored out of the choice.

\subsection{Research Plan}

The previous sections describe \ac{vpl} but is missing an efficient strategy for
encoding sets of \pl{} formulas in a \ac{vpl} formula. The proposed thesis will
directly address this gap:

\paragraph{Encoding strategies} This item will produce an efficient algorithm
that combines a set of \pl{} formulas into a single \ac{vpl} formula. Efficient
has two meanings: It should produce a \ac{vpl} formula in reasonable time, and
it should produce a \ac{vpl} formula that has measurably less variation if
possible. Such an algorithm is desirable for two reasons. First, it is
practically important; a result of a previous study of variational
satisfiability solving~\cite{10.1145/3382025.3414965} was that the greater the
\emph{sharing ratio}, the ratio of plain to total terms in a variational
formula, the faster the \ac{vpl} formula was solved, on average. Thus, by
developing a more efficient encoding algorithm, the performance of the solver is
less volatile. Second, it lowers the barrier of use, with such an algorithm, the
a new user because the new user need not understand \ac{vpl}, rather they only
need to identify the problem set they desire to solve.

A naive encoding algorithm is easy to construct: one wraps all formulas in
unique choices and then uses equivalency rules to increase the sharing ratio.
However, it is likely that better algorithms exist. I envision an algorithm that
utilizes a strategy similar to Huffman coding~\cite{4051119} to find similar
formulas to merge.

This work is able to be done in parallel to much of the other deliverables but
intersects with two other items. First, a proof of variation preservation,
\autoref{proof-deliverable}, for a variational solver must include a similar
proof for the encoding strategy. Second, the encoding strategy will affect
solver performance and thus also affect the evaluation,
\autoref{eval-deliverable}. Lastly, the performance of the encoding strategy
itself will require a set of data to be evaluated on. I discuss harvesting such
a set of data from real world software product lines in \autoref{sec:vsmt},
although enough real world data is already available to begin on this item.

%%% Local Variables:
%%% mode: latex
%%% TeX-master: "../thesis"
%%% End:

\section{Proposal Contribution 2: Variational Satisfiability Solving}
~\label{sec:vsat} A core contributions of the proposed thesis is the design and
architecture of a variational satisfiability solver. In this section, I review
the architecture of a prototype solver called \vsat{} and conclude the section
by summarizing work left to do on \autoref{phase-change-deliverable}, \ie{},
assessing the hardness of variational satisfiability problems.

\subsection{Design}
Research on \ac{sat} and incremental \ac{sat} is fast moving with novel \ac{sat}
solvers regularly competing in the International \ac{sat}
Competition~\cite{interSatComp}, to leverage these results we design variational
satisfiability solving to target the incremental interface specified in the
SMT-LIB2~\cite{BarFT-RR-17} standard. Targeting the standard provides several
benefits: an implementation of a variational solver is free to choose any
SMT-LIB2~\cite{BarFT-RR-17} compliant solver ranging from experimental solvers
such as cvc4~\cite{10.1007/978-3-642-22110-1_14}, to industrial strength solvers
such as z3~\cite{10.1007/978-3-540-78800-3_24}. Since the implementation is
\ac{sat} solver agnostic, a variational solver can be run as a
\emph{meta-solver}, \ie{}, a \ac{sat} solver that interleaves, or simultaneously
uses several \ac{sat} solvers to solve a \ac{sat} problem, especially in
asynchronous workloads. Lastly, any result from the \ac{sat} or \ac{smt}
community that enters the standard is supported, this includes new \ac{smt}
theories.

In addition to coupling to SMTLIB2, the variational \ac{sat} solver should not
place further burden on the end-user. Thus users of a variational solver should
not be required to understand the choice calculus in order to interpret their
results, a variational solver's output should not contain choices. This design
principle in combination with \autoref{encoding-strat-deliverable} completes the
approach and would allow an end-users to receive the benefits of this work
without additional upfront cost.

\subsection{Architecture}

This section provides an informal description of variational satisfiability
solving and variational models, I provide the formalization in the next section.
A variational satisfiability solver is a compiler from the domain of variational
formulas to SMT-LIB2 programs. Throughout this section, I use SMTLIB2 snippets
to describe variational solving concepts in terms of an incremental solver.
While I target SMTLIB2, conforming to the standard is not an essential
requirement. Any solver that exposes an incremental API as defined by
minisat~\cite{10.1007/978-3-319-09284-3_16} can be used to implement variational
satisfiability solving following the same architecture and semantics.

\begin{figure}
  \begin{subfigure}[t]{0.5\linewidth}
    \tikzstyle{block}    = [draw,fill=white!20,node distance = 1.5cm,align=center]
\tikzstyle{inEdge}   = [fill=white, text width=1cm]
\tikzstyle{overEdge} = [midway,above]
\tikzstyle{input}    = [fill=white!20,node distance = 2.2cm,align=center, text width=1cm]
\tikzstyle{double} = [draw, anchor=text, rectangle split,rectangle split parts=2]
% diameter of semicircle used to indicate that two lines are not connected
\tikzstyle{branch}=[fill,shape=circle,minimum size=3pt,inner sep=0pt]
\tikzstyle{pinstyle} = [pin edge={to-,thin,black}]
\begin{center}
\begin{tikzpicture}[>=latex]

  % The loop
  \node[block]  (solve) {Reification\\Engine};
  \node[block, below right=of solve] (vmodel) {VModel\\Constructor};
  \node[block, below=1.05cm of solve] (vcore) {Reduction\\Engine};
  \node[block, right=of solve] (base) {Base\\Solver};

  % input nodes
  \node (input) [input, left=0.7cm of vcore] {$\kf{Query}$ $\kf{formula}$};
  \node (vf)    [input, left=0.7cm of solve] {$\kf{variant}$ $\kf{formula}$};


  % outputs
  \node[input, right=0.75cm of vmodel] (output) {$\kf{VModel}$};

  \draw
  %% Inputs
  (input) edge[->] node[overEdge] {} (vcore)
  (vf)    edge[dotted,->] node[overEdge] {} (solve)

  %% outputs
  (vmodel) edge[->] node[overEdge] {} (output)
  (base)   edge[->,sloped] node[overEdge,right,rotate=-90,xshift=0.8em]
  {$\kf{plain~models}$} (vmodel)
  (base)   edge[->,sloped,bend right=10] node[overEdge,right,rotate=-90] {} (vmodel)
  (base)   edge[->,sloped,bend right=20] node[overEdge,right,rotate=-90] {} (vmodel)
  (base)   edge[->,sloped,bend left=10] node[overEdge,right,rotate=-90] {} (vmodel)
  (base)   edge[->,sloped,bend left=20] node[overEdge,right,rotate=-90] {} (vmodel)

  %% Loop
  (solve) edge[->,sloped, bend left=25] node[overEdge,right,text width=1.5cm,rotate=90] {$s$, $v_{1} \vee v_{2}$, $v_{1} \wedge v_{2}$} (vcore)
  (vcore) edge[->,sloped, bend left=25] node[overEdge,left,rotate=-90] {$\kf{VCore}$, $\unit{}$} (solve)
  (solve) edge[->,in=100,out=160,loop, min distance=10mm] node[overEdge,xshift=-2em,yshift=0.2em] {$v \wedge \sem[C]{\chc[D]{e_{1}, e_{2}}}$} (solve)
  (solve) edge[dashed,->,in=70,out=15,loop, min distance=10mm] node[overEdge,xshift=4.75em,yshift=0.2em,text width = 4.5cm] {$v \wedge \sem[C\ \cup\ \{(D, \tru{})\}]{\chc[D]{e_{1}, e_{2}}}$, $v \wedge \sem[C\ \cup\ \{(D, \fls{})\}]{\chc[D]{e_{1}, e_{2}}}$} (solve)

  %% Loop escape
  (solve) edge[->] node[overEdge] {} (base)
  (solve) edge[->,sloped,bend right=5] node[overEdge] {} (base)
  (solve) edge[->,sloped,bend right=10] node[overEdge] {} (base)
  (solve) edge[->,sloped,bend left=5] node[overEdge] {} (base)
  (solve) edge[->,sloped,bend left=10] node[overEdge] {\unit{}} (base)
  % (vmodel) edge[->,sloped,bend right=10]  node[overEdge] {$r$, $s$, $t$} (base)
  % (vmodel) edge[->,sloped,bend left=10]  node[overEdge,xshift=0.1cm] {$\neg v$, $v_{1} \vee v_{2}$} (acc)
  % (vmodel) edge[->, loop above]  node[overEdge] {$\unit{} \wedge v$, $v \wedge \unit{}$} (vmodel)
  % (acc) edge[->, loop below] node[overEdge,below] {$r$, $\neg s$, $s_{1} \wedge s_{2}$, $s_{1} \vee s_{2}$} (acc)
  % (acc) edge[->,sloped, bend left=25] node[overEdge,below] {$s$} (vmodel)
  % (base) edge[->,sloped, bend right=25] node[overEdge,below] {$\unit{}$} (vmodel)
  % (vmodel) edge[->] node[overEdge] {} (vcore)
  ;

\end{tikzpicture}%
\end{center}

    \vspace{3.5ex}
    \caption{System overview of a variational solver.}%
    \label{impl:overview}
  \end{subfigure}
  \begin{subfigure}[t]{0.5\linewidth}
    \tikzstyle{block}    = [draw,fill=white!20,node distance = 1.75cm,align=center]
\tikzstyle{inEdge}   = [fill=white, text width=1cm]
\tikzstyle{overEdge} = [midway,above]
\tikzstyle{input}    = [fill=white!20,node distance = 2.2cm,align=center, text width=1cm]
\tikzstyle{double} = [draw, anchor=text, rectangle split,rectangle split parts=2]
% diameter of semicircle used to indicate that two lines are not connected
\tikzstyle{branch}=[fill,shape=circle,minimum size=3pt,inner sep=0pt]
\tikzstyle{pinstyle} = [pin edge={to-,thin,black}]
\begin{center}
\begin{tikzpicture}[>=latex]

  % input nodes
  \node[input] (input) {$\kf{Query}$ $\kf{formula}$};
  \node[block, right= 0.55cm of input] (ilInput) {to IL};

  % The loop
  \node[block, right=0.55cm of ilInput] (eval) {Evaluation};
  \node[block, below right=1.5cm of eval] (acc) {Accumulation};
  \node[block, below left=1.5cm of eval] (base) {Base\\Solver};

  % output
  \node[input, above left=0.75cm and 0.05 of eval] (vcore) {};

  % nowhere node
  \node[input, above right=0.75cm and 0.05cm of eval] (nowhere) {};

  % escape edges
  \draw
  %% inputs
  (input) edge[->] node[overEdge] {} (ilInput)
  (ilInput) edge[->] node[overEdge] {} (eval)

  %% Eval
  (eval) edge[->,sloped,bend right=10]  node[overEdge] {$r$, $s$, $t$} (base)
  (eval) edge[->,sloped,bend left=10]  node[overEdge,xshift=0.1cm] {$\neg v$, $v_{1} \vee v_{2}$} (acc)
  (eval) edge[->, loop above]  node[overEdge, text width=1cm] {$v_{1} \wedge v_{2}$} (eval)
  (eval) edge[<-,sloped,bend right=25] node[overEdge,right,rotate=-57,text width=1.5cm,xshift=0.5em] {$s$, $v_{1} \vee v_{2}$, $v_{1} \wedge v_{2}$} (nowhere)

  %% acc
  (acc) edge[->, loop below, transform canvas={xshift=6mm}] node[overEdge,below, text width = 1cm] {$\neg v$, $v_{1} \vee v_{2}$, $v_{1} \wedge v_{2}$} (acc)
  (acc) edge[->, loop below, transform canvas={xshift=-6mm}] node[overEdge,below, text width = 1cm] {$r$, $\neg s$, $s_{1} \vee s_{2}$, $s_{1} \wedge s_{2}$} (acc)
  (acc) edge[->,sloped, bend left=25] node[overEdge,below, text width = 1.3cm] {$s$, $v_{1} \wedge v_{2}$, $\ v_{1}\vee v_{2}$} (eval)
  (base) edge[->,sloped, bend right=25] node[overEdge,below] {$\unit{}$} (eval)

  %% output
  (eval) edge[->,sloped, bend left = 25] node[overEdge,left,rotate=54] {$\kf{VCore}$} (vcore) ;

\end{tikzpicture}%
\end{center}

    \caption{Overview of the reduction engine.}
    \label{impl:vcore}
  \end{subfigure}
  \caption{}
\end{figure}

A \ac{vpl} formula is solved using a recursive approach, decoupling the handling
of plain terms from the handling of variational terms. The idea is to define a
process to evaluate plain terms and skip choices, then define another process
that can only configures choices thus introducing new plain terms to the formula
that can be recursively processed. The base case is a variant, at which point a
model can be queried and the assertion stack can be popped to backtrack to solve
another variant.

I present an overview of a variational solver as a state diagram in
\autoref{impl:overview} that operates on the input's abstract syntax tree.
Labels on incoming edges denote inputs to a state and labels on outgoing edges
denote return values; we show only inputs for recursive edges; labels separated
by a comma share the edge. We omit labels that can be derived from the logical
properties of connectives, such as commutativity of $\vee$ and $\wedge$.
Similarly, we omit base case edge labels for choices and describe these cases in
the text. The solver has four subsystems: The \emph{reduction engine} processes
plain terms and generates a formula ready for reification called a
\emph{variational core}. The \emph{reification engine} configures choices in a
variational core. The \textit{base solver} is the incremental solver used to
produce plain models. Finally, the \emph{variational model constructor}
synthesizes a single variational model from the set of plain models returned by
the base solver.

The solver inputs a \vpl{} formula, called a \emph{query formula}, and an
optional input, called a \emph{variation context} (\vc{}). A \vc{} is a
propositional formula of dimensions that restricts the solver to a subset of
variants.
%
The variational solver translates the query formula to a formula in an
intermediate language (IL) that the reduction and reification engines operate
over; its syntax is given below.
%
\[
  v \hquad\Coloneqq\hquad \unit{}
  \hquad|\hquad t
  \hquad|\hquad s
  \hquad|\hquad \neg v
  \hquad|\hquad v \wedge v
  \hquad|\hquad v \vee v
  \hquad|\hquad \chc[D]{e,e}
\]
%
The IL includes two kinds of terminals not present in the input query formulas:
plain sub-terms that can be reduced symbolically will be replaced by a
\emph{symbolic reference} $s$, and sub-terms that have been sent to the base
solver will be represented by the unit value \unit{}.
%
Note that choices contain unprocessed expressions ($e$) as alternatives.


\paragraph{Derivation of a Variational Core}%
\label{ssec:impl:accum}

A variational core is an IL formula that captures the variational structure of
a query formula. Plain terms will either be placed on the assertion stack or
will be symbolically reduced, leaving only logical connectives, symbolic
references, and choices.
%
Consider the query formula $f = ((a \wedge b) \wedge \chc[A]{e_{1}, e_{2}}) \wedge
((p \wedge \neg q) \vee \chc[B]{e_{3}, e_{4}})$. Translated to an IL formula,
$f$ has four references ($a$, $b$, $p$, $q$) and two choices. The reduction
engine shown in \autoref{impl:vcore} will produce a variational core that will
assert $(a \wedge b)$ in the base solver, thus pushing it onto the assertion
stack and create a symbolic reference for $(p \wedge \neg q)$. This is done in
two states: \emph{evaluation}, which issues commands to the base solver to
process plain terms, and \emph{accumulation} which is called by evaluation to
create symbolic references.

Generating the core begins with evaluation. Evaluation will match on the root
node: $\wedge$, of $f$ and recur following the $v_1 \wedge v_2$ edge, where
%
$v_1=(a\wedge b)\wedge\chc[A]{e_1,e_2}$ and
$v_2=(p\wedge \neg q) \vee \chc[B]{e_3, e_4}$.
%
The recursion processes the left child first. Thus, evaluation will again
match on $\wedge$ of $v_{1}$ creating another recursive call with $v_{1}' =
(a\wedge b)$ and $v_{2}' = \chc[A]{e_1,e_2}$. Finally, the base case is reached
with a last recursive call where $v_{1}'' = a$, and $v_{2}'' = b$. At the base
case both $a$ and $b$ are references, thus evaluation will send $a$ to the
base solver, following the $\kf{r, s, t}$ edge, which returns $\unit{}$ for
the left child. The right child follows the same process yielding $\unit{}
\wedge \unit{}$; since the assertion stack implicitly conjuncts all assertions,
$\unit{} \wedge \unit{}$ will be further reduced to $\unit{}$ and returned as
the result of $v_{1}'$,
indicating that both children have been pushed to the base solver. This
leaves $v_{1}' = \unit{}$ and $v_{2}' = \chc[A]{e_1,e_2}$. $v_{2}'$ is a base
case for choices and cannot be reduced in evaluation, and so $\unit{} \wedge
\chc[A]{e_1,e_2}$, will be reduced to just $\chc[A]{e_1,e_2}$ as the result for
$v_{1}$.

In evaluation, conjunctions can be split because of the behavior of the
assertion stack and the and-elimination property of $\wedge$. Disjunctions and
negations cannot be split in this way because both cannot be performed if a child
node has been lost to the solver, e.g., $\neg \unit{}$. Thus, in accumulation, we
construct symbolic terms to represent entire sub-trees, ensuring information is
not lost, but still allowing for the sub-tree to be evaluated if it is sound to
do so.

The right child, $v_2=(p\wedge \neg q) \vee \chc[B]{e_3, e_4}$ requires
accumulation. Evaluation will match on the root $\vee$, and send $(p\wedge \neg
q) \vee \chc[B]{e_3, e_4}$ to accumulation via the $v_{1} \vee v_{2}$ edge.
Accumulation has two recursive edges, one to create symbolic references (with
labels $r, s, \hdots$), and one to recur to values. Accumulation matches the
root $\vee$ and recurs on the self-loop with edge $v_{1} \vee v_{2}$, $v_{1} =
(p\wedge \neg q)$, and $v_{2} = \chc[B]{e_3, e_4}$. Processing the left child
first, accumulation will recur again with $v_{1}' = p$ and $v_{2}' = \neg q$.
$v_{1}' = p$ is a base case for references, thus a unique symbolic reference
$s_{p}$ is generated for $p$, following the self-loop with label $r$ and
returned as the result for $v_{1}'$. $v_{2}'$ will follow the self-loop with
label $\neg v$ to recur through $\neg$ to $q$, where a symbolic term $s_{q}$
will be generated and returned. This yields $\neg s_{q}$, which follows the
$\neg s$ edge to be processed into a new symbolic term, yielding the result for
$v_{2}'$ as $s_{\neg q}$. With both results $v_{1} = s_{p}\wedge s_{\neg q}$,
accumulation will match on $\wedge$ \emph{and} both $s_{p}$ and $s_{\neg q}$ to
accumulate the entire sub-tree to a single symbolic term, $s_{s_{p} \wedge
  s_{\neg q}}$, which will be returned as the result for $v_{1}$. $v_{2}$ is a
base case, hence accumulation will return $s_{s_{p} \wedge s_{\neg q}} \vee
\chc[B]{e_3, e_4}$ to evaluation. Evaluation will conclude with
$\chc[A]{e_1,e_2}$ as the result for the left child of $\wedge$ and $s_{s_{p}
  \wedge s_{\neg q}} \vee \chc[B]{e_3, e_4}$ for the right child, yielding
$\chc[A]{e_1,e_2} \wedge (s_{s_{p} \wedge s_{\neg q}} \vee \chc[B]{e_3, e_4})$ as
the variational core of $\kf{f}$.

A variational core is derived to save redundant work. If solved naively, plain
sub-formulas of $f$, such as $a \wedge b$ and $p \wedge \neg q$, would be
processed once for each variant even though they are unchanged. To save
computation evaluation moves sub-formulas into the solver state to be reused
among different variants, and accumulation caches sub-formulas that cannot be
immediately evaluated to be evaluated later.

Symbolic references are variables in the reduction engine's memory that
represent a sub-tree of the query formula. From the perspective of the base
solver a symbolic reference represents a set of programmatic statements. For
example, $s_{pq}$ represents three declarations in the base solver:
%
\begin{lstlisting}[columns=flexible,keepspaces=true]
(declare-const p Bool)           ;; $s_{pq}$ represents
(declare-const q Bool)           ;; several declarations
(declare-fun $s_{ab}$ () Bool (or p (not c)))
\end{lstlisting}

The program language shown here is lisp~\cite{10.1145/367177.367199} as defined
in the SMTLIB2 standard. Function application begins with an open parenthesis,
where the first symbol in the parenthesis is the name of the function, every
symbol after the first is an argument to that function. In the above snippet, we
see three function calls, two which declare constants in the program for \pV{}
and \qV{}, both with type $\kf{Bool}$, and one which declares a new function
which takes no input and returns a $\kf{Bool}$.

Similar to symbolic references, a variational core is a sequence of statements
in the base solver with holes $\Diamond$. For example, the representation of
$\kf{VCore_{f}}$:
%
\begin{lstlisting}[columns=flexible,keepspaces=true]
(assert (and a b))                ;; $a \wedge b$ on the assertion stack
(declare-const $\Diamond_{A}$)                ;; choice A
 $\vdots$                                ;; many declares may occur
(assert $\Diamond_{A}$)                       ;; many assertions may occur
 $\vdots$                                ;; $s_{pq}$
(declare-fun $s_{pq}$ () Bool (and p q))
(declare-const $\Diamond_{B}$)                ;; choice B
 $\vdots$
(assert (or $s_{ab}$ $\Diamond_{B}$))               ;; assert waiting on $\sem[C]{\chc[B]{e_{3},e_{4}}}$
\end{lstlisting}
%
Each hole is filled by configuring a choice and may require multiple statements
to process the alternative as the alternative could introduce several new
variables or functions.

\paragraph{Solving the Variational Core}

The reduction engine performs the work at each recursive step. Whereas the
reification engine defines transitions between the recursive steps by
manipulating the configuration. In \vpl{}, a configuration was formalized as a
function, for variational solvers we use a set of tuples $\{\kf{(D \times
  \mathbb{B})}\}$. \autoref{impl:overview} shows two self-loops for the
reification engine corresponding to the reification of choices. The edges from
the reification engine to the reduction engine are transitions taken after a
choice is removed, where new plain terms have been introduced and thus a new
core is derived. If the user supplied a variation context, then it is used to
construct an initial configuration. Finally, a model is called from the base
solver when the reduction engine returns \unit{}, indicating that a variant has
been found.

We display a subset of edges of the reification engine using the $\wedge$
connective. In general, these edges will be duplicated for each binary logical
connective, e.g., $\vee$. The left edge, is taken when a choice is observed in
the variational core: $v \wedge \sem[C]{\chc[D]{e_{1}, e_{2}}}$ and $D \in C$.
This edge reduces choices with dimension $D$ to an alternative, which are then
translated to IL\@. The right edge is dashed to indicate assertion stack
manipulation, and is taken when $D \notin C$. For this edge, the configuration
is mutated for both alternatives: $C \cup \{(D, \tru{})\}$, and $C \cup \{(D,
\fls{})\}$, and the recursive call is wrapped with a \texttt{push}, and
\texttt{pop} command. To the base solver, this branching is a linear sequence of
assertion stack manipulations that performs backtracking behavior, for example
the representation of $\kf{f}$ is:
%
\begin{lstlisting}[columns=flexible,keepspaces=true]
 $\vdots$          ;; declares and assertions from VCore
(push 1)    ;; a configuration on B has occurred
 $\vdots$          ;; new declarations for left alternative
(declare-fun $s$ () Bool (or $s_{pq}$ $[\Diamond_{B} \rightarrow s_{B_{T}}]\Diamond_{B}$))  ;; fill
(assert $s$)
 $\vdots$          ;; recursive processing
(pop 1)     ;; return for the right alternative
(push 1)    ;; repeat for right alternative
\end{lstlisting}
%
Where the hole $\Diamond_{B}$\footnote{the notation $[x \rightarrow v]p$ should
  be read ``replace all free occurrences of $\kf{x}$ in $\kf{p}$ with
  $\kf{v}$'', it derives from work on explicit substitution from the programming
  languages community~\cite{10.5555/509043}}, will be filled with a newly
defined variable $s_{B_{T}}$ that represents the left alternative's formula.

\paragraph{Variational Models}%
\label{ssec:vmodels}
%
Classic \ac{sat} models map variables to Boolean values; variational models map
variables to variational contexts that record the variants where the variable
was assigned \tru{}. The variational context for a variable $r$ is denoted as
\vc{r}, and a variational model reserves a special variable called \SatVar{} to
track the configurations that were found satisfiable.
%
\begin{figure}[h]
  \centering
  \begin{subfigure}[t]{\textwidth}
  \begin{tabbing}
    \qquad \quad \= $\aV{} \rightarrow$ \tru{} \\
    \> $\bV{} \rightarrow$ \fls{} \\
    \> $\cV{} \rightarrow$ \tru{} \\
    \> $\pV{} \rightarrow$ \tru{} \\
    \> $\qV{} \rightarrow$ \fls{} \\
    \quad $C_{FF}$ = \{(\AV{}, \fls{}), (\BV{}, \fls{})\} \\
  \end{tabbing}
\end{subfigure}%
\begin{subfigure}[t]{\textwidth}
  \begin{tabbing}
    \qquad \quad \= $\aV{} \rightarrow$ \tru{} \\
    \> $\bV{} \rightarrow$ \fls{} \\
    \> $\cV{} \rightarrow$ \tru{} \\
    \> $\pV{} \rightarrow$ \tru{} \\
    \> $\qV{} \rightarrow$ \fls{} \\
    \quad $C_{FT}$ = \{(\AV{}, \fls{}), (\BV{}, \tru{})\} \\
  \end{tabbing}
\end{subfigure}%
\begin{subfigure}[t]{\textwidth}
  \begin{tabbing}
    \qquad \quad \= $\aV{} \rightarrow$ \tru{} \\
    \> $\bV{} \rightarrow$ \fls{} \\
    \\
    \> $\pV{} \rightarrow$ \fls{} \\
    \> $\qV{} \rightarrow$ \tru{} \\
    \quad $C_{TT}$ = \{(\AV{}, \tru{}), (\BV{}, \tru{})\} \\
  \end{tabbing}
\end{subfigure}%
  \caption{Possible plain models for variants of $\kf{f}$.}%
  \label{fig:models:plain}
\end{figure}
\begin{figure}[h]
  \centering
  \begin{subfigure}[t]{\textwidth}
  \begin{tabbing}
  \qquad \qquad \= $\_Sat \rightarrow\ (\neg \AV{} \wedge \neg \BV{}) \vee (\neg \AV{} \wedge \BV{}) \vee (\AV{} \wedge \BV)$ \\
  \> $\aV{} \rightarrow\ (\neg \AV{} \wedge \neg \BV{}) \vee (\neg \AV{} \wedge \BV{}) \vee (\AV{} \wedge \BV)$ \\
  \> $\bV{} \rightarrow\ \fls{}$ \\
  \> $\cV{} \rightarrow\ (\neg{} \AV{} \wedge{} \neg{} \BV{}) \vee{} (\neg{} \AV{} \wedge{} \BV{})$ \\
  \> $\pV{} \rightarrow\ (\neg \AV{} \wedge \neg \BV{}) \vee (\neg \AV{} \wedge \BV{})$ \\
  \> $\qV{} \rightarrow\ (\AV{} \wedge \BV)$
\end{tabbing}
\end{subfigure}

  \caption{Variational model corresponding to the plain models in
    \autoref{fig:models:plain}.}%
  \label{fig:models:var}
\end{figure}
%
As an example, consider an altered version of the query formula from the
previous section $f = ((\aV{} \wedge \neg \bV{}) \wedge \chc[A]{\aV{}
  \rightarrow \neg \pV{}, \cV{}}) \wedge ((\pV{} \wedge \neg \qV{}) \vee
\chc[B]{\qV{}, \pV{}})$. We can easily see that one variant, with configuration
$\{(\AV{},\tru{}), (\BV{},\fls{})\}$ is unsatisfiable. If the remaining variants
are satisfiable, then three models would result, as illustrated in
\autoref{fig:models:plain}; the corresponding variational model is shown in
\autoref{fig:models:var}.

We see that \Satfmf{} consists of three disjuncted terms, one for each
satisfiable variant. Variational models are flexible; a satisfiable assignment
of the query formula can be found by calling \sat{} on \Satfmf{}. Assuming the
model $C_{FT} = \{(\AV{}, \fls{}), (\BV{}, \tru{})\}$ is returned, one can find
a variable's value through substitution with the configuration; for example,
substituting the returned model on \vc{c} yields:
%
\begin{align*}
  \cV{} \rightarrow\ & (\neg \AV{} \wedge \neg \BV{}) \vee (\neg \AV{} \wedge \BV{}) & \text{\vc{} for \cV{}} \\
  \cV{} \rightarrow\ & (\neg \fls{} \wedge \neg \tru{}) \vee (\neg \fls{} \wedge \tru{}) & \text{Substitute \fls{} for \AV{}, \tru{} for \BV{}} \\
  \cV{} \rightarrow\ & \tru{} & \text{Result}
\end{align*}%
%
Furthermore, to find variants where a variable \cV{} is satisfiable reduces to
$\kf{SAT(\vc{\cV{}})}$

Variational models are constructed incrementally by merging each new plain model
returned by the solver into the variational model. A merge requires the current
configuration, the plain model, and current \vc{} of a variable. Variables are
initialized to \fls{}. For each variable $i$ in the model, if $i$'s assignment
is \tru{} in the plain model, then the configuration is translated to a
variation context and disjuncted with \vc{i}. For example, to merge the
$C_{FT}$'s plain model to the variational model in \autoref{fig:models:var},
$C_{FT}$'s configuration is converted to $\neg \AV{} \wedge \BV{}$. This clause
is disjuncted for variables assigned \tru{} in the plain model: \vc{\aV{}},
\vc{\cV{}}, and \vc{\pV{}}, even if they are new (e.g., \cV{}). Variables
assigned \fls{} are skipped, thus \vc{\qV{}} remains \fls{}. For example, in the
next model $C_{TT}$, \cV{} is \fls{} thus \vc{\cV{}} remains unaltered, while
\vc{\qV{}} flips to \tru{} hence \vc{\qV{}} records $\AV{} \wedge \BV{}$.
Variables such as \bV{}, whose \vc{}'s stay \fls{} are called \textit{constant}.


Variational models are constructed in \ac{dnf}, and form a monoid under with
$\vee$ as the semigroup operation, and \fls{} as the unit value. I take note of
this for mathematically inclined readers because it has important ramifications
for the asynchronous version of variational satisfiability solvers.

\subsection{Formalization of Variational Satisfiability Solving}
This subsection presents a selection of inference rules that specify behavior
described in the previous subsection. Many inference rules are similar to others
due to commutativity of boolean operators and thus I only present an interesting
subset. These inference rules will be significantly altered in the proposed
thesis to fully generalize to \ac{smt} problems.

\begin{figure}
  \begin{mathpar}
  %%% Computation rules
  \inferrule*[Right=Ac-Gen]
  { r \notin \kf{dom}(\Delta) \\
    \texttt{spawn($\Delta, r$)} = (\Delta', s)}
  {(\Delta, r) \mapsto (\Delta', s)}

  \inferrule*[Right=Ac-Ref]
  { \Delta(r) = s}
  {(\Delta, r) \mapsto (\Delta, s)}
  \\
  %%% Negation rules
  \inferrule*[Right=Ac-Neg]
  { \texttt{negate($\Delta, s$)} = (\Delta', s')}
  { (\Delta, \neg s) \mapsto (\Delta', s')}
\\
%   \hspace{0.7cm}
%   \inferrule*[Right=Ac-C]
%   { }
%   {(\Delta, \chc[D]{e_{1}, e_{2}}) \mapsto (\Delta, \chc[D]{e_{1}, e_{2}})}
%
% \begin{mathpar}
%   \inferrule*[Right=Ac-Neg-C]
%   { }
%   { (\Delta, \neg \chc[D]{e_{1},e_{2}}) \mapsto (\Delta, \chc[D]{\neg e_{1}, \neg e_{2}}) }
% % \end{mathpar}
%   \hspace{0.9cm}
%
% % \begin{mathpar}
%   \inferrule*[Right=Ac-Unit]
%   { }
%   { (\Delta, \bullet) \mapsto (\Delta, \unit{})}
% \end{mathpar}
%
  \inferrule*[Right=Ac-SOr]
  { \texttt{or($\Delta, s_{1}, s_{2}$)} = (\Delta', s') }
  {(\Delta, s_{1} \vee s_{2}) \mapsto (\Delta', s')}

  \inferrule*[Right=Ac-SAnd]
  { \texttt{and($\Delta, s_{1}, s_{2}$)} = (\Delta', s') }
  {(\Delta, s_{1} \wedge s_{2}) \mapsto (\Delta', s')}
\\
%
% \begin{mathpar}
%   \hspace{-1.5cm}
%   \inferrule*[Right=Ac-DM-VOr]
%   { }
%   { (\Delta, \neg (e_{1} \vee e_{2})) \mapsto (\Delta, \neg e_{1} \wedge
%     \neg e_{2}) }
% \end{mathpar}
%
% \begin{mathpar}
%   \hspace{-1.5cm}
%   \inferrule*[Right=Ac-DM-VAnd]
%   { }
%   { (\Delta, \neg (e_{1} \wedge e_{2})) \mapsto (\Delta, \neg e_{1}
%     \vee \neg e_{2}) }
% \end{mathpar}
%
%%% And with Choice rules
% \begin{mathpar}
%   \hspace{-1.5cm}
%   \inferrule*[Right=Ac-VAnd-ChcL]
%   {\Delta, e \mapsto \Delta', s}
%   { \Delta, \chc[D]{e_{1},e_{2}} \wedge e \mapsto \Delta', \chc[D]{e_{1},e_{2}} \wedge s}
% \end{mathpar}
%
% \begin{mathpar}
%   \hspace{-1.5cm}
%   \inferrule*[Right=Ac-VAnd-ChcR]
%   {\Delta, e \mapsto \Delta', s}
%   { \Delta, e \wedge \chc[D]{e_{1},e_{2}} \mapsto \Delta', s \wedge \chc[D]{e_{1},e_{2}} }
% \end{mathpar}
%
  %%% Or with Choice rules
% \begin{mathpar}
%   \hspace{-1.5cm}
%   \inferrule*[Right=Ac-VOr-ChcL]
%   {\Delta, e \mapsto \Delta', s}
%   { \Delta, \chc[D]{e_{1},e_{2}} \vee e \mapsto \Delta' \,
%     \chc[D]{e_{1},e_{2}} \vee s}
% \end{mathpar}
%
% \begin{mathpar}
%   \hspace{-1.5cm}
%   \inferrule*[Right=Ac-VOr-ChcR]
%   {\Delta, e \mapsto \Delta', s}
%   { \Delta, e \vee \chc[D]{e_{1},e_{2}} \mapsto \Delta', s \vee \chc[D]{e_{1},e_{2}} }
% \end{mathpar}
%
%%% Congruence rules with symbolic execution
% \end{mathpar}
%
% \begin{mathpar}
%   \hspace{-1.5cm}
%   \inferrule*[Right=Ac-VOr-Value]
%   {(\Delta, e_{1}) \mapsto (\Delta_{1}, s_{1}) \\ (\Delta_{1}, e_{2}) \mapsto
%     (\Delta_{2}, s_{2}) \\  (\Delta_{2}, s_{1} \vee s_{2}) \leadsto
%     (\Delta', s)}
%   { (\Delta, e_{1} \vee e_{2}) \mapsto (\Delta', s)}
% \end{mathpar}
%
% \begin{mathpar}
%   %%% Congruence rules with symbolic execution
%   \inferrule*[Right=Ac-VAnd]
%   {(\Delta, v_{1}) \mapsto (\Delta_{1}, v_{1}) \\ (\Delta_{1}, v_{2}) \mapsto
%     (\Delta', v_{2})}
%   {(\Delta, v_{1} \wedge v_{2}) \mapsto (\Delta', v_{1} \wedge v_{2})}
%
%   \inferrule*[Right=Ac-VOr]
%   {(\Delta, v_{1}) \mapsto (\Delta_{1}, v_{1}) \\
%     (\Delta_{1}, v_{2}) \mapsto (\Delta', v_{2})}
%   {(\Delta, v_{1} \vee v_{2}) \mapsto (\Delta', v_{1} \vee v_{2})}
% \end{mathpar}%
%
\end{mathpar}
%
% \begin{mathpar}
% \inferrule*[Right=Ac-And]
% {(\Delta, v_{1}) \mapsto (\Delta_{1}, v_{1}') \\ (\Delta_{1}, v_{2}) \mapsto (\Delta_{2}, v_{2}')}
% % ---------------------------------------------------------------
% { (\Delta, v_{1} \wedge v_{2}) \mapsto (\Delta_{2}, v_{1}' \wedge v_{2}') }
% \end{mathpar}
% %
% \begin{mathpar}
% \inferrule*[Right=Ac-Or]
% {(\Delta, v_{1}) \mapsto (\Delta_{1}, v_{1}') \\ (\Delta_{1}, v_{2}) \mapsto (\Delta_{2}, v_{2}')}
% % ---------------------------------------------------------------
% { (\Delta, v_{1} \vee v_{2}) \mapsto (\Delta_{2}, v_{1}' \vee v_{2}') }

% \end{mathpar}

  \caption{Selected accumulation semantics on IL formulas.}%
  \label{impl:accum}
\end{figure}

Accumulation is defined in \autoref{impl:accum} as a relation of the form
$(\Delta,v)\mapsto(\Delta,v)$, where $\Delta$ is a symbolic store, and $v$ is
the syntactic category representing the set of all possible IL formulas; the
tuples on the left and right are interpreted as input and output, respectively.
Accumulation interacts with the symbolic execution engine via primitive
operations represented in \texttt{typewriter} font.
%
The \rn{Acc-Gen} rule generates new symbolic references using \texttt{spawn},
which are looked up by \rn{Acc-Ref}. The \rn{Acc-And} and \rn{Acc-Or} rules
reduce symbolic sub-formulas to a new symbolic reference.
%
This selection of rules do the work of reducing formulas to symbolic references.
The remaining rules simply push negation down expressions and propagate
accumulation over the $\wedge$ and $\vee$-connectives.

%
\begin{figure}
  \input{Figures/Vsat_Impl_evaluation_rules}
  \caption{Selected evaluation semantics over \vpl{} formulas.}%
  \label{impl:eval}
\end{figure}

Evaluation is defined in \autoref{impl:eval} as a relation of the form
$(\Theta,v)\rightarrowtail(\Theta,v)$, where $\Theta=(\Gamma,\Delta)$ and
$\Gamma$ represents the base solver state.
%
As before, we show only a significant subset of the rules here. The rules
\rn{Ev-Term} and \rn{Ev-Sym} push new clauses to the base solver using the
primitive \texttt{assert} operation. The \rn{Ev-UL} and \rn{Ev-UR} implement
left and right unit, reducing conjunctions where one side has been processed by
the base solver.
%
Of special note is the difference between the \rn{Ev-Or} and \rn{Ev-And} rules.
While \rn{Ev-And} is a straightforward congruence rule, \rn{Ev-Or} instead
processes its arguments using accumulation ($\mapsto$). Disjunctions are a
source of back-tracking in variational solving, and thus the solver cannot
evaluate the left-hand side without evaluating the right, both of which may
contain choices, hence evaluation must switch to accumulation, as we informally
described in the previous subsection.

\begin{figure}
  % \begin{tabbing}
%   % \begin{align*}
%   {\sc CoreChoices}$\ :\ C \rightarrow vm \rightarrow ev \rightarrow vm$ \\
%   {\sc CoreChoices}$\ C\ mdls\ v$\\
%   \qquad case $v$ of \\
%   \qquad \qquad \= Unit \qquad \qquad \= = return \mdls{} \\
%   \> \chc[d]{e,e'} \> = CoreChoicesHelper C \dimd{} \mdls{} (Evaluate e) (Evaluate e') \\
%   \> \chc[d]{e,e'} $\wedge\ ev\ $ \> = do \\
%   \> \> \quad \=  $vE \leftarrow$ Evaluate $e$ \\
%   \> \> \> $vE' \leftarrow$ Evaluate $e'$ \\
%   \> \> \> CoreChoicesHelper C \dimd{} \mdls{} ($vE \wedge\ ev$) ($vE' \wedge ev$) \\
%   \> $ev\ \wedge\ $\chc[d]{e,e'} \> = do \\
%   \> \> \quad \=  $vE \leftarrow$ Evaluate $e$ \\
%   \> \> \> $vE' \leftarrow$ Evaluate $e'$ \\
%   \> \> \> CoreChoicesHelper C \dimd{} \mdls{} ($ev \wedge\ vE$) ($ev \wedge eV'$) \\
%   \> \chc[d]{e,e'} $\vee\ ev\ $ \> = do \\
%   \> \> \quad \=  $vE \leftarrow$ Evaluate $e$ \\
%   \> \> \> $vE' \leftarrow$ Evaluate $e'$ \\
%   \> \> \> CoreChoicesHelper C \dimd{} \mdls{} ($vE \vee\ ev$) ($vE' \vee ev$) \\
%   \> $ev\ \vee\ $\chc[d]{e,e'} \> = do \\
%   \> \> \quad \=  $vE \leftarrow$ Evaluate $e$ \\
%   \> \> \> $vE' \leftarrow$ Evaluate $e'$ \\
%   \> \> \> CoreChoicesHelper C \dimd{} \mdls{} ($ev \vee\ vE$) ($ev \vee eV'$) \\
%   \> $ev$ \> = do \\
%   \> \> \quad \= $vE \leftarrow$ FindPChoice (Evaluate $ev$) \\
%   \> \> \> solveChoices C \dimd{} \mdls{} ($vE$)  \\
% \end{tabbing}
\begin{mathpar}
  %%% Computation rules
  \inferrule*[Right=Gen]
  { \texttt{getModel($\Gamma, \Delta$)} = m' }
  { (C, \Theta, m, \unit{}) \Downarrow_{i} (C, \Theta, \oplus(C, m', m), \unit{}) }
% \end{mathpar}
\hspace{0.5cm}
% \begin{mathpar}
  \inferrule*[Right=Sym]
  { (\Theta, s) \rightarrowtail (\Theta', \unit{})}
  { (C, \Theta, m, s) \Downarrow_{i} (C, \Theta', m, \unit{})
  }
\end{mathpar}
%
\begin{mathpar}
% \inferrule*[Right=Cr-And]
% { (\Phi, v_{1}) \Downarrow_{i} (\Phi_{1}, v_{1}') \\
%     (\Phi_1, v_{2}) \Downarrow_{1} (\Phi', v_{2}') \\
%   }
%   { (\Phi, v_{1} \wedge v_{2}) \Downarrow_{i} (\Phi', v_{1}' \wedge v_{2}')
%   }
% \\%
%   \hspace{0.5cm}
  %% solve a symbolic reference execute
  \inferrule*[Right=Cr-Or]
  { (\Delta, v_{1} \vee v_{2}) \mapsto (\Delta', v)}
  { (C, \Gamma, \Delta, m, v_{1} \vee v_{2}) \Downarrow_{i} (C, \Gamma, \Delta',
    m, v) }
\\%
\inferrule*[Right=Cr-And-T]
{
  (D, \true) \in C \\
  (C, \Theta, m, v \wedge \texttt{toIR$(e_{1})$}) \Downarrow_{i} (C, \Theta, m, v')
}
{
  (C, \Theta, m, v \wedge \chc[D]{e_{1}, e_{2}}) \Downarrow_{i} (C, \Theta, m, v')
}

  % \inferrule*[Right=Cr-And-F]
  % {
  %   C\ d = False \\
  %   ((\Gamma, \Delta), e_{2}) \rightarrowtail ((\Gamma_{2}, \Delta_{2}), v_{2}) \\
  %   ((C, \Gamma_{2}, \Delta_{2}), v \wedge v_{2}) \Downarrow_{i} m
  % }
  % {
  %   ((C, \Gamma, \Delta), v \wedge \chc[d]{e_{1}, e_{2}}) \Downarrow_{i} m
  % }
%
  % \inferrule*[Right=Cr-Or-T]
  % {
  %   C(d) = True \\
  %   ((\Gamma, \Delta), e_{1}) \mapsto ((\Gamma_{1}, \Delta_{1}), v_{1}) \\
  %   ((C, \Gamma_{1}, \Delta_{1}), v \vee v_{1}) \Downarrow_{i} m
  % }
  % {
  %   ((C, \Gamma, \Delta), v \vee \chc[d]{e_{1}, e_{2}}) \Downarrow_{i} m
  % }
  %
\inferrule*[Right=Cr-Or-T]
{
  (D, \true) \in C \\
  (C, \Theta, m, v \vee \texttt{toIR$(e_{1})$}) \Downarrow_{i} (C, \Theta, m, v')
}
{
  (C, \Theta, m, v \vee \chc[D]{e_{1}, e_{2}}) \Downarrow_{i} (C, \Theta, m, v')
}
  % \inferrule*[Right=Cr-Or-F]
  % {
  %   C(D) = false \\
  %   ((C, \Gamma, \Delta), v \vee \texttt{toIR$(e_{2})$}) \Downarrow_{i} m
  % }
  % {
  %   ((C, \Gamma, \Delta), v \vee \chc[D]{e_{1}, e_{2}}) \Downarrow_{i} m
  % }
\\
%   \inferrule*[Right=Cr-COr]
%   {
%     \hspace{1.3cm}
%     ((C \cup \{(D, \kernfix{true})\}, \Gamma, \Delta), v \vee \chc[D]{e_{1}, e_{2}}) \Downarrow_{i+1} m_{1} \\
%     D \notin C \\
%     ((C \cup \{(D, \kernfix{false})\}, \Gamma, \Delta), v \vee \chc[D]{e_{1}, e_{2}}) \Downarrow_{i+1} m_{2} \\
%   }
%   {
%     ((C, \Gamma, \Delta), v \vee \chc[D]{e_{1}, e_{2}}) \Downarrow_{i} m_{1}
%     \oplus m_{2}
%   }
% \\
  \inferrule*[Right=Cr-CAnd]
  {
    % \hspace{1.5cm}
    D \notin C \\
    % \hspace{-0.5cm}
    (C \cup \{(D, \true)\}, \Theta, m,  v \wedge \chc[D]{e_{1}, e_{2}})
    \Downarrow_{i+1} (C', \Theta', m',  \unit{}) \\
    \hspace{2.25cm}
    (C \cup \{(D, \false)\}, \Theta, m,  v \wedge \chc[D]{e_{1}, e_{2}}) \Downarrow_{i+1} (C'', \Theta'', m'',  \unit{})\\
  }
  {
    (C, \Theta, m, v \wedge \chc[D]{e_{1}, e_{2}}) \Downarrow_{i} (C'',
    \Theta'', m' \cup m'',  \unit{})
  }
%
\end{mathpar}

  \caption{Selected variational solving semantics on cores.}%
  \label{impl:choice-eval}
\end{figure}

Solving the core is defined in \autoref{impl:choice-eval}, as a relation
%
$(C, \Gamma, \Delta, m,v) \Downarrow_{i} (C, \Gamma, \Delta, m, v)$, where $C$,
is a set which represents the configuration of the \vpl{} formula, and $m$
represents the variational model, which is initialized as empty.
%
The count of \texttt{push}'s on the assertion stack are represented with the
counter $i$. The solving process reifies choices by manipulating the
configuration and uses accumulation and evaluation to process terms. The \vc{}
input to the solver pre-populates the configuration, thereby restricting the
solver to a subset of variants. When no \vc{} is input, the configuration is
initialized as empty. \rn{Cr-CAnd} processes novel choices by manipulating the
configuration and performing a \texttt{push} in the base solver; resulting
variational models are merged via an element-wise $\vee$, shown as $\cup$.
Choices are removed through \rn{Cr-And-T} and \rn{Cr-Or-T} by selection on the
alternative. Once the choice is removed, the nested clauses are translated to
the intermediate language through the \texttt{toIL} primitive and processed by
accumulation and evaluation. A model is called from the base solver with
\rn{Gen}, once the core, and thus query formula is reduced to \unit{}. Note
again, \rn{Cr-Or} switches to accumulation to ensure sound results, we have
omitted congruence rules such as \rn{Cr-And}. Similarly, we omit rules which are
commutative versions of those shown here, namely: rules which process the left
branch of connectives, rules which select the $\kf{false}$ alternatives,
and the \rn{Or} version of \rn{Cr-CAnd}.

\subsection{Research Plan}

\paragraph{Estimating the difficulty of finding satisfiability} This item will
produce a method to determine the \emph{hardness} of solving a \ac{vpl} formula
for satisfiability. A well known result in the random-\ac{sat} community is the
phenomenon of a \emph{phase transition}~\cite{Gent94thesat} in randomly
generated \ac{sat} problems. The phase transition of \ac{sat} problems is an
inflection point in the probability of finding a satisfying assignment as the
ratio of clauses to variables is varied. Conceptually, one may think of the
phase transition as a method to estimate the difficultly in solving a \ac{sat}
problem. If there are many variables relative to clauses, then the \ac{sat}
problem is likely easy to solve as it is under-constrained, in contrast, if
there are too few variables relative to clauses then it is over-constrained and
thus easy to compute as unsatisfiable.

Difficult problems are balanced with respect to the clause variable ratio and
thus are at the phase transition point of a high probability of finding
satisfiability to finding unsatisfiability. Estimating the difficulty of a
\ac{sat} problem is thus theoretically useful to isolate sets of problems to
study and progress the state of the art, but also practically useful, because an
end-user may estimate the difficulty of a problem and choose to \emph{not} solve
it.

This item will replicate the analysis from the random-\ac{sat} community to
determine if the phase transition exists for variational satisfiability
problems. It is clear that the phase transition exists for variants, but having
a method to assess the phase transition point \emph{in terms of} \ac{vpl}
formulas would provide the aforementioned benefits for variational
satisfiability solvers.

%%% Local Variables:
%%% mode: latex
%%% TeX-master: "../thesis"
%%% End:

\section{Proposal Contribution 3: Variational Satisfiable-Modulo Theory Solving}
~\label{sec:vsmt} The final contribution to the thesis is generalizing the
prototype variational solver to solve \ac{smt} problems. Like all objects that
require craft, the architecture and design benefit greatly from lessons learned
in the first prototype, I cover these advances below and conclude the section
with a discussion of remaining work.

\subsection{Motivation}
This section discusses the motivation for a variational \ac{smt} solver.
Specifically, for a variational \ac{smt} solver that \emph{does not} use the
theory interface of \ac{smt} solvers to reason about variation. Satisfiable
modulo theory solvers, by virtue of abstracting satisfiability solving over
background theories, are useful in multitudes of domains including:
verification~\cite{boogiepl-a-typed-procedural-language-for-checking-object-oriented-programs},
test generation and bug
finding\cite{Cadar:2008:KUA:1855741.1855756,Godefroid:2012:SWF:2090147.2094081},
planning or scheduling
applications\cite{10.1109/RTSS.2010.25,10.1145/2038642.2038689}, interactive
proof
assistants~\cite{10.1007/978-3-540-78800-3_24,10.1007/978-3-319-08867-9_49}, and
many more~\cite{10.5555/1391237}.

The motivation in creating a variational \ac{smt} solver is identical to the
motivation for a variational \ac{sat} solvers for a subset of theories \ie{},
the incremental interface is automated, the user need not hand-program the
solver, performance benefits are now possible by virtue of a static explicit
encoding. Furthermore, \ac{smt} solvers provide a good platform for research on
variation. Understanding and implementing variational effects and effect
handlers is an active and unsolved area of research on variation and variational
programming. Essentially, the problem is soundly tracking side-effects such as
file I/O or state mutation in the context of variation.

Variational \ac{smt} solvers side step this issue by building upon a ground
theory of \emph{uninterpreted functions}. Uninterpreted functions are functions
that have no apriori meaning, such as a function which defines a constant value,
as opposed to a function like + which apriori means to add two integers. Thus
functions in the \ac{smt} domain, including those in background theorys are
total and side-effect free, making \ac{smt} solvers an attractive target for
variational research.

Besides the motivation deriving from variational research and inherited from the
variational \ac{sat} solver, a variational \ac{smt} solver is desirable because
it simplifies the use of a \ac{smt} solvers in real-world applications.
%
For example, consider the following snippet of a C-like language that uses
contract-based verification to ensure correctness:
%
\begin{lstlisting}[columns=flexible,keepspaces=true]
@precondition: $x_{in} > y_{in}$
void swap(int x, int y) {
  x := x + y;
  y := x - y;
  x := x - y;
}
\end{lstlisting}
%
With a plain \ac{smt} solver one can prove that this code does indeed swap the
variables by constructing an \ac{smt} problem that includes the following
constraint $x_{out} = y_{in} \wedge y_{out} = x_{in}$, in addition to encoding
the precondition, any constraints derived from the function body, and
constraints derived from any post condition. However, in practice, code bases
are variational artifacts, either through explicit variation annotations such as
C preprocessor \texttt{\#ifdef}s or through implicit practices such as branching
and forking in version control systems. For example, in addition to the snippet
above, one might have another variant that requires verification, perhaps to
prevent a bug or as a safety check after a refactor:
%
\begin{lstlisting}[columns=flexible,keepspaces=true]
@precondition: $x_{in} > y_{in}$
@precondition: $x_{in} > -1$     // new
int swap(int x, int y) {
  x := x + y;
  y := x - y;
  x := x - y;
  return 1;                // new
}
\end{lstlisting}
%
In this variant, we see two minor additions; an additional precondition, $x_{in}
> -1$, and a return value, note that most of the code has not changed. With a
variational \ac{smt} solver one could use choices to express this difference and
verify \emph{both} variants instead of each variant at a time. Thus, a
variational \ac{smt} solver, by virtue of being variation-aware, would allow
verification tools to directly express variation using choices and verify the
entire code base or software system, rather than each of its' variants one at a
time.

Verifying the entire code base is possible using plain \ac{smt} solvers but the
situation is analogous to the difference between a programming language which
has the concept of looping, compared to a language that is not
\emph{loop-aware}. One might still express loops in the latter language, \eg{},
with \texttt{Goto} or \texttt{Gosub} primitives, but doing so is more
error-prone, and more difficult than using a construct such as \texttt{While}
that encapsulates and expresses the concept of looping.

One approach to construct a variational \ac{smt} solver is to add a background
\emph{theory of variation} to a plain \ac{smt} solver. This approach has many
desirable properties; the \vc{} that determines the variants of interest could
be expressed in the solver, the solver would enforce the synchronization of
choices, new theories could be supported, and the difference between a
variational \ac{sat} solver and variational \ac{smt} solver would only be the
inclusion of other theories in a problem.

Unfortunately, there are several issues which make this approach undesirable.
First, it is overly solver specific. Some solvers such as openSMT~\cite{openSMT}
are architected specifically so the end-user can include custom theories, other
solver such a yices\cite{10.1007/978-3-319-08867-9_49},
cvc4\cite{10.1007/978-3-642-22110-1_14} and
z3~\cite{10.1007/978-3-540-78800-3_24} have varying degrees of support. Yices
and z3 provide no support\footnote{z3 had plugin support for custom theories but
  this feature was removed because model construction became problematic. The z3
  developers now suggest treating z3 as a black box and building around it, just
  as this thesis proposes. See:
  https://stackoverflow.com/questions/46508907/smt-solver-with-custom-theories},
while cvc4 includes an API for custom theories but with incomplete documention.
Second, adding a theory of variation to the solver limits the asynchronous
implementation to the asynchronous attributes of the \ac{smt} solver. Lastly,
the interaction between plain and the hypothetical variational theory is not
clear. The interaction between combination of theories in \ac{smt} solvers is an
active area of research~\cite{10.5555/1550723}. By including a theory of
variation, which is a \emph{meta-theory}, \ie{}, a theory which operates on
other theories to construct variation-aware theories, these problems are
exacerbated. Thus, while it is possible to construct a theory of variation, we
leave this to future research. Instead, we choose to create the variational
prototype as a proof of concept variational \ac{smt} solver that uses a plain
\ac{smt} solver as a black-box.


\subsection{\ac{vpl} Extensions}
Extending the variational solver for \ac{smt} background theories requires
non-trivial extensions to variational propositional logic, and consequently the
intermediate language the solver operates upon. In \ac{smt} solving, Boolean
values correspond to constraints over individual variables which range over
different domains, such as arrays, arithmetic, bitvectors or strings. To support
\ac{smt} theories the variational \ac{smt} solver must be able to abstract these
theories and reason about them in a variational context. We show a simple
extension of \ac{vpl} to include integer arithmetic, and conclude the section by
extending the variational \ac{smt} solver with an array theory.

To begin, we require formalizations of \ac{smt} background theories, for our
purposes, we'll represent any \ac{smt} theory as a 2-tuple, consisting of a
formal grammar \G{}, and a semantic function with type $\sem{\cdot} : \G{}
\rightarrow \mathbb{B}$\footnote{This formulation follows conventions from the
  programming language community. We choose this formulation to build upon the
  notation and background in \autoref{sec:vpl}. The \ac{sat}/\ac{smt} community
  would represent this a set called a \emph{signature} that consists of words in
  the theory called \emph{atoms}, functions which operate on the words, such as
  $+$ and $<$ and functions which maps sentences to Booleans called
  \emph{predicates}}. For the remainder of the proposal we represent all
grammars in Backus-Naur form. For example, consider a simple background theory
of integer addition, subtraction and two inequalities:
%
\begin{syntax}
  i & \in{} & \mathbb{Z}
  & \textit{Integer literals} \\[1.5ex]

  a & \Coloneqq{} & i    & \textit{Terminal} \\
  & | & a + a     & \textit{Addition} \\
  & | & a -{} a     & \textit{Subtraction} \\
  & | & a < a     & \textit{Less Than} \\
  & | & a > a   & \textit{Greater Than} \\
  % & | & \chc[D]{a,a} & \textit{Choice} \\
\end{syntax}
%
Note that with this formulation, the semantic function is partial, the theory
should only allow syntax trees that have inequalities at the root such that a
Boolean is the only type of value that can result. With the semantic function
and grammar, the integer theory can be integrated into \ac{vpl}. For example,
one can imagine the following sound formula: $\aV{} \wedge \neg \sem{((1 + 5) <
  1729)} \vee \cV{}$

With the definition of an \ac{smt} background theory we define a
\emph{variation-aware} background theory as a 2-tuple that consists of a
variation-aware grammar, $\G{}^{\chcL\chcR}$, and a variation-aware semantic
function $\sem{\cdot} : C \rightarrow \G{} \rightarrow \mathbb{B}$. The
alterations to the semantic function are minimal, requiring a configuration as
the extra input to track choices. The semantics follow from the semantic
function described in \autoref{fig:cc:cfg} only distributing over integer
connectives instead of logical connectives such as $\wedge$ and $\vee$.
Converting a grammar to a variation-aware grammar depends on the grammar at
hand. In this case, the theory is a context free grammar and thus the only
difference is adding choices as a recursive case:
%
\begin{syntax}
  i & \in{} & \mathbb{Z}
  & \textit{Integer literals} \\[1.5ex]

  \lift{a} & \Coloneqq{} & i    & \textit{Terminal} \\
  & | & \chc[D]{\lift{a},\lift{a}} & \textit{\textbf{Choice}} \\
  & | & \lift{a} + \lift{a}     & \textit{Addition} \\
  & | & \lift{a} - \lift{a}     & \textit{Subtraction} \\
  & | & \lift{a} < \lift{a}     & \textit{Less Than} \\
  & | & \lift{a} > \lift{a}   & \textit{Greater Than} \\
\end{syntax}
%
Note that we could define the variation-aware theory only with a variation-aware
domain $\lift{\iV{}}$ which would add choices to the set of integers. Doing so
would allow the variation-aware grammar to express expressions such as
$\chc[A]{1, 2} + 4$, \emph{but not} $\chc[A]{10 + \chc[A]{2,3}, 3} + 2$ because
choices would only be allowed to range over integers and not expression. We'll
use this behavior in the following array example.

More complicated theories, such as
arrays~\cite{demoura2009generalized,Mccarthy62towardsa} require more careful
handling. The array theory parameterizes an array with a type to determine the
type of the array's elements, and includes only two functions: $\kf{select :
  Array\ \nats\ X \rightarrow \nats \rightarrow X}$, which given an array and a
natural number index, creates a constraint that an element $x \in X$, is at
index $n \in \nats{}$, in the input array. \newline Similarly, $\kf{store :
  Array\ \nats\ X \rightarrow \nats \rightarrow X \rightarrow Array\ \nats\ X}$
constructs a constraint that at index $n \in \nats{}$, the input array contains
value $\kf{x \in X}$. In SMTLIB2, these constraints obey the following law
$\forall a\ \in \kf{Array\ \nats\ X},\ \forall i\ \in \nats,\ e \in X,\ \kf{(=\
  (select\ (store\ a\ i~e)\ i)\ e)}$. A simple formulation then for an array
theory is:
%
\begin{syntax}
  a & \in & Array\ \nats{}\ X & \textit{all possible arrays} \\
  i & \in & \nats & \textit{Natural Numbers} \\
  x & \in & X & \textit{set of elements} \\[1.5ex]

  arr & \Coloneqq & select\ a\ i    & \textit{Selection} \\
  % & | & \chc[D]{\lift{a},\lift{a}} & \textit{Choice} \\
  & | & store\ a\ i\ x & \textit{Storage} \\
  % & | & \lift{a} + \lift{a}     & \textit{Addition} \\
\end{syntax}
%
The semantic function for this grammar is stateful, such that it can track the
array and its constraints. The prototype \ac{smt} solver offloads this work to
the underlying incremental solver and instead places holes in the array
constraints.

We present a variation-aware grammar, \lift{arr}, which maintains the same
interface, \ie{}, \emph{store} and \emph{select}, but operates on variation-aware
domains such as \lift{\nats}, \lift{X}:
%
\begin{syntax}
  a & \in{} & \lift{(Array\ \nats{}\ X)} & \textit{choice of all possible arrays} \\
  i & \in{} & \lift{\nats} & \textit{choice of Natural Numbers} \\
  x & \in{} & \lift{X} & \textit{choice of elements} \\[1.5ex]

  \lift{arr} & \Coloneqq{} & select\ a\ i    & \textit{Selection} \\
  % & | & \chc[D]{\lift{a},\lift{a}} & \textit{Choice} \\
    & | & store\ a\ i\ x & \textit{Storage} \\
    % & | & \lift{a} + \lift{a}     & \textit{Addition} \\
\end{syntax}
%
This is a design decision, one could easily add an array theory which allows for
a choice of \emph{select} or \emph{store}, similar to \lift{a}, in addition to
including variation-aware domains. Furthermore, we could restrict the language
by choosing to only use a variation-aware element domain \lift{X}, which would
yield an array of choices, or only a variation-aware domain of arrays,
\lift{(Array\ \nats{}\ X)}, yielding a choice of arrays. Both are possible with
this formulation, for example, consider the following variational \ac{smt}
program:
%
\begin{lstlisting}[columns=flexible,keepspaces=true]
(declare-const e1 Int)
(declare-const e2 Int)
(declare-const a1 (Array Int Int))                ;; an array of integers
(declare-const a2 (Array Int Int))                ;; second array of integers
(assert (= (store a2      3 A$\chcL$1202,2718$\chcR$) a2))     ;; an array of choices
(assert (= (store A$\chcL$a1,a2$\chcR$ 3 1729       ) a1))     ;; a choice of arrays
(assert (= (select A$\chcL$a1,a2$\chcR$ 3) e1))
(assert (= (select a2      3) e2))
(check-sat)
(get-model)
\end{lstlisting}
%
We see that there are two integer arrays, $\kf{a1}$ and $\kf{a2}$, and three
choices: one choice which chooses an array in \emph{store}, another which
chooses an element to store in \emph{store} in $\kf{a2}$, and a third choice to
determine which array $\kf{e1}$ retrieves its value from. All choices are
parameterized by the dimension $\kf{A}$ yielding two variants, and results are
returned in the $\kf{e1}$ and $\kf{e2}$ variables. The variational core for this
program would simply replace the choices with holes:
%
\begin{lstlisting}[columns=flexible,keepspaces=true]
 $\vdots$
(assert (= (store $\Diamond_{A}$ 3 1729) a1))  ;; a choice of arrays
(assert (= (store a2 3 $\Diamond_{A}$)   a2))  ;; an array of choices
(assert (= (select $\Diamond_{A}$ 3) e1))
 $\vdots$
\end{lstlisting}
%
To solve such a program, the variational \ac{smt} solver will compile to SMTLIB2
wrapping variation-aware statements and statements affected by variation-aware
statements, such as \texttt{(get-model)}, with a \texttt{push} and \texttt{pop}
instruction:
\begin{lstlisting}[columns=flexible,keepspaces=true]
(declare-const e1 Int)
(declare-const e2 Int)
(declare-const a1 (Array Int Int))      ;; an array of integers
(declare-const a2 (Array Int Int))      ;; second array of integers
(push)                                  ;; a configuration on A has occurred
(assert (= (store a2 3 1202) a2))
(assert (= (store a1 3 1729) a1))
(assert (= (select a1 3) e1))           ;; e1 set to 1729
(assert (= (select a2 3) e2))           ;; e2 set to 1202
(check-sat)
(get-model)
(pop)
(push)                                  ;; Right alternative of A
(assert (= (store a2 3 2718) a2))
(assert (= (store a2 3 1729) a1))       ;; a1 unifies with a2
(assert (= (select a1 3) e1))           ;; e1 set to 1729
(assert (= (select a2 3) e2))           ;; e2 set to 2718
(check-sat)
(get-model)
(pop)
\end{lstlisting}

\subsection{Variational \ac{smt} Models}
To support \ac{smt} theories, variational models must be abstract enough to
handle values other than Booleans. Functionally, variational \ac{smt} models
must satisfy several constraints: the variational \ac{smt} model must be more
memory efficient than storing all models returned by the solver naively. The
varational \ac{smt} model must allow users to find satisfying values for a
variant. The model must allow users to find all variants at which a variable has
a particular value or range of values. Furthermore, several useful properties of
varational models, as presented in \autoref{ssec:vmodels}, should be maintained:
The model is non-variational; hence the user does not need to understand the
choice calculus in order to understand their results. The model produces results
that can be fed into a plain \ac{sat} solver. The model can be built
incrementally and without regard to the ordering of results because it forms a
commutative monoid under $\{\fls{}, \vee\}$. The model maps variables to a context free
grammar and can thus be parsed quickly\footnote{The use of ``quickly'' here
  means linear or quadratic time using context-free grammar parser such as an
  Earley parser~\cite{10.1145/362007.362035}}.

To maintain these properties and satisfy the functional requirements, our
strategy for variational \ac{smt} models is to create a mapping of variables to
\ac{smt} expressions. By virtue of this strategy, variables are disallowed from
changing types across the set of variants and hence disallowed from changing
types as the result of a choice in the variational model. For any variable in
the model, we assume the type returned by the base solver is correct, and store
the satisfying value in a linked list constructed \emph{if-statements}.
Specifically, we use the function $\kf{ite} : \mathbb{B} \rightarrow T
\rightarrow T$ as the \texttt{cons} operation to build the list. $\kf{ite}$ is
defined in the ground theory of Booleans as defined in the SMTLIB2 standard. All
variables are initialized as \texttt{undefined} until a value is found in a
variant. To ensure the correct value of a variable corresponds to the
appropriate variant, we translate the configuration that determines the variant
to a variation context, and place the appropriate value in the \emph{then}
branch, with the else branch linking to the previous expression.

Consider the following variational \ac{smt} problem extended with an integer
arithmetic theory: $f = (\chc[A]{\iV{}, 13} - \cV{} < \bV + 10) \rightarrow
\chc[B]{\aV{}, \cV{} > \iV{}}$. \fV{} contains two unique choices, $\kf{A}$,
$\kf{B}$, and thus represents four variants. In this case, the expression is
under-constrained and so each variant will be found satisfiable.
%
\begin{figure}[h]
  \centering
  \input{Figures/Vsmt_plain_models}
  \caption{Possible plain models for variants of $\kf{f}$.}%
  \label{fig:vsmt:models:plain}
\end{figure}
\begin{figure}[h]
  \centering
  \input{Figures/Vsmt_variational_model}
  \caption{Variational model corresponding to the plain models in
    \autoref{fig:vsmt:models:plain}.}%
  \label{fig:vsmt:models:var}
\end{figure}

\autoref{fig:vsmt:models:plain} show possible plain models for $\kf{f}$ with the
corresponding variational \ac{smt} model display in
\autoref{fig:vsmt:models:var}. We've added line breaks to emphasize the branches
the $\kf{then}$ and $\kf{else}$ branches of the $\kf{ite}$ SMTLIB2 primitive.

This formulation maintains the user requirements of the model. We maintain a
special variable $\_Sat$ to track the variants that were found satisfiable. In
this case all variants are satisfiable and thus we have four clauses over
dimensions in disjunctive normal form. If a user has a configuration then they
only need to perform substitution to determine the value of a variable under
that configuration. For example, if the user were interested in the value of
\iV{} in the $\{(\AV{}, \tru{}), (\BV{}, \tru{})\}$ variant they would
substitute the configuration into \vc{\iV{}} and recover 0 from the first
$\kf{ite}$ case. To find the variants at which a variable has a value a user can
employ a \ac{smt} solver, add \vc{\iV{}} as a constraint, and query for a model.

This also maintains the desirable properties of variational \ac{sat} models
while allowing any type known to the \ac{smt} solver. The variational \ac{smt}
model does not require knowledge of choice calculus or variation, it is still
monoidal, although not a commutative monoid, and can be built in any order as
long as there are no duplicate variants; a scenario that is impossible by the
property of synchronization on choices.

However, variational \ac{sat} models clearly compressed results by preventing
duplicate values with constant variables. In contrast, the variational \ac{smt}
model allows for duplicate values as long as those values are parameterized by
disjoint variants. For example, both \iV{} and \cV{} contain duplicate values,
but only one: \iV{} is easy to check in $O(1)$ time as the duplicates are
sequential in \vc{\iV{}} and can thus be checked during model construction. Such
a case would be easily avoided by tracking the set of all values a variable has
been assigned in all variants. However, we chose to keep variational models as
simple as possible and therefore only present the minimum required machinery.

\subsection{Requirements and Design Principles}
As before the items that are completed are marked with a \checkmark{}. The final
thesis will address each item:
\begin{enumerate}
\item The user must be able to incrementally add to a variational core. This
  item recovers some of the incrementality lost from the synthesis of a \ac{vpl}
  formula. I hypothesize two possible solutions: Given a variational core, a
  user can add new clauses to the core under the condition that no bound
  variable is removed in the new variational core. We require this constraint
  because if a user attempted to remove a variable in the variational core then
  the solver would need to unpack the symbolic references which could lead to
  unsound results. A second method for incrementality comes from work on
  CLU\cite{10.1145/69622.357182}, under this method one would derive a
  variational core and then serialize or marshal it to disk, effectively caching
  the core, and the solver state, for future use or transmission to another
  solver instance.
\item \checkmark{} Provide a \emph{general} method to solve a variational core.
  A variational \ac{smt} solver can be extended with many background theories.
  Depending which theories the user requires, many possible formulas with
  varying types (or \emph{sorts} in the \ac{sat} literature) and operators can
  be present in a query formula and thus in the variational core. Hence, the
  variational \ac{smt} solver must have a general method to reason about these
  types and operations. The improvement is to use a Huet zipper~\cite{huet_1997}
  to capture operators as a context, such as negation, as an Algebraic Data
  Type. With a zipper, one can traverse the variational core and delay the
  semantic of the operator thus processing the rest of the tree to a symbolic
  variable. Then, evaluating with any operator is reduced to only the
  denotational semantics of the operator, preserving compositionality and
  completeness.
\item \checkmark{} The variational \ac{smt} solver should be able to solve a
  query formula concurrently or in a single-threaded mode. This item change
  requires changing the semantics of choices. A benefit of the static and
  explicit approach of representing variation using the choice calculus is that
  we can alter the denotational semantics of a choice very easily. In the
  prototype \ac{sat} solver, the semantics of a choice was wrapping both
  alternatives with a \texttt{push} and \texttt{pop} call, for the variational
  \ac{smt} solver, the semantics of a choice are extended to also capture the
  solver state and transmit a continuation over an asynchronous
  channel~\cite{Marlow2012} to a worker thread. In order to ensure sound
  results, we exploit the monoidal design of variational models to ensure that
  the variational model is insensitive to the order results are produced from
  the base solver.
\end{enumerate}

\subsection{Research Plan}
While I have made significant progress in pursuit of the aforementioned design
goals. Several deliverables are still to be completed:

\paragraph{Optimizations based on nanopass compiler research} In the worst case,
a variational core will be evaluated $2^{|\kf{D}|}$ times where $|\kf{D}|$ is
the number of unique dimensions in the query formula. Therefore, any
optimizations that can simplify the variational core once are likely to have an
observable impact on performance. We employ a nanopass compiler architecture to
increase the flexibility of the approach, since varying background theories are
likely to produce different optimizations. This work has not begun but is
architected for in the prototype variational \ac{smt} solver. The proposed
thesis will answer the following research questions:
\begin{enumerate}
\item Which optimizations produce a corresponding effect on runtime performance
  for both real world case studies and random \ac{smt} problems?
\item For any given optimization, what is the magnitude
  of the effect?
\item Is the effect on performance sensitive to the order optimizations are
  applied?
\end{enumerate}

\paragraph{Evaluation of solver performance}
Previous work on the variational \ac{sat} solver~\cite{10.1145/3382025.3414965}
used two real-world case studies~\cite{MNS+:SPLC17} from the \ac{spl} community
for an empirical semantic. Thus, while these case studies are representative of
practical use cases more evaluation could be done. The proposed thesis will
perform a more robust evaluation and will consist of a mix of real-world and
randomly generated data.

Specifically I will reuse the aforementioned case studies, both of which include
\ac{smt} versions, and will use a forked version of a variational type checker,
TypeChef~\cite{KKHL:FOSD10}, to log \ac{sat} problems produced by lexing,
parsing and type checking the Linux kernel~\cite{linux} and
Busybox~\cite{busybox} open source projects. The forked version of TypeChef is
complete but does not scale for the Linux kernel at time of this writing, and
thus requires a refactor to use a more sophisticated logging infrastructure such
as a database.

With this data, a convincing evaluation of the variational \ac{smt} solver
architecture and design is possible. However, the majority of data will only
exercise the solver for Boolean propositional formulas. To test the architecture
in the \ac{smt} use case, the study will use the \ac{smt} versions of the
previous case studies and will generate mixed\cite{Gent94thesat} random \ac{smt}
problems according to established methods in the random-\ac{sat} community.

This data serves several other purposes. It will be used to evaluate performance
of the encoding strategies from \autoref{encoding-strat-deliverable} on the
prototype \ac{smt} solver. It will serve as the test data to assess the impact
of sharing on solver performance. The non-random portion will be made publicly
available as a dataset of real-world variational \ac{sat} and \ac{smt} problems,
and as a dataset of related \ac{sat} problems, thus increasing the impact of
this thesis.

\paragraph{Proof of variation preservation}
A proof of variation preservation is a key contribution of the thesis. This work
is partially complete with a proof of progress for accumulation and evaluation,
and a lemma that plain formulas always result in a \unit{} value in Agda. The
strategy is to show progress and preservation over the inference rules deriving
and solving a variational core. Implicit in this is to show variational
preservation for constructing a \ac{vpl} formula. There properties left to prove
are as follows:
\begin{enumerate}
% \item \emph{A minimal variational core has a normal form}. I conjecture that a
%   minimal (in the sense of least terms) variational cores consists only of
%   choices, joined by disjunctions and at most a symbolic term. By the operations
%   defined in \autoref{sec:vsat}, a disjunction is the only connective in the
%   intermediate language that cannot be removed. Similarly choices are the only
%   values which are not accumulated or evaluated. Thus, one can imagine a general
%   case where a minimal variational core---minimal with respect to the encoding of
%   choices---can be found by converting the core into disjunctive normal form and
%   accumulating any remaining symbolic terms into

%   composed of conjunctions, disjunctions,
%   symbolics and choices, but a more minimal core might be possible where all
%   conjunctions are removed, the core is reordered and any remaining symbolic
%   terms are further accumulated.
\item \emph{Encoding preservation: An encoding algorithm preserves the
    uniqueness of variants}. In other words whichever encoding algorithm is used
  to synthesize the set of problems to \ac{vpl}, that encoding algorithm does
  not lose variants.
\item \emph{Variational core progress: Given a \ac{vpl} formula a variational
    core is always derivable}
\item \emph{Variational core preservation: For a set of \pl{} problems, the
    corresponding variational core can recover that set of problems and if a
    problem in the initial set was satisfiable, then the corresponding variant
    is also satisfiable}. For any variational core, the set of \ac{sat} or
  \ac{smt} problems can be recovered by enumerating all variants with a total
  configuration and substituting symbolic terms with their concrete
  representations. The resulting formula should then be semantically equivalent
  to the original \ac{sat} or \ac{smt} problem. This item is a direct
  reproduction of the quick-check properties that verify the prototype \ac{sat}
  and \ac{smt} solver's as sound.
\item \emph{Variational model Preservation: A variational model produces a
    satisfying assignment for every satisfiable variant}. For any variant, a
  variational model always results in a satisfying assignments to the
  corresponding \ac{sat} or \ac{smt} problem. Similar to the previous item this
  is the second half of
\end{enumerate}


%%% Local Variables:
%%% mode: latex
%%% TeX-master: "../main"
%%% End:


\section{Related Work}
~\label{chapter:related-work}
%
We have succeeded in creating a variation-aware \ac{sat} and \ac{smt} solver by
utilizing plain \ac{sat} and \ac{smt} solvers. This chapter situates this work
in the larger research context. \autoref{section:related-work:similar-solvers}
discusses other satisfiability solvers that reuse information and provides a
small history of incremental \ac{sat} solving.
\autoref{section:related-work:applications} discusses possible applications for
the variational solver.
\autoref{section:related-work:reasoning-about-variability} discusses other
methods to reason about variability in software product-lines and situates this
work in that domain. Lastly, \autoref{section:related-work:variational-systems}
compares our approach to other variation-aware systems that have been invented
over the last decade.
%
\section{Comparison to other solvers and execution models}
\label{section:related-work:similar-solvers}
%
This work is most similar to the Green solver by \citet{VGD:FSE12}, which also
constructs a \ac{sat} solver that exploits shared terms and prevents redundant
computation. However, the projects differ in important ways. Visser et al.'s
solver is oriented for program analysis and does not use incremental \ac{sat}
solving. Rather, it employs heuristics to find canonical forms of sliced
programs, and caches solver results on these canonical forms in a key-value
store~\citep{redis}. In contrast, variational \ac{sat} solving is domain
agnostic, solves \ac{sat} problems expressed in \ac{vpl}, returns a variational
model, and uses incremental \ac{sat} solving.

It is also possible to view incremental \ac{sat} and \ac{smt} solvers and the
incremental \ac{sat} problem as variational systems and as a variational
problem. Both are concerned with efficiently solving instances of problems which
by definition share terms and are therefore related, and thus variational. We
provide a small history of incremental \ac{sat} here as it is related work by
being the target language of our compiler.

The incremental \ac{sat} problem was first defined by \citet{hooker_1993}, with
successive refinements of techniques by~\citet{branch-bound}, and with the
assertion stack idea developed in~\citet{kim2000solving}\todo{double check}. The
incremental \ac{sat} problem was devised as a solution to verification and
optimization problems in electronic design automation such as covering
problems~\cite{10.1145/217474.217603}, detecting delay
faults\cite{10.1145/343647.343801}, and model
checking\cite{Clarke:1986:AVF:5397.5399}. The first incremental solver to gain
traction was \texttt{SATIRE} created by \citet{10.1145/378239.379019}.

Just two years later, \citet{10.1007/978-3-540-24605-3_37} made a major advance
in incremental \ac{sat} with \texttt{MiniSat} by defining, documenting, and
popularizing the implementation techniques required for an incremental \ac{sat}
solver. \texttt{MiniSat} was the result of work on two other solver's called
\texttt{SATZOO} and \texttt{SATNIK}, and seemed to hit a sweet spot in the
design space. It simplified the existing notions of incrementality from the
state of the art incremental solvers \texttt{SATIRE} and
\texttt{PBS}~\cite{10.1145/774572.774638} and combined propagation strategies
from the \texttt{Chaff}~\cite{Moskewicz:2001:CEE:378239.379017} solver such as
conflict-driven backtracking\cite{Zhang:2001:ECD:603095.603153} and dynamic
variable ordering~\cite{Moskewicz:2001:CEE:378239.379017}. These combinations
lead to a solver that was performent, and whose implementation was small and
communicative. That same year, the first SMTLIB standard would be proposed by
\citet{SMT-LIBformat} although incremental \ac{sat} commands would not be
incorporated until the 2.0 version\cite{BarST-RR-10} in 2010.

The use of choices in the variational solvers is similar to the concept of
\textit{facets} by~\cite{austin2012multiple} and \textit{faceted execution}
by~\cite{Schmitz2018FacetedSM,Micinski2018AbstractingFE,10.1145/2465106.2465121},
in that both choices and facets syntactically demarcate terms in an object
language that must be specially handled, and yet must also operate with terms
outside of the choice or facet. Facets are very similar to choices, facets use a
label to determine branches (or alternatives in our language), facets are
synchronized by these labels, facets are treated as tree-data structures, and
facets are similarly treated as undetermined until they are reified.

\citet{10.1145/3243734.3243806} define the faceted secure execution framework
\texttt{Multef}, which tries to avoid repeated execution of non-faceted values
just as this work tries to gain performance through avoiding repeated execution
of plain values. \texttt{Multref} does this by forking executions threads when a
novel facet is encountered. This strategy avoids redundant execution before the
facet is found but still has redundant or repeated computations inside the fork.
In contrast, our methods of accumulation, evaluation, and utilization's of a
zipper succeeds in only evaluating plain terms a single time and reusing that
information across variants. The facets have been employed to policy-agnostic
programming models and information flow control~\cite{IFC}, thus our methods
might leak too much information to be useful in that domain.

However, there are other striking similarities, \citet{optimisingFacets}
improves the performance of \texttt{Multref} by defining rewrite rules which
manipulate facets similarly to the equivalence laws presented for choices in
\autoref{fig:cc:eqv}. For example, Algehed et al. removes redundant facets
through a rewrite rule called \texttt{Choice Irrelevance}, which is isomorphic
to the \rn{IDemp} rule in \autoref{fig:cc:eqv}. Another case is definition of
\emph{Squashes} which finds dead branches in nested facets. Squashes are
similarly isomorphic to our discussion of \emph{dominating choices} in
\autoref{section:vpl:semantics}.



%%% Local Variables:
%%% mode: latex
%%% TeX-master: "../../thesis"
%%% End:
%
\section{Applications}
\label{section:conclusion:applications}
%
Variational \ac{sat} and \ac{smt} solving provides an improved user interface
and possible performance gains for variational \ac{sat} and \ac{smt} problems.
However, the space of variational \ac{sat} and \ac{smt} problems is largely
unexplored, as viewing problems as inherently \emph{variational} is only just
beginning to gain awareness outside of the software product-line and variational
programming languages communities. In this section we describe areas for
possible applications.

\citet{TTS+:VariVolution19} define two fundamental dimensions of variation:
variation in \emph{time}, where software is revised over some unit of time with
the intent that the new version will replace the old version; and variation in
\emph{space} where variants are meant to co-exist simultaneously. Our approach
to variational \ac{sat} and \ac{smt} solving is able to express both kinds of
variation with the caveat that all points of variation are known \emph{before}
running the solver. Thus, applications that utilize a plain \ac{sat} solver, do
not need to discover variation during run-time and that must negotiate variation
in time or space are possible applications for variational \ac{sat} or \ac{smt}
solver.

Problems in this domain include scheduling problems~\cite{BBH+09} which need to
account for a counterfactual event; for example, scheduling a set of jobs on a
number of machines but also accounting for one or several machines being unable
to take jobs. Such a problem is directly expressible in \ac{vpl} where each
dimensions corresponds to a machine being online, or a machine being disabled.
Another classic \ac{sat} application is circuit layout and hardware verification
problems~\cite{BBH+09}. In this domain, \ac{sat} solvers are used as the
back-end engine to answer safety and live-ness questions; questions such as a
given system can never reach a certain state or a system will always reach some
given state after a certain state is reached~\cite{BBH+09}. This work could be
directly applied to such problems; for example one might have two or more
circuits which share significant regions and yet are distinct products with
distinct behavior. Performing hardware verification on each circuit would
produce two related \ac{sat} problems, where the shared portions are redundantly
calculated. Thus, one can imagine translating the set of \ac{sat} problems to a
\ac{vpl} formula and solving them with a variational solver. Another direct
application would be performing hardware verification in the presence of
patches, one might encode speculative analyses to ensure desirable properties in
the hardware if regions or elements in the circuit are completely removed,
significantly patched, or stop operating. The particulars in this domain are
open research questions, however given the findings in this thesis large
performance gains are possible through the use of a variational \ac{sat} or
\ac{smt} solvers.

%
Software variability is a natural application domain for this work. The
variability of SPLs or configurable software is often reduced to propositional
logic~\citep{B05,CW07,MWCC08} for analysis
purposes~\citep{BSRC10,TAK+:CSUR14,GBT+19}. Many analyses have been implemented
using \ac{sat} solving such as~\cite{TAK+:CSUR14}, including feature-model
analysis~\citep{BSRC10,GBT+19}, parsing~\citep{KGR+:OOPSLA11}, dead-code
analysis~\citep{TLSS:EuroSys11}, code simplification~\citep{RGA+:ICSE15}, type
checking~\citep{TBKC07}, consistency checking~\citep{CP06}, dataflow
analysis~\citep{LKA+:ESECFSE13}, model checking~\citep{CCS+13},
variability-aware execution~\citep{NKN:ICSE14}, testing~\citep{MMCA:IST14},
product sampling~\citep{MKR+:ICSE16,VAT+:SPLC18}, product
configuration~\citep{SIMA:ASE13}, optimization of non-functional
properties~\citep{SRK+:SQJ12}, and variant-preserving
refactoring~\citep{FMS+:SANER17}. While each of these analyses gives rise to
multiple \ac{sat} problems for even a single analysis run, the authors typically
do not discuss how they are solved. We argue that many could benefit from
variational solving.

More generally, any scenario that involves solving many related \ac{sat}
problems, and where all of these problems are known or can be generated in
advance, is a potential application for variational \ac{sat} solving.
%
Such situations arise in program analysis~\citep{VGD:FSE12}, and especially in
\emph{speculative} program analyses that involve generating and exploring huge
numbers of variations of a program, for example, as in
counterfactual~\citep{CE14popl} and migrational~\citep{CCW18icfp,CCEW18popl}
typing. Furthermore, we believe that variational solving could provide a basis
for similar speculative analyses on feature models.


%%% Local Variables:
%%% mode: latex
%%% TeX-master: "../../thesis"
%%% End:
%
\section{Reasoning about Variability in \ac{spl}}
\label{section:related-work:reasoning-about-variability}
%
Since \ac{sat} solving is so common in software variability applications, many
strategies have been developed to reduce effort in this domain.

Similar to variational formulas, \citet{NMS+:GPCE18} encode several versions of
a feature model in a single formula. We reuse their benchmark as part of our
evaluation as described in
\autoref{section:case-studies:experimental-methodology}; a direct comparison
with their approach is nuanced and discussed in
\autoref{section:case-studies:results-and-discussion}.
%
While their work focuses on feature-model analysis only, variational formulas
and variational solving can be applied to many application areas.

In the context of family-based type checking~\citep{TAK+:CSUR14}, others have
discussed merging multiple \ac{sat} problems into one.
%
Most work in this area use a \emph{local} approach where \ac{sat} problems are
solved as they are encountered during typing; in contrast, \emph{global}
approaches collect \ac{sat} checks into a single problem that is solved at the
end of the analysis.
%
While the global approach improves efficiency by increasing reuse of learned
clauses in the solver, it loses the ability to identify \emph{which} variants
contain type errors~\citep{ASLK:FOSD10,HZS:SCP11}.
%
% Through variational models,
Variational solving can achieve the reuse benefits of the global approach
without sacrificing the precision of the local approach.


Since the size of \ac{sat} problems in software variability applications is often
dominated by the feature model, researchers tried to reduce the size of
satisfiability problems by delaying consideration of the feature model until
after the analysis and only using it rule out false
positives~\citep{BMB+:PLDI13,CCS+13,LKA+:ESECFSE13}, a technique known as late
feature-model consideration~\citep{TAK+:CSUR14}.
%
\citet{BMB+:PLDI13} found that this technique increases the overall efficiency
of static analysis~\citep{BMB+:PLDI13}, while \citet{CCS+13} found that it
actually decreases efficiency of family-based model checking. Variational
solving is orthogonal to these approaches since the feature model can be
excluded from a variational formula and then used later to rule out false
positives.


Feature models can also be reduced in size to speed up analyses, for example,
by slicing~\citep{ACLF:ASE11,KST+:SPLC16} or decomposition~\citep{SKT+:ICSE16}.
It is largely unexplored how much such reductions can improve efficiency, but the
analysis will still involve multiple similar \ac{sat} problems, which can
benefit from variational solving.


A final approach is to avoid \ac{sat} problems by using modal implications
graphs~\citep{KTS+:ICSE18}, which support faster reasoning. The idea is to
encode as many software variability constraints as possible in such graphs,
then use a \ac{sat} solver only for the remaining constraints.
%
The construction of modal implication graphs already requires solving \ac{sat}
problems, but this cost is amortized if many \ac{sat} queries will be solved
during the analysis, as \citet{KTS+:ICSE18} found for configuration processes.

Lastly, our idea of representing variation in a non-traditional formula (a
\ac{vpl} formula in our case) is similar to the approach by
\cite{10.1145/3442391.3442405}, which uses quantified boolean formulas to encode
variation, and quantified boolean \ac{sat} solvers to detect anomalies in
context-aware feature models. Notably, this approach has the benefit of avoiding
incremental \ac{sat} solving altogether.



%%% Local Variables:
%%% mode: latex
%%% TeX-master: "../../thesis"
%%% End:
%
\section{Variational or Variation-Aware Systems}
~\label{section:related-work:variational-systems}
%
Variational \ac{sat} solving is the latest in a line of work that uses the
choice calculus to investigate variation as a computational phenomena. This body
of work ranges from data structures, to graphics, to full fledged systems such
as the system presented in this thesis. Due to the nature of variational
problems, many variational or variation-aware systems employ \ac{sat} and
\ac{smt} solvers. We collect and discuss these contributions here beginning with
variational data structures.

There is relatively little work on variational data structures.
\citet{EWC13fosd} describes a general strategy for constructing a variational
data structure. \citet{Walk14onward} expands on this strategy and attempts to
formalize ad-hoc implementations used in variational systems such as
TypeChef~\cite{KKHL:FOSD10} and SuperC~\cite{GG:PLDI12}. For this section we
focus on recent advancements implementing performent variational stacks and
lists. The goal for variational data structures is to construct a data structure
which describes a set of non-variational data structures efficiently. The
variational artifact is the implementation of the variational data structure,
and the variants in this domain are the plain versions of the data structure or
plain values that result from operating on the data structure. The challenge
then is to devise a variational data structure that describes and contains the
variation, and provides a set of operations to manipulate the data structure
that are as close to the performance of their plain counterparts as possible.

A fundamental tension in this domain is exemplified by work on variational
stacks by~\citet{MMWWK17vamos}. Meng et al.\ identify two kinds of possible
variational stacks depending on the location of variation on may have either: a
stack of choices, or a choice of stacks. However, their analysis on a general
implementation strategy was inconclusive, rather they found that depending on
the implementation strategy runtime performance could be affected by as much as
20\%. Furthermore, the variation in their experiment is coarse grained, \ie{},
the sharing ratio is high. Thus, Meng et al.\ utilized heuristics (optimizations
in their paper) which further improved performance for both implementations by
43\%.

The work on variational stacks yields an alternative implementation strategy for
variational \ac{sat} solvers. We have carefully designed our variational
\ac{sat} and \ac{smt} solvers with a goal to utilize a plain base solver. We
could have done otherwise and implemented a variational solver directly. With
variational stacks the variational solver could utilize a variational assertion
stack and we would avoid the need for a zipper in choice removal. Such an
implementation is worth considering although by developing an independent solver
we lose any benefits brought by the \ac{sat}/\ac{smt} communities and lose the
general recipe for constructing a variation-aware system \emph{using} its plain
counterpart.

Similar to variational stacks, \citet{SE17fosd} successfully implemented
variational lists. Smeltzer and Erwig devise six implementations of variational
lists with one implementation, the \emph{suffix list} coming from previous
work~\cite{EW11gttse}. Smeltzer and Erwig's study leads to some surprising
results. Out of their six implementations they found that for some
implementations, simple functions such as \texttt{head} (which returns the first
element of the list) are slower than the brute force counterpart, because the
implementation may be required to traverse the whole list to resolve the
variation. However, they do conclude that one implementation a \emph{segment
  list} yields reasonable performance given the data in their study. The segment
list is an interesting result as the idea behind the design is to encode
variation as a \emph{sequence of segments}, where a segment is either a choice
or a sequence of plain elements. This idea should sound familiar as accumulation
and symbolic values are essentially pointers to sequences of plain terms.
Smeltzer and Erwig also observe that the sharing ratio has a measurable impact
on performance (a finding we also observed) and thus minimizing or manipulating
choices to increase the ratio is important; a result that has also been observed
in software product-lines by~\citet{ARW+:ICSE13} and~\citet{KRE+:FOSD12}.

In addition to data structures there has been research on applications of the
choice calculus to graphics~\citep{ES18diagrams},type
systems~\citep{CCEW18popl,CCW18icfp,CEW:TOPLAS14,CEW12icfp}, and error
messages~\citep{CES17jvlc,CE14popl,CEW12icfp,CES14hcc}. For the remainder of
this section we focus on variational or variation-aware systems.

This work is not the first to construct a variational or variation-aware system.
Notably, \citet{KKHL:FOSD10} used the choice calculus and variational data
structures to type check every possible Linux kernel. Thus constructing a
variational parsing~\cite{KGR+:OOPSLA11}, a variational
lexer~\cite{Kastner11partialpreprocessing}, type
system\cite{LKA+:ESECFSE13,KOE:OOPSLA12} and control-flow and data-flow
analyses\cite{LKA+:ESECFSE13}. Similarly, \citet{GG:PLDI12} variationally parse
the Linux kernel by utilizing variational data structures and \emph{choice
  nodes} in the abstract syntax tree. TypeChef is notable for several reasons:
its implementation is a direct inspiration for our baseline algorithm \vTop{}
which uses an incremental \ac{sat} solver but only exhibits sharing
\emph{before} a choice is discovered. This kind of sharing, called \emph{prefix
  sharing} by \citet{SE17fosd} is the de-facto standard in software product-line
applications which employ incremental \ac{sat} solvers. Given the results of
this thesis, large performance gains are possible if our results are
Representative with the use of a variational \ac{sat} or \ac{smt} solver.
TypeChef is also notable for its two step approach, first it parses source code
to find \cpp{if-def} annotations, and stores these in files called
\emph{presence conditions}. Presence conditions are isomorphic to variation
contexts, both are \pl{} formula's over dimensions (or conditions of the
\cpp{if-def}) which determine a variant. Using the presence conditions, TypeChef
annotates choice nodes to determine which variant the leaves of the choice node
belong. Then TypeChef extends the symbol table of a C program to contain types
which are conditional based on the presence conditions. This allows a variable's
type to change from one variant to another. Each type checking operation is then
lifted to handle the variational cases and then type checking checks the
variation-aware types to ensure every each variant type checks. Similar to our
use of variation contexts, TypeChef allows a \emph{variability model} which
specifies variants that should be type checked by conjoining the model with the
presence conditions.

In his PhD dissertation, \citet{M14,JensDebugging} constructs a variational
interpreter called VarexJ and a variational bytecode transformer called VarexC,
to achieve a variational execution and debugging framework. The framework tries
to maximize sharing in two ways: It directly utilizes the choice calculus to
represent local points of variation and achieve a \emph{fine-grained approach},
this allows the framework to share program states and keep a unified heap.
Additionally, the framework achieves instruction-level sharing among
control-flows between variants. It achieves this by implementing a variational
scheduler, which seeks to order the execution of program statements to optimize
sharing. We achieve this same effect through the interaction between
accumulation, evaluation and choice removal with the wrapped primitive
operations. Interestingly, Meinicke identifies redundant \ac{sat} calls as a
major bottleneck in the variational execution framework. Specifically they
determine redundant \rn{check-sat} calls as the most expensive. To reduce the
redundant calls the variational execution framework caches calls to the solver
thus only employing the solver for new queries. Chu-Pan Wong used VarexJ to do
speculative mutation testing and automated program repair~\cite{ChuPanThesis}.

% Lastly, Meinicke's work was put to use by \citet{ChuPanThesis} to induce
% \emph{speculative variation} for automatic program repair and mutation testing.
% We only touch on this work briefly as it applies a variational system rather
% than creating one. However, the work utilizes the aforementioned variational
% execution framework and a \ac{sat} solver to reduce the search space of
% interesting mutants to test.

Lastly, choice calculus has been successfully applied to databases to construct
a complete approach to variational databases including a variational database
management system, a variational query language, and variational tables.
science, such as databases~\citep{ATW17dbpl,ATW18poly},


%%% Local Variables:
%%% mode: latex
%%% TeX-master: "../../thesis"
%%% End:


%%% Local Variables:
%%% mode: latex
%%% TeX-master: "../../thesis"
%%% End:

\section{Research plan}
~\label{sec:research-plan}
\autoref{sec:prop-contr} lists the deliverables for this thesis. Each item has
been discussed in more depth in previous sections. This section summarizes these
items and provides an itemized list of each deliverable, broken down into
discrete tasks, and their corresponding time estimations. The time estimates are
guidelines with the real schedule being dictated by our publication schedule.

\subsection{Encoding strategies of \ac{vpl} formulae}
The naive and naive-with-optimizations encoding algorithms are complete. The
remaining work is a literature review for other representations of boolean
formulae which may be useful, and to implement a Huffman based encoding
algorithm and assess its performance:

\begin{enumerate}
\item Construct a list of alternative representations of boolean formulae which
  may lend itself to encoding strategies
\item\label{huffman-todo} Implement the Huffman coding algorithm over boolean
  formulae, recursively merge formulas into a \ac{vpl} formula based on a
  similarity metric.
\end{enumerate}

\textbf{Estimated time to complete: 1-2 weeks}

\subsection{A method to determine the hardness of \ac{vpl} formulae}
The time consuming parts of this item: creating a random generator suitable for
the analysis and creating a benchmark platform to test are complete. The only
items remaining are to perform the analysis.

\begin{enumerate}
\item\label{phase-todo} Review literature from random-ac{sat} community on observing phase
  transition for both random $k$-\ac{sat} and mixed-\ac{sat} problems.
\item Replicate the transition detection analysis from \autoref{phase-todo}.
\end{enumerate}

\textbf{Estimated time to complete: 1 week}


\subsection{Variational \ac{smt} Solving}
A prototype asynchronous variational \ac{smt} solver is complete and operational
but several auxiliary tasks are still required. I expect this item to require
the majority of time besides writing the the proposed thesis.

\begin{enumerate}
\item\label{ui-todo} Implement an enhanced user interface which treats the
  solver like a server rather than a batch process.
\item\label{vpl-todo} Write the formal grammar for \ac{smt}-flavored \ac{vpl}.
  The current formalization exists in Haskell code and thus must be translated
  into a publishable format.
\item\label{vmodel-todo} Write the formalization for \ac{smt}-flavored variational models. An
  identical item to \autoref{vpl-todo} only with variational models.
\item \label{solve-todo} Write the denotational semantics for \ac{smt}-flavored
  variational solving algorithm. An identical item to \autoref{vpl-todo} only
  for the solving algorithm.
\item \label{proof-todo} Construct a proof for variation preserving semantics
  using the formalizations in items 2-4.
\item\label{nano-review-todo} Review literature on nanopass compilers to
  construct a set of possible optimizations.
\item\label{opts-todo} Implement optimizations from \autoref{nano-review-todo}
\item \label{conf-todo} Implement configurations of the variational solver to
  provide an interface suitable for benchmarking.
\end{enumerate}

Estimated time to complete 1-4: 1-2 weeks

Estimated time to complete \autoref{proof-todo}: 4 weeks (concurrent with other work)

Estimated time to complete items 6-7: 2 weeks

\textbf{Total estimated time to complete: 6-8 weeks}

\subsection{Assembling Case Study Data}
Most of the work left to assemble the case study data involves refactoring and
running the forked TypeChef system on the Linux kernel; the Busybox case study
dataset is finished. Parsers for the \ac{spl} case studies and the Busybox/Linux
results are already complete and have been tested.

\begin{enumerate}
\item\label{tc-todo} Implement a database caching system in TypeChef to record \ac{sat}
  problems at scale
\item\label{run-linux-todo} Run the Linux case study to generate \ac{sat}
  problems.
\item\label{eval-linux-todo} Evaluate the Linux and Busybox case studies using
  encoding strategies from \autoref{huffman-todo}.
\end{enumerate}

Estimated time to complete \autoref{tc-todo}: 2-3 weeks (with help from
collaborator Paul Maximillian Bittner)

Estimated time to complete \autoref{run-linux-todo}: 1 week

Estimated time to complete \autoref{eval-linux-todo}: 1-2 weeks

\textbf{Total estimated time: 4-6 weeks}


%%% Local Variables:
%%% mode: latex
%%% TeX-master: "../main"
%%% End:


\section{Conclusion}
\label{chapter:conclusion}
%
This thesis has presented variational satisfiability and satisfiability-modulo
theory solving. In \autoref{chapter:introduction} we defined the success of this
thesis as applying the concept of variation in the domain of satisfiability
solving to create a variational satisfiability solver. The solver must
explicitly express the concept of variation in a user-facing language and must
be performent with respect to the performance of plain satisfiability solvers.
By this definition we have succeeded in demonstrating these ideas work in
practice in the domain of satisfiability solving. We have not only shown that
through the application of the choice calculus variation can be directly
expressed by the end-user, but also that runtime performance may be improved
because local points of variation are made explicit.
%
To conclude the thesis we review the important contributions in
\autoref{section:conclusion:cont-summary}.
\autoref{section:conclusion:future-work} provides immediate directions for
future work.

\section{Summary of Contributions}
\label{section:conclusion:cont-summary}

The main contribution of this work is the formalization of variational
satisfiability solver. In \autoref{chapter:introduction} we defined the success
of this thesis as applying the concept of variation in the domain of
satisfiability solving to create a variational satisfiability solver. The solver
must explicitly expresses the concept of variation in a user-facing language and
must be reasonably performent with respect to the performance of plain
satisfiability solvers. By this definition we have succeeded.\todo{make sure
  this was defined in intro}

In \autoref{chapter:vpl} we formalized a many-valued logic to express
variational \ac{sat} problems, demonstrated an application of the choice
calculus with propositional logic as the object language. We defined the
denotational semantics of configuration over the logic, and fundamental concepts
such as variants and synchronization.

In \autoref{chapter:vsat} we formalized our approach to variational
satisfiability solving based on this logic. Our approach is to variationalize
non-variational solvers by constructing a compiler to a standardized input
format. We saw that this approach has many desirable properties: The stages of
accumulation, evaluation, and choice removal cleanly separate concerns. The
sharing of plain terms is guaranteed between variants because we use a zipper to
capture evaluation contexts. Since our design integrates plain base solvers, our
variational solver can take advantage of advances made by the \ac{sat} and
\ac{smt} communities. Lastly, we proved that our design is confluent, thereby
showing that the variational solver is variation-preserving and thus sound.

In \autoref{chapter:vsmt} we extended the architecture to handle non-Boolean
constraints. We saw that extensions over the term language follow a pattern: One
wraps the primitive base solver operations to handle symbolic values, then
defines a congruence rule to process the recur on the left child of the
relation, and finally defines a computation rule that calls the wrapped
primitive to combine two symbolic values thereby producing a fold over the
relation. We presented two extensions; one over integer constraints, and one
over array based constraints. Since, symbolic values are untyped, we carefully
constructed the extended logic to make type errors inexpressible, we could have
otherwise chose to employ a simple type system as the \acl{smtlib} standard
does. Lastly, we saw that this extension pattern works even for background
theories that seem difficult, because our architecture processes plain terms
before variational terms due to the ordering between accumulation and choice
removal.

In \autoref{chapter:case-studies}, we built two prototype variational solvers
called \vsat{} and \vsmt{}. We evaluated the solvers over two real-world
datasets. We observed that variational solving does produce speedups over
standard use of an incremental solver when solving many variants. The
variational solvers produce this performance speedup by reusing shared terms and
avoiding redundant computation. We also observed that the base solver does have
an impact on runtime performance. Therefore an advantage of our method is that
it is base solver agnostic, and thus implementations can choose whichever
\acl{smtlib} solver is performent for its problem domain. However, we found that
outside of its use case---when solving only a single variant---variational
solving did show a performance overhead that was statistically significant for
one dataset. Lastly, our finding that the sharing ratio is positively correlated
to runtime performance repeats similar findings in the variational literature.

%
\section{Future Work}
\label{section:conclusion:future-work}

There are numerous avenues of future work ranging from novel applications, to
refining the implementations, to extended the solvers with new features. In this
section we collect and discuss the most promising future work beginning with
tool extensions and ending with abstracting this work to domains other than
satisfiability solving.

\subsection{Utilization of Variational Cores}
Variational cores are an important and foundational concept for the variational
solver and consequently for the variationalization recipe. Recall that the
purpose of variational cores was threefold: First, to condense the query formula
such that the variational terms were the majority of terms in the core. Second,
to simplify the choice removal process by reducing the amount of traversal
required to process the choices. Third, to enforce sharing between variants as
the contexts captured by the core were are reused during choice removal.

This last point is key, because variational cores in combination with the
accumulation and evaluation stores, completely capture the context of a formula
they can be reused in novel ways. For example, one might serialize a variational
core and associated stores to disk, effectively caching the core for future use.
Such a feature would enable desirable user facing features: the solver could
restart without losing information and thus might be useful for debugging or
exploration, if the variational cores require a lot of processing time to
generate this time would be amortized, or if the application domain only builds
on previous versions of the same formulas, then the variational core could be
consistently reused for every new version.

For example, consider the case of a feature model which evolves every month for
several months, similarly to the \fin{} and \auto{} datasets. Since the feature
model, and consequently the \ac{vpl} formula evolves over time, the previous
variational core could be modified to reflect the changes for the new formula.
Adding new constraints is straightforward; one would simply nest the previous
variational core in a conjunction context (\inAndL{\kf{core}}{\kf{~new}}) with
the new core and reuse the previous stores when generating the new core to
ensure sharing. A more difficult problem is removing constraints or variables in
the previous core. Both removing constraints and removing variables is
problematic as the variable or constraint could have been accumulated into a
symbol value or several symbolic values. One could traverse a dependency graph
to find all references of the variable and symbolic value, and then seek to
replace those references with a unit value, such as \tru{} for $\wedge$ or
\fls{} for $\vee$. However this immediately leads to the problematic case where
the variable or symbolic to be removed is in a $\neg$ context. There is no unit
value where $\neg$ does not have meaning and thus we cannot remove arbitrary
variables from a variational core.

In addition to manipulating or storing variational cores, future variational
solvers might utilize them as a convenient messaging format. Throughout this
thesis, we have assumed and have only considered systems which process all
variants in a single base solver instance, however this need not be the case.
Instead, when a choice in focus during choice removal one might choose to solve
the true alternative variants in a different solver and all the false
alternatives in the same solver. For example, a user might know that all true
alternative variants have particularly good performance characteristics for
boolector, while all false variants have good characteristics for yices. Since
we compile to \acl{smtlib} script such a feature is possible with few changes to
our method of variational solving. To add such a feature, a future variational
solver would allow the user to select particular solvers over the input \vc{} or
the configuration for a query formula.

\subsection{Further \ac{smt} background Theories and Tool Extensions}
\ac{sat} and \ac{smt} solvers are attractive targets for research on variational
languages. As of this writing, designing a language with variational
side-effects is an open research problem. The essential problem is tracking
effects for particular variants across the interface between a variational-aware
system and a plain system. For example, imagine writing a file to disk in one
variant and deleting a different file in another variant. Since the file system
has no concept of variation or variant, the variational system is not able to
guarantee variants are isolated, and therefore variants may interact in
undesirable and difficult to predict ways. \ac{sat} and \ac{smt} solvers side
step this limitation as they are side-effect free systems. There is simply no
way to read a file from disc in an \acl{smtlib} script. Similarly, classes of
traditional run-time errors, such as dividing by zero, are not possible. If a
script divides by zero then the script will not simply not unify and an
\rn{unsat} will be returned.

Due to the attractive properties of \ac{sat} and \ac{smt} solvers for
variational research, a straightforward avenue of future work is to continue to
investigate efficient variational folds by further extending the variational
solvers. Modern \ac{sat} and \ac{smt} solvers allow quantified constraints
following first-order logic. In this thesis, we have only considered
unquantified constraints, and thus the interaction of between quantified
constraints and choices is an open research problem.

Similarly, we have demonstrated extensions for core background theories, but
there are many features of plain solvers that would be desirable additions to
variational solvers. Such features include generation of variational
unsatisfiable-cores. An unsatisfiable core is a subset of constraints that
prevent the \ac{sat} or \ac{smt} from unifying. Unsatisfiable cores are
desirable for many problems. For example, one might desire to find the clique in
a \ac{sat} encoded weighted graph which prevents a traversal under some cost
limit. Or one might desire to find the sub-set of features in a feature model
that prevent classes of products from being built.

Enabling variational unsatisfiable cores is possible with our approach of
accumulation, evaluation and choice removal. The key requirement would be to
ensure that the plain, \rn{get-unsat-core} command occurs inside the
\rn{push}/\rn{pop} block for a given variant. Thus far we have only seen the
\rn{get-model} command have this property. So a straightforward extension is to
create a syntactic category that contains useful plain commands in this context,
such as \rn{get-model} or \rn{get-unsat-core}, which would be issued to the base
solver once a variant has been reduced to \unit{}. Another approach is to create
full fledged variational \acl{smtlib} language, instead of expressions of
variational constraints as we have presented here. Constructing such a
variational \acl{smtlib} language is likely to save work for future extensions.
The language would be identical to \acl{smtlib} except that \rn{push}/\rn{pop}
would not be exposed to the user (or would only be enabled with an option), and
choices would be included in the language just as we have included the for
\ac{vpl} and \evpl{}.

Lastly, a promising area of future work is constructing an asynchronous
variational \ac{sat} and \ac{smt} solver. During our experience bench-marking
the variational prototype solvers we found that the majority of the time spent
in the base solver is spent querying for a model. Furthermore, each variant
waits until they can be processed by the base solver. For example, consider the
formula $\fV{} = \kf{\chc[A]{a,b} \wedge{} \chc[B]{c,d}}$, \fV{} has four
satisfiable variants. Our prototype solvers choose true alternatives first
(recurring down the left child of a relation), thus the order of the variants in
the base solver will be \sem[\set{(A,\tru{}), (B,\tru{})}]{f},
\sem[\set{(A,\tru{}), (B, \fls{})}]{f}, \sem[\set{(A,\fls{}), (B,\tru{})}]{f},
\sem[\set{(A, \fls{}), (B, \fls{})}]{f}. Notice that each false variant waits
for its true variant before being considered, for example every variant with
$\set{(A,\fls{})} \in C$ is processed \emph{after} variants where
$\set{(A,\tru{})} \in C$, and similarly so for the $\kf{B}$ dimension. Due to
this ordering, the runtime cost of solving false variants includes the cost of
solving the true variants, unless the variation context excludes true variants.
However the problem is tractable, instead of using \rn{push} and \rn{pop} to
represent variation, we could instead fork a new solver thread and solve all
$(A, \fls{})$ variants on that solver thread, or mix independent solver
instances and incremental solving.

We have created three versions of asynchronous prototype solvers but have not
succeeded in constructing a generalized sound asynchronous variational solver,
and thus do not provide a formalization. Constructing an asynchronous solver is
relatively straightforward. Since variational models form monoids, the order in
which plain models are added to the variational model isn't important.
Similarly, since variational cores capture the evaluation context at a given
time, transmitting variational cores to other solver instances is also
straightforward.

The problem for asynchronous solvers is ensuring that the ordering of
alternatives is maintained and consequently that variants remain isolated from
each other. For example, a simple model might be to have a pool of producer base
solver instances and a pool of consumers instances. The producer instances could
derive variational cores, and the consumers would take a variational core and a
configuration, and find the next choice that is not in the configuration or
generate a model. The two pool model's appeal is its simplicity, however a
subtle bugs are introduced due to the interaction between variation and
asynchronous workloads.

Assume we have a formula with three unique dimensions $A$, $B$, and $C$ which
will be processed in that order, \ie{} the same order as the variants of \fV{}
above. Since the order of alternatives is no longer deterministic we might
encounter a case where we are stuck or have mixed variants. Consider the case
where there are an unbalanced number of consumers and producers, with consumers
significantly outnumbering produces. Now consider a scenario where a consumer
thread has consumed the \set{(A, \tru{}), (B,\fls{})} core and then finds a
choice with a $C$ dimension. This thread must wait for a request from a producer
thread to mutate its local configuration, thereby configuring for an alternative
and continuing to solve. Suppose the consumer observes a request to consume
\set{(C,\tru{})}, does so, and produces a model for that variant. Now, the
consumer will backtrack with a \rn{pop} call and wait for another request from a
producer for $\kf{(C, \fls{})}$. However, this is an asynchronous environment
and so this thread may have out paced other threads. Thus the next request might
be to consume \set{(B,\tru{})}, and now we are stuck. If the consumer accepts
the request we will have mixed two variants, \set{(B,\fls{})} and \set(B,\tru{})
on this thread yielding incorrect results, if the consumer does not take the
request then we could end in a deadlock if the scenario is repeated for each
consumer.

Such an example is contrived but occurs with asynchronous communication and must
be accounted for. The fix is for each thread to track which variant it has
solved and maintain a stack to track the ordering of choices. We must ensure
that the choices are solved in order such that if a request comes to solve a
\set{(A, \tru{})} variant, and the thread has consumed the variational core with
\set{(A, \fls{})} then the thread must issue as many \rn{pop}s as needed to
backtrack. By tracking this information we can avoid deadlocks, and malformed
variants and still gain the benefits of concurrent solving which could be
substantial especially for large variational formulas. Whether the performance
gains outweigh the costs is an open research problem. It simply could be the
case that the runtime cost of forking, inter-process communication, and the cost
of avoiding poor performing scenarios, such as more than one pop, does not
outweigh the performance gains from asynchronously finding plain models.

\subsection{Automated \ac{vpl} Formulas}
Thus far we have only considered a \ac{vpl} or \evpl{} formula as input to a
variational solver. This format is likely to be inconvienient as end-users
consider sets of \ac{sat} problems. Thus, a useful extension for these users is
to change the input from a \ac{vpl} formula to a set of \ac{sat} problems the
user is interested in. With the set of \ac{sat} problems, one could synthesize a
\ac{vpl} formula with a sharing ratio that is \emph{good enough} and then run
the solver on that \ac{vpl} formula. For the rest of this section, we'll refer
to the problem of synthesizing a \emph{good} \ac{vpl} formula from a set of
\ac{sat} formulas the \emph{synthesis problem}.

There are several considerations to highlight. First, we found that the sharing
ratio of a formula positively correlates to run-time performance in
\autoref{chapter:case-studies}, echoing results from previous research on
variation. Therefore, the synthesis algorithm should try to maximize the sharing
ratio as it chooses which variants to combine in a choice. Second, minimizing
the number of choices is high priority for the algorithm. Our results indicate
that the run-time of the variational solver grows linearly in the number of
variants to solve (hence exponentially in the number of unique dimensions), thus
adding a single new choice doubles the number of variants and the expected
run-time. Rather than provide an algorithm that find the \emph{best} \ac{vpl}
formula, we instead describe a greedy algorithm that tries to find a reasonable
\ac{vpl} formula. An algorithm that finds the \emph{best} \ac{vpl} formula,
\eg{} one which maximizes the sharing ratio while minimizing the number of
choices is an open research problem. We suspect it is at least NP-hard (likely
by demonstrating that the Binary Decision Diagram variable ordering problem karp
reduces to the \ac{vpl} synthesis problem), although we have not begun to
investigate the problem space.

The synthesis problem is a search problem over a total undirected graph of
possible formula combinations. Each vertex in the graph is a \ac{sat} formula or
\ac{vpl} formula that can be combined and is connected to every other vertex.
Edges represent the possible combinations of two vertices and is weighted with a
\emph{fitness metric} indicating a good or bad match. Good or bad in this domain
indicates a high degree of sharing between two vertices. Our approach is to
traverse the graph and greedily select the best combination between two
vertices. Combinations mutate the graph. The old vertices are replaced with the
combined vertex, the old edges of the combined vertices are relaxed and new
edges connect the combined vertex to every other vertex in the graph. The
algorithm then repeats until only a single vertex remains in the graph.

There are two sub-procedures in the algorithm: A procedure to combine vertices,
and a procedure to calculate the fitness metric between two vertices. To combine
two vertices we generate a unique dimension, nest one vertex in the true
alternative, and the other in the false alternative. This simple combination
procedure results in poor sharing as the choice will always be at the root of
the abstract syntax tree of the resulting \ac{vpl} formula. Thus to increase the
sharing ratio, we drive the choice towards the leaves of the abstract syntax
tree of the \ac{vpl} formula using the equivalency laws in \autoref{fig:cc:eqv}.

Next we need a procedure that inputs two \ac{sat} or \ac{vpl} formulas, and
returns a fitness metric. There are several possible algorithms; ranging from
string edit distance, to a tree edit distance over the abstract syntax trees of
the \ac{sat} or \ac{vpl} formulas. String comparison algorithms such Levenshtein
distance\cite{Levenshtein_SPD66} or Hamming distance~\cite{H:BST50} are
promising as both have implementations which run in polynomial time, assuming an
encoding from the \ac{sat} problems to strings is computationally feasible.
Graph edit distance is a more direct approach but is NP-Complete with an
approximate solution that is APX-hard~\cite{hardnessOfGraphEditDistance}.
However, most edit distance algorithms work well in practice, and it is likely
that the graph comparisons in this domain are simpler than comparisons which
occur in the worst case, \eg{}, over enormous graphs such as those found in
social networks. Furthermore there are many heuristics such as longest-common
sub-string which might produce metrics that are good enough for reasonable
sharing ratios. The exact design of the informal algorithm described here is
left as an open research problem.

\subsection{Abstracting the Variationalization Recipe to Other Domains}
Our approach to creating a variation-aware system by using the plain version of
that system is not specific to satisfiability solvers. The only portion of our
work that is particular to satisfiability solvers is code generation in the base
solver. In essence, our method is a variational left-fold over a variational
language. Thus, one might reuse the ideas of accumulation, evaluation, choice
removal and variational cores in other domains. In particular, the recipe for
variationalization to other domains is clear: To variationalize a plain system
one needs to define the variational artifact for the domain, and a method to
express variation in that system. Our variational artifact was a \ac{vpl}
formula and we chose to use scopes from the \acl{smtlib} standard to express
variation in the plain \ac{sat} solver. Then, one needs a method to express
segments of plain terms and preserve sharing between variants in the plain
system, our approach was to define symbolic values and utilize the internal
cache of the plain solvers to preserve sharing. Lastly, one needs a way to
retrieve results and combine plain results in any order, just as we defined
monoidal variational models.

Using this recipe one can imagine a variational prolog which reuses the work
presented in this thesis. For such a language, the variational artifact would be
a prolog-like programming language with choices. Expressing segments of plain
terms with symbolic values could be directly reused from this thesis. Similarly,
the variational result would be nearly identical to the variational models
presented in \autoref{section:vsat:models}. Embedding variation in prolog is the
difficult part although there are several possibilities.
SWI-prolog~\cite{wielemaker:2011:tplp} defines a special kind of predicate
called \emph{dynamic predicates}. Dynamic predicates indicate to the prolog
interpreter that the predicate may change during execution. Changing the
predicate during execution is performed using two primitives, \emph{assertz} and
\emph{retract}. Thus prolog defines a way to assert a constraint in the
interpreter and then refine the constraint as needed, and so dynamic predicates
may serve as a viable primitive for variation in the prolog interpreter. Another
promising embedding is using delimited continuations. In \autoref{chapter:vsat}
we hypothesized that because a Heut zipper is used for choice removal, using
delimited continuations is also feasible as Huet zippers have been shown to be
isomorphic to delimited continuation~\cite{olegZippers}. Fortunately prolog has
first class support for delimited
continuations~\cite{DBLP:journals/tplp/SchrijversDDW13} and thus choice removal
could be done in the base prolog interpreter rather than at the variation-aware
level. Using delimited continuations could greatly reducing the complexity of
creating a variational prolog, so much so that it might be possible to define
variational prolog as a library rather than a separate entity. The exact details
for a variational implementation are not clear but creating a variational prolog
is a promising avenue of future work.

%%% Local Variables:
%%% mode: latex
%%% TeX-master: "../../thesis"
%%% End:
%

% \ac{sat} and \ac{smt} solvers are ubiquitous and powerful tools in computer
% science and software engineering. Incremental \ac{sat} and \ac{smt} provide an
% interface that supports solving many related problems efficiently. However, the
% interface could be automated and improved.

% The goal of this thesis is to explore the design and architecture of a
% variational satisfiability solver that automates and improves on the incremental
% \ac{sat} interface. Through the application of the choice calculus the interface
% can be automated for satisfiability problems, the solver interaction can be
% formalized and made asynchronous, and the solver is able to directly express
% variation in a problem domain.

% The thesis will present a complete approach to variational satisfiability and
% satisfiability modulo theory solving based on incremental solving. It will
% include a method to automatically encode a set of Boolean formulae into a
% variational propositional formula. A method for detecting the difficulty of
% solving such a variational propositional formula. A data set suitable for future
% research in the \ac{sat}, \ac{spl} and variation research communities. A
% variational satisfiability solver, an asynchronous variational satisfiable
% modulo theory solver and a proof of variational preservation.

%%% Local Variables:
%%% mode: latex
%%% TeX-master: "../../thesis"
%%% End:

%%%%%%%%%%%%%%%%%%%%%%%%%%%%%%% Acronyms %%%%%%%%%%%%%%%%%%%%%%%%%%%%%%%%%%%%%%%
\begin{acronym}[]
  \acro{cnf}[CNF]{conjunctive normal form}
  \acro{dmf}[DMF]{disjunctive minimal form}
  \acro{dnf}[DNF]{disjunctive normal form}
  \acro{dsl}[DSL]{domain specific language}
  \acro{bdd}[BDD]{Binary Decision Diagram}
  \acro{sat}[SAT]{satisfiability solving}
  \acro{smt}[SMT]{satisfiability-modulo theories}
  \acro{spl}[SPL]{software product-lines}
  \acro{cdcl}[CDCL]{conflict-driven clause learning}
  \acro{dpll}[DPLL]{Davis-Putnam-Logemann-Loveland}
  \acro{cs}[CS]{clause sharing}
  \acro{vpl}[VPL]{variational propositional logic}
  \acro{cse}[CSE]{common subexpression elimination}
\end{acronym}

%%%%%%%%%%%%%%%%%%%%%%%%%%%% Bibilography %%%%%%%%%%%%%%%%%%%%%%%%%%%%%%%%%%%%%%%
\begin{footnotesize}
  \bibliographystyle{unsrt}
  \bibliography{bib/abrv, bib/jeff,bib/eric,bib/martin,bib/spl,bib/satsolvers}
\end{footnotesize}
%%%%%%%%%%%%%%%%%%%%%%%%%%%% Bibilography %%%%%%%%%%%%%%%%%%%%%%%%%%%%%%%%%%%%%%%

\end{document}
