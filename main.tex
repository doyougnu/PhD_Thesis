% -*- mode: latex; -*-
\documentclass[11pt]{article}

\topmargin      0.0in
\headheight     0.0in
\headsep        0.0in
\oddsidemargin  0.0in
\evensidemargin 0.0in
\textheight     9.0in
\textwidth      6.5in

%%%%%%%%%%%%%%%%%%%%%%%%%%%%%%% Packages %%%%%%%%%%%%%%%%%%%%%%%%%%%%%%%%%%%%%%%
\usepackage{graphics}
\usepackage{epsfig}
\usepackage{times}
\usepackage{amsmath}
\usepackage{amssymb}                 %% for more arrows like rightarrowtail
\usepackage{amsfonts}                %% \mathbb
\usepackage{mathtools}               %% for coloneqq and others
\usepackage{cases}                   %% better math brackets
\usepackage{mathpartir}              %%% for inference rules
\usepackage[nolist]{acronym}
\usepackage{hyperref}                %% for auto references
\usepackage{xpunctuate}              %% for punctuation's after macros
\usepackage{cite}                    %% for bibliography ranges
\usepackage{subcaption}              %% For complex figures with subfigures/subcaptions
\usepackage{wrapfig}                 %% wrapping figures around text
\usepackage{listings}                %% for source code lstlisting
\usepackage{xcolor}                  %% For listings
\usepackage{caption}
\usepackage{tikz}                    %% assertion stack visualization
\usetikzlibrary{matrix}
\usetikzlibrary{arrows}
\usetikzlibrary{positioning}
\usetikzlibrary{shapes.multipart}
\usetikzlibrary{automata}

\usepackage{lib/cc}
\usepackage{lib/lambda}
\usepackage{lib/paperCommands}
%%%%%%%%%%%%%%%%%%%%%%%%%%%%%%% Packages %%%%%%%%%%%%%%%%%%%%%%%%%%%%%%%%%%%%%%%

\lstset{%
  frame=top,frame=bottom,
  basicstyle=\small\normalfont\ttfamily,    % the size of the fonts that are used for the code
  stepnumber=1,                           % the step between two line-numbers. If it is 1 each line will be numbered
  numbersep=10pt,                         % how far the line-numbers are from the code
  tabsize=2,                              % tab size in blank spaces
  extendedchars=true,                     %
  breaklines=true,                        % sets automatic line breaking
  captionpos=t,                           % sets the caption-position to top
  mathescape=true,
  showstringspaces=false,
  escapeinside={(*@}{@*)},
}%

\title{{\bf Variational Satisfiability Solving} \\
\it PhD. Thesis}%
\author{{\bf Jeffrey M. Young}  \\
Department of Electrical Engineering and Computer Science \\
Oregon State University\\
{\small youngjef@oregonstate.edu}
}
\date{\today}

\begin{document}
\pagestyle{plain}
\pagenumbering{roman}
\maketitle

\begin{abstract}
  Over the last two decades, satisfiability and satisfiability-modulo theory
(SAT/SMT) solvers have grown powerful enough to be general purpose reasoning
engines throughout software engineering and computer science. However, most
practical use cases of SAT/SMT solvers require not just solving a single SAT/SMT
problem, but solving sets of related SAT/SMT problems. This discrepancy was
directly addressed by the SAT/SMT community with the invention of incremental
SAT/SMT solving. However, incremental SAT/SMT solvers require end-users to hand
write a program which dictates the terms that are shared between problems and
terms which are unique. By placing the onus on end-users, incremental solvers
couple the end-users' solution to the end-users' \emph{exact} sequence of
SAT/SMT problems---making the solution overly specific---and require the
end-user to write extra infrastructure to coordinate or handle the results.

This dissertation argues that the aforementioned problems result from accidental
complexity produced by solving a problem that is \emph{variational} without the
concept of \emph{variation}, similar to problematic use of \rn{goto} statements
in the absence of \rn{while} loop constructs. To demonstrate the argument, this
thesis applies theory from \emph{variational} programming to the domain of
SAT/SMT solvers to create the first variational SAT solver and solve
aforementioned problems.
%
To do so, the thesis formalizes a variational propositional logic and specifies
variational SAT solving as a compiler, which compiles variational SAT problems
to non-variational SAT that are processed by an industrial strength SAT solver.
It shows that the compiler is an instance of a variational fold and uses that
fact to extend the variational SAT solver to an asynchronous variational SMT
solver.
% Finally, it defines a general algorithm to construct a single
% variational string from a set of non-variational strings.

%%% Local Variables:
%%% mode: latex
%%% TeX-master: "../thesis"
%%% End:
\end{abstract}

\pagenumbering{arabic}

\section{Introduction}
~\label{sec:introduction}

%% get to the problem:
%%% incremental satisfiability solvers solve sets of related problems efficiently
%%% but the interface could be improved in two ways
%%% first the interface require hand programming the solver, leading to
%%% brittle solutions
%%% secondly, the specifies control flow operations that dictate runtime
%%% behavior, thus any static optimizations enabled by domain knowledge cannot
%%% be used.

Classic \ac{sat}, which solves the boolean satisfiability
problem~\cite{10.5555/1550723} has been one of the largest success stories in
computer science over the last two decades. Although \ac{sat} solving is known
to be NP-complete~\cite{10.1145/800157.805047}, \ac{sat} solvers based on
\ac{cdcl}~\cite{Marques-Silva:1999:GSA:304491.304506,Silva:1997:GNS:244522.244560,10.5555/1867406.1867438}
have been able to solve boolean formulae with millions variables quickly enough
for use in real-world applications\cite{10.5555/1557461}. Leading to their
proliferation into several fields of scientific inquiry ranging from software
engineering to
Bioinformatics~\cite{10.1007/11814948_16,10.1007/978-3-642-31612-8_12}.

However, the majority of research in the \ac{sat} community focuses on solving a
single \ac{sat} problem as fast as possible, yet many practical applications of
\ac{sat}
solvers~\cite{silva1997robust,10.1007/3-540-44798-9_4,10.1145/378239.379019,10.1145/1698759.1698762,Een_asingle-instance,een2003temporal,10.5555/1998496.1998520}
require solving a set of related \ac{sat}
problems~\cite{10.1007/3-540-44798-9_4, silva1997robust, een2003temporal}. To
take just one example, \ac{spl} utilizes \ac{sat} solvers for a diverse range of
analyses including: automated feature model
analysis~\cite{useBTRC05,GBT+19,TAK+:CSUR14}, feature model
sampling~\cite{MKR+:ICSE16,VAT+:SPLC18}, anomaly
detection~\cite{AKTS:FOSD16,KAT:TR16,MNS+:SPLC17}, and dead code
analysis~\cite{TLSS:EuroSys11}.

This gap between the \ac{sat} research community and the practical use cases of
\ac{sat} solvers is well known. To address the gap, modern solvers attempt to
propagate information from one solving instance, on one problem, to future
instances in the problem set. Initial attempts focused on
\ac{cs}~\cite{10.1007/3-540-44798-9_4,10.1145/378239.379019} where learned
clauses from one problem in the problem set are propagated forward to future
problems. Modern solvers are based on a major breakthrough that occurred with
\emph{incremental \ac{sat} under assumptions}, introduced in
Minisat~\cite{10.1007/978-3-540-24605-3_37}.

Incremental \ac{sat} under assumptions, made two major contributions: a
performance contribution, where several pieces of information including learned
clauses, restart and clause-detection heuristics are carried forward. A
usability contribution; Minisat exposed an interface which allowed the end-user
to directly program the solver. Through the interface the user can add or remove
clauses and dictate which clauses or variables are shared and which are unique
in the problem set.

Despite the success of incremental \ac{sat}, the incremental interface can be
improved in two ways: First, by requiring the user to direct the solver, the
users' solution is specific to the exact set of satisfiability problems at hand,
thus the programmed solution is specific to the problem set and therefore to the
solver input. Second, should the user be interested in the assignment of
variables under which the problem at hand was found to be satisfiable, then the
user must create additional infrastructure to track results; which again couples
to the input and is therefore difficult to reuse.

In this thesis, I hypothesize usability and performance improvements to
incremental \ac{sat} are possible by applying recent work on \emph{variation},
and \emph{variational
  programming}~\cite{EW11gttse,EW11tosem,HW16fosd,CEW16ecoop,Walk14onward},
which defines a theory of variation and formalizes a language to expresses
variation called the \emph{choice calculus}. With the choice calculus, the
aforementioned set of problems is able to be expressed statically as a
\emph{variational artifact}. With this representation, the interface to
incremental solving can be automated through a \emph{variational interpreter},
furthermore by identifying and isolating the variational nature in incremental
solvers, optimizations derived from the choice calculus become possible.

The goal of my research is to explore the design space and architecture of a
\emph{variational satisfiability solver} that uses research on variation in the
context of incremental \ac{sat} solving. The rest of this section expands on
these claims. \autoref{sec:prop-contr} lists the specific contributions this
thesis will make and outlines the rest of this document.

\subsection{Proposed Contributions}%
\label{sec:prop-contr}
The high-level goal of my research is to formalize and construct a variational
satisfiability solver that understands and can solve \ac{sat} problems that
contain \emph{variational values} in addition to boolean values. In pursuit of
this goal, my thesis will make the following contributions, items which are
complete at time of this writing are indicated with a \checkmark:
\begin{enumerate}
\item\label{vpl-deliverable} \emph{Variational propositional logic}: \ac{sat}
  solvers input and operate on sentences in propositional
  logic\cite{10.5555/1324777}. Variational satisfiability solvers, in order to
  reason about variation, must input sentences in a propositional logic that is
  \textit{variational}, \ie{} a many-valued logic~\cite{Rescher1969-RESML} which
  contains variational values, called \emph{choices} in addition to boolean
  values.

  The formulation of \ac{vpl} is requisite and central to the high level goal of
  designing a variational satisfiability solver. Furthermore, \ac{vpl} serves
  two other functions: It provides an avenue for future work through the
  formalization of variation in the domain of propositional logic for
  variational satisfiability solvers. It provides a foundation for research on
  variation in propositional logic outside of the considerations of
  satisfiability solvers.

  This work is nearly complete. The logic has been formalized and successfully
  used in a prototype variational solver\cite{10.1145/3382025.3414965}.
  \autoref{sec:vpl} introduces \ac{vpl} and describes the following
  contributions which are directly enabled by it:

  \begin{enumerate}
  \item \checkmark{} A set of variation preserving equivalences. Similar to the well known
    propositional logic equivalences, such as DeMorgan's law, these equivalences
    allow a variational solver to refactor input possibly yielding simpler
    variational sentences.
  \item\label{encoding-strat-deliverable} An efficient algorithm for translating
    a set of propositional formulae into a single \ac{vpl} formula. The
    prototype variational \ac{sat} solver used a naive algorithm, and
    preliminary results showed that the encoding impacts solver performance.
    Hence, finding a more efficient encoding algorithm is desirable. This work
    is yet to be done but there are two promising routes forward. First, a naive
    algorithm which interleaves syntactic equivalences to produce a \ac{vpl}
    formula that is easier to solve. Second, an algorithm similar to Huffman
    codes\cite{4051119} to translate the \ac{sat} problems into a data
    structure, then use heuristics to select high quality candidates to combine.
    With such an algorithm the end-user of the variational solver only needs to
    input their problem sequence rather than a \ac{vpl} formula.
  \end{enumerate}

\item\label{vsat-deliverable} \emph{A variational satisfiability solver}: This
  is the central contribution of my thesis. It is completed and is published in
  a peer-reviewed conference~\cite{10.1145/3382025.3414965} paper. Preliminary
  results are promising but based on only two case studies from the \ac{spl}
  community.

  \autoref{sec:vsat} discusses these results and provides an overview of the
  variational solving algorithm. The following contributions are based on this
  work:
  \begin{enumerate}
  \item \checkmark{} Formalization of a variational \ac{sat} solving algorithm
    that inputs a \ac{vpl} formula and outputs a \emph{variational model}.
  \item \checkmark{} Formalization of variational models; that is satisfying
    assignments of values to variables in input formula that succinctly
    represent results in the context of variation.
  \item \checkmark{} A method for determining the amount of variation in a given
    \ac{vpl} formula.
  \item\label{phase-change-deliverable} A method for determining the relative
    hardness of a \ac{vpl} formula based on work in the random-\ac{sat}
    community~\cite{Gent94thesat}. This item is orthogonal to all other items
    and thus can be done in parallel.
  \end{enumerate}

\item\label{vsmt-deliverable} \emph{A concurrent variational \ac{smt} solver}:
  Contingent on \autoref{vsat-deliverable}, \emph{satisfiability modulo
    theories} extends \ac{sat} solvers such that they are able to reason about
  logical formulas in combination to \textit{background theories}, such as
  arithmetic or arrays. Furthermore, with variation statically represented in a
  \ac{vpl} formula, the \ac{sat} or \ac{smt} procedure can be made asynchronous
  leading to speedups on multi-core machines. The approach is to change the
  semantics of a choice; in the prototype \ac{sat} solver each choice blocks
  future \ac{sat} problems from being solved, by creating an asynchronous
  solving algorithm these future problems are unblocked and can be processed
  earlier.

  This item is an extension of the central contributions of the thesis. There
  are two extensions to the previous work to construct a variational \ac{smt}
  solver and one to make it asynchronous.

  First, the extensions to \ac{vpl} abstract logical connectives in \ac{vpl}
  allowing for theories which conclude to a Boolean value, such as arithmetic
  inequalities, and thus can be reasoned about in a \ac{smt} solver. Second,
  variational models are similarly extended, rather than assuming only Boolean
  values, the extension allows for polymorphic results through the use of
  SMTLIB2 compliant functions.

  Third, I extend the semantics of choices in the variational \ac{sat} solver to
  include atomic concurrent operations. When a choice is observed the solver
  state is copied and sent to a thread with instructions to compute continue the
  computation.

  This work is completed but unpublished. \autoref{sec:vsmt} expands on this
  item and discusses the evaluation of the prototype variational \ac{smt} solver
  with additional case studies. The following summarizes the expected
  contributions:
  \begin{enumerate}
  \item \checkmark{} Formalize the extension of \ac{vpl} with \ac{smt} theories.
  \item \checkmark{} Formalize the extension of variational models to express
    \ac{smt} results.
  \item \checkmark{} Formalize the asynchronous variational solving algorithm
  \item\label{nanopass-deliverable} A set of optimizations based on work on
    nanopass compilers~\cite{10.1145/2500365.2500618} from the scheme
    programming language community~\cite{r7rs-scheme}. The goal is to leverage
    \ac{vpl} equivalence rules and other compiler optimizations, such as
    inlining, on SMTLIB2 programs, thus optimizing variational SMT programs. The
    prototype variational \ac{smt} solver is architected as a nanopass compiler
    and thus is able to perform optimizations as a single pass over the input
    formula. However, no optimizations are performed as of yet, although all
    requisite items for this work to begin are done.
  \item \label{eval-deliverable}An empirical evaluation of solver performance. The empirical evaluation
    will reuse the datasets the prototype \ac{sat} solver was evaluated on. In
    addition, three new data sets will be added, two by harvesting \ac{sat}
    problems from work on variational lexing, parsing, and type
    checking~\cite{KKHL:FOSD10} real world software such as
    Busybox~\cite{busybox} and the Linux kernel~\cite{linux}, and one by
    generating variational \ac{smt} problems. This dataset will be used several
    times in the thesis and will be made public. First, as a foundation to test
    the encoding strategies from \autoref{encoding-strat-deliverable}. Second,
    to evaluation the optimizations from \autoref{nanopass-deliverable} and
    third, to evaluate the performance of the single threaded and multi-threaded
    variational \ac{smt} solver. This work is partially complete, random
    generation of variational \ac{sat} and \ac{smt} problems is done, as is the
    Busybox dataset. The remaining work is to scale the logging solution to
    handle the Linux kernel.
  \end{enumerate}

\item \label{proof-deliverable} \textit{Proof of variation preservation}: A
  proof of variation preservation is a proof that the results of the variational
  solvers are sound, \ie{}, for any variant $\kf{v}$, if a variational solver
  finds $\kf{VSat(v)} = \kf{True}$, then $\kf{Sat(v)} = \kf{True}$. Both
  \autoref{vsat-deliverable} and \autoref{vsmt-deliverable} are verified sound
  via property-based testing~\cite{10.1145/351240.351266} but the variational
  solving algorithm itself has not been proven sound up to the soundness of the
  underlying incremental solver. This work is in progress using the proof
  assistant Agda\cite{10.1145/2841316} and is expected to yield such a proof.
  \autoref{sec:vsmt} discusses this item further and lists the specific tasks
  left to do.
\end{enumerate}

\subsection{Significance and Potential Impact}%
\label{sec:sign-potent-impact}
The goal of this thesis is to explore the design and architecture of a
variational satisfiability solver. The solver should allow end-users to input a
set of propositional formulae and output a model that is useful \emph{without}
requiring the end-user to understand or be aware of research on variation.

This work is applied programming language theory in the domain of satisfiability
solvers. Many analyses in the software product-lines community use incremental
\ac{sat} solvers. By creating a variational \ac{sat} solver it is likely that
such analyses would directly benefit from this work, and thus advance the state
of the art.

For researchers in the incremental satisfiability solving community, this work
serves as an avenue to construct new incremental \ac{sat} solvers which
efficiently solve classes of problems that deal with variation, by exploiting
results from the programming language community.

For researchers studying variation the significance and impact is several fold.
By utilizing results in variational research, this work adds validity to
variational theory and serves as an empirical case study. At the time of this
writing, and to my knowledge, this work is the first to directly use results in
the variational research community to parallelize a variation unaware tool. Thus
by directly handling variation, this work demonstrates possible benefits, such
as parallelism, researchers in other domains may attain and thereby magnifies
the impact of any results produced by the variational research community.
Furthermore, the result of my thesis, a variational \ac{sat} solver, provides a
new logic and tool to reason about variation itself.

For researchers in other domains, a requisite result in constructing a
variational satisfiability solver is a variational compiler; which translates
\ac{vpl} to a solver-domain programming language. Thus, while my thesis is
focused on the domain of \ac{sat} solvers, this work describes a first of its
kind variational compiler whose architecture may be reused to create new
variation-aware tools such as build systems or programming languages. Such
compilers could directly benefit from \autoref{nanopass-deliverable} as this
item describes performance improvements that \emph{are only possible} with an
explicit and static representation of variation.

%%% Local Variables:
%%% mode: latex
%%% TeX-master: "../main"
%%% End:

\section{Background}
~\label{sec:background}

This section provides necessary background on incremental \ac{sat} solving. All
descriptions follow the SMT-LIB2~\cite{BarFT-SMTLIB} standard and describe
incremental solvers as a black box eliding internal details of any specific
solver which adheres to the standard.

%
\begin{figure}[h]
  \begin{subfigure}[t]{.45\textwidth}
    \tikzstyle{block} = [draw,fill=blue!20,minimum size=2em, node distance=1cm]
% diameter of semicircle used to indicate that two lines are not connected
\def\radius{.7mm}
\tikzstyle{branch}=[fill,shape=circle,minimum size=3pt,inner sep=0pt]

\begin{tikzpicture}[>=latex']

    % Draw blocks, inputs and outputs

    % \foreach \y in {1,2,3,4,5} {

        % blocks
        \node[block] at (2,-1) (prod) {$SAT$};
        \node[block, name=stging, below of = prod] {$SAT$};
        \node[block, name=devel, below of = stging] {$SAT$};

        % input nodes
        \node[left of = devel, xshift=-10pt] (input1) {\rV{}};
        \node[left of = stging, xshift=-10pt] (input2) {\qV{}};
        \node[left of = prod, xshift=-10pt] (input3) {\pV{}};

        % inputs
        \draw[->] (input3) -- (prod);
        \draw[->] (input2) -- (stging);
        \draw[->] (input1) -- (devel);

        % outputs
        \draw[->] (prod.east) -- +(0.5,0);
        \draw[->] (stging.east) -- +(0.5,0);
        \draw[->] (devel.east) -- +(0.5,0);


        % \node at (3.5,-3) (result1) {$result$};
        \node[right of = stging, xshift=20pt] (input2) {$result_{\qV{}}$};
        \node[right of = devel, xshift=20pt] (input2) {$result_{\rV{}}$};
        \node[right of = prod, xshift=20pt] (input3) {$result_{\pV{}}$};
    % }
    % \node[block] at (2,-6) (block6) {$f_6$};
    % \draw[->] (block6.east) -- +(0.5,0);

    % % Calculate branch point coordinate
    % \path (input1) -- coordinate (branch) (block1);

    % % Define a style for shifting a coordinate upwards
    % % Note the curly brackets around the coordinate.
    % \tikzstyle{s}=[shift={(0mm,\radius)}]
    % % It would be natural to use the yshift or xshift option, but that does
    % % not seem to work when shifting coordinates.

    % \draw[->] (branch) node[branch] {}{ % draw branch junction
    %         \foreach \c in {2,3,4,5} {
    %             % Draw semicircle junction to indicate that the lines are
    %             % not connected. The intersection between the lines are
    %             % calculated using the convenient -| syntax. Since we want
    %             % the semicircle to have its center where the lines intersect,
    %             % we have to shift the intersection coordinate using the 's'
    %             % style to account for this.
    %             [shift only] -- ([s]input\c -| branch) arc(90:-90:\radius)
    %             % Note the use of the [shift only] option. It is not necessary,
    %             % but I have used it to ensure that the semicircles have the
    %             % same size regardless of scaling.
    %         }
    %     } |- (block6);
\end{tikzpicture}
    \caption{Brute force procedure, no reuse between solver calls.}%
    \label{fig:bkg:bf}
  \end{subfigure}%
  \vfill
  \begin{subfigure}[t]{.45\textwidth}
    \tikzstyle{block} = [draw,fill=blue!20,node distance = 3.2cm, minimum size=2em]

% diameter of semicircle used to indicate that two lines are not connected
\tikzstyle{branch}=[fill,shape=circle,minimum size=3pt,inner sep=0pt]

\begin{tikzpicture}[>=latex']

    % Draw blocks, inputs and outputs

    % \foreach \y in {1,2,3,4,5} {
        % blocks
        \node[block] at (2,-1) (prod) {$SAT$};
        \node[block, name=stging, below=3.3cm of prod] {$SAT$};
        \node[block, name=devel, below=1.3cm of stging] {$SAT$};

        % input nodes
        % \node[left of = devel, xshift=-15pt] (input1) {$development$};
        % \node[left of = stging] (input2) {$\turnBlue{staging}$};
        \node[left of =prod,xshift=-10pt] (input3) {\pV{}};

        % inputs
        \draw[->] (input3) -- (prod);
        \draw[->] (prod) -- node[xshift = 60pt, align=left] {%
          $ \begin{aligned}
            \texttt{pop}  &\quad \eV{}\\
            \texttt{pop}  &\quad \cV{}\\
            \texttt{pop}  &\quad \bV{}\\
            \texttt{push} &\quad (\bV{} \vee \neg \iV{})  \\
            \texttt{push} &\quad \cV{} \\
            \texttt{push} &\quad (\gV{} \rightarrow \cV{}) \\
          \end{aligned}
          $
        } (stging);
\draw[->] (stging) -- node[xshift = 85pt, align=left] {%
  $\begin{aligned}
    & \texttt{resetAssertionStack}\\
    &\texttt{push} \quad \zV{} \leftrightarrow (a \wedge b \wedge c \wedge e)
    % \text{pop} &\quad \kf{pti} \rightarrow c_{i.1}\\
    % \text{pop} &\quad (\kf{spectre\_v2} \vee \kf{l1tf}) \leftrightarrow \\&\quad (c_{0} \wedge (\kf{nospec\_store\_bypass\text{-}disable} \rightarrow f_{j})\\
    % \text{pop} &\quad (c_{0.0} \wedge c_{1} \wedge \ldots c_{n})\\
  \end{aligned}
  $
} (devel);

        % outputs
        \draw[->] (prod.east) -- +(0.5,0);
        \draw[->] (stging.east) -- +(0.5,0);
        \draw[->] (devel.east) -- +(0.5,0);


        % \node at (3.5,-3) (result1) {$result$};
        % \node[right of = stging, xshift=15pt] (input2) {$result$};
        % \node[right of = devel, xshift=15pt] (input2) {$result$};
        % \node at (3.5,-1) (input3) {$result$};

        \node[right of = stging, xshift=18pt] (input2) {$result_{\qV{}}$};
        \node[right of = devel, xshift=18pt] (input2) {$result_{\rV{}}$};
        \node at (3.8,-1) (input3) {$result_{\pV{}}$};
    % }
    % \node[block] at (2,-6) (block6) {$f_6$};
    % \draw[->] (block6.east) -- +(0.5,0);

    % % Calculate branch point coordinate
    % \path (input1) -- coordinate (branch) (block1);

    % % Define a style for shifting a coordinate upwards
    % % Note the curly brackets around the coordinate.
    % \tikzstyle{s}=[shift={(0mm,\radius)}]
    % % It would be natural to use the yshift or xshift option, but that does
    % % not seem to work when shifting coordinates.

    % \draw[->] (branch) node[branch] {}{ % draw branch junction
    %         \foreach \c in {2,3,4,5} {
    %             % Draw semicircle junction to indicate that the lines are
    %             % not connected. The intersection between the lines are
    %             % calculated using the convenient -| syntax. Since we want
    %             % the semicircle to have its center where the lines intersect,
    %             % we have to shift the intersection coordinate using the 's'
    %             % style to account for this.
    %             [shift only] -- ([s]input\c -| branch) arc(90:-90:\radius)
    %             % Note the use of the [shift only] option. It is not necessary,
    %             % but I have used it to ensure that the semicircles have the
    %             % same size regardless of scaling.
    %         }
    %     } |- (block6);
\end{tikzpicture}
    \caption{Incremental procedure, reuse defined by pop and push calls.}%
      % sat calls share state that is determined by }
    \label{fig:bkg:inc}
  \end{subfigure}
  \caption{}%
  \label{fig:bkg}
\end{figure}
%

Suppose, we have three related propositional formulas that we want to solve.
%
\begin{align*}
  p =\ a \wedge b \wedge c \wedge e && q=\ a \wedge (b \vee \neg \iV{}) \wedge c \wedge (\gV{} \rightarrow c) && r =~z \leftrightarrow (a \wedge b \wedge c \wedge e)
\end{align*}
%
\pV{} is simply a conjunction of variables. In \qV{}, relative to \pV{}, we can
see that two variables are added, \iV{}, \gV{}, one variable is removed \eV{},
and and there are two new clauses: $(b \vee \neg \iV{})$ and $(\gV{} \rightarrow
c)$, both of which possibly affect the values of \bV{} and \cV{}. In \rV{}, the
variables and constraints introduced in \pV{} are further constrained to a new
variable, \zV{}.

Suppose one wants to find a satisfying assignment for each formula. Using a
classic \ac{sat} solver results in the procedure illustrated in
\autoref{fig:bkg:bf}; where \sat{} solving is a batch process and no information
is reused. Alternatively, a procedure using an incremental \sat{} solver is
illustrated in \autoref{fig:bkg:inc}; in this scenario, all of the formulas are
solved by single solver instance where terms are programmatically added and
removed from the solver throughout the process. The ability to add and remove
terms from the solvers is enabled by a data structure within the incremental
\sat{} solver called an \textit{assertion stack}. The assertion stack is a stack
of declarations, definitions, or formulas that determine the \textit{context} of
the solver. A solver context is the union of all global variable definitions and
everything on the assertion stack. A program may add an assertion to the stack
via the \texttt{push} operation and remove from the top via a \texttt{pop}
operation~\cite{10.1007/978-3-319-09284-3_16}. A call to
\texttt{resetAssertionStack} pops everything on the stack.

In an efficient process one would initially add as many \emph{shared} terms as
possible; \pV{} in this example. Then check for satisfiability, request a model,
and manipulate the assertion stack to reach the next problem of interest; \qV{}
in this case. Notice that to reach the next problem, \qV{}, from \pV{}, several
operations are required: \eV{} and \cV{} must be removed, \bV{} must be updated,
and the new clauses must be introduced. To reach \rV{} from \qV{} all assertions
would need to be popped to add \zV{}, then re-pushed.

%%% Local Variables:
%%% mode: latex
%%% TeX-master: "../main"
%%% End:

\section{Proposal Contribution 1: VPL =\@ Variation + Propositional Logic}
~\label{sec:vpl} In this section, I present the syntax and semantics of
variational propositional logic. While this section fulfills the majority of
\autoref{vpl-deliverable} I restate it here to serve as background for
\autoref{sec:vsat}, \autoref{sec:vsmt} and the choice calculus. I conclude the
section by summarizing work left to do.
% \revised{and provide background to the choice calculus as the logic is
% formalized.}
%
The logic is a conservative extension of classic two-valued logic
(\pl{})\footnote{Notation for propositional logic comes from work on many-valued
  logic, see~\cite{Rescher1969-RESML}.} with a \emph{choice} construct from the
choice calculus~\cite{EW11tosem,Walk13thesis}, a formal language for describing
variation.

% We define the syntax and semantics of \vpl{} in Section~\ref{sec:logic:vpl} and
% demonstrate its use by encoding the example from Section~\ref{sec:bkgrnd} into
% \vpl{} in Section~\ref{sec:logic:enc}.

\begin{figure}[h!]
  \begin{subfigure}[t]{\textwidth}
    \centering
    \begin{syntax}
  % D & \Coloneqq & \text{(any dimension name)} & \textit{Dimension} \\ \\
  % t & \Coloneqq & r & \textit{Variable reference} \\
  % & | & \mathit{T} & \textit{True} \\
  % & | & \mathit{F} & \textit{False} \\ \\

  t & \Coloneqq & r \quad|\quad \tru \quad|\quad \fls
    & \textit{Variables and Boolean literals} \\[1.5ex]

  f & \Coloneqq & t    & \textit{Terminal} \\
    & | & \neg f       & \textit{Negate} \\
    & | & f \vee f     & \textit{Or} \\
    & | & f \wedge f   & \textit{And} \\
    & | & \chc[D]{f,f} & \textit{Choice} \\
\end{syntax}

    \caption{Syntax of \vpl{}.}%
    \label{fig:cc:stx}
  \end{subfigure}
  \begin{subfigure}[t]{\textwidth}
    \begin{align*}
  \sem[]{\cdot} &: f\rightarrow C \rightarrow f
    \qquad\qquad \text{where } C = D\rightarrow\mathbb{B}_\bot \\
  \sem[C]{t}             &= t \\
  \sem[C]{\neg f}        &= \neg \sem[C]{f} \\
  \sem[C]{f_1\wedge f_2} &= \sem[C]{f_1}\wedge\sem[C]{f_2}\\
  \sem[C]{f_1\vee f_2}   &= \sem[C]{f_1}\vee\sem[C]{f_2}\\
  \sem[C]{\chc[D]{f_1,f_2}} &=
    \begin{cases}
      \sem[C]{f_1}                       & C(D) = \true \\
      \sem[C]{f_2}                       & C(D) = \false \\
      \chc[D]{\sem[C]{f_1},\sem[C]{f_2}} & C(D) = \bot \\
    \end{cases}
\end{align*}

    \centering
    \caption{Configuration semantics of \vpl{}.}%
    \label{fig:cc:cfg}
  \end{subfigure}
  \begin{subfigure}[t]{\textwidth}
    \begin{align*}
  \chc[D]{f,f}
    & \equiv f
    & \rn{Idemp} \\
  \chc[D]{\chc[D]{f_1,f_2},f_3}
    & \equiv \chc[D]{f_1,f_3}
    & \rn{Dom-L} \\
  \chc[D]{f_1,\chc[D]{f_2,f_3}}
    & \equiv \chc[D]{f_1,f_3}
    & \rn{Dom-R} \\
  \chc[D_1]{\chc[D_2]{f_1,f_2},\chc[D_2]{f_3,f_4}}
    & \equiv \chc[D_2]{\chc[D_1]{f_1,f_3},\chc[D_1]{f_2,f_4}}
    & \rn{Swap} \\
  \chc[D]{\neg f_1,\neg f_2}
    & \equiv \neg\chc[D]{f_1,f_2}
    & \rn{Neg} \\
  \chc[D]{f_1\vee f_3,\;f_2\vee f_4}
    & \equiv \chc[D]{f_1,f_2}\vee\chc[D]{f_3,f_4}
    & \rn{Or} \\
  \chc[D]{f_1\wedge f_3,\;f_2\wedge f_4}
    & \equiv \chc[D]{f_1,f_2}\wedge\chc[D]{f_3,f_4}
    & \rn{And} \\
  \chc[D]{f_1\wedge f_2,f_1}
    & \equiv f_1\wedge\chc[D]{f_2,\tru}
    & \rn{And-L} \\
  \chc[D]{f_1\vee f_2,f_1}
    & \equiv f_1\vee\chc[D]{f_2,\fls}
    & \rn{Or-L} \\
  \chc[D]{f_1,f_1\wedge f_2}
    & \equiv f_1\wedge \chc[D]{\tru,f_2}
    & \rn{And-R} \\
  \chc[D]{f_1,f_1\vee f_2}
    & \equiv f_1\vee\chc[D]{\fls,f_2}
    & \rn{Or-R}
\end{align*}

    \centering
    \caption{\vpl{} equivalence laws}%
    \label{fig:cc:eqv}
    \vspace{0.4cm}
  \end{subfigure}
\caption{Formal definition of \vpl{}.}%
\label{fig:cc}
\end{figure}
%
% The idea behind the choice calculus extended logic (\vpl{}) is to construct a
% logic that \TODO{is this a word?}deterministically expresses what is shared, and
% what is distinct, in a set of formulas in classic two valued logic (\pl{}).
% Semantically, one can think of \vpl{} as describing, in a deterministic, and
% constrained way, what could be and what must be, where a term in \pl{} must be
% but a term in \vpl{} \textit{could be} optional. Where as you might have encode
% a sentence such as: All men are mortal; Socrates is a man; therefore Socrates is
% mortal in \pl{}, a corresponding sentence in \vpl{} could be: it could be that
% either Socrates or Anaximander is a man; all men are mortal; thus it could be that
% Socrates or Anaximander is a man. Notice the ending of our \vpl{} statement:
% \textit{is a man}, this is purposeful and highlights that in \vpl{} a point of
% variation simultaneously represents two options.
%

\subsection{Syntax}
%
The syntax of variational propositional logic is given in \autoref{fig:cc:stx}.
It extends the propositional formula notation of \pl{} with a single new
connective called a \emph{choice} from the choice calculus.
%
A choice $\chc[D]{f_1,f_2}$ represents either $f_1$ or $f_2$ depending on the
Boolean value of its \emph{dimension} $D$. We call $f_1$ and $f_2$ the
\emph{alternatives} of the choice.
%
Although dimensions are Boolean variables, the set of dimensions is disjoint
from the set of variables from \pl{}, which may be referenced in the leaves of
a formula. We use lowercase letters to range over variables and uppercase
letters for dimensions.
%
The syntax of \vpl{} does not include derived logical connectives, such as
$\rightarrow$ and $\leftrightarrow$. However, such forms can be defined from
other primitives and are assumed throughout the rest of the proposal.

\subsection{Semantics}
%
Conceptually, a variational formula represents several propositional logic
formulas at once, which can be obtained by resolving all of the choices. For
researchers unfamiliar with work on variation, it is useful to think of \ac{vpl}
as analogous to \cpp{ifdef}-annotated \pl{}, where choices correspond to a
disciplined~\cite{LKA:AOSD11} application of \cpp{ifdef} annotations.
%
From a logical perspective, following the many-valued logic of
Kleene~\cite{Rescher1969-RESML}, the intuition behind \ac{vpl} is that a choice is
a placeholder for two equally possible alternatives that is deterministically
resolved by reference to an external environment.
%
In this sense, \ac{vpl} deviates from other many-valued logics, such as modal
logic~\cite{sep-logic-modal}, because a choice \emph{waits} until there is
enough information to choose an alternative (i.e., until the formula is
\emph{configured}).

The \emph{configuration semantics} of \ac{vpl} is given in \autoref{fig:cc:cfg}
and describes how choices are eliminated from a formula. The semantics are
parameterized by a \emph{configuration}\ $C$, which is a partial function from
dimensions to Boolean values.
%
The first four cases of the semantics simply propagate configuration down the
formula, terminating at the leaves. The case for choices is the interesting one:
if the dimension of the choice is defined in the configuration, then the choice
is replaced by its left or right alternative corresponding to the associated
value of the dimension in the configuration. If the dimension is undefined in
the configuration, then the choice is left intact and configuration propagates
into the choice's alternatives.


If a configuration $C$ eliminates all choices in a formula $f$, we call $C$
\emph{total} with respect to $f$. If $C$ does \emph{not} eliminate all choices
in $f$ (i.e., a dimension used in $f$ is undefined in $C$), we call $C$
\emph{partial} with respect to $f$.
%
We call a choice-free formula \emph{plain}, and call the set of all plain
formulas that can be obtained from $f$ (by configuring it with every possible
total configuration) the \emph{variants} of $f$.
%
% Every plain variant of a \vpl{} formula is a \pl{} formula, demonstrating the
% intuition that a \vpl{} formula conceptually represents many plain \pl{}
% formula.


To illustrate the semantics of \vpl{}, consider the formula
$p\wedge\chc[A]{q,r}$, which has two variants: $p\wedge q$ when $C(A)=\true$
and $p\wedge r$ when $C(A)=\false$.
%
From the semantics, it follows that choices in the same dimension are
\emph{synchronized} while choices in different dimensions are
\emph{independent}. For example, $\chc[A]{p,q}\wedge\chc[B]{r,s}$ has four
variants, while $\chc[A]{p,q}\wedge\chc[A]{r,s}$ has only two ($p\wedge r$ and
$q\wedge s$).
%
It also follows from the semantics that nested choices in the same dimension
contain redundant alternatives; that is, inner choices are \emph{dominated} by
outer choices in the same dimension. For example, $\chc[A]{p,\chc[A]{r,s}}$ is
equivalent to $\chc[A]{p,s}$ since the alternative $r$ cannot be reached by any
configuration.
% \begin{theorem}[\vpl{} reducible to \pl{}]
%   \label{thm:vplToPl}
%   For any configuration $C$ and any formula $e$, if $C$ is
%   valid and total with respect to $e$, then $\sem[C]{e} \in \pl{}$
% \end{theorem}
%
% \begin{proof}
%   This follows directly from the semantics of configuration in
%   Figure~\ref{fig:cc:cfg}, and Definition~\ref{tot:conf}. The proof is done
%   by structural induction and case analysis; because we have a total
%   configuration, and the configuration semantic function is a total function,
%   every choice and its' configured alternatives, will be recursively reified for
%   $e$. Then by the definition of \vpl{} a formula which lacks choices is by
%   definition in \pl{}.
% \end{proof}
%
% \begin{lemma}[Variants are plain]
%   By Theorem~\ref{thm:vplToPl} and the fact that variants are found via total
%   configurations
% \end{lemma}
%
As the previous example illustrates, the representation of a \ac{vpl} formula is
not unique; that is, the same set of variants may be encoded by different
formulas. \autoref{fig:cc:eqv} defines a set of equivalence laws for \ac{vpl}
formulas. These laws follow directly from the configuration semantics in
\autoref{fig:cc:cfg} and can be used to derive semantics-preserving
transformations of \ac{vpl} formulas.
%
For example, we can simplify the formula $\chc[A]{p\vee q, p\vee r}$ by first
applying the \rn{Or} law to obtain $\chc[A]{p,p}\vee\chc[A]{q,r}$, then applying
the \rn{Idemp} law to the first argument to obtain $p\vee\chc[A]{q,r}$ in which
the redundant $p$ has been factored out of the choice.

\subsection{Research Plan}

The previous sections describe \ac{vpl} but is missing an efficient strategy for
encoding sets of \pl{} formulas in a \ac{vpl} formula. The proposed thesis will
directly address this gap:

\paragraph{Encoding strategies} This item will produce an efficient algorithm
that combines a set of \pl{} formulas into a single \ac{vpl} formula. Efficient
has two meanings: It should produce a \ac{vpl} formula in reasonable time, and
it should produce a \ac{vpl} formula that has measurably less variation if
possible. Such an algorithm is desirable for two reasons. First, it is
practically important; a result of a previous study of variational
satisfiability solving~\cite{10.1145/3382025.3414965} was that the greater the
\emph{sharing ratio}, the ratio of plain to total terms in a variational
formula, the faster the \ac{vpl} formula was solved, on average. Thus, by
developing a more efficient encoding algorithm, the performance of the solver is
less volatile. Second, it lowers the barrier of use, with such an algorithm, the
a new user because the new user need not understand \ac{vpl}, rather they only
need to identify the problem set they desire to solve.

A naive encoding algorithm is easy to construct: one wraps all formulas in
unique choices and then uses equivalency rules to increase the sharing ratio.
However, it is likely that better algorithms exist. I envision an algorithm that
utilizes a strategy similar to Huffman coding~\cite{4051119} to find similar
formulas to merge.

This work is able to be done in parallel to much of the other deliverables but
intersects with two other items. First, a proof of variation preservation,
\autoref{proof-deliverable}, for a variational solver must include a similar
proof for the encoding strategy. Second, the encoding strategy will affect
solver performance and thus also affect the evaluation,
\autoref{eval-deliverable}. Lastly, the performance of the encoding strategy
itself will require a set of data to be evaluated on. I discuss harvesting such
a set of data from real world software product lines in \autoref{sec:vsmt},
although enough real world data is already available to begin on this item.

%%% Local Variables:
%%% mode: latex
%%% TeX-master: "../main"
%%% End:

\section{Proposal Contribution 2: Variational Satisfiability Solving}
~\label{sec:vsat} A core contributions of the proposed thesis is the design and
architecture of a variational satisfiability solver. In this section, I review
the architecture of a prototype solver called \vsat{} and conclude the section
by summarizing work left to do on \autoref{phase-change-deliverable}, \ie{},
assessing the hardness of variational satisfiability problems.

\subsection{Design}
Research on \ac{sat} and incremental \ac{sat} is fast moving with novel \ac{sat}
solvers regularly competing in the International \ac{sat}
Competition~\cite{interSatComp}, to leverage these results we design variational
satisfiability solving to target the incremental interface specified in the
SMT-LIB2~\cite{BarFT-RR-17} standard. Targeting the standard provides several
benefits: an implementation of a variational solver is free to choose any
SMT-LIB2~\cite{BarFT-RR-17} compliant solver ranging from experimental solvers
such as cvc4~\cite{10.1007/978-3-642-22110-1_14}, to industrial strength solvers
such as z3~\cite{10.1007/978-3-540-78800-3_24}. Since the implementation is
\ac{sat} solver agnostic, a variational solver can be run as a
\emph{meta-solver}, \ie{}, a \ac{sat} solver that interleaves, or simultaneously
uses several \ac{sat} solvers to solve a \ac{sat} problem, especially in
asynchronous workloads. Lastly, any result from the \ac{sat} or \ac{smt}
community that enters the standard is supported, this includes new \ac{smt}
theories.

In addition to coupling to SMTLIB2, the variational \ac{sat} solver should not
place further burden on the end-user. Thus users of a variational solver should
not be required to understand the choice calculus in order to interpret their
results, a variational solver's output should not contain choices. This design
principle in combination with \autoref{encoding-strat-deliverable} completes the
approach and would allow an end-users to receive the benefits of this work
without additional upfront cost.

\subsection{Architecture}

This section provides an informal description of variational satisfiability
solving and variational models, I provide the formalization in the next section.
A variational satisfiability solver is a compiler from the domain of variational
formulas to SMT-LIB2 programs. Throughout this section, I use SMTLIB2 snippets
to describe variational solving concepts in terms of an incremental solver.
While I target SMTLIB2, conforming to the standard is not an essential
requirement. Any solver that exposes an incremental API as defined by
minisat~\cite{10.1007/978-3-319-09284-3_16} can be used to implement variational
satisfiability solving following the same architecture and semantics.

\begin{figure}
  \begin{subfigure}[t]{0.5\linewidth}
    \tikzstyle{block}    = [draw,fill=white!20,node distance = 3.5cm,align=center]
\tikzstyle{inEdge}   = [fill=white, text width=1cm]
\tikzstyle{overEdge} = [midway,above]
\tikzstyle{input}    = [fill=white!20,node distance = 2.2cm,align=center, text width=1cm]
\tikzstyle{double} = [draw, anchor=text, rectangle split,rectangle split parts=2]
% diameter of semicircle used to indicate that two lines are not connected
\tikzstyle{branch}=[fill,shape=circle,minimum size=3pt,inner sep=0pt]
\tikzstyle{pinstyle} = [pin edge={to-,thin,black}]
\begin{center}
\begin{tikzpicture}[>=latex]

  % The loop
  \node[block]  (solve) {Reification\\Engine};
  \node[block, below right=2.05cm and 3.125cm of solve] (vmodel) {VModel\\Constructor};
  \node[block, below=2.05cm of solve] (vcore) {Reduction\\Engine};
  \node[block, right=of solve] (base) {Base\\Solver};

  % input nodes
  \node (input) [input, left=0.7cm of vcore] {$\kf{Query}$ $\kf{formula}$};
  \node (vf)    [input, left=0.7cm of solve] {$\kf{variant}$ $\kf{formula}$};


  % outputs
  \node[input, right=0.75cm of vmodel] (output) {$\kf{VModel}$};

  \draw
  %% Inputs
  (input) edge[->] node[overEdge] {} (vcore)
  (vf)    edge[dotted,->] node[overEdge] {} (solve)

  %% outputs
  (vmodel) edge[->] node[overEdge] {} (output)
  (base)   edge[->,sloped] node[overEdge,left,rotate=89,xshift=3.0cm]
  {$\kf{plain~models}$} (vmodel)
  (base)   edge[->,sloped,bend right=10] node[overEdge,right,rotate=-90] {} (vmodel)
  (base)   edge[->,sloped,bend right=20] node[overEdge,right,rotate=-90] {} (vmodel)
  (base)   edge[->,sloped,bend left=10] node[overEdge,right,rotate=-90] {} (vmodel)
  (base)   edge[->,sloped,bend left=20] node[overEdge,right,rotate=-90] {} (vmodel)

  %% Loop
  (solve) edge[->,sloped, bend left=25] node[overEdge,right,text width=5.5cm,rotate=90] {$s$, $v_{1} \vee v_{2}$, $v_{1} \wedge v_{2}$} (vcore)
  (vcore) edge[->,sloped, bend left=25] node[overEdge,left,rotate=-90] {$\kf{VCore}$, $\unit{}$} (solve)
  (solve) edge[->,in=100,out=160,loop, min distance=20mm] node[overEdge,xshift=-2em,yshift=0.2em] {$v \wedge \sem[C]{\chc[D]{e_{1}, e_{2}}}$} (solve)
  (solve) edge[dashed,->,in=70,out=15,loop, min distance=20mm] node[overEdge,xshift=4.75em,yshift=0.2em,text width = 5.5cm] {$v \wedge \sem[C\ \cup\ \{(D, \true)\}]{\chc[D]{e_{1}, e_{2}}}$, $v \wedge \sem[C\ \cup\ \{(D, \false)\}]{\chc[D]{e_{1}, e_{2}}}$} (solve)

  %% Loop escape
  (solve) edge[->] node[overEdge] {} (base)
  (solve) edge[->,sloped,bend right=5] node[overEdge] {} (base)
  (solve) edge[->,sloped,bend right=10] node[overEdge] {} (base)
  (solve) edge[->,sloped,bend left=5] node[overEdge] {} (base)
  (solve) edge[->,sloped,bend left=10] node[overEdge] {\unit{}} (base)
  % (vmodel) edge[->,sloped,bend right=10]  node[overEdge] {$r$, $s$, $t$} (base)
  % (vmodel) edge[->,sloped,bend left=10]  node[overEdge,xshift=0.1cm] {$\neg v$, $v_{1} \vee v_{2}$} (acc)
  % (vmodel) edge[->, loop above]  node[overEdge] {$\unit{} \wedge v$, $v \wedge \unit{}$} (vmodel)
  % (acc) edge[->, loop below] node[overEdge,below] {$r$, $\neg s$, $s_{1} \wedge s_{2}$, $s_{1} \vee s_{2}$} (acc)
  % (acc) edge[->,sloped, bend left=25] node[overEdge,below] {$s$} (vmodel)
  % (base) edge[->,sloped, bend right=25] node[overEdge,below] {$\unit{}$} (vmodel)
  % (vmodel) edge[->] node[overEdge] {} (vcore)
  ;

\end{tikzpicture}%
\end{center}

    \vspace{3.5ex}
    \caption{System overview of a variational solver.}%
    \label{impl:overview}
  \end{subfigure}
  \begin{subfigure}[t]{0.5\linewidth}
    \tikzstyle{block}    = [draw,fill=white!20,node distance = 1.75cm,align=center]
\tikzstyle{inEdge}   = [fill=white, text width=1cm]
\tikzstyle{overEdge} = [midway,above]
\tikzstyle{input}    = [fill=white!20,node distance = 2.2cm,align=center, text width=1cm]
\tikzstyle{double} = [draw, anchor=text, rectangle split,rectangle split parts=2]
% diameter of semicircle used to indicate that two lines are not connected
\tikzstyle{branch}=[fill,shape=circle,minimum size=3pt,inner sep=0pt]
\tikzstyle{pinstyle} = [pin edge={to-,thin,black}]
\begin{center}
\begin{tikzpicture}[>=latex]

  % input nodes
  \node[input] (input) {$\kf{Query}$ $\kf{formula}$};
  \node[block, right= 0.55cm of input] (ilInput) {to IL};

  % The loop
  \node[block, right=0.55cm of ilInput] (eval) {Evaluation};
  \node[block, below right=1.5cm of eval] (acc) {Accumulation};
  \node[block, below left=1.5cm of eval] (base) {Base\\Solver};

  % output
  \node[input, above left=0.75cm and 0.05 of eval] (vcore) {};

  % nowhere node
  \node[input, above right=0.75cm and 0.05cm of eval] (nowhere) {};

  % escape edges
  \draw
  %% inputs
  (input) edge[->] node[overEdge] {} (ilInput)
  (ilInput) edge[->] node[overEdge] {} (eval)

  %% Eval
  (eval) edge[->,sloped,bend right=10]  node[overEdge] {$r$, $s$, $t$} (base)
  (eval) edge[->,sloped,bend left=10]  node[overEdge,xshift=0.1cm] {$\neg v$, $v_{1} \vee v_{2}$} (acc)
  (eval) edge[->, loop above]  node[overEdge, text width=1cm] {$v_{1} \wedge v_{2}$} (eval)
  (eval) edge[<-,sloped,bend right=25] node[overEdge,right,rotate=-57,text width=1.5cm,xshift=0.5em] {$s$, $v_{1} \vee v_{2}$, $v_{1} \wedge v_{2}$} (nowhere)

  %% acc
  (acc) edge[->, loop below, transform canvas={xshift=6mm}] node[overEdge,below, text width = 1cm] {$\neg v$, $v_{1} \vee v_{2}$, $v_{1} \wedge v_{2}$} (acc)
  (acc) edge[->, loop below, transform canvas={xshift=-6mm}] node[overEdge,below, text width = 1cm] {$r$, $\neg s$, $s_{1} \vee s_{2}$, $s_{1} \wedge s_{2}$} (acc)
  (acc) edge[->,sloped, bend left=25] node[overEdge,below, text width = 1.3cm] {$s$, $v_{1} \wedge v_{2}$, $\ v_{1}\vee v_{2}$} (eval)
  (base) edge[->,sloped, bend right=25] node[overEdge,below] {$\unit{}$} (eval)

  %% output
  (eval) edge[->,sloped, bend left = 25] node[overEdge,left,rotate=54] {$\kf{VCore}$} (vcore) ;

\end{tikzpicture}%
\end{center}

    \caption{Overview of the reduction engine.}
    \label{impl:vcore}
  \end{subfigure}
  \caption{}
\end{figure}

A \ac{vpl} formula is solved using a recursive approach, decoupling the handling
of plain terms from the handling of variational terms. The idea is to define a
process to evaluate plain terms and skip choices, then define another process
that can only configures choices thus introducing new plain terms to the formula
that can be recursively processed. The base case is a variant, at which point a
model can be queried and the assertion stack can be popped to backtrack to solve
another variant.

I present an overview of a variational solver as a state diagram in
\autoref{impl:overview} that operates on the input's abstract syntax tree.
Labels on incoming edges denote inputs to a state and labels on outgoing edges
denote return values; we show only inputs for recursive edges; labels separated
by a comma share the edge. We omit labels that can be derived from the logical
properties of connectives, such as commutativity of $\vee$ and $\wedge$.
Similarly, we omit base case edge labels for choices and describe these cases in
the text. The solver has four subsystems: The \emph{reduction engine} processes
plain terms and generates a formula ready for reification called a
\emph{variational core}. The \emph{reification engine} configures choices in a
variational core. The \textit{base solver} is the incremental solver used to
produce plain models. Finally, the \emph{variational model constructor}
synthesizes a single variational model from the set of plain models returned by
the base solver.

The solver inputs a \vpl{} formula, called a \emph{query formula}, and an
optional input, called a \emph{variation context} (\vc{}). A \vc{} is a
propositional formula of dimensions that restricts the solver to a subset of
variants.
%
The variational solver translates the query formula to a formula in an
intermediate language (IL) that the reduction and reification engines operate
over; its syntax is given below.
%
\[
  v \hquad\Coloneqq\hquad \unit{}
  \hquad|\hquad t
  \hquad|\hquad s
  \hquad|\hquad \neg v
  \hquad|\hquad v \wedge v
  \hquad|\hquad v \vee v
  \hquad|\hquad \chc[D]{e,e}
\]
%
The IL includes two kinds of terminals not present in the input query formulas:
plain sub-terms that can be reduced symbolically will be replaced by a
\emph{symbolic reference} $s$, and sub-terms that have been sent to the base
solver will be represented by the unit value \unit{}.
%
Note that choices contain unprocessed expressions ($e$) as alternatives.


\paragraph{Derivation of a Variational Core}%
\label{ssec:impl:accum}

A variational core is an IL formula that captures the variational structure of
a query formula. Plain terms will either be placed on the assertion stack or
will be symbolically reduced, leaving only logical connectives, symbolic
references, and choices.
%
Consider the query formula $f = ((a \wedge b) \wedge \chc[A]{e_{1}, e_{2}}) \wedge
((p \wedge \neg q) \vee \chc[B]{e_{3}, e_{4}})$. Translated to an IL formula,
$f$ has four references ($a$, $b$, $p$, $q$) and two choices. The reduction
engine shown in \autoref{impl:vcore} will produce a variational core that will
assert $(a \wedge b)$ in the base solver, thus pushing it onto the assertion
stack and create a symbolic reference for $(p \wedge \neg q)$. This is done in
two states: \emph{evaluation}, which issues commands to the base solver to
process plain terms, and \emph{accumulation} which is called by evaluation to
create symbolic references.

Generating the core begins with evaluation. Evaluation will match on the root
node: $\wedge$, of $f$ and recur following the $v_1 \wedge v_2$ edge, where
%
$v_1=(a\wedge b)\wedge\chc[A]{e_1,e_2}$ and
$v_2=(p\wedge \neg q) \vee \chc[B]{e_3, e_4}$.
%
The recursion processes the left child first. Thus, evaluation will again
match on $\wedge$ of $v_{1}$ creating another recursive call with $v_{1}' =
(a\wedge b)$ and $v_{2}' = \chc[A]{e_1,e_2}$. Finally, the base case is reached
with a last recursive call where $v_{1}'' = a$, and $v_{2}'' = b$. At the base
case both $a$ and $b$ are references, thus evaluation will send $a$ to the
base solver, following the $\kf{r, s, t}$ edge, which returns $\unit{}$ for
the left child. The right child follows the same process yielding $\unit{}
\wedge \unit{}$; since the assertion stack implicitly conjuncts all assertions,
$\unit{} \wedge \unit{}$ will be further reduced to $\unit{}$ and returned as
the result of $v_{1}'$,
indicating that both children have been pushed to the base solver. This
leaves $v_{1}' = \unit{}$ and $v_{2}' = \chc[A]{e_1,e_2}$. $v_{2}'$ is a base
case for choices and cannot be reduced in evaluation, and so $\unit{} \wedge
\chc[A]{e_1,e_2}$, will be reduced to just $\chc[A]{e_1,e_2}$ as the result for
$v_{1}$.

In evaluation, conjunctions can be split because of the behavior of the
assertion stack and the and-elimination property of $\wedge$. Disjunctions and
negations cannot be split in this way because both cannot be performed if a child
node has been lost to the solver, e.g., $\neg \unit{}$. Thus, in accumulation, we
construct symbolic terms to represent entire sub-trees, ensuring information is
not lost, but still allowing for the sub-tree to be evaluated if it is sound to
do so.

The right child, $v_2=(p\wedge \neg q) \vee \chc[B]{e_3, e_4}$ requires
accumulation. Evaluation will match on the root $\vee$, and send $(p\wedge \neg
q) \vee \chc[B]{e_3, e_4}$ to accumulation via the $v_{1} \vee v_{2}$ edge.
Accumulation has two recursive edges, one to create symbolic references (with
labels $r, s, \hdots$), and one to recur to values. Accumulation matches the
root $\vee$ and recurs on the self-loop with edge $v_{1} \vee v_{2}$, $v_{1} =
(p\wedge \neg q)$, and $v_{2} = \chc[B]{e_3, e_4}$. Processing the left child
first, accumulation will recur again with $v_{1}' = p$ and $v_{2}' = \neg q$.
$v_{1}' = p$ is a base case for references, thus a unique symbolic reference
$s_{p}$ is generated for $p$, following the self-loop with label $r$ and
returned as the result for $v_{1}'$. $v_{2}'$ will follow the self-loop with
label $\neg v$ to recur through $\neg$ to $q$, where a symbolic term $s_{q}$
will be generated and returned. This yields $\neg s_{q}$, which follows the
$\neg s$ edge to be processed into a new symbolic term, yielding the result for
$v_{2}'$ as $s_{\neg q}$. With both results $v_{1} = s_{p}\wedge s_{\neg q}$,
accumulation will match on $\wedge$ \emph{and} both $s_{p}$ and $s_{\neg q}$ to
accumulate the entire sub-tree to a single symbolic term, $s_{s_{p} \wedge
  s_{\neg q}}$, which will be returned as the result for $v_{1}$. $v_{2}$ is a
base case, hence accumulation will return $s_{s_{p} \wedge s_{\neg q}} \vee
\chc[B]{e_3, e_4}$ to evaluation. Evaluation will conclude with
$\chc[A]{e_1,e_2}$ as the result for the left child of $\wedge$ and $s_{s_{p}
  \wedge s_{\neg q}} \vee \chc[B]{e_3, e_4}$ for the right child, yielding
$\chc[A]{e_1,e_2} \wedge (s_{s_{p} \wedge s_{\neg q}} \vee \chc[B]{e_3, e_4})$ as
the variational core of $\kf{f}$.

A variational core is derived to save redundant work. If solved naively, plain
sub-formulas of $f$, such as $a \wedge b$ and $p \wedge \neg q$, would be
processed once for each variant even though they are unchanged. To save
computation evaluation moves sub-formulas into the solver state to be reused
among different variants, and accumulation caches sub-formulas that cannot be
immediately evaluated to be evaluated later.

Symbolic references are variables in the reduction engine's memory that
represent a sub-tree of the query formula. From the perspective of the base
solver a symbolic reference represents a set of programmatic statements. For
example, $s_{pq}$ represents three declarations in the base solver:
%
\begin{lstlisting}[columns=flexible,keepspaces=true]
(declare-const p Bool)           ;; $s_{pq}$ represents
(declare-const q Bool)           ;; several declarations
(declare-fun $s_{ab}$ () Bool (or p (not c)))
\end{lstlisting}

The program language shown here is lisp~\cite{10.1145/367177.367199} as defined
in the SMTLIB2 standard. Function application begins with an open parenthesis,
where the first symbol in the parenthesis is the name of the function, every
symbol after the first is an argument to that function. In the above snippet, we
see three function calls, two which declare constants in the program for \pV{}
and \qV{}, both with type $\kf{Bool}$, and one which declares a new function
which takes no input and returns a $\kf{Bool}$.

Similar to symbolic references, a variational core is a sequence of statements
in the base solver with holes $\Diamond$. For example, the representation of
$\kf{VCore_{f}}$:
%
\begin{lstlisting}[columns=flexible,keepspaces=true]
(assert (and a b))                ;; $a \wedge b$ on the assertion stack
(declare-const $\Diamond_{A}$)                ;; choice A
 $\vdots$                                ;; many declares may occur
(assert $\Diamond_{A}$)                       ;; many assertions may occur
 $\vdots$                                ;; $s_{pq}$
(declare-fun $s_{pq}$ () Bool (and p q))
(declare-const $\Diamond_{B}$)                ;; choice B
 $\vdots$
(assert (or $s_{ab}$ $\Diamond_{B}$))               ;; assert waiting on $\sem[C]{\chc[B]{e_{3},e_{4}}}$
\end{lstlisting}
%
Each hole is filled by configuring a choice and may require multiple statements
to process the alternative as the alternative could introduce several new
variables or functions.

\paragraph{Solving the Variational Core}

The reduction engine performs the work at each recursive step. Whereas the
reification engine defines transitions between the recursive steps by
manipulating the configuration. In \vpl{}, a configuration was formalized as a
function, for variational solvers we use a set of tuples $\{\kf{(D \times
  \mathbb{B})}\}$. \autoref{impl:overview} shows two self-loops for the
reification engine corresponding to the reification of choices. The edges from
the reification engine to the reduction engine are transitions taken after a
choice is removed, where new plain terms have been introduced and thus a new
core is derived. If the user supplied a variation context, then it is used to
construct an initial configuration. Finally, a model is called from the base
solver when the reduction engine returns \unit{}, indicating that a variant has
been found.

We display a subset of edges of the reification engine using the $\wedge$
connective. In general, these edges will be duplicated for each binary logical
connective, e.g., $\vee$. The left edge, is taken when a choice is observed in
the variational core: $v \wedge \sem[C]{\chc[D]{e_{1}, e_{2}}}$ and $D \in C$.
This edge reduces choices with dimension $D$ to an alternative, which are then
translated to IL\@. The right edge is dashed to indicate assertion stack
manipulation, and is taken when $D \notin C$. For this edge, the configuration
is mutated for both alternatives: $C \cup \{(D, \tru{})\}$, and $C \cup \{(D,
\fls{})\}$, and the recursive call is wrapped with a \texttt{push}, and
\texttt{pop} command. To the base solver, this branching is a linear sequence of
assertion stack manipulations that performs backtracking behavior, for example
the representation of $\kf{f}$ is:
%
\begin{lstlisting}[columns=flexible,keepspaces=true]
 $\vdots$          ;; declares and assertions from VCore
(push 1)    ;; a configuration on B has occurred
 $\vdots$          ;; new declarations for left alternative
(declare-fun $s$ () Bool (or $s_{pq}$ $[\Diamond_{B} \rightarrow s_{B_{T}}]\Diamond_{B}$))  ;; fill
(assert $s$)
 $\vdots$          ;; recursive processing
(pop 1)     ;; return for the right alternative
(push 1)    ;; repeat for right alternative
\end{lstlisting}
%
Where the hole $\Diamond_{B}$\footnote{the notation $[x \rightarrow v]p$ should
  be read ``replace all free occurrences of $\kf{x}$ in $\kf{p}$ with
  $\kf{v}$'', it derives from work on explicit substitution from the programming
  languages community~\cite{10.5555/509043}}, will be filled with a newly
defined variable $s_{B_{T}}$ that represents the left alternative's formula.

\paragraph{Variational Models}%
\label{ssec:vmodels}
%
Classic \ac{sat} models map variables to Boolean values; variational models map
variables to variational contexts that record the variants where the variable
was assigned \tru{}. The variational context for a variable $r$ is denoted as
\vc{r}, and a variational model reserves a special variable called \SatVar{} to
track the configurations that were found satisfiable.
%
\begin{figure}[h]
  \centering
  \begin{subfigure}[t]{\textwidth}
  \begin{tabbing}
    \qquad \quad \= $\aV{} \rightarrow$ \tru{} \\
    \> $\bV{} \rightarrow$ \fls{} \\
    \> $\cV{} \rightarrow$ \tru{} \\
    \> $\pV{} \rightarrow$ \tru{} \\
    \> $\qV{} \rightarrow$ \fls{} \\
    \quad $C_{FF}$ = \{(\AV{}, \fls{}), (\BV{}, \fls{})\} \\
  \end{tabbing}
\end{subfigure}%
\begin{subfigure}[t]{\textwidth}
  \begin{tabbing}
    \qquad \quad \= $\aV{} \rightarrow$ \tru{} \\
    \> $\bV{} \rightarrow$ \fls{} \\
    \> $\cV{} \rightarrow$ \tru{} \\
    \> $\pV{} \rightarrow$ \tru{} \\
    \> $\qV{} \rightarrow$ \fls{} \\
    \quad $C_{FT}$ = \{(\AV{}, \fls{}), (\BV{}, \tru{})\} \\
  \end{tabbing}
\end{subfigure}%
\begin{subfigure}[t]{\textwidth}
  \begin{tabbing}
    \qquad \quad \= $\aV{} \rightarrow$ \tru{} \\
    \> $\bV{} \rightarrow$ \fls{} \\
    \\
    \> $\pV{} \rightarrow$ \fls{} \\
    \> $\qV{} \rightarrow$ \tru{} \\
    \quad $C_{TT}$ = \{(\AV{}, \tru{}), (\BV{}, \tru{})\} \\
  \end{tabbing}
\end{subfigure}%
  \caption{Possible plain models for variants of $\kf{f}$.}%
  \label{fig:models:plain}
\end{figure}
\begin{figure}[h]
  \centering
  \begin{subfigure}[t]{\textwidth}
  \begin{tabbing}
  \qquad \qquad \= $\_Sat \rightarrow\ (\neg \AV{} \wedge \neg \BV{}) \vee (\neg \AV{} \wedge \BV{}) \vee (\AV{} \wedge \BV)$ \\
  \> $\aV{} \rightarrow\ (\neg \AV{} \wedge \neg \BV{}) \vee (\neg \AV{} \wedge \BV{}) \vee (\AV{} \wedge \BV)$ \\
  \> $\bV{} \rightarrow\ \fls{}$ \\
  \> $\cV{} \rightarrow\ (\neg{} \AV{} \wedge{} \neg{} \BV{}) \vee{} (\neg{} \AV{} \wedge{} \BV{})$ \\
  \> $\pV{} \rightarrow\ (\neg \AV{} \wedge \neg \BV{}) \vee (\neg \AV{} \wedge \BV{})$ \\
  \> $\qV{} \rightarrow\ (\AV{} \wedge \BV)$
\end{tabbing}
\end{subfigure}

  \caption{Variational model corresponding to the plain models in
    \autoref{fig:models:plain}.}%
  \label{fig:models:var}
\end{figure}
%
As an example, consider an altered version of the query formula from the
previous section $f = ((\aV{} \wedge \neg \bV{}) \wedge \chc[A]{\aV{}
  \rightarrow \neg \pV{}, \cV{}}) \wedge ((\pV{} \wedge \neg \qV{}) \vee
\chc[B]{\qV{}, \pV{}})$. We can easily see that one variant, with configuration
$\{(\AV{},\tru{}), (\BV{},\fls{})\}$ is unsatisfiable. If the remaining variants
are satisfiable, then three models would result, as illustrated in
\autoref{fig:models:plain}; the corresponding variational model is shown in
\autoref{fig:models:var}.

We see that \Satfmf{} consists of three disjuncted terms, one for each
satisfiable variant. Variational models are flexible; a satisfiable assignment
of the query formula can be found by calling \sat{} on \Satfmf{}. Assuming the
model $C_{FT} = \{(\AV{}, \fls{}), (\BV{}, \tru{})\}$ is returned, one can find
a variable's value through substitution with the configuration; for example,
substituting the returned model on \vc{c} yields:
%
\begin{align*}
  \cV{} \rightarrow\ & (\neg \AV{} \wedge \neg \BV{}) \vee (\neg \AV{} \wedge \BV{}) & \text{\vc{} for \cV{}} \\
  \cV{} \rightarrow\ & (\neg \fls{} \wedge \neg \tru{}) \vee (\neg \fls{} \wedge \tru{}) & \text{Substitute \fls{} for \AV{}, \tru{} for \BV{}} \\
  \cV{} \rightarrow\ & \tru{} & \text{Result}
\end{align*}%
%
Furthermore, to find variants where a variable \cV{} is satisfiable reduces to
$\kf{SAT(\vc{\cV{}})}$

Variational models are constructed incrementally by merging each new plain model
returned by the solver into the variational model. A merge requires the current
configuration, the plain model, and current \vc{} of a variable. Variables are
initialized to \fls{}. For each variable $i$ in the model, if $i$'s assignment
is \tru{} in the plain model, then the configuration is translated to a
variation context and disjuncted with \vc{i}. For example, to merge the
$C_{FT}$'s plain model to the variational model in \autoref{fig:models:var},
$C_{FT}$'s configuration is converted to $\neg \AV{} \wedge \BV{}$. This clause
is disjuncted for variables assigned \tru{} in the plain model: \vc{\aV{}},
\vc{\cV{}}, and \vc{\pV{}}, even if they are new (e.g., \cV{}). Variables
assigned \fls{} are skipped, thus \vc{\qV{}} remains \fls{}. For example, in the
next model $C_{TT}$, \cV{} is \fls{} thus \vc{\cV{}} remains unaltered, while
\vc{\qV{}} flips to \tru{} hence \vc{\qV{}} records $\AV{} \wedge \BV{}$.
Variables such as \bV{}, whose \vc{}'s stay \fls{} are called \textit{constant}.


Variational models are constructed in \ac{dnf}, and form a monoid under with
$\vee$ as the semigroup operation, and \fls{} as the unit value. I take note of
this for mathematically inclined readers because it has important ramifications
for the asynchronous version of variational satisfiability solvers.

\subsection{Formalization of Variational Satisfiability Solving}
This subsection presents a selection of inference rules that specify behavior
described in the previous subsection. Many inference rules are similar to others
due to commutativity of boolean operators and thus I only present an interesting
subset. These inference rules will be significantly altered in the proposed
thesis to fully generalize to \ac{smt} problems.

\begin{figure}
  \begin{mathpar}
  %%% Computation rules
  \inferrule*[Right=Ac-Gen]
  { r \notin \kf{dom}(\Delta) \\
    \texttt{spawn($\Delta, r$)} = (\Delta', s)}
  {(\Delta, r) \mapsto (\Delta', s)}

  \inferrule*[Right=Ac-Ref]
  { \Delta(r) = s}
  {(\Delta, r) \mapsto (\Delta, s)}
  \\
  %%% Negation rules
  \inferrule*[Right=Ac-Neg]
  { \texttt{negate($\Delta, s$)} = (\Delta', s')}
  { (\Delta, \neg s) \mapsto (\Delta', s')}
\\
%   \hspace{0.7cm}
%   \inferrule*[Right=Ac-C]
%   { }
%   {(\Delta, \chc[D]{e_{1}, e_{2}}) \mapsto (\Delta, \chc[D]{e_{1}, e_{2}})}
%
% \begin{mathpar}
%   \inferrule*[Right=Ac-Neg-C]
%   { }
%   { (\Delta, \neg \chc[D]{e_{1},e_{2}}) \mapsto (\Delta, \chc[D]{\neg e_{1}, \neg e_{2}}) }
% % \end{mathpar}
%   \hspace{0.9cm}
%
% % \begin{mathpar}
%   \inferrule*[Right=Ac-Unit]
%   { }
%   { (\Delta, \bullet) \mapsto (\Delta, \unit{})}
% \end{mathpar}
%
  \inferrule*[Right=Ac-SOr]
  { \texttt{or($\Delta, s_{1}, s_{2}$)} = (\Delta', s') }
  {(\Delta, s_{1} \vee s_{2}) \mapsto (\Delta', s')}

  \inferrule*[Right=Ac-SAnd]
  { \texttt{and($\Delta, s_{1}, s_{2}$)} = (\Delta', s') }
  {(\Delta, s_{1} \wedge s_{2}) \mapsto (\Delta', s')}
\\
%
% \begin{mathpar}
%   \hspace{-1.5cm}
%   \inferrule*[Right=Ac-DM-VOr]
%   { }
%   { (\Delta, \neg (e_{1} \vee e_{2})) \mapsto (\Delta, \neg e_{1} \wedge
%     \neg e_{2}) }
% \end{mathpar}
%
% \begin{mathpar}
%   \hspace{-1.5cm}
%   \inferrule*[Right=Ac-DM-VAnd]
%   { }
%   { (\Delta, \neg (e_{1} \wedge e_{2})) \mapsto (\Delta, \neg e_{1}
%     \vee \neg e_{2}) }
% \end{mathpar}
%
%%% And with Choice rules
% \begin{mathpar}
%   \hspace{-1.5cm}
%   \inferrule*[Right=Ac-VAnd-ChcL]
%   {\Delta, e \mapsto \Delta', s}
%   { \Delta, \chc[D]{e_{1},e_{2}} \wedge e \mapsto \Delta', \chc[D]{e_{1},e_{2}} \wedge s}
% \end{mathpar}
%
% \begin{mathpar}
%   \hspace{-1.5cm}
%   \inferrule*[Right=Ac-VAnd-ChcR]
%   {\Delta, e \mapsto \Delta', s}
%   { \Delta, e \wedge \chc[D]{e_{1},e_{2}} \mapsto \Delta', s \wedge \chc[D]{e_{1},e_{2}} }
% \end{mathpar}
%
  %%% Or with Choice rules
% \begin{mathpar}
%   \hspace{-1.5cm}
%   \inferrule*[Right=Ac-VOr-ChcL]
%   {\Delta, e \mapsto \Delta', s}
%   { \Delta, \chc[D]{e_{1},e_{2}} \vee e \mapsto \Delta' \,
%     \chc[D]{e_{1},e_{2}} \vee s}
% \end{mathpar}
%
% \begin{mathpar}
%   \hspace{-1.5cm}
%   \inferrule*[Right=Ac-VOr-ChcR]
%   {\Delta, e \mapsto \Delta', s}
%   { \Delta, e \vee \chc[D]{e_{1},e_{2}} \mapsto \Delta', s \vee \chc[D]{e_{1},e_{2}} }
% \end{mathpar}
%
%%% Congruence rules with symbolic execution
% \end{mathpar}
%
% \begin{mathpar}
%   \hspace{-1.5cm}
%   \inferrule*[Right=Ac-VOr-Value]
%   {(\Delta, e_{1}) \mapsto (\Delta_{1}, s_{1}) \\ (\Delta_{1}, e_{2}) \mapsto
%     (\Delta_{2}, s_{2}) \\  (\Delta_{2}, s_{1} \vee s_{2}) \leadsto
%     (\Delta', s)}
%   { (\Delta, e_{1} \vee e_{2}) \mapsto (\Delta', s)}
% \end{mathpar}
%
% \begin{mathpar}
%   %%% Congruence rules with symbolic execution
%   \inferrule*[Right=Ac-VAnd]
%   {(\Delta, v_{1}) \mapsto (\Delta_{1}, v_{1}) \\ (\Delta_{1}, v_{2}) \mapsto
%     (\Delta', v_{2})}
%   {(\Delta, v_{1} \wedge v_{2}) \mapsto (\Delta', v_{1} \wedge v_{2})}
%
%   \inferrule*[Right=Ac-VOr]
%   {(\Delta, v_{1}) \mapsto (\Delta_{1}, v_{1}) \\
%     (\Delta_{1}, v_{2}) \mapsto (\Delta', v_{2})}
%   {(\Delta, v_{1} \vee v_{2}) \mapsto (\Delta', v_{1} \vee v_{2})}
% \end{mathpar}%
%
\end{mathpar}
%
% \begin{mathpar}
% \inferrule*[Right=Ac-And]
% {(\Delta, v_{1}) \mapsto (\Delta_{1}, v_{1}') \\ (\Delta_{1}, v_{2}) \mapsto (\Delta_{2}, v_{2}')}
% % ---------------------------------------------------------------
% { (\Delta, v_{1} \wedge v_{2}) \mapsto (\Delta_{2}, v_{1}' \wedge v_{2}') }
% \end{mathpar}
% %
% \begin{mathpar}
% \inferrule*[Right=Ac-Or]
% {(\Delta, v_{1}) \mapsto (\Delta_{1}, v_{1}') \\ (\Delta_{1}, v_{2}) \mapsto (\Delta_{2}, v_{2}')}
% % ---------------------------------------------------------------
% { (\Delta, v_{1} \vee v_{2}) \mapsto (\Delta_{2}, v_{1}' \vee v_{2}') }

% \end{mathpar}

  \caption{Selected accumulation semantics on IL formulas.}%
  \label{impl:accum}
\end{figure}

Accumulation is defined in \autoref{impl:accum} as a relation of the form
$(\Delta,v)\mapsto(\Delta,v)$, where $\Delta$ is a symbolic store, and $v$ is
the syntactic category representing the set of all possible IL formulas; the
tuples on the left and right are interpreted as input and output, respectively.
Accumulation interacts with the symbolic execution engine via primitive
operations represented in \texttt{typewriter} font.
%
The \rn{Acc-Gen} rule generates new symbolic references using \texttt{spawn},
which are looked up by \rn{Acc-Ref}. The \rn{Acc-And} and \rn{Acc-Or} rules
reduce symbolic sub-formulas to a new symbolic reference.
%
This selection of rules do the work of reducing formulas to symbolic references.
The remaining rules simply push negation down expressions and propagate
accumulation over the $\wedge$ and $\vee$-connectives.

%
\begin{figure}
  \begin{mathpar}
  %%% Computation rules
  \inferrule*[Right=Ev-Term]
  { \texttt{assert($(\Gamma, \Delta), t$)} = \Gamma' }
  {(\Gamma, \Delta, t) \rightarrowtail (\Gamma', \Delta, \unit{}) }

  \inferrule*[Right=Ev-Sym]
  { \texttt{assert($(\Gamma, \Delta), s$)} = \Gamma'}
  { (\Gamma, \Delta, s) \rightarrowtail (\Gamma', \Delta, \unit{}) }
\end{mathpar}
% \inferrule*[Right=Ev-Chc]
%   { }
%   {(\Theta, \chc[D]{e_{1}, e_{2}}) \rightarrowtail (\Theta, \chc[D]{e_{1}, e_{2}})}
%
%   \hspace{0.5cm}
%   \inferrule*[Right=Ev-Unit]
%   { }
%   { (\Theta,\unit{}) \rightarrowtail (\Theta, \unit{}) }

  \begin{mathpar}
  %%%\unit{} elimination rules
  \inferrule*[Right=Ev-UL]
  {  }
  { (\Theta,\unit{} \wedge v) \rightarrowtail (\Theta, v) }

  \inferrule*[Right=Ev-UR]
  { }
  { (\Theta, v \wedge\unit{}) \rightarrowtail (\Theta, v)}
\end{mathpar}
\begin{mathpar}
%   \inferrule*[Right=Ev-DM-VOr]
%   { (\Theta ,\neg e_{1} \wedge \neg e_{2}) \rightarrowtail (\Theta', v)}
%   { (\Theta, \neg (e_{1} \vee e_{2})) \rightarrowtail (\Theta', v)}
%
%   \hspace{1cm}
%   \inferrule*[Right=Ev-DM-VAnd]
%   { (\Theta, \neg e_{1} \vee \neg e_{2}) \rightarrowtail (\Theta', v)}
%   { (\Theta, \neg (e_{1} \wedge e_{2})) \rightarrowtail (\Theta', v)}
%
  %%% Negation rules
  % \inferrule*[Right=Ev-NegS]
  % { (\Delta, \neg s) \mapsto (\Delta', s') \\
  %   ((\Gamma, \Delta'), s') \rightarrowtail (\Theta, \unit{})
  % }
  % { ((\Gamma, \Delta), \neg s) \rightarrowtail (\Theta, \unit{})}
%   \inferrule*[Right=Ev-Neg-Chc]
%   { }
%   { (\Theta, \neg \chc[D]{e_{1},e_{2}}) \rightarrowtail (\Theta, \chc[D]{\neg e_{1}, \neg e_{2}}) }
%
  \inferrule*[Right=Ev-Neg]
  { (\Delta,\neg v) \mapsto (\Delta', v')
    % ((\Gamma, \Delta') ,\neg v) \rightarrowtail (\Theta, v')
  }
  { (\Gamma, \Delta, \neg v) \rightarrowtail (\Gamma, \Delta', v') }
\end{mathpar}
  %%% And with Choice rules
% \begin{mathpar}
%   \hspace{-1.5cm}
%   \inferrule*[Right=Ev-And-ChcL]
%   { \Gamma, \Delta,e \rightarrowtail \Gamma', \Delta',e'}
%   { \Gamma, \Delta,\chc[D]{e_{1},e_{2}} \wedge e \rightarrowtail \Gamma',
%     \Delta',\chc[D]{e_{1},e_{2}} \wedge e' }
% \end{mathpar}
% \begin{mathpar}
%   \hspace{-1.5cm}
%   \inferrule*[Right=Ev-And-ChcR]
%   { \Gamma, \Delta,e \rightarrowtail \Gamma', \Delta',e'}
%   { \Gamma, \Delta,e \wedge \chc[D]{e_{1},e_{2}} \rightarrowtail \Gamma',
%     \Delta',e' \wedge \chc[D]{e_{1},e_{2}} }
%   % {l \rightarrowtail l'}
%   % { l \wedge \chc[D]{e_{1},e_{2}} \rightarrowtail l' \wedge \chc[D]{e_{1},e_{2}}}
% \end{mathpar}
%   %%% Or with Choice rules
% \begin{mathpar}
%   \hspace{-1.5cm}
%   \inferrule*[Right=Ev-Or-ChcL]
%   {\Delta,e \mapsto \Delta',v}
%   { \Gamma, \Delta,\chc[D]{e_{1},e_{2}} \vee e \rightarrowtail \Gamma,
%     \Delta',\chc[D]{e_{1},e_{2}} \vee v}
% \end{mathpar}
% \begin{mathpar}
%   \hspace{-1.5cm}
%   \inferrule*[Right=Ev-Or-ChcR]
%   {\Delta,e \mapsto \Delta',v}
%   { \Gamma, \Delta,e \vee \chc[D]{e_{1},e_{2}} \rightarrowtail \Gamma,
%     \Delta',v \vee \chc[D]{e_{1},e_{2}} }
% \end{mathpar}
% \begin{mathpar% }
%   \inferrule*[Right=Ev-Or-Value]
%   {(\Delta, e_{1}) \mapsto (\Delta_{1}, s_{1}) \\
%     (\Delta_{1}, e_{2}) \mapsto (\Delta_{2}, s_{2}) \\
%     (\Delta_{2}, s_{1} \vee s_{2}) \mapsto (\Delta_{s}, s) \\
%     ((\Gamma, \Delta_{s}), s) \twoheadrightarrow (\Theta,\unit{})
%   }
%   { ((\Gamma, \Delta),e_{1} \vee e_{2}) \rightarrowtail (\Theta,\unit{})}
% \end{mathpar}
  %%% Congruence rules with symbolic execution
\begin{mathpar}
  \inferrule*[Right=Ev-Or]
  { (\Delta, v_{1}) \mapsto (\Delta', v_{1}') \\
    (\Delta',v_{2}) \mapsto (\Delta'', v_{2}')}
  { (\Gamma, \Delta, v_{1} \vee v_{2}) \rightarrowtail (\Gamma, \Delta'', v_{1}' \vee v_{2}')}
\end{mathpar}%
\begin{mathpar}
  \inferrule*[Right=Ev-And]
  {(\Theta, v_{1}) \rightarrowtail (\Theta', v_{1}') \\
    (\Theta', v_{2}) \rightarrowtail (\Theta'', v_{2}')
  }
  { (\Theta, v_{1} \wedge v_{2}) \rightarrowtail (\Theta'', v_{1}' \wedge v_{2}')
  }
\end{mathpar}%

  \caption{Selected evaluation semantics over \vpl{} formulas.}%
  \label{impl:eval}
\end{figure}

Evaluation is defined in \autoref{impl:eval} as a relation of the form
$(\Theta,v)\rightarrowtail(\Theta,v)$, where $\Theta=(\Gamma,\Delta)$ and
$\Gamma$ represents the base solver state.
%
As before, we show only a significant subset of the rules here. The rules
\rn{Ev-Term} and \rn{Ev-Sym} push new clauses to the base solver using the
primitive \texttt{assert} operation. The \rn{Ev-UL} and \rn{Ev-UR} implement
left and right unit, reducing conjunctions where one side has been processed by
the base solver.
%
Of special note is the difference between the \rn{Ev-Or} and \rn{Ev-And} rules.
While \rn{Ev-And} is a straightforward congruence rule, \rn{Ev-Or} instead
processes its arguments using accumulation ($\mapsto$). Disjunctions are a
source of back-tracking in variational solving, and thus the solver cannot
evaluate the left-hand side without evaluating the right, both of which may
contain choices, hence evaluation must switch to accumulation, as we informally
described in the previous subsection.

\begin{figure}
  % \begin{tabbing}
%   % \begin{align*}
%   {\sc CoreChoices}$\ :\ C \rightarrow vm \rightarrow ev \rightarrow vm$ \\
%   {\sc CoreChoices}$\ C\ mdls\ v$\\
%   \qquad case $v$ of \\
%   \qquad \qquad \= Unit \qquad \qquad \= = return \mdls{} \\
%   \> \chc[d]{e,e'} \> = CoreChoicesHelper C \dimd{} \mdls{} (Evaluate e) (Evaluate e') \\
%   \> \chc[d]{e,e'} $\wedge\ ev\ $ \> = do \\
%   \> \> \quad \=  $vE \leftarrow$ Evaluate $e$ \\
%   \> \> \> $vE' \leftarrow$ Evaluate $e'$ \\
%   \> \> \> CoreChoicesHelper C \dimd{} \mdls{} ($vE \wedge\ ev$) ($vE' \wedge ev$) \\
%   \> $ev\ \wedge\ $\chc[d]{e,e'} \> = do \\
%   \> \> \quad \=  $vE \leftarrow$ Evaluate $e$ \\
%   \> \> \> $vE' \leftarrow$ Evaluate $e'$ \\
%   \> \> \> CoreChoicesHelper C \dimd{} \mdls{} ($ev \wedge\ vE$) ($ev \wedge eV'$) \\
%   \> \chc[d]{e,e'} $\vee\ ev\ $ \> = do \\
%   \> \> \quad \=  $vE \leftarrow$ Evaluate $e$ \\
%   \> \> \> $vE' \leftarrow$ Evaluate $e'$ \\
%   \> \> \> CoreChoicesHelper C \dimd{} \mdls{} ($vE \vee\ ev$) ($vE' \vee ev$) \\
%   \> $ev\ \vee\ $\chc[d]{e,e'} \> = do \\
%   \> \> \quad \=  $vE \leftarrow$ Evaluate $e$ \\
%   \> \> \> $vE' \leftarrow$ Evaluate $e'$ \\
%   \> \> \> CoreChoicesHelper C \dimd{} \mdls{} ($ev \vee\ vE$) ($ev \vee eV'$) \\
%   \> $ev$ \> = do \\
%   \> \> \quad \= $vE \leftarrow$ FindPChoice (Evaluate $ev$) \\
%   \> \> \> solveChoices C \dimd{} \mdls{} ($vE$)  \\
% \end{tabbing}
\begin{mathpar}
  %%% Computation rules
  \inferrule*[Right=Gen]
  { \texttt{getModel($\Gamma, \Delta$)} = m' }
  { (C, \Theta, m, \unit{}) \Downarrow_{i} (C, \Theta, \oplus(C, m', m), \unit{}) }
% \end{mathpar}
\hspace{0.5cm}
% \begin{mathpar}
  \inferrule*[Right=Sym]
  { (\Theta, s) \rightarrowtail (\Theta', \unit{})}
  { (C, \Theta, m, s) \Downarrow_{i} (C, \Theta', m, \unit{})
  }
\end{mathpar}
%
\begin{mathpar}
% \inferrule*[Right=Cr-And]
% { (\Phi, v_{1}) \Downarrow_{i} (\Phi_{1}, v_{1}') \\
%     (\Phi_1, v_{2}) \Downarrow_{1} (\Phi', v_{2}') \\
%   }
%   { (\Phi, v_{1} \wedge v_{2}) \Downarrow_{i} (\Phi', v_{1}' \wedge v_{2}')
%   }
% \\%
%   \hspace{0.5cm}
  %% solve a symbolic reference execute
  \inferrule*[Right=Cr-Or]
  { (\Delta, v_{1} \vee v_{2}) \mapsto (\Delta', v)}
  { (C, \Gamma, \Delta, m, v_{1} \vee v_{2}) \Downarrow_{i} (C, \Gamma, \Delta',
    m, v) }
\\%
\inferrule*[Right=Cr-And-T]
{
  (D, \true) \in C \\
  (C, \Theta, m, v \wedge \texttt{toIR$(e_{1})$}) \Downarrow_{i} (C, \Theta, m, v')
}
{
  (C, \Theta, m, v \wedge \chc[D]{e_{1}, e_{2}}) \Downarrow_{i} (C, \Theta, m, v')
}

  % \inferrule*[Right=Cr-And-F]
  % {
  %   C\ d = False \\
  %   ((\Gamma, \Delta), e_{2}) \rightarrowtail ((\Gamma_{2}, \Delta_{2}), v_{2}) \\
  %   ((C, \Gamma_{2}, \Delta_{2}), v \wedge v_{2}) \Downarrow_{i} m
  % }
  % {
  %   ((C, \Gamma, \Delta), v \wedge \chc[d]{e_{1}, e_{2}}) \Downarrow_{i} m
  % }
%
  % \inferrule*[Right=Cr-Or-T]
  % {
  %   C(d) = True \\
  %   ((\Gamma, \Delta), e_{1}) \mapsto ((\Gamma_{1}, \Delta_{1}), v_{1}) \\
  %   ((C, \Gamma_{1}, \Delta_{1}), v \vee v_{1}) \Downarrow_{i} m
  % }
  % {
  %   ((C, \Gamma, \Delta), v \vee \chc[d]{e_{1}, e_{2}}) \Downarrow_{i} m
  % }
  %
\inferrule*[Right=Cr-Or-T]
{
  (D, \true) \in C \\
  (C, \Theta, m, v \vee \texttt{toIR$(e_{1})$}) \Downarrow_{i} (C, \Theta, m, v')
}
{
  (C, \Theta, m, v \vee \chc[D]{e_{1}, e_{2}}) \Downarrow_{i} (C, \Theta, m, v')
}
  % \inferrule*[Right=Cr-Or-F]
  % {
  %   C(D) = false \\
  %   ((C, \Gamma, \Delta), v \vee \texttt{toIR$(e_{2})$}) \Downarrow_{i} m
  % }
  % {
  %   ((C, \Gamma, \Delta), v \vee \chc[D]{e_{1}, e_{2}}) \Downarrow_{i} m
  % }
\\
%   \inferrule*[Right=Cr-COr]
%   {
%     \hspace{1.3cm}
%     ((C \cup \{(D, \kernfix{true})\}, \Gamma, \Delta), v \vee \chc[D]{e_{1}, e_{2}}) \Downarrow_{i+1} m_{1} \\
%     D \notin C \\
%     ((C \cup \{(D, \kernfix{false})\}, \Gamma, \Delta), v \vee \chc[D]{e_{1}, e_{2}}) \Downarrow_{i+1} m_{2} \\
%   }
%   {
%     ((C, \Gamma, \Delta), v \vee \chc[D]{e_{1}, e_{2}}) \Downarrow_{i} m_{1}
%     \oplus m_{2}
%   }
% \\
  \inferrule*[Right=Cr-CAnd]
  {
    % \hspace{1.5cm}
    D \notin C \\
    % \hspace{-0.5cm}
    (C \cup \{(D, \true)\}, \Theta, m,  v \wedge \chc[D]{e_{1}, e_{2}})
    \Downarrow_{i+1} (C', \Theta', m',  \unit{}) \\
    \hspace{2.25cm}
    (C \cup \{(D, \false)\}, \Theta, m,  v \wedge \chc[D]{e_{1}, e_{2}}) \Downarrow_{i+1} (C'', \Theta'', m'',  \unit{})\\
  }
  {
    (C, \Theta, m, v \wedge \chc[D]{e_{1}, e_{2}}) \Downarrow_{i} (C'',
    \Theta'', m' \cup m'',  \unit{})
  }
%
\end{mathpar}

  \caption{Selected variational solving semantics on cores.}%
  \label{impl:choice-eval}
\end{figure}

Solving the core is defined in \autoref{impl:choice-eval}, as a relation
%
$(C, \Gamma, \Delta, m,v) \Downarrow_{i} (C, \Gamma, \Delta, m, v)$, where $C$,
is a set which represents the configuration of the \vpl{} formula, and $m$
represents the variational model, which is initialized as empty.
%
The count of \texttt{push}'s on the assertion stack are represented with the
counter $i$. The solving process reifies choices by manipulating the
configuration and uses accumulation and evaluation to process terms. The \vc{}
input to the solver pre-populates the configuration, thereby restricting the
solver to a subset of variants. When no \vc{} is input, the configuration is
initialized as empty. \rn{Cr-CAnd} processes novel choices by manipulating the
configuration and performing a \texttt{push} in the base solver; resulting
variational models are merged via an element-wise $\vee$, shown as $\cup$.
Choices are removed through \rn{Cr-And-T} and \rn{Cr-Or-T} by selection on the
alternative. Once the choice is removed, the nested clauses are translated to
the intermediate language through the \texttt{toIL} primitive and processed by
accumulation and evaluation. A model is called from the base solver with
\rn{Gen}, once the core, and thus query formula is reduced to \unit{}. Note
again, \rn{Cr-Or} switches to accumulation to ensure sound results, we have
omitted congruence rules such as \rn{Cr-And}. Similarly, we omit rules which are
commutative versions of those shown here, namely: rules which process the left
branch of connectives, rules which select the $\kf{false}$ alternatives,
and the \rn{Or} version of \rn{Cr-CAnd}.

\subsection{Research Plan}

\paragraph{Estimating the difficulty of finding satisfiability} This item will
produce a method to determine the \emph{hardness} of solving a \ac{vpl} formula
for satisfiability. A well known result in the random-\ac{sat} community is the
phenomenon of a \emph{phase transition}~\cite{Gent94thesat} in randomly
generated \ac{sat} problems. The phase transition of \ac{sat} problems is an
inflection point in the probability of finding a satisfying assignment as the
ratio of clauses to variables is varied. Conceptually, one may think of the
phase transition as a method to estimate the difficultly in solving a \ac{sat}
problem. If there are many variables relative to clauses, then the \ac{sat}
problem is likely easy to solve as it is under-constrained, in contrast, if
there are too few variables relative to clauses then it is over-constrained and
thus easy to compute as unsatisfiable.

Difficult problems are balanced with respect to the clause variable ratio and
thus are at the phase transition point of a high probability of finding
satisfiability to finding unsatisfiability. Estimating the difficulty of a
\ac{sat} problem is thus theoretically useful to isolate sets of problems to
study and progress the state of the art, but also practically useful, because an
end-user may estimate the difficulty of a problem and choose to \emph{not} solve
it.

This item will replicate the analysis from the random-\ac{sat} community to
determine if the phase transition exists for variational satisfiability
problems. It is clear that the phase transition exists for variants, but having
a method to assess the phase transition point \emph{in terms of} \ac{vpl}
formulas would provide the aforementioned benefits for variational
satisfiability solvers.

%%% Local Variables:
%%% mode: latex
%%% TeX-master: "../main"
%%% End:

\section{Proposal Contribution 3: Variational Satisfiable-Modulo Theory Solving}
~\label{sec:vsmt} The final contribution to the thesis is generalizing the
prototype variational solver to solve \ac{smt} problems. Like all objects that
require craft, the architecture and design benefit greatly from lessons learned
in the first prototype, I cover these advances below and conclude the section
with a discussion of remaining work.

\subsection{Motivation}
This section discusses the motivation for a variational \ac{smt} solver.
Specifically, for a variational \ac{smt} solver that \emph{does not} use the
theory interface of \ac{smt} solvers to reason about variation. Satisfiable
modulo theory solvers, by virtue of abstracting satisfiability solving over
background theories, are useful in multitudes of domains including:
verification~\cite{boogiepl-a-typed-procedural-language-for-checking-object-oriented-programs},
test generation and bug
finding\cite{Cadar:2008:KUA:1855741.1855756,Godefroid:2012:SWF:2090147.2094081},
planning or scheduling
applications\cite{10.1109/RTSS.2010.25,10.1145/2038642.2038689}, interactive
proof
assistants~\cite{10.1007/978-3-540-78800-3_24,10.1007/978-3-319-08867-9_49}, and
many more~\cite{10.5555/1391237}.

The motivation in creating a variational \ac{smt} solver is identical to the
motivation for a variational \ac{sat} solvers for a subset of theories \ie{},
the incremental interface is automated, the user need not hand-program the
solver, performance benefits are now possible by virtue of a static explicit
encoding. Furthermore, \ac{smt} solvers provide a good platform for research on
variation. Understanding and implementing variational effects and effect
handlers is an active and unsolved area of research on variation and variational
programming. Essentially, the problem is soundly tracking side-effects such as
file I/O or state mutation in the context of variation.

Variational \ac{smt} solvers side step this issue by building upon a ground
theory of \emph{uninterpreted functions}. Uninterpreted functions are functions
that have no apriori meaning, such as a function which defines a constant value,
as opposed to a function like + which apriori means to add two integers. Thus
functions in the \ac{smt} domain, including those in background theorys are
total and side-effect free, making \ac{smt} solvers an attractive target for
variational research.

Besides the motivation deriving from variational research and inherited from the
variational \ac{sat} solver, a variational \ac{smt} solver is desirable because
it simplifies the use of a \ac{smt} solvers in real-world applications.
%
For example, consider the following snippet of a C-like language that uses
contract-based verification to ensure correctness:
%
\begin{lstlisting}[columns=flexible,keepspaces=true]
@precondition: $x_{in} > y_{in}$
void swap(int x, int y) {
  x := x + y;
  y := x - y;
  x := x - y;
}
\end{lstlisting}
%
With a plain \ac{smt} solver one can prove that this code does indeed swap the
variables by constructing an \ac{smt} problem that includes the following
constraint $x_{out} = y_{in} \wedge y_{out} = x_{in}$, in addition to encoding
the precondition, any constraints derived from the function body, and
constraints derived from any post condition. However, in practice, code bases
are variational artifacts, either through explicit variation annotations such as
C preprocessor \texttt{\#ifdef}s or through implicit practices such as branching
and forking in version control systems. For example, in addition to the snippet
above, one might have another variant that requires verification, perhaps to
prevent a bug or as a safety check after a refactor:
%
\begin{lstlisting}[columns=flexible,keepspaces=true]
@precondition: $x_{in} > y_{in}$
@precondition: $x_{in} > -1$     // new
int swap(int x, int y) {
  x := x + y;
  y := x - y;
  x := x - y;
  return 1;                // new
}
\end{lstlisting}
%
In this variant, we see two minor additions; an additional precondition, $x_{in}
> -1$, and a return value, note that most of the code has not changed. With a
variational \ac{smt} solver one could use choices to express this difference and
verify \emph{both} variants instead of each variant at a time. Thus, a
variational \ac{smt} solver, by virtue of being variation-aware, would allow
verification tools to directly express variation using choices and verify the
entire code base or software system, rather than each of its' variants one at a
time.

Verifying the entire code base is possible using plain \ac{smt} solvers but the
situation is analogous to the difference between a programming language which
has the concept of looping, compared to a language that is not
\emph{loop-aware}. One might still express loops in the latter language, \eg{},
with \texttt{Goto} or \texttt{Gosub} primitives, but doing so is more
error-prone, and more difficult than using a construct such as \texttt{While}
that encapsulates and expresses the concept of looping.

One approach to construct a variational \ac{smt} solver is to add a background
\emph{theory of variation} to a plain \ac{smt} solver. This approach has many
desirable properties; the \vc{} that determines the variants of interest could
be expressed in the solver, the solver would enforce the synchronization of
choices, new theories could be supported, and the difference between a
variational \ac{sat} solver and variational \ac{smt} solver would only be the
inclusion of other theories in a problem.

Unfortunately, there are several issues which make this approach undesirable.
First, it is overly solver specific. Some solvers such as openSMT~\cite{openSMT}
are architected specifically so the end-user can include custom theories, other
solver such a yices\cite{10.1007/978-3-319-08867-9_49},
cvc4\cite{10.1007/978-3-642-22110-1_14} and
z3~\cite{10.1007/978-3-540-78800-3_24} have varying degrees of support. Yices
and z3 provide no support\footnote{z3 had plugin support for custom theories but
  this feature was removed because model construction became problematic. The z3
  developers now suggest treating z3 as a black box and building around it, just
  as this thesis proposes. See:
  https://stackoverflow.com/questions/46508907/smt-solver-with-custom-theories},
while cvc4 includes an API for custom theories but with incomplete documention.
Second, adding a theory of variation to the solver limits the asynchronous
implementation to the asynchronous attributes of the \ac{smt} solver. Lastly,
the interaction between plain and the hypothetical variational theory is not
clear. The interaction between combination of theories in \ac{smt} solvers is an
active area of research~\cite{10.5555/1550723}. By including a theory of
variation, which is a \emph{meta-theory}, \ie{}, a theory which operates on
other theories to construct variation-aware theories, these problems are
exacerbated. Thus, while it is possible to construct a theory of variation, we
leave this to future research. Instead, we choose to create the variational
prototype as a proof of concept variational \ac{smt} solver that uses a plain
\ac{smt} solver as a black-box.


\subsection{\ac{vpl} Extensions}
Extending the variational solver for \ac{smt} background theories requires
non-trivial extensions to variational propositional logic, and consequently the
intermediate language the solver operates upon. In \ac{smt} solving, Boolean
values correspond to constraints over individual variables which range over
different domains, such as arrays, arithmetic, bitvectors or strings. To support
\ac{smt} theories the variational \ac{smt} solver must be able to abstract these
theories and reason about them in a variational context. We show a simple
extension of \ac{vpl} to include integer arithmetic, and conclude the section by
extending the variational \ac{smt} solver with an array theory.

To begin, we require formalizations of \ac{smt} background theories, for our
purposes, we'll represent any \ac{smt} theory as a 2-tuple, consisting of a
formal grammar \G{}, and a semantic function with type $\sem{\cdot} : \G{}
\rightarrow \mathbb{B}$\footnote{This formulation follows conventions from the
  programming language community. We choose this formulation to build upon the
  notation and background in \autoref{sec:vpl}. The \ac{sat}/\ac{smt} community
  would represent this a set called a \emph{signature} that consists of words in
  the theory called \emph{atoms}, functions which operate on the words, such as
  $+$ and $<$ and functions which maps sentences to Booleans called
  \emph{predicates}}. For the remainder of the proposal we represent all
grammars in Backus-Naur form. For example, consider a simple background theory
of integer addition, subtraction and two inequalities:
%
\begin{syntax}
  i & \in{} & \mathbb{Z}
  & \textit{Integer literals} \\[1.5ex]

  a & \Coloneqq{} & i    & \textit{Terminal} \\
  & | & a + a     & \textit{Addition} \\
  & | & a -{} a     & \textit{Subtraction} \\
  & | & a < a     & \textit{Less Than} \\
  & | & a > a   & \textit{Greater Than} \\
  % & | & \chc[D]{a,a} & \textit{Choice} \\
\end{syntax}
%
Note that with this formulation, the semantic function is partial, the theory
should only allow syntax trees that have inequalities at the root such that a
Boolean is the only type of value that can result. With the semantic function
and grammar, the integer theory can be integrated into \ac{vpl}. For example,
one can imagine the following sound formula: $\aV{} \wedge \neg \sem{((1 + 5) <
  1729)} \vee \cV{}$

With the definition of an \ac{smt} background theory we define a
\emph{variation-aware} background theory as a 2-tuple that consists of a
variation-aware grammar, $\G{}^{\chcL\chcR}$, and a variation-aware semantic
function $\sem{\cdot} : C \rightarrow \G{} \rightarrow \mathbb{B}$. The
alterations to the semantic function are minimal, requiring a configuration as
the extra input to track choices. The semantics follow from the semantic
function described in \autoref{fig:cc:cfg} only distributing over integer
connectives instead of logical connectives such as $\wedge$ and $\vee$.
Converting a grammar to a variation-aware grammar depends on the grammar at
hand. In this case, the theory is a context free grammar and thus the only
difference is adding choices as a recursive case:
%
\begin{syntax}
  i & \in{} & \mathbb{Z}
  & \textit{Integer literals} \\[1.5ex]

  \lift{a} & \Coloneqq{} & i    & \textit{Terminal} \\
  & | & \chc[D]{\lift{a},\lift{a}} & \textit{\textbf{Choice}} \\
  & | & \lift{a} + \lift{a}     & \textit{Addition} \\
  & | & \lift{a} - \lift{a}     & \textit{Subtraction} \\
  & | & \lift{a} < \lift{a}     & \textit{Less Than} \\
  & | & \lift{a} > \lift{a}   & \textit{Greater Than} \\
\end{syntax}
%
Note that we could define the variation-aware theory only with a variation-aware
domain $\lift{\iV{}}$ which would add choices to the set of integers. Doing so
would allow the variation-aware grammar to express expressions such as
$\chc[A]{1, 2} + 4$, \emph{but not} $\chc[A]{10 + \chc[A]{2,3}, 3} + 2$ because
choices would only be allowed to range over integers and not expression. We'll
use this behavior in the following array example.

More complicated theories, such as
arrays~\cite{demoura2009generalized,Mccarthy62towardsa} require more careful
handling. The array theory parameterizes an array with a type to determine the
type of the array's elements, and includes only two functions: $\kf{select :
  Array\ \nats\ X \rightarrow \nats \rightarrow X}$, which given an array and a
natural number index, creates a constraint that an element $x \in X$, is at
index $n \in \nats{}$, in the input array. \newline Similarly, $\kf{store :
  Array\ \nats\ X \rightarrow \nats \rightarrow X \rightarrow Array\ \nats\ X}$
constructs a constraint that at index $n \in \nats{}$, the input array contains
value $\kf{x \in X}$. In SMTLIB2, these constraints obey the following law
$\forall a\ \in \kf{Array\ \nats\ X},\ \forall i\ \in \nats,\ e \in X,\ \kf{(=\
  (select\ (store\ a\ i~e)\ i)\ e)}$. A simple formulation then for an array
theory is:
%
\begin{syntax}
  a & \in & Array\ \nats{}\ X & \textit{all possible arrays} \\
  i & \in & \nats & \textit{Natural Numbers} \\
  x & \in & X & \textit{set of elements} \\[1.5ex]

  arr & \Coloneqq & select\ a\ i    & \textit{Selection} \\
  % & | & \chc[D]{\lift{a},\lift{a}} & \textit{Choice} \\
  & | & store\ a\ i\ x & \textit{Storage} \\
  % & | & \lift{a} + \lift{a}     & \textit{Addition} \\
\end{syntax}
%
The semantic function for this grammar is stateful, such that it can track the
array and its constraints. The prototype \ac{smt} solver offloads this work to
the underlying incremental solver and instead places holes in the array
constraints.

We present a variation-aware grammar, \lift{arr}, which maintains the same
interface, \ie{}, \emph{store} and \emph{select}, but operates on variation-aware
domains such as \lift{\nats}, \lift{X}:
%
\begin{syntax}
  a & \in{} & \lift{(Array\ \nats{}\ X)} & \textit{choice of all possible arrays} \\
  i & \in{} & \lift{\nats} & \textit{choice of Natural Numbers} \\
  x & \in{} & \lift{X} & \textit{choice of elements} \\[1.5ex]

  \lift{arr} & \Coloneqq{} & select\ a\ i    & \textit{Selection} \\
  % & | & \chc[D]{\lift{a},\lift{a}} & \textit{Choice} \\
    & | & store\ a\ i\ x & \textit{Storage} \\
    % & | & \lift{a} + \lift{a}     & \textit{Addition} \\
\end{syntax}
%
This is a design decision, one could easily add an array theory which allows for
a choice of \emph{select} or \emph{store}, similar to \lift{a}, in addition to
including variation-aware domains. Furthermore, we could restrict the language
by choosing to only use a variation-aware element domain \lift{X}, which would
yield an array of choices, or only a variation-aware domain of arrays,
\lift{(Array\ \nats{}\ X)}, yielding a choice of arrays. Both are possible with
this formulation, for example, consider the following variational \ac{smt}
program:
%
\begin{lstlisting}[columns=flexible,keepspaces=true]
(declare-const e1 Int)
(declare-const e2 Int)
(declare-const a1 (Array Int Int))                ;; an array of integers
(declare-const a2 (Array Int Int))                ;; second array of integers
(assert (= (store a2      3 A$\chcL$1202,2718$\chcR$) a2))     ;; an array of choices
(assert (= (store A$\chcL$a1,a2$\chcR$ 3 1729       ) a1))     ;; a choice of arrays
(assert (= (select A$\chcL$a1,a2$\chcR$ 3) e1))
(assert (= (select a2      3) e2))
(check-sat)
(get-model)
\end{lstlisting}
%
We see that there are two integer arrays, $\kf{a1}$ and $\kf{a2}$, and three
choices: one choice which chooses an array in \emph{store}, another which
chooses an element to store in \emph{store} in $\kf{a2}$, and a third choice to
determine which array $\kf{e1}$ retrieves its value from. All choices are
parameterized by the dimension $\kf{A}$ yielding two variants, and results are
returned in the $\kf{e1}$ and $\kf{e2}$ variables. The variational core for this
program would simply replace the choices with holes:
%
\begin{lstlisting}[columns=flexible,keepspaces=true]
 $\vdots$
(assert (= (store $\Diamond_{A}$ 3 1729) a1))  ;; a choice of arrays
(assert (= (store a2 3 $\Diamond_{A}$)   a2))  ;; an array of choices
(assert (= (select $\Diamond_{A}$ 3) e1))
 $\vdots$
\end{lstlisting}
%
To solve such a program, the variational \ac{smt} solver will compile to SMTLIB2
wrapping variation-aware statements and statements affected by variation-aware
statements, such as \texttt{(get-model)}, with a \texttt{push} and \texttt{pop}
instruction:
\begin{lstlisting}[columns=flexible,keepspaces=true]
(declare-const e1 Int)
(declare-const e2 Int)
(declare-const a1 (Array Int Int))      ;; an array of integers
(declare-const a2 (Array Int Int))      ;; second array of integers
(push)                                  ;; a configuration on A has occurred
(assert (= (store a2 3 1202) a2))
(assert (= (store a1 3 1729) a1))
(assert (= (select a1 3) e1))           ;; e1 set to 1729
(assert (= (select a2 3) e2))           ;; e2 set to 1202
(check-sat)
(get-model)
(pop)
(push)                                  ;; Right alternative of A
(assert (= (store a2 3 2718) a2))
(assert (= (store a2 3 1729) a1))       ;; a1 unifies with a2
(assert (= (select a1 3) e1))           ;; e1 set to 1729
(assert (= (select a2 3) e2))           ;; e2 set to 2718
(check-sat)
(get-model)
(pop)
\end{lstlisting}

\subsection{Variational \ac{smt} Models}
To support \ac{smt} theories, variational models must be abstract enough to
handle values other than Booleans. Functionally, variational \ac{smt} models
must satisfy several constraints: the variational \ac{smt} model must be more
memory efficient than storing all models returned by the solver naively. The
varational \ac{smt} model must allow users to find satisfying values for a
variant. The model must allow users to find all variants at which a variable has
a particular value or range of values. Furthermore, several useful properties of
varational models, as presented in \autoref{ssec:vmodels}, should be maintained:
The model is non-variational; hence the user does not need to understand the
choice calculus in order to understand their results. The model produces results
that can be fed into a plain \ac{sat} solver. The model can be built
incrementally and without regard to the ordering of results because it forms a
commutative monoid under $\{\fls{}, \vee\}$. The model maps variables to a context free
grammar and can thus be parsed quickly\footnote{The use of ``quickly'' here
  means linear or quadratic time using context-free grammar parser such as an
  Earley parser~\cite{10.1145/362007.362035}}.

To maintain these properties and satisfy the functional requirements, our
strategy for variational \ac{smt} models is to create a mapping of variables to
\ac{smt} expressions. By virtue of this strategy, variables are disallowed from
changing types across the set of variants and hence disallowed from changing
types as the result of a choice in the variational model. For any variable in
the model, we assume the type returned by the base solver is correct, and store
the satisfying value in a linked list constructed \emph{if-statements}.
Specifically, we use the function $\kf{ite} : \mathbb{B} \rightarrow T
\rightarrow T$ as the \texttt{cons} operation to build the list. $\kf{ite}$ is
defined in the ground theory of Booleans as defined in the SMTLIB2 standard. All
variables are initialized as \texttt{undefined} until a value is found in a
variant. To ensure the correct value of a variable corresponds to the
appropriate variant, we translate the configuration that determines the variant
to a variation context, and place the appropriate value in the \emph{then}
branch, with the else branch linking to the previous expression.

Consider the following variational \ac{smt} problem extended with an integer
arithmetic theory: $f = (\chc[A]{\iV{}, 13} - \cV{} < \bV + 10) \rightarrow
\chc[B]{\aV{}, \cV{} > \iV{}}$. \fV{} contains two unique choices, $\kf{A}$,
$\kf{B}$, and thus represents four variants. In this case, the expression is
under-constrained and so each variant will be found satisfiable.
%
\begin{figure}[h]
  \centering
  \begin{subfigure}[t]{\textwidth}
  \begin{tabbing}
    \qquad \quad \= $\iV{} \rightarrow$ -1 \\
    \> $\cV{} \rightarrow$ 0 \\
    \\
    \\
    \quad $C_{FF}$ = \{(\AV{}, \fls{}), (\BV{}, \fls{})\} \\
  \end{tabbing}
\end{subfigure}%
\begin{subfigure}[t]{\textwidth}
  \begin{tabbing}
    \\
    \\
    \qquad \quad \= $\aV{} \rightarrow$ \tru{} \\
    \\
    \quad $C_{FF}$ = \{(\AV{}, \fls{}), (\BV{}, \tru{})\} \\
  \end{tabbing}
\end{subfigure}%
\begin{subfigure}[t]{\textwidth}
  \begin{tabbing}
    \qquad \quad \= $\iV{} \rightarrow$ 0 \\
    \> $\cV{} \rightarrow$ 1 \\
    \\
    \\
    \quad $C_{FT}$ = \{(\AV{}, \tru{}), (\BV{}, \fls{})\} \\
  \end{tabbing}
\end{subfigure}%
\begin{subfigure}[t]{\textwidth}
  \begin{tabbing}
    \qquad \quad \= $\iV{} \rightarrow$ 0 \\
    \> $\cV{} \rightarrow$ 0 \\
    \\
    \> $\bV{} \rightarrow$ -10 \\
    \quad $C_{TT}$ = \{(\AV{}, \tru{}), (\BV{}, \tru{})\} \\
  \end{tabbing}
\end{subfigure}%
  \caption{Possible plain models for variants of $\kf{f}$.}%
  \label{fig:vsmt:models:plain}
\end{figure}
\begin{figure}[h]
  \centering
  \begin{subfigure}[t]{\textwidth}
  \begin{tabbing}
  \qquad \qquad \= $\_Sat \rightarrow (\neg \AV{} \wedge \neg \BV{}) \vee (\neg \AV{} \wedge \BV{}) \vee (\AV{} \wedge \neg \BV{}) \vee (\AV{} \wedge \BV)$ \\
  \> \iV{}\quad\hspace{1.7ex}\=$\rightarrow$ ($\kf{ite}$ ($\AV{} \wedge \BV{}$)\; \=0 \\
  \> \> \> ($\kf{ite}$\; ($\AV{} \wedge \neg \BV{}$)\; \=0 \\
  \> \> \> \> ($\kf{ite}$\; ($\neg \AV{} \wedge \neg \BV{}$)\; -1 $\kf{Undefined}$))) \\

  \> \cV{}\quad\hspace{1.7ex}\=$\rightarrow$ ($\kf{ite}$ ($\AV{} \wedge \BV{}$)\; \=0 \\
  \> \> \> ($\kf{ite}$\; ($\AV{} \wedge \neg \BV{}$)\; \=1 \\
  \> \> \> \> ($\kf{ite}$\; ($\neg \AV{} \wedge \neg \BV{}$)\; 0 $\kf{Undefined}$))) \\

  \> \aV{}\quad\hspace{1.7ex}\=$\rightarrow$ ($\kf{ite}$ ($\neg \AV{} \wedge \BV{}$)\; \tru{} $\kf{Undefined}$)\\
  \> \bV{}\quad\hspace{1.7ex}\=$\rightarrow$ ($\kf{ite}$ ($\AV{} \wedge \BV{}$)\; -10 $\kf{Undefined}$)
\end{tabbing}
\end{subfigure}

  \caption{Variational model corresponding to the plain models in
    \autoref{fig:vsmt:models:plain}.}%
  \label{fig:vsmt:models:var}
\end{figure}

\autoref{fig:vsmt:models:plain} show possible plain models for $\kf{f}$ with the
corresponding variational \ac{smt} model display in
\autoref{fig:vsmt:models:var}. We've added line breaks to emphasize the branches
the $\kf{then}$ and $\kf{else}$ branches of the $\kf{ite}$ SMTLIB2 primitive.

This formulation maintains the user requirements of the model. We maintain a
special variable $\_Sat$ to track the variants that were found satisfiable. In
this case all variants are satisfiable and thus we have four clauses over
dimensions in disjunctive normal form. If a user has a configuration then they
only need to perform substitution to determine the value of a variable under
that configuration. For example, if the user were interested in the value of
\iV{} in the $\{(\AV{}, \tru{}), (\BV{}, \tru{})\}$ variant they would
substitute the configuration into \vc{\iV{}} and recover 0 from the first
$\kf{ite}$ case. To find the variants at which a variable has a value a user can
employ a \ac{smt} solver, add \vc{\iV{}} as a constraint, and query for a model.

This also maintains the desirable properties of variational \ac{sat} models
while allowing any type known to the \ac{smt} solver. The variational \ac{smt}
model does not require knowledge of choice calculus or variation, it is still
monoidal, although not a commutative monoid, and can be built in any order as
long as there are no duplicate variants; a scenario that is impossible by the
property of synchronization on choices.

However, variational \ac{sat} models clearly compressed results by preventing
duplicate values with constant variables. In contrast, the variational \ac{smt}
model allows for duplicate values as long as those values are parameterized by
disjoint variants. For example, both \iV{} and \cV{} contain duplicate values,
but only one: \iV{} is easy to check in $O(1)$ time as the duplicates are
sequential in \vc{\iV{}} and can thus be checked during model construction. Such
a case would be easily avoided by tracking the set of all values a variable has
been assigned in all variants. However, we chose to keep variational models as
simple as possible and therefore only present the minimum required machinery.

\subsection{Requirements and Design Principles}
As before the items that are completed are marked with a \checkmark{}. The final
thesis will address each item:
\begin{enumerate}
\item The user must be able to incrementally add to a variational core. This
  item recovers some of the incrementality lost from the synthesis of a \ac{vpl}
  formula. I hypothesize two possible solutions: Given a variational core, a
  user can add new clauses to the core under the condition that no bound
  variable is removed in the new variational core. We require this constraint
  because if a user attempted to remove a variable in the variational core then
  the solver would need to unpack the symbolic references which could lead to
  unsound results. A second method for incrementality comes from work on
  CLU\cite{10.1145/69622.357182}, under this method one would derive a
  variational core and then serialize or marshal it to disk, effectively caching
  the core, and the solver state, for future use or transmission to another
  solver instance.
\item \checkmark{} Provide a \emph{general} method to solve a variational core.
  A variational \ac{smt} solver can be extended with many background theories.
  Depending which theories the user requires, many possible formulas with
  varying types (or \emph{sorts} in the \ac{sat} literature) and operators can
  be present in a query formula and thus in the variational core. Hence, the
  variational \ac{smt} solver must have a general method to reason about these
  types and operations. The improvement is to use a Huet zipper~\cite{huet_1997}
  to capture operators as a context, such as negation, as an Algebraic Data
  Type. With a zipper, one can traverse the variational core and delay the
  semantic of the operator thus processing the rest of the tree to a symbolic
  variable. Then, evaluating with any operator is reduced to only the
  denotational semantics of the operator, preserving compositionality and
  completeness.
\item \checkmark{} The variational \ac{smt} solver should be able to solve a
  query formula concurrently or in a single-threaded mode. This item change
  requires changing the semantics of choices. A benefit of the static and
  explicit approach of representing variation using the choice calculus is that
  we can alter the denotational semantics of a choice very easily. In the
  prototype \ac{sat} solver, the semantics of a choice was wrapping both
  alternatives with a \texttt{push} and \texttt{pop} call, for the variational
  \ac{smt} solver, the semantics of a choice are extended to also capture the
  solver state and transmit a continuation over an asynchronous
  channel~\cite{Marlow2012} to a worker thread. In order to ensure sound
  results, we exploit the monoidal design of variational models to ensure that
  the variational model is insensitive to the order results are produced from
  the base solver.
\end{enumerate}

\subsection{Research Plan}
While I have made significant progress in pursuit of the aforementioned design
goals. Several deliverables are still to be completed:

\paragraph{Optimizations based on nanopass compiler research} In the worst case,
a variational core will be evaluated $2^{|\kf{D}|}$ times where $|\kf{D}|$ is
the number of unique dimensions in the query formula. Therefore, any
optimizations that can simplify the variational core once are likely to have an
observable impact on performance. We employ a nanopass compiler architecture to
increase the flexibility of the approach, since varying background theories are
likely to produce different optimizations. This work has not begun but is
architected for in the prototype variational \ac{smt} solver. The proposed
thesis will answer the following research questions:
\begin{enumerate}
\item Which optimizations produce a corresponding effect on runtime performance
  for both real world case studies and random \ac{smt} problems?
\item For any given optimization, what is the magnitude
  of the effect?
\item Is the effect on performance sensitive to the order optimizations are
  applied?
\end{enumerate}

\paragraph{Evaluation of solver performance}
Previous work on the variational \ac{sat} solver~\cite{10.1145/3382025.3414965}
used two real-world case studies~\cite{MNS+:SPLC17} from the \ac{spl} community
for an empirical semantic. Thus, while these case studies are representative of
practical use cases more evaluation could be done. The proposed thesis will
perform a more robust evaluation and will consist of a mix of real-world and
randomly generated data.

Specifically I will reuse the aforementioned case studies, both of which include
\ac{smt} versions, and will use a forked version of a variational type checker,
TypeChef~\cite{KKHL:FOSD10}, to log \ac{sat} problems produced by lexing,
parsing and type checking the Linux kernel~\cite{linux} and
Busybox~\cite{busybox} open source projects. The forked version of TypeChef is
complete but does not scale for the Linux kernel at time of this writing, and
thus requires a refactor to use a more sophisticated logging infrastructure such
as a database.

With this data, a convincing evaluation of the variational \ac{smt} solver
architecture and design is possible. However, the majority of data will only
exercise the solver for Boolean propositional formulas. To test the architecture
in the \ac{smt} use case, the study will use the \ac{smt} versions of the
previous case studies and will generate mixed\cite{Gent94thesat} random \ac{smt}
problems according to established methods in the random-\ac{sat} community.

This data serves several other purposes. It will be used to evaluate performance
of the encoding strategies from \autoref{encoding-strat-deliverable} on the
prototype \ac{smt} solver. It will serve as the test data to assess the impact
of sharing on solver performance. The non-random portion will be made publicly
available as a dataset of real-world variational \ac{sat} and \ac{smt} problems,
and as a dataset of related \ac{sat} problems, thus increasing the impact of
this thesis.

\paragraph{Proof of variation preservation}
A proof of variation preservation is a key contribution of the thesis. This work
is partially complete with a proof of progress for accumulation and evaluation,
and a lemma that plain formulas always result in a \unit{} value in Agda. The
strategy is to show progress and preservation over the inference rules deriving
and solving a variational core. Implicit in this is to show variational
preservation for constructing a \ac{vpl} formula. There properties left to prove
are as follows:
\begin{enumerate}
% \item \emph{A minimal variational core has a normal form}. I conjecture that a
%   minimal (in the sense of least terms) variational cores consists only of
%   choices, joined by disjunctions and at most a symbolic term. By the operations
%   defined in \autoref{sec:vsat}, a disjunction is the only connective in the
%   intermediate language that cannot be removed. Similarly choices are the only
%   values which are not accumulated or evaluated. Thus, one can imagine a general
%   case where a minimal variational core---minimal with respect to the encoding of
%   choices---can be found by converting the core into disjunctive normal form and
%   accumulating any remaining symbolic terms into

%   composed of conjunctions, disjunctions,
%   symbolics and choices, but a more minimal core might be possible where all
%   conjunctions are removed, the core is reordered and any remaining symbolic
%   terms are further accumulated.
\item \emph{Encoding preservation: An encoding algorithm preserves the
    uniqueness of variants}. In other words whichever encoding algorithm is used
  to synthesize the set of problems to \ac{vpl}, that encoding algorithm does
  not lose variants.
\item \emph{Variational core progress: Given a \ac{vpl} formula a variational
    core is always derivable}
\item \emph{Variational core preservation: For a set of \pl{} problems, the
    corresponding variational core can recover that set of problems and if a
    problem in the initial set was satisfiable, then the corresponding variant
    is also satisfiable}. For any variational core, the set of \ac{sat} or
  \ac{smt} problems can be recovered by enumerating all variants with a total
  configuration and substituting symbolic terms with their concrete
  representations. The resulting formula should then be semantically equivalent
  to the original \ac{sat} or \ac{smt} problem. This item is a direct
  reproduction of the quick-check properties that verify the prototype \ac{sat}
  and \ac{smt} solver's as sound.
\item \emph{Variational model Preservation: A variational model produces a
    satisfying assignment for every satisfiable variant}. For any variant, a
  variational model always results in a satisfying assignments to the
  corresponding \ac{sat} or \ac{smt} problem. Similar to the previous item this
  is the second half of
\end{enumerate}


%%% Local Variables:
%%% mode: latex
%%% TeX-master: "../main"
%%% End:


\section{Related Work}
~\label{sec:related-work}
Related work has been discussed throughout the previous sections. This section
collects related work that was not previously covered. To my knowledge, this
work is the first to translate the specific ideas of a nanopass architecture,
and a variational compiler, to the \ac{sat}/\ac{smt} domains. Furthermore, at
time of this writing this work is the first to investigate variational
concurrency.

% \paragraph{Incremental Solving}
% The main goal of variational solving is to solve multiple satisfiability
% problems efficiently. In the SAT community, this problem was mainly addressed
% before by means of incremental solving, which is supported by many modern SAT or
% SMT solvers. With incremental solving, we can feed a solver with formulas,
% compute satisfiability, and then compute satisfiability again after adding or
% removing formulas. This way, the solver can reuse learned knowledge about the
% part of the formulas that remains in the solver. However, when applying
% incremental solving in such an adhoc fashion, it is likely that the full reuse
% potential is not exploited. In contrast, with variational solving, we create a
% variational formula first, such that variation becomes explicit and can be
% utilized as all variation points are known prior to the solving. While we have
% experienced speed-ups of variational solving over incremental solving, it
% requires that all those satisfiability problems are known in advance.

% \paragraph{Similar Solvers, Related Techniques}
%
This work is most similar to \cite{Visser:2012:GRR:2393596.2393665}, which also
constructs a \ac{sat} solver that exploits shared terms and prevents redundant
computation. However, the projects differ in important ways. Visser et al.'s
solver is oriented for program analysis and does not use incremental \ac{sat}
solving. Rather, it uses heuristics to find canonical forms of sliced programs,
and caches solver results on these canonical forms in a key-value
store~\cite{redis}. In contrast, variational \ac{sat} solving is domain
agnostic, solves \ac{sat} problems expressed in \ac{vpl}, returns a variational
model, and uses incremental \ac{sat} solving.


Variational \ac{sat} solving is the latest in a line of work that uses the
choice calculus to investigate variation as a computational phenomena. The
choice calculus has been successfully applied to diverse areas of computer
science, such as databases~\cite{ATW17dbpl,ATW18poly},
graphics~\cite{ES18diagrams}, data
structures~\cite{MMWWK17vamos,Walk14onward,SE17fosd,EWC13fosd}, type
systems~\cite{CCEW18popl,CCW18icfp,CEW12icfp,CEW14toplas}, error
messages~\cite{CES17jvlc,CE14popl,CEW12icfp,CES14hcc}, and now satisfiability
solving.
%
Our use of choices is similar to the concept of
\textit{facets}~\cite{austin2012multiple} and \textit{faceted
execution}~\cite{Schmitz2018FacetedSM,Micinski2018AbstractingFE,10.1145/2465106.2465121},
which have been successfully applied to information-flow security and
policy-agnostic programming.
%
% Facets, like choices, provide a syntactic form to parameterize a value or
% expression in a language, but differ from choices by having specific meaning
% in the domain of information security, namely to hide information between
% public and private information consumers. Choices, in contrast, are given
% meaning by the object language (i.e., the application domain.), thus one can
% view choices as a syntactic abstraction of facets, and may recover faceted
% execution through the construction of a variational interpreter.

The use of compiler optimization techniques in \ac{sat} solvers is not novel,
for example \ac{cse} and variable elimination has been successfully implemented
in \ac{sat} solvers\cite{10.1007/11499107_5,10.1007/978-3-319-23219-5_23}. Other
pre-processing techniques such as removing blocked
clauses~\cite{10.5555/1928380.1928406}, and detecting
autarkies~\cite{10.1007/978-3-540-79719-7_18} have been very successful at
automatically increasing performance of the \ac{sat} or \ac{smt} solver. From
this perspective, our use of choices is a pre-processing technique to speed up
incremental solving over sets of related problems.

Some solvers such as z3~\cite{10.1007/978-3-540-78800-3_24}, allow users to
program the heuristics used by the solver to find choose efficient solving
techniques, z3 in particular calls these constructs \emph{strategies}.
Strategies are thus similar to many optimizing compiler
techniques~\cite{10.5555/6448}, only applied to the \ac{sat} domain

% \paragraph{Applications for Variational Solving}
%
% \itodo{Thomas: Ideally, we would give an overview on applications that could
% benefit from incremental/variational solving. For example: program analysis
% \cite{VGD:FSE12}. Another application, namely reasoning about software
% variability, is described below in more detail.}
%
% Software variability, as explored in this paper, is a natural application domain
% for our work. The variability of SPLs or configurable software is often reduced
% to propositional logic~\cite{B05,CW07,MWCC08} for analysis
% purposes~\cite{BSRC10,TAK+:CSUR14,GBT+19}. Many analyses have been implemented
% using \ac{} solving~\cite{TAK+:CSUR14}, including feature-model
% analysis~\cite{BSRC10,GBT+19}, parsing~\cite{KGR+:OOPSLA11}, dead-code
% analysis~\cite{TLSS:EuroSys11}, code simplification~\cite{RGA+:ICSE15}, type
% checking~\cite{TBKC07}, consistency checking~\cite{CP06}, dataflow
% analysis~\cite{LKA+:ESECFSE13}, model checking~\cite{CCS+13}, variability-aware
% execution~\cite{NKN:ICSE14}, testing~\cite{MMCA:IST14}, product
% sampling~\cite{MKR+:ICSE16,VAT+:SPLC18}, product
% configuration~\cite{SIMA:ASE13}, optimization of non-functional
% properties~\cite{SRK+:SQJ12}, and variant-preserving
% refactoring~\cite{FMS+:SANER17}. While each of these analyses gives rise to
% multiple \ac{sat} problems for even a single analysis run, the authors typically
% do not discuss how they are solved. We argue that many could benefit from
% variational solving.

% More generally, any scenario that involves solving many related \ac{sat}
% problems, and where all of these problems are known or can be generated in
% advance, is a potential application for variational \ac{sat} solving.
% %
% Such situations arise in program analysis~\cite{VGD:FSE12}, and especially in
% \emph{speculative} program analyses that involve generating and exploring huge
% numbers of variations of a program, for example, as in
% counterfactual~\cite{CE14popl} and migrational~\cite{CCW18icfp,CCEW18popl}
% typing. Furthermore, we believe that variational solving provides a basis for
% such speculative analyses on feature models.
%
% where Generating type-error suggestions can be reduced to a variational
% \ac{} problem with choices isolating describe variants where a suggestion is
% applied. \itodo{jeff: other applications: variational route planning,
% variational circuit layout, variational (or counterfactual) scheduling}

%%% Local Variables:
%%% mode: latex
%%% TeX-master: "../main"
%%% End:

\section{Research plan}
~\label{sec:research-plan}
\autoref{sec:prop-contr} lists the deliverables for this thesis. Each item has
been discussed in more depth in previous sections. This section summarizes these
items and provides an itemized list of each deliverable, broken down into
discrete tasks, and their corresponding time estimations. The time estimates are
guidelines with the real schedule being dictated by our publication schedule.

\subsection{Encoding strategies of \ac{vpl} formulae}
The naive and naive-with-optimizations encoding algorithms are complete. The
remaining work is a literature review for other representations of boolean
formulae which may be useful, and to implement a Huffman based encoding
algorithm and assess its performance:

\begin{enumerate}
\item Construct a list of alternative representations of boolean formulae which
  may lend itself to encoding strategies
\item\label{huffman-todo} Implement the Huffman coding algorithm over boolean
  formulae, recursively merge formulas into a \ac{vpl} formula based on a
  similarity metric.
\end{enumerate}

\textbf{Estimated time to complete: 1-2 weeks}

\subsection{A method to determine the hardness of \ac{vpl} formulae}
The time consuming parts of this item: creating a random generator suitable for
the analysis and creating a benchmark platform to test are complete. The only
items remaining are to perform the analysis.

\begin{enumerate}
\item\label{phase-todo} Review literature from random-ac{sat} community on observing phase
  transition for both random $k$-\ac{sat} and mixed-\ac{sat} problems.
\item Replicate the transition detection analysis from \autoref{phase-todo}.
\end{enumerate}

\textbf{Estimated time to complete: 1 week}


\subsection{Variational \ac{smt} Solving}
A prototype asynchronous variational \ac{smt} solver is complete and operational
but several auxiliary tasks are still required. I expect this item to require
the majority of time besides writing the the proposed thesis.

\begin{enumerate}
\item\label{ui-todo} Implement an enhanced user interface which treats the
  solver like a server rather than a batch process.
\item\label{vpl-todo} Write the formal grammar for \ac{smt}-flavored \ac{vpl}.
  The current formalization exists in Haskell code and thus must be translated
  into a publishable format.
\item\label{vmodel-todo} Write the formalization for \ac{smt}-flavored variational models. An
  identical item to \autoref{vpl-todo} only with variational models.
\item \label{solve-todo} Write the denotational semantics for \ac{smt}-flavored
  variational solving algorithm. An identical item to \autoref{vpl-todo} only
  for the solving algorithm.
\item \label{proof-todo} Construct a proof for variation preserving semantics
  using the formalizations in items 2-4.
\item\label{nano-review-todo} Review literature on nanopass compilers to
  construct a set of possible optimizations.
\item\label{opts-todo} Implement optimizations from \autoref{nano-review-todo}
\item \label{conf-todo} Implement configurations of the variational solver to
  provide an interface suitable for benchmarking.
\end{enumerate}

Estimated time to complete 1-4: 1-2 weeks

Estimated time to complete \autoref{proof-todo}: 4 weeks (concurrent with other work)

Estimated time to complete items 6-7: 2 weeks

\textbf{Total estimated time to complete: 6-8 weeks}

\subsection{Assembling Case Study Data}
Most of the work left to assemble the case study data involves refactoring and
running the forked TypeChef system on the Linux kernel; the Busybox case study
dataset is finished. Parsers for the \ac{spl} case studies and the Busybox/Linux
results are already complete and have been tested.

\begin{enumerate}
\item\label{tc-todo} Implement a database caching system in TypeChef to record \ac{sat}
  problems at scale
\item\label{run-linux-todo} Run the Linux case study to generate \ac{sat}
  problems.
\item\label{eval-linux-todo} Evaluate the Linux and Busybox case studies using
  encoding strategies from \autoref{huffman-todo}.
\end{enumerate}

Estimated time to complete \autoref{tc-todo}: 2-3 weeks (with help from
collaborator Paul Maximillian Bittner)

Estimated time to complete \autoref{run-linux-todo}: 1 week

Estimated time to complete \autoref{eval-linux-todo}: 1-2 weeks

\textbf{Total estimated time: 4-6 weeks}


%%% Local Variables:
%%% mode: latex
%%% TeX-master: "../main"
%%% End:


\section{Conclusion}
~\label{sec:conclusion}
\ac{sat} and \ac{smt} solvers are ubiquitous and powerful tools in computer
science and software engineering. Incremental \ac{sat} and \ac{smt} provide an
interface that supports solving many related problems efficiently. However, the
interface could be automated and improved.

The goal of this thesis is to explore the design and architecture of a
variational satisfiability solver that automates and improves on the incremental
\ac{sat} interface. Through the application of the choice calculus the interface
can be automated for satisfiability problems, the solver interaction can be
formalized and made asynchronous, and the solver is able to directly express
variation in a problem domain.

The thesis will present a complete approach to variational satisfiability and
satisfiability modulo theory solving based on incremental solving. It will
include a method to automatically encode a set of Boolean formulae into a
variational propositional formula. A method for detecting the difficulty of
solving such a variational propositional formula. A data set suitable for future
research in the \ac{sat}, \ac{spl} and variation research communities. A
variational satisfiability solver, an asynchronous variational satisfiable
modulo theory solver and a proof of variational preservation.

%%% Local Variables:
%%% mode: latex
%%% TeX-master: "../thesis"
%%% End:

%%%%%%%%%%%%%%%%%%%%%%%%%%%%%%% Acronyms %%%%%%%%%%%%%%%%%%%%%%%%%%%%%%%%%%%%%%%
\begin{acronym}[]
  \acro{cnf}[CNF]{conjunctive normal form}
  \acro{dmf}[DMF]{disjunctive minimal form}
  \acro{dnf}[DNF]{disjunctive normal form}
  \acro{dsl}[DSL]{domain specific language}
  \acro{bdd}[BDD]{Binary Decision Diagram}
  \acro{sat}[SAT]{satisfiability solving}
  \acro{smt}[SMT]{satisfiability-modulo theories}
  \acro{spl}[SPL]{software product-lines}
  \acro{cdcl}[CDCL]{conflict-driven clause learning}
  \acro{dpll}[DPLL]{Davis-Putnam-Logemann-Loveland}
  \acro{cs}[CS]{clause sharing}
  \acro{vpl}[VPL]{variational propositional logic}
  \acro{cse}[CSE]{common subexpression elimination}
\end{acronym}

%%%%%%%%%%%%%%%%%%%%%%%%%%%% Bibilography %%%%%%%%%%%%%%%%%%%%%%%%%%%%%%%%%%%%%%%
\begin{footnotesize}
  \bibliographystyle{unsrt}
  \bibliography{bib/abrv, bib/jeff,bib/eric,bib/martin,bib/spl,bib/satsolvers}
\end{footnotesize}
%%%%%%%%%%%%%%%%%%%%%%%%%%%% Bibilography %%%%%%%%%%%%%%%%%%%%%%%%%%%%%%%%%%%%%%%

\end{document}
